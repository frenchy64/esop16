\section{Experience}

Typed Clojure has been used in production Clojure systems for 2 years.

%\subsection{Let-aliasing}
%
%\begin{mathpar}
%  \footnotesize
%\infer [T-LocalAlias]
%{ \Theta[\x{}] = \object{}
%  \\
%  \inpropenv {\propenv{}} {\isprop {\t{}} {\object{}}}
%  \\\\
%  \s{} = {\falsy} }
%{ \judgement {\Theta; \propenv{}} 
%             {\hastype {\x{}} {\t{}}}
%             {\filterset {{\notprop {\s{}} {\object{}}}} {{\isprop {\s{}} {\object{}}}}}
%             {\object{}}
%                   }
%
%\infer [T-LetAlias]
%{ \judgement {\Theta; \propenv{}} {\hastype {\e{1}} {\s{}}} {\filterset {\thenprop {\prop{1}}} {\elseprop {\prop{1}}}}
%             {\object{1}}
%  \\\\
%  \object{1} \notequal \emptyobject{}
%  \\\\
%  \judgement
%       {\Theta[\x{} \mapsto \object{1}];
%         \propenv{}}
%             {\hastype {\e{}} {\t{}}} {\filterset {\thenprop {\prop{}}} {\elseprop {\prop{}}}}
%             {\object{}} 
%             }
%{ \judgement {\Theta; \propenv{}} {\hastype {\letexp {\x{}} {\e{1}} {\e{}}} {\t{}}}
%             {\filterset {\thenprop {\prop{}}} {\elseprop {\prop{}}}}
%             {\object{}} 
%             }
%\end{mathpar}

\subsection{HVec dispatch}

\subsection{Higher-rank polymorphism}

Typed Clojure supports f-bounded polymorphism and higher-rank
types.
% why include them?
% examples, involving generic type operators for monads & conduits
% useful for generic programming, similar to Hashell type classes

\subsection{Using negative filters}

Occurrence typing plays an important role in Typed Racket and Typed Clojure.
By maintaining a \emph{proposition environment} of propositions relating types to
bindings, we can update bindings with more accurate types as programs progress.
It follows that there is some correspondence between propositions and types,
characterised by the \emph{update} function, which takes a type and a proposition
and returns a type which updates the input type using the proposition.

There is a straightforward relationship between ``positive'' propositions and types.
For example 
{\tt (update Number (is Integer 0))}
updates Number by Integer, which is Integer, because Integer <: Number.

The relationship between ``negative'' propositions and types is not always obvious.
A common proposition in Typed Clojure is (! (U nil false) a): the proposition that
local binding ``a'' is \emph{not} of type (U nil false).
This problem is most visible in expressions like {\tt (filter identity coll)}, where
``identity'' has a ``then'' proposition that has negative information: (! (U nil false) 0),
which reads, the 0th argument of identity does not contain (U nil false).

\subsubsection{Arrays}
\label{sec:arrays}

Supporting statically sound interactions with Java arrays is a goal
of Typed Clojure. This is complicated by Java's decision to make
arrays covariant in their argument, a well documented source of static
unsoundness. Bracha~\cite{Bra98} summarises Java's approach to maintaining
soundness at runtime, which involves all array writes being checked by
runtime assertions.

This approach fits Java's type system, but we can do better in a more powerful
type system like Typed Clojure. Our goal is to catch all type-incorrect array
writes at compile time so the type system can do more to help Clojure programmers
use arrays, especially those being passed from foreign Java code.

Our basic approach is to make our array types \emph{bivariant}. Array types
look like {\ArrayTwo {\t{w}} {\t{r}}} and
are reminiscent of functions or pipes: having a contravariant parameter for input (writing)
and a covariant parameter for output (reading).
This type can write type {\t{w}} and read type {\t{r}}.

Most commonly, an array type is invariant in its parameter; it can
write and read input of the same type.
We can get the same effect by setting our input and output
parameters to the same type. For example, {\ArrayTwo {\Number} {\Number}}
(or equivalently, {\Array {\Number}})
in Typed Clojure is similar to invariant array types of \Number in languages like Scala.

The biggest gain in using a separate input parameter is the ability
to specify \emph{read-only} arrays. Crucially, our type system features an
explicit bottom type \lstinline|Nothing|, enabling a read-only \lstinline|Number| array
to be of type \lstinline|(Array2 Nothing Number)|.

To realise why defining read-only arrays are useful, we need to examine
what makes array covariance unsound in Java.
\begin{verbatim}
FIXME
Array covariance about the type of an array so the consumer
of an array cannot tell the actual type of the array when examining a type
signature.
\end{verbatim}

\begin{lstlisting}
...
public static Number[] getNumberArray() {
  Number[] n = new Integer[10];
  return n;
}
...
\end{lstlisting}

To the casual consumer \emph{getNumberArray} returns an array that can both
read and write \lstinline|Number|s. However it is clear from the implementation
that attempting to write say a \lstinline|Double| to this array will result
in a runtime error.

\begin{verbatim}
...
Number[] myArray = getNumberArray();
myArray[0] = 1.1;
/* Exception in thread "main" 
   java.lang.ArrayStoreException: 
   java.lang.Double */
...
\end{verbatim}

Notice that this is a runtime error, and Java's type system has not helped
statically prevent it.
This could cause a similar issue for other statically-typed languages offering
interoperability with Java. 

To prevent these sorts of runtime exceptions in Typed Clojure, we declare
all arrays from unknown sources to be \emph{read-only}. Put differently,
the only way to define a writeable array is to create it in type-checked Clojure
code.

\begin{lstlisting}
(let [n (CovariantArray/getNumberArray)]
  (aset n 0 1.1))

; Polymorphic static method clojure.lang.RT/aset could not be 
; applied to arguments:
; Domains: 
;         (Array2 i o) clojure.core.typed/AnyInteger i
; 
; Arguments:
;         (Array2 Nothing java.lang.Number) int (Value 1.1)
; 
; with expected type:
;         Any
\end{lstlisting}

The type inferred for the local \lstinline|n| is \lstinline|(Array2 Nothing Number)|
which tells the type system: it is never safe to write to this array, but
it is safe to assume \lstinline|Number|s can be read from this array.

To emphasise, Typed Clojure throws a static type error. Errors like this help Clojure programmers
use foreign Java libraries more correctly.

\begin{verbatim}
Note that Java libraries are often large 
and complex and programmers will probably
enjoy the extra help from the type system.
\end{verbatim}
