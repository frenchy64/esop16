\section{Experience}
\label{sec:experience}

Typed Clojure is implemented as a Clojure library---\coretyped{}.
In contrast to Racket, Clojure does not provide extension
points to the macroexpander. 
To satisfy our goals of providing
Typed Clojure as a library that works with the latest version of the Clojure
compiler, \coretyped{} is implemented as an external static analysis pass
that must be explicitly invoked by the programmer, and not as an
integral part of the Clojure compilation process. 
%Therefore, \coretyped{} is in a sense a linter.

This means that type checking is  optional. 
On the positive side, \coretyped{} is flexible to the needs of a dynamically
typed programmer, encouraging experimentation with programs that may not
type check.
On the negative side, programmers must remember to type check their namespaces.
Also, the compiler cannot depend on types, making
type-based optimisation is impossible. 
%If this were not the case, we could dispose of type-hints
%altogether, and simply use static types to resolve reflection.

%\subsection{Let-aliasing}
%
%\begin{mathpar}
%  \footnotesize
%\infer [T-LocalAlias]
%{ \Theta[\x{}] = \object{}
%  \\
%  \inpropenv {\propenv{}} {\isprop {\t{}} {\object{}}}
%  \\\\
%  \s{} = {\falsy} }
%{ \judgement {\Theta; \propenv{}} 
%             {\hastype {\x{}} {\t{}}}
%             {\filterset {{\notprop {\s{}} {\object{}}}} {{\isprop {\s{}} {\object{}}}}}
%             {\object{}}
%                   }
%
%\infer [T-LetAlias]
%{ \judgement {\Theta; \propenv{}} {\hastype {\e{1}} {\s{}}} {\filterset {\thenprop {\prop{1}}} {\elseprop {\prop{1}}}}
%             {\object{1}}
%  \\\\
%  \object{1} \notequal \emptyobject{}
%  \\\\
%  \judgement
%       {\Theta[\x{} \mapsto \object{1}];
%         \propenv{}}
%             {\hastype {\e{}} {\t{}}} {\filterset {\thenprop {\prop{}}} {\elseprop {\prop{}}}}
%             {\object{}} 
%             }
%{ \judgement {\Theta; \propenv{}} {\hastype {\letexp {\x{}} {\e{1}} {\e{}}} {\t{}}}
%             {\filterset {\thenprop {\prop{}}} {\elseprop {\prop{}}}}
%             {\object{}} 
%             }
%\end{mathpar}

%\subsection{Further Extensions}
%
%In addition to the key features we present in this paper,
%\coretyped{} supports other extensions to handle additional Clojure
%features. 
%
%\smallsection{Datatypes, Records and Protocols}
%Clojure features datatypes and protocols. Datatypes are Java classes
%declared final with public final fields. They can implement Java interfaces
%or protocols, which are similar to interfaces but already-defined classes
%and \nil{} may extend protocols.
%%
%Typed Clojure can reason about most of these features,
%including the ability to define polymorphic datatypes and protocols and
%utilising the Java type system to help check implemented interface methods.

%\smallsection{Intersection Types}
%Typed Clojure includes simple intersection types which do no sophisticated
%reasoning with the dual subtyping rules to unions.
%
%In some cases this makes types more expressive. Say we know \clj{x} has some
%universally quantified type \clj{a} and we learn \clj{x} is a \clj{Number}.
%Without intersection types, we must choose which piece of information to forget.
%In Typed Clojure, \clj{x} is simply of type \clj{(I x Number)}.
%
%\smallsection{Mutation and Polymorphism}
%Clojure supports mutable references with software-transactional-memory
%which Typed Clojure defines \emph{bivariantly}---with write and read type parameters
%as in the atomic reference \clj{(Atom2 Int Int)} which can write and read \clj{Int}.
%Typed Clojure also supports parametric polymorphism, including
%Typed Racket's variable-arity polymorphism~\cite{stf-esop}, 
%which enables us to assign a type to functions like \clj{swap!} (\figref{main:fig:swap!}),
%which takes a mutable \emph{atom},
%a function and extra arguments, and swaps into the atom the result of
%applying the function to the atom's current value and the extra arguments.
%
%\begin{figure}
%\begin{minted}{clojure}
%(ann clojure.core/swap! (All [w r b ...] 
%                          [(Atom2 w r) [r b ... b -> w] b ... b -> w]))
%(swap! (atom :- Num 1) + 2 3);=> 6 (atom contains 6)
%\end{minted}
%%\inputminted[firstline=5,lastline=5]{clojure}{code/demo/src/demo/atom.clj}
%\caption{Type annotation and example call of \clj{swap!}}
%\label{main:fig:swap!}
%\end{figure}

\subsection{Evaluation}
\label{sec:casestudy}

Throughout this paper, we have focused on three interrelated type
system features: heterogenous maps, Java interoperability, and
multimethods. Our hypothesis is that these features are widely used in
existing Clojure programs, in interconnecting way, and that handling
them as we have done is required to type check realistic Clojure
programs.

To evaluate this hypothesis, we analyzed two existing Typed Clojure
code bases, one from the open-source community, and one from a company
that uses Typed Clojure in production. For our data gathering, we
instrumented the Typed Clojure type checker to record how often
various features were used. 

\paragraph{feeds2imap}
feeds2imap is an open source library written in Typed Clojure. It provides an
RSS reader using the \emph{java.mail} framework.

% static call (:check/:static-call) = 74
% - user   5
% - inlined (:check/static-call-clojure-lang-probably-inline) 69
% static field = 13
% - user 1
% - inlined (:check/static-field-clojure-lang-probably-inline) 12
% new (:check/:new) = 11
% - user 6
% - inlined (:check/new-clojure-lang-probably-inline) 5
% instance call = 53
% - body  20
% - inlined (:check/instance-call-clojure-lang-probably-inline) 33
% total 151
% - user  32
Of 11 typed namespaces containing 825 lines of code,
there are 32 Java interactions.
The majority are method calls, consisting of 
20 (62\%) instance methods and 5 (16\%) static methods.
The rest consists of 1 (3\%) static field access, and
6 (19\%) constructor calls---there are no instance field
accesses.

%  from :check/find-val-type-with-hmap* numbers
There are 27 lookup operations on HMap types, of which 20 (74\%) resolve to
mandatory entries, 6 (22\%) to optional entries,
and 1 (4\%) is an unresolved lookup. 
No lookups involved fully specified maps.

% :collect/:def     93
% :check/checked-def 52
From 93 \clj{def} expressions in typed code, 52 (56\%) are checked,
with a rate of 1
Java interaction for 1.6 checked top-level definitions, and
1 HMap lookup to 1.9 checked top-level definitions.
That leaves 41 (44\%) unchecked vars, mainly due to partially complete 
porting to Typed Clojure, but in some cases due to unannotated third-party 
libraries.

No typed multimethods are defined or used. 
% :collect/defalias-is-HMap      7
% :invoke-special-collect/(quote clojure.core.typed/def-alias*)     18
Of 18 total type aliases,
7 (39\%) contained one HMap type, and none contained unions of HMaps---on further
inspection there was no HMap entry used to dictate control flow, often
handled by multimethods.
This is unusual in our experience, and
is perhaps explained by feeds2imap mainly wrapping
wrapping existing \emph{javax.mail} functionality.

\paragraph{CircleCI}
CircleCI %% \footnote{\url{https://circleci.com/}}
provides continuous integration services built with a mixture of open-
and closed-source.
Typed Clojure was used at CircleCI in production systems for two years \cite{CircleCIUsesTC},
maintaining
87 namespaces and 19,000 lines of code.
%an experience we summarise in \secref{sec:limitations}.
%
%CircleCI provided the first author access to the main closed-source backend system written in Clojure
%and Typed Clojure.
%We conducted a study of the effectiveness of Typed Clojure in practice.
%There is no clear metric for quantifying typed Clojure code, since untyped code
%can be freely mixed and some seemingly typed namespaces are not checked
%regularly. 
%We manually type checked all namespaces that depend on \clj{clojure.core.typed}
%and considered those with type errors as untyped.
%We then searched the remaining typed code for unsafe Typed Clojure operations like
%var annotations with \clj{:no-check} and the \clj{tc-ignore} macro,
%which instruct Typed Clojure to ignore the specified code,
%and also considered those untyped.
%Furthermore, we manually collected and inspected all top-level annotations and
%classified them.
%
%We determined that

%% Out of 588 top-level var annotations, 270 (46\%) were checked annotations of
%% functions defined in typed code,
%% 129 (22\%) annotations assigned types to external libraries 
%% and the remaining 189 (32\%) annotated `unchecked' user code.
%Some of the type-annotated definitions were so annotated by the first
%author and contributed back to CircleCI.
%HMaps were a valuable feature, with 38 (59\%) out of 64 total type aliases
%featuring them; see \egref{example:circleci} for an instance.
%
%Because of various shortcomings of \coretyped{}, all 57 \clj{defmethod}
%expressions in typed namespaces were unchecked.
%
%811 top-level var annotations
%
%% Due to a lack of checked multimethods,
%% the first author ported 11 previously-untyped multimethods to Typed Clojure, also checking 
%% 89 methods.

The CircleCI code base contains 11 checked multimethods.
 All 11 dispatch functions
are on a HMap key containing a keyword, in a similar style to
\egref{example:desserts-on-meal}.
Correspondingly, all 89 methods are associated with a keyword dispatch value.
The argument type was in all cases a single HMap type, however,
rather than a union type.
In our experience from porting other libraries, this is unusual.

% 87 typed namespaces
% :check/gen-analysis     87

% :check/find-val-type-with-hmap    328
% :check/find-val-type-with-hmap-present    208
% :check/find-val-type-with-hmap-with-optional     70
% :check/find-val-type-with-hmap-fall-through     30
% :check/find-val-type-with-hmap-absent     20
% :check/find-val-has-complete      2
% :merge/complete-used-on-right      5

Of 328 lookup operations on HMaps,
208 (64\%) resolve to mandatory keys,
70 (21\%) to optional keys,
20 (6\%) to absent keys, and
30 (9\%) lookups are unresolved.
%
% :collect/defalias-is-HMap     62
% :invoke-special-collect/(quote clojure.core.typed/def-alias*)     95 
Of 95 total type aliases defined with \clj{defalias},
62 (65\%) involved one or more HMap types.
%
%% :new-special/(quote clojure.lang.MultiFn)     11
%
%% :check/:static-call    525
%% :check/static-call-clojure-lang-probably-inline    499
%% = 26 user
%
%% :check/:instance-call    510
%% :check/instance-call-clojure-lang-probably-inline    474
%% = 36 user
%
%% :check/:new    159
%% :check/new-clojure-lang-probably-inline    121
%% = 38
%
%% :check/:static-field     92
%% :check/static-field-clojure-lang-probably-inline     87
%% = 5
%
%% 26 + 36 + 38 + 5 = 105
%
%% :invoke-special-collect/(quote clojure.core.typed/non-nil-return*) 35
%% :invoke-special-collect/(quote clojure.core.typed/nilable-param*)  1
%
Out of 105 Java interactions, 26 (25\%) are static methods, 36 (34\%)
are instance methods, 38 (36\%) are constructors, and 5 (5\%) are static
fields. 35 methods are overriden to return non-nil, and 1 method 
overridden to accept nil---suggesting that
\coretyped{} disallowing \clj{nil} as a method argument by default
is justified.

% :check/checked-def  407
% :check/checked-MultiFn-addMethod 57
% :instance-method-special/(quote clojure.lang.MultiFn/addMethod)     89
% = 464 checked definitions
Of 464 checked top-level definitions (which consists of
57 \clj{defmethod} calls and 407 \clj{def} expressions),
1 HMap lookup occurs per 1.4 top-level definitions,
and 1 Java interaction occurs every 4.4 top-level definitions.

% :check/def-not-checking-definition   1352
% :check/checked-def  407
% = 1759
% :collect/:def   1834
% = 1427 unchecked
From 1834 \clj{def} expressions in typed code,
%87 typed namespaces,
only 407 (22\%) were checked.
That leaves 1427 (78\%) which have unchecked definitions, either by an explicit \clj{:no-check} annotation
or \clj{tc-ignore} to suppress type checking,
or the \clj{warn-on-unannotated-vars} option, which skips \clj{def} expressions
that lack expected types via \clj{ann}.
From a brief investigation,
reasons include unannotated third-party libraries,
work-in-progress conversions to Typed Clojure,
unsupported Clojure idioms, 
and hard-to-check code.

\paragraph{Lessons}
Based on our empirical survey, HMaps and Java interoperability support
are vital features used on average more than once per typed
function. 
%
Multimethods are less common
in our case studies. The CircleCI code base contains only 26 multimethods,
a low number in our experience.

%The
%data therefore validates our choice of a type system that supports
%expressive multimethod definition and acknowledges the relationship
%between these seemingly-distinct features. 

%
%The other lesson from our case studies and from other interactions
%with Typed Clojure users, it is clear the main barrier to entry to
%Typed Clojure for large systems is the requirement to annotate
%functions outside the borders of typed code.  We hope that this
%can be addressed by making annotations available for popular
%libraries.

\subsection{Limitations}
\label{sec:limitations}

After a 2 year trial, our second case study  disabled type checking~\cite{CircleCIBlog}
because of issues with the Typed Clojure implementation.% We summarise the main issues.

\smallsection{Performance}
CircleCI reports a 3GHz i7 MacBook Pro takes 2 minutes
to check a single namespace. Typed Clojure does not cache results of
transitive dependencies for correctness, and we conjecture adding this feature will significantly
improve re-checking performance.
Preserving static soundness in the context of arbitrary code reloading is a largely unexplored area.

\smallsection{Library annotations}
Annotations for external code is rarely available, so a large part of the
untyped-typed porting process is reverse engineering libraries.

\smallsection{Unsupported idioms}
While the current set of features is vital to checking Clojure code, as we have seen
in \secref{sec:casestudy} there is still much work to do.
In practice, common Clojure functions are too polymorphic for the current implementation
or theory to account for. For example, \clj{comp} provides variable-arity function composition,
and further research is required to check callsites.

%\smallsection{Java Arrays}
%Java arrays are known to be statically unsound.
%\cite{Bra98} summarises the approach taken to regain runtime soundness, which involves
%checking array writes at runtime.
%
%Typed Clojure implements an experimental partial solution, making arrays \emph{bivariant},
%separating the write and read types into contravariant and covariant parameters.
%If the array originates from typed code, then we may track the write and read
%parameters statically. Currently arrays from foreign sources
%have their write parameter set to to \Bot{}, protecting typed code from writing
%something of incorrect type. However there are currently no casting mechanisms to 
%convince Typed Clojure the foreign array is writeable.

%\smallsection{Array-backed sequences}
%Typed Clojure assumes sequences are immutable. This is almost always true, however
%for performance reasons,
%sequences created from Java arrays (and Iterables) reflect future writes to the array 
%in the `immutable' sequence. While disturbing and a clear unsoundness in Typed Clojure,
%this has not yet been an issue in practice and is strongly discouraged as undefined behavior:
%``Robust programs should not mutate arrays or Iterables that have seqs on them.''~\cite{CljSeqDoc}.
%
%\smallsection{Typed-untyped interoperation}
%Currently, interactions between typed and untyped Clojure code are unchecked
%which can violate the expectations of Typed Clojure.
%Gradual typing~\cite{thf06,siek06:_gradual} ensures sound interoperability between typed and untyped code by enforcing
%invariants of the type system via run-time contracts.
% We hope to add support
%for gradual typing in the future.



%OLD

%\subsection{Using negative filters}
%
%Occurrence typing plays an important role in Typed Racket and Typed Clojure.
%By maintaining a \emph{proposition environment} of propositions relating types to
%bindings, we can update bindings with more accurate types as programs progress.
%It follows that there is some correspondence between propositions and types,
%characterised by the \emph{update} function, which takes a type and a proposition
%and returns a type which updates the input type using the proposition.
%
%There is a straightforward relationship between ``positive'' propositions and types.
%For example 
%{\tt (update Number (is Integer 0))}
%updates Number by Integer, which is Integer, because Integer <: Number.
%
%The relationship between ``negative'' propositions and types is not always obvious.
%A common proposition in Typed Clojure is (! (U nil false) a): the proposition that
%local binding ``a'' is \emph{not} of type (U nil false).
%This problem is most visible in expressions like {\tt (filter identity coll)}, where
%``identity'' has a ``then'' proposition that has negative information: (! (U nil false) 0),
%which reads, the 0th argument of identity does not contain (U nil false).
%
%\subsubsection{Arrays}
%\label{sec:arrays}
%
%Supporting statically sound interactions with Java arrays is a goal
%of Typed Clojure. This is complicated by Java's decision to make
%arrays covariant in their argument, a well documented source of static
%unsoundness. Bracha~\cite{Bra98} summarises Java's approach to maintaining
%soundness at runtime, which involves all array writes being checked by
%runtime assertions.
%
%This approach fits Java's type system, but we can do better in a more powerful
%type system like Typed Clojure. Our goal is to catch all type-incorrect array
%writes at compile time so the type system can do more to help Clojure programmers
%use arrays, especially those being passed from foreign Java code.
%
%Our basic approach is to make our array types \emph{bivariant}. Array types
%look like {\ArrayTwo {\t{w}} {\t{r}}} and
%are reminiscent of functions or pipes: having a contravariant parameter for input (writing)
%and a covariant parameter for output (reading).
%This type can write type {\t{w}} and read type {\t{r}}.
%
%Most commonly, an array type is invariant in its parameter; it can
%write and read input of the same type.
%We can get the same effect by setting our input and output
%parameters to the same type. For example, {\ArrayTwo {\Number} {\Number}}
%(or equivalently, {\Array {\Number}})
%in Typed Clojure is similar to invariant array types of \Number in languages like Scala.
%
%The biggest gain in using a separate input parameter is the ability
%to specify \emph{read-only} arrays. Crucially, our type system features an
%explicit bottom type \lstinline|Nothing|, enabling a read-only \lstinline|Number| array
%to be of type \lstinline|(Array2 Nothing Number)|.
%
%To realise why defining read-only arrays are useful, we need to examine
%what makes array covariance unsound in Java.
%\begin{verbatim}
%FIXME
%Array covariance about the type of an array so the consumer
%of an array cannot tell the actual type of the array when examining a type
%signature.
%\end{verbatim}
%
%\begin{lstlisting}
%...
%public static Number[] getNumberArray() {
%  Number[] n = new Integer[10];
%  return n;
%}
%...
%\end{lstlisting}
%
%To the casual consumer \emph{getNumberArray} returns an array that can both
%read and write \lstinline|Number|s. However it is clear from the implementation
%that attempting to write say a \lstinline|Double| to this array will result
%in a runtime error.
%
%\begin{verbatim}
%...
%Number[] myArray = getNumberArray();
%myArray[0] = 1.1;
%/* Exception in thread "main" 
%   java.lang.ArrayStoreException: 
%   java.lang.Double */
%...
%\end{verbatim}
%
%Notice that this is a runtime error, and Java's type system has not helped
%statically prevent it.
%This could cause a similar issue for other statically-typed languages offering
%interoperability with Java. 
%
%To prevent these sorts of runtime exceptions in Typed Clojure, we declare
%all arrays from unknown sources to be \emph{read-only}. Put differently,
%the only way to define a writeable array is to create it in type-checked Clojure
%code.
%
%\begin{lstlisting}
%(let [n (CovariantArray/getNumberArray)]
%  (aset n 0 1.1))
%
%; Polymorphic static method clojure.lang.RT/aset could not be 
%; applied to arguments:
%; Domains: 
%;         (Array2 i o) clojure.core.typed/AnyInteger i
%; 
%; Arguments:
%;         (Array2 Nothing java.lang.Number) int (Value 1.1)
%; 
%; with expected type:
%;         Any
%\end{lstlisting}
%
%The type inferred for the local \lstinline|n| is \lstinline|(Array2 Nothing Number)|
%which tells the type system: it is never safe to write to this array, but
%it is safe to assume \lstinline|Number|s can be read from this array.
%
%To emphasise, Typed Clojure throws a static type error. Errors like this help Clojure programmers
%use foreign Java libraries more correctly.
%
%\begin{verbatim}
%Note that Java libraries are often large 
%and complex and programmers will probably
%enjoy the extra help from the type system.
%\end{verbatim}
