% rubber: set arguments --shell-escape
% arara: pdflatex: {shell: true}
%-----------------------------------------------------------------------------
%
%               Template for sigplanconf LaTeX Class
%
% Name:         sigplanconf-template.tex
%
% Purpose:      A template for sigplanconf.cls, which is a LaTeX 2e class
%               file for SIGPLAN conference proceedings.
%
% Guide:        Refer to "Author's Guide to the ACM SIGPLAN Class,"
%               sigplanconf-guide.pdf
%
% Author:       Paul C. Anagnostopoulos
%               Windfall Software
%               978 371-2316
%               paul@windfall.com
%
% Created:      15 February 2005
%
%-----------------------------------------------------------------------------


\documentclass[preprint,9pt,twocolumn,nocopyrightspace,authoryear]{sigplanconf}

% The following \documentclass options may be useful:

% preprint      Remove this option only once the paper is in final form.
% 10pt          To set in 10-point type instead of 9-point.
% 11pt          To set in 11-point type instead of 9-point.
% authoryear    To obtain author/year citation style instead of numeric.

\usepackage{common}
\usepackage{minted}
\include{bibliography.bib}
\usepackage{chngcntr}
\usepackage{url}

\newcounter{ex}
\newenvironment{exmp}{
\refstepcounter{ex}%
\par\vspace{.8em}\hfill\framebox{\small Example~\theex}\nopagebreak
\\[-2.4em]\nopagebreak}
{\relax}


\begin{document}

\special{papersize=8.5in,11in}
\setlength{\pdfpageheight}{\paperheight}
\setlength{\pdfpagewidth}{\paperwidth}

\conferenceinfo{CONF 'yy}{Month d--d, 20yy, City, ST, Country} 
\copyrightyear{20yy} 
\copyrightdata{978-1-nnnn-nnnn-n/yy/mm} 
\doi{nnnnnnn.nnnnnnn}

\newcommand{\clj}[1]{\mintinline{clojure}{#1}}
\newcommand{\java}[1]{\mintinline{java}{#1}}
\newcommand{\rkt}[1]{\mintinline{racket}{#1}}

% Uncomment one of the following two, if you are not going for the 
% traditional copyright transfer agreement.

%\exclusivelicense                % ACM gets exclusive license to publish, 
                                  % you retain copyright

%\permissiontopublish             % ACM gets nonexclusive license to publish
                                  % (paid open-access papers, 
                                  % short abstracts)

%\titlebanner{banner above paper title}        % These are ignored unless
%\preprintfooter{short description of paper}   % 'preprint' option specified.

\title{Practical Optional Types for Clojure}

\authorinfo{}{}{\vspace*{-2mm}}
%\authorinfo{Ambrose Bonnaire-Sergeant}
%           {Indiana University}
%           {abonnair@indiana.edu}
%\authorinfo{Rowan Davies}
%           {University of Western Australia}
%           {rowan.davies@uwa.edu.au}
%\authorinfo{Sam Tobin-Hochstadt}
%           {Indiana University}
%           {samth@indiana.edu}

\maketitle

\begin{abstract}
Typed Clojure is an optional type system for Clojure, a dynamic
language in the Lisp family that targets the JVM. Typed Clojure
enables Clojure programmers to gain greater confidence in the
correctness of their code via static type checking while remaining in
the Clojure world, and has acquired significant adoption in the
Clojure community. Typed Clojure's type
system repurposes Typed Racket's
\emph{occurrence typing}, an approach to statically reasoning about
predicate tests, and also includes several new type system features to
handle existing Clojure idioms.

In this paper, we describe Typed Clojure and present these type system
extensions, focusing on three features widely used in Clojure. 
%
 First, multimethods provide extensible
operations, and their Clojure semantics turns out to have a surprising
synergy with the underlying occurrence typing framework.
%
Second, Java
interoperability is central to Clojure's mission but introduces
challenges such as ubiquitous \texttt{null}; Typed Clojure handles
Java interoperability while ensuring the absence of null-pointer
exceptions in typed programs. 
%
Third, Clojure programmers
idiomatically use immutable dictionaries for data structures; Typed
Clojure handles this with multiple forms of
heterogeneous dictionary types.
%

We provide a formal model of the Typed Clojure type system
incorporating these and other features, with a proof of
soundness. Additionally, Typed Clojure is now in use by numerous
corporations and developers working with Clojure, and we report on 
a quantiative analysis of the use of the features described in this
paper in two substantial code bases.


%% Clojure is a dynamically typed language hosted on the Java 
%% Virtual Machine.
%% Typed Racket is a valuable starting point for
%% a gradual type system that targets Clojure.
%% Building a similar type system for a new language gives the
%% designer some flexibility to repurpose and extend features.
%% This paper gives an overview of Typed Clojure, concentrating
%% on the extensions and differences from Typed Racket. We also
%% show where Typed Racket's features were particularly useful
%% for type checking non-trivial Clojure idioms.
\end{abstract}

%% \category{CR-number}{subcategory}{third-level}

%% \keywords
%% optional types, Clojure

\section{Clojure with static typing}

% current situation 

The popularity of dynamically-typed languages in software
development, combined with a recognition that types often improve
programmer productivity, software reliability, and performance, has
led to the recent development of a wide variety of optional and
gradual type systems aimed at checking existing programs written in
existing languages.  These include Microsoft's TypeScript for
JavaScript, Facebook's Hack for PHP and Flow for JavaScript, and MyPy
for Python among the optional systems, and Typed Racket, Reticulated
Python, and GradualTalk among gradually-typed systems.\footnote{We
  reserve the term ``gradual typing'' for systems such as Typed Racket which soundly
  interoperate between typed and untyped code; systems like Typed Clojure or TypeScript which do not
  enforce type invariants we describe as ``optionally typed''.}
%FIXME: add citations for all those systems.

One key lesson of these systems, indeed a lesson known to early
developers of optional type systems such as StrongTalk, is that type
systems for existing languages must be designed to work with the
features and idioms of the target language. Often this takes the form
of a core language, be it of functions or classes and objects,
together with extensions to handle distinctive language features.


We synthesize these lessons to present \emph{Typed Clojure}, an
optional type system for Clojure. Typed Clojure builds on the core
type checking approach of Typed Racket, an existing gradual type
system for Racket. However, Typed Clojure extends this basic framework
in multiple ways to accommodate the unique idioms and features of
Clojure, producing an expressive synthesis of ideas and demonstrating
a surprising coincidence between multiple dispatch in
Clojure and Typed Racket's  occurrence typing framework. 

The essence of Typed Clojure, of course, is Clojure, a dynamically
typed language in the Lisp family built to run on the Java Virtual
Machine (JVM) which has recently gained popularity as an alternative
JVM language.  It offers the flexibility of a Lisp dialect, including
macros, emphasizes a functional style via a
standard library of immutable data structures, and provides
interoperability with existing Java code, allowing programmers to use
existing Java libraries without leaving Clojure.
%
Since its initial release in 2007, Clojure has been widely adopted for
``backend'' development in places where its support for parallelism,
functional programming, and Lisp-influenced abstraction is desired on
the JVM. As a result, it now has an extensive base of existing untyped
programs, whose developers can now benefit from Typed Clojure. As
a result, Typed Clojure is used in industry, experience we discuss
in this paper.

\begin{figure}
\inputminted[firstline=5]{clojure}{code/demo/src/demo/parent2.clj}
\caption{A simple Typed Clojure program}
\label{fig:ex1}
\end{figure}

Figure~\ref{fig:ex1} presents a simple program demonstrating many
aspects of our system, from simple type annotations to explicit
handling of Java's \java{null} (written \clj{nil}) in interoperation, as well as an
extended form of occurrence typing and method resolution of
Java interoperability based on static type information.

The \clj{parent} function has the type 
$$
\clj{['{:file (U nil File)} -> (U nil Str)]}
$$
which means that it takes a hash table whose \clj{:file} key maps to either
\clj{nil} or a \clj{File}, and it produces either \clj{nil} or a
\clj{String}. The \clj{parent} function uses the \clj{:file} keyword
as an accessor to get the file, checks that it isn't \clj{nil}, and
then obtains the parent by making a Java method call.

In the remainder of this paper, we describe how Typed Clojure's
central innovations, including Java interoperability, multimethods,
and heterogeneously-typed immutable maps, enable this example and many
others. We begin with an example-driven presentation of the main type
system features in \secref{sec:overview}. We then incrementally
present a core calculus for Typed Clojure covering all of these
features together in \secref{sec:formal} and prove type
soundness (\secref{sec:metatheory}). We then discuss the full
implementation of Typed Clojure, dubbed \coretyped{}, which extends
the formal model in many ways, and the experience gained from its use
in \secref{sec:experience}. Finally, we discuss related work and
conclude.

\subsection{Contributions}

These are our main contributions.

\begin{itemize}
  \item We motivate, describe and evaluate Typed Clojure, an optional type system for Clojure
        that understands existing Clojure idioms.
  \item We present the theory and implementation of a type system
        based on \emph{occurrence typing} that statically prevents null-pointer 
        exceptions in typed code in the presence of Java interoperability, 
        applicable to existing JVM languages.
  \item We prove type soundness for Typed Clojure.
\end{itemize}


\section{Overview of Typed Clojure}

\label{sec:overview}

We now begin a tour of the central features of Typed Clojure,
beginning with Clojure itself. In our presentation, we will make 
use of the full Typed Clojure system to illustrate the key type system
ideas, before studying the core features in detail in
section~\ref{sec:formal}. 

\subsection{Clojure}

Clojure~\cite{Hic08} is a Lisp built to run on the
Java Virtual Machine with exemplary support for concurrent programming
and immutable data structures. It emphasizes mostly-functional
programming, restricting imperative updates to a limited set of
structures which have specific thread synchronization behaviour. By
default, it provides fast implementations of immutable lists, vectors,
and hash tables, which are used for most data structures, although it
also provides means for defining new records.

One of Clojure's primary advantages is easy interoperation with
existing Java libraries. It automatically generates appropriate JVM
bytecode to make Java method and constructor calls, and treats Java
values as any other Clojure value. However, this smooth
interoperability comes at the cost of pervasive \java{null}, which
leads to the possibility of null pointer exceptions---a drawback we
address in Typed Clojure.

%\paragraph{Clojure Syntax}
%
%We describe new syntax as they appear in each example, but
%begin with include the essential basics of Clojure syntax.
%
%\clj{nil} is exactly Java's \java{null}.
%Parentheses indicate \emph{applications}, brackets
%delimit
%\emph{vectors}, braces
%delimit
%\emph{hash-maps}
%and double quotes delimit \emph{Java strings}.
%\emph{Symbols} begin with an alphabetic character,
%and a colon prefixed symbol like \clj{:a} is a \emph{keyword}.
%
%\emph{Commas} are always \emph{whitespace}.

\subsection{Typed Clojure}

Here is a simple program in Typed Clojure.

We define a global variable \clj{greet} as a one-argument function
taking and returning a string.

\begin{minted}{clj}
(ann  greet [Str -> Str])
(defn greet [n]
  (str "Hello, " n "!"))
(greet "Grace") ; :- Str
;=> "Hello, Grace!"
\end{minted}

The function may take strings, but providing \clj{nil} (Clojure's name for Java's \java{null})
is a static type error --- \clj{nil} is not a string in Typed Clojure.

\begin{minted}{clj}
(greet 
  nil ; Type Error:
      ;   Expected Str, given nil
  )
\end{minted}

\paragraph{Unions} We can make the annotation more permissive with \emph{ad-hoc unions}
to allow \clj{nil}.

\begin{minted}{clj}
(ann  greet-nil [(U nil Str) -> Str])
(defn greet-nil [n]
  (str "Hello" (when n (str ", " n)) "!"))
(greet-nil "Donald") ; :- Str
;=> "Hello, Donald!"
(greet-nil nil)      ; :- Str
;=> "Hello!"
\end{minted}

All Clojure values are true except \clj{nil} and \clj{false}, so the
comma is only added when the argument is non-nil.

\paragraph{Flow analysis} Typed Clojure uses occurrence typing to
model type-based control flow.
In \clj{greetings}, a \clj{when} expression ensures \clj{repeat}
is never passed a \clj{nil} argument.

\begin{minted}{clj}
(ann  greetings [Str (U nil Int) -> Str])
(defn greetings [n i]
  (str "Hello, "
       (when i  ; when i is a non-nil integer
         (apply str (repeat i "hello, ")))
       n "!"))
(greetings "Donald" 2)  ; :- Str
;=> "Hello, hello, hello, Donald!"
(greetings "Grace" nil) ; :- Str
;=> "Hello, Grace!"
\end{minted}

Removing the \clj{when} expression is a static type error
--- \clj{repeat} cannot be passed \clj{nil}.

\begin{minted}{clj}
(ann  greetings-bad [Str (U nil Int) -> Str])
(defn greetings-bad [n i]
  (str "Hello, "
       (apply str 
         (repeat 
           i ; Type Error:
             ;   Expected Int, given (U nil Int).
           "hello, "))
       n "!"))
\end{minted}



%\subsection{Type System Basics}
%
%\citet{TF10}
%presented Typed Racket with occurrence typing,
%a technique for deriving type information from conditional control flow.
%They introduced the concept of occurrence typing 
%with the following example.
%
%\inputminted[firstline=3]{racket}{code/tr/example1.rkt}
%
%This function takes a value that is either \emph{\#f} % mintinline really hates #
%or a number, represented by an \emph{untagged} union type.
%The `then' branch has an implicit invariant
%that \rkt{x} is a number, which is automatically inferred with occurrence typing
%and type checked without further annotations.
%
%We chose to build on the ideas and implementation
%of Typed Racket to implement a type system targeting Clojure for several reasons.
%Initially, the similarities between Racket and Clojure drew us to
%investigate the effectiveness of repurposing occurrence typing
%for a Clojure type system---both languages share a Lisp heritage,
%similar standard functions 
%(for instance \clj{map}
%in both languages is variable-arity)
%and idioms.
%While Typed Racket is gradually typed and has sophisticated
%dynamic semantics for cross-language interaction, we 
%chose to first implement
%the static semantics
%with the hope to extend Typed Clojure to be gradually typed at a future date.
%Finally,
%Typed Racket's combination of bidirectional checking
%and occurrence typing presents a successful model for 
%type checking dynamically typed programs without compromising
%soundness, which is appealing over success typing~\cite{Lindahl:2006:PTI}
%which cannot prove strong properties about programs
%and soft typing~\cite{CF91}
%which has proved too complicated in practice.
%
%Here is the above program in Typed Clojure.
%\begin{exmp}
%\inputminted[firstline=5]{clojure}{code/demo/src/demo/eg1.clj}
%\label{example:conditionalflow}
%\end{exmp}
%
%The \clj{fn} macro (provided by core.typed) supports optional annotations by 
%adding
%\clj{:-} and a type after a parameter
%position
%or binding vector 
%to annotate parameter types
%and return types respectively.
%\clj{number?} is
%a Java \java{instanceof} test of \clj{java.lang.Number}.
%As in Typed Racket, \clj{U} creates an \emph{untagged union} type, which can take
%any number of types.
%
%Typed Clojure can already check all of the examples in~\citet{TF10}---the 
%rest of this section describes the extensions necessary
%to check Clojure code.


\subsection{Java interoperability}
\label{sec:overviewjavainterop}

Clojure supports interoperability with Java, including the ability to
call constructors, invoke methods, and access fields.

\begin{minted}{clj}
  (.getParent (new File "a/b"))  ; :- (U nil Str)
  ;=> "a"
\end{minted}

If a specific constructor, method, or field cannot be found based on the
static types of its arguments, a type error is thrown.

\begin{minted}{clj}
  (fn [f] 
    (.getParent f) ; Type Error:
                   ;   Unresolved interop: getParent
    )
\end{minted}

Function arguments default to \clj{Any}, the most permissive type. Ascribing
a parameter type helps Typed Clojure find a specific method.

%Calls to Java methods and fields have prefix notation
%like \clj{(.method target args*)} and \clj{(.field target)} respectively,
%with method and field names prefixed with a dot and methods taking some number of arguments.

\begin{exmp}
\inputminted[firstline=6,lastline=8]{clojure}{code/demo/src/demo/parent3.clj}
\end{exmp}

%\begin{exmp}
%\inputminted[firstline=18,lastline=19]{clojure}{code/demo/src/demo/parent3.clj}
%\end{exmp}

The conditional guards from dereferencing \clj{nil}, and as before
removing it is a static type error.

\begin{minted}{clj}
(defn parent-bad-in [f :- (U nil File)]
  (.getParent f) ; Type Error:
                 ;   Cannot call instance method 
                 ;   java.io.File/getParent on type 
                 ;   (U nil File).
  )
\end{minted}

Any Java method returning a reference can also return \java{null} ---
Typed Clojure rejects programs that assume otherwise.

\begin{minted}{clj}
(defn parent-bad-out [f :- File] :- Str
  (.getParent f) ; Type Error:
                 ;   Expected Str, given (U nil Str).
  )
\end{minted}

%Typed Clojure and Java treat \java{null} differently.
%In Clojure, where it is known as \clj{nil}, Typed Clojure assigns it an explicit type
%called \clj{nil}. In Java \java{null} is implicitly a member of any reference type.
%This means the Java static type \java{String} is equivalent to
%\clj{(U nil String)} in Typed Clojure.
%
%Reference types in Java are nullable, so to guarantee a method call does not
%leak \java{null} into a Typed Clojure program we
%must assume methods can return \clj{nil}.

In contrast, JVM invariants guarantee constructors return a non-null reference.\footnote{\url{http://docs.oracle.com/javase/specs/jls/se7/html/jls-15.html#jls-15.9.4}}

\begin{exmp}
\inputminted[firstline=15,lastline=16]{clojure}{code/demo/src/demo/parent3.clj}
\end{exmp}

By default Typed Clojure conservatively assumes method and constructor arguments to be \emph{non-nullable},
but can be configured globally for particular non-target positions if needed.

\subsection{Multimethods}

\label{sec:multioverview}

A \emph{multimethod} is a kind of extensible function. It combines
a \emph{dispatch function} with one or more \emph{methods}.
When invoked, the arguments are first supplied to the dispatch function, yielding
a \emph{dispatch value}. A method is then chosen
based on the dispatch value---this method is applied to the arguments
to finally return a value for the entire expression.

Recall the multimethod in figure~\ref{fig:ex1}, duplicated below.

\begin{minted}{clj}
(ann pname [(U File String) -> (U nil String)])
(defmulti pname class)
(defmethod pname String [s] (pname (new File s)))
(defmethod pname File [f] (.getName f))
\end{minted}

%Its dispatching function is
%\clj{class}, with two methods associated with dispatch values \clj{java.lang.String} and \clj{java.io.File}
%respectively.

The dispatch function \clj{class}---associated at multimethod creation with \clj{defmulti}---dictates 
whether the \clj{String} or \clj{File} method is chosen---both installed via \clj{defmethod}.

The multimethod dispatch rules involve
\clj{isa?}, a hybrid predicate which includes a subclassing check for classes and
an equality check for most other values.

\begin{minted}{clojure}
  (isa? String Object) ;=> true
  (isa? Object String) ;=> false
  ; keywords are interned symbols with a : prefix
  (isa? :a :a) ;=> true
  (isa? :a :b) ;=> false
\end{minted}

The current dispatch value and---in turn---each method's associated dispatch value
is supplied to \clj{isa?}. If exactly one method returns true, it is chosen.

To demonstrate,
  \clj{(pname "STAINS/JELLY")}
chooses the \clj{String} method because
\clj{(isa? (class "STAINS/JELLY") String)}
is true and
\clj{(isa? (class "STAINS/JELLY") File)}
is false. Then the \clj{String} method body
\clj{(pname (new File "STAINS/JELLY"))}
chooses the \clj{File} method for opposite reasons,
resulting in 
\begin{minted}{clojure}
(.getName (new File "STAINS/JELLY"))  ; :- (U nil Str)
;=> "JELLY"
\end{minted}

%The following Typed Clojure program is semantically identical to figure~\ref{fig:ex1}.
%
%\begin{minted}{clj}
%(ann pname [(U Str File) -> (U nil Str)])
%(defn pname [x]
%  ; dispatch value calculated by applying dispatch
%  ; function `class` to argument `x`.
%  (cond
%    ; if (class x) subclasses String, but not File
%    (and (isa? (class x) String)
%         (not (isa? (class x) File)))
%    ; then choose the String method
%    (pname (new File x))
%
%    ; else if (class x) subclasses File, but not String
%    (and (isa? (class x) File)         
%         (not (isa? (class x) String)))
%    ; then choose the File method
%    (.getName x)
%    :else (throw (Exception. "No match"))))
%\end{minted}
%
%An unambiguous match leads to the corresponding method being applied to the arguments,
%giving the final result.

Dispatching on a non-class value is common, as well as specifying
a \clj{:default} method for when no methods are matched.

\begin{minted}{clojure}
(ann hi [Kw -> Str])
(defmulti hi identity)
(defmethod hi :en [_] "hello")
(defmethod hi :fr [_] "bonjour")
(defmethod hi :default [_] "um...")

(str (hi :en) " " (hi :fr)) ;=> "hello bonjour"
(hi :bocce)   ;=> "um..."
\end{minted}

The \clj{hi} multimethod dispatches on its argument via the identity function.

%\subsection{Multimethods}
%
%A multimethod in Clojure is a function with a \emph{dispatch
%function} and a \emph{dispatch table} of methods. Multimethods are created with {\clj{defmulti}}.
%\inputminted[firstline=5,lastline=6]{clojure}{code/demo/src/demo/rep.clj}
%The multimethod \clj{path} has type \clj{[Any -> (U nil String)]}, an initially empty \emph{dispatch table}
%and \emph{dispatch function} \clj{class}, a function that
%returns the class of its argument or \clj{nil} if passed \clj{nil}.
%
%We can use {\clj{defmethod}} to install a method to \clj{path}.
%\inputminted[firstline=7,lastline=7]{clojure}{code/demo/src/demo/rep.clj}
%Now the dispatch table maps
%the \emph{dispatch value} \clj{String} to the function
%\clj{(fn [x] x)}. 
%We add another method
%which maps
%\clj{File} to the function
%\clj{(fn [x] (.getPath x))}
%in the dispatch table.
%\inputminted[firstline=8,lastline=8]{clojure}{code/demo/src/demo/rep.clj}
%
%After installing both methods, the call 
%$$
%\clj{(path (new File "dir/a"))}
%$$
%dispatches to the second method we installed because
%$$
%\clj{(isa? (class "dir/a") String)}
%$$
%is true, and finally returns 
%$$
%\clj{((fn [x] (.getPath x)) "dir/a")}.
%$$

%We include the above sequence of definitions as \egref{example:rep}.
%
%\begin{Code}
%\begin{exmp}
%\inputminted[firstline=5,lastline=10]{clojure}{code/demo/src/demo/rep.clj}
%\label{example:rep}
%\end{exmp}
%\end{Code}
%
%Typed Clojure does not predict if a runtime dispatch will be successful---\clj{(path :a)} 
%type checks because \clj{:a} agrees with the parameter type \clj{Any},
%but throws an error at runtime.

%\paragraph{Multiple dispatch} \clj{isa?} is special with vectors---vectors of the
%same length recursively call \clj{isa?} on the elements pairwise.
%\begin{minted}{clojure}
%  (isa? [Keyword Keyword] [Object Object]) ;=> true
%\end{minted}
%
%\inputminted[firstline=6,lastline=23]{clojure}{code/demo/src/demo/eg7.clj}
%
%\egref{example:multidispatch}
%simulates multiple dispatch by dispatching on
%a vector containing the class of both arguments. \clj{open}
%takes two arguments which can be strings or files and returns
%a new file that concatenates their paths.
%
%We call three different \clj{File} constructors, each known at compile-time
%via type hints.
%Multiple dispatch follows the same kind of reasoning as \egref{example:incmap},
%except we update multiple bindings simultaneously.

\subsection{Heterogeneous hash-maps}

Immutable hash-maps with keyword keys play a major role in Clojure programming.

\begin{minted}{clojure}
(def breakfast {:en "waffles" :fr "croissants"})

breakfast ; :- (HMap :mandatory {:en Str :fr Str}
          ;          :complete? true)

(get breakfast :en)    ; :- Str
;=> "waffles"
; keywords are functions that look themselves up
(:fr breakfast)        ; :- Str
;=> "croissants"
(:bocce breakfast) ; :- nil
;=> nil
\end{minted}

HMap types model the most common usages of keyword maps.
The inferred type for \clj{breakfast} holds two kinds of information---the known entries \clj{:en}
and \clj{:fr} have string values (expressed as a \clj{:mandatory} entry) and that 
every other key is absent (by \clj{:complete?} being \clj{true}).

HMap's default to being partially specified---\clj{:complete?} defaults to \clj{false}. The HMap shorthand 
\clj{'{:en Str :fr Str}}
forgets 
information about absent keys, only providing \clj{:mandatory}
information.

\begin{minted}{clojure}
(ann lunch '{:en Str :fr Str})
(def lunch {:en "muffin" :fr "baguette"})
(:en lunch)    ; :- Str
;=> "muffin"
(:fr lunch)    ; :- Str
;=> "baguette"
; Unknown lookups are now less accurate
(:bocce lunch) ; :- Any
;=> nil
\end{minted}

\paragraph{HMaps in practice} The next example is extracted from a production system at CircleCI,
a company with a large production Typed Clojure system
(section~\ref{sec:casestudy} presents a case study).

\begin{minted}{clojure}
(defalias RawKeyPair
  "Unencrypted keypair -- extra keys disallowed"
  (HMap :mandatory {:public-key RawKey,
                    :private-key RawKey},
        :complete? true))

(defalias EncKeyPair
  "Encrypted keypair -- extra keys disallowed"
  (HMap :mandatory {:public-key RawKey,
                    :enc-private-key EncKey},
        :complete? true))

(ann enc-keypair [RawKeyPair -> EncKeyPair])
(defn enc-keypair "Encrypt an unencrypted keypair"
  [{pk :private-key :as kp}] ; original map is kp
  (assoc 
    ; remove unencrypted private key
    (dissoc kp :private-key)
    ; add encrypted private key
    :enc-private-key (encrypt pk)))
\end{minted}
%\begin{exmp}
%\inputminted[firstline=10,lastline=22]{clojure}{code/demo/src/demo/key.clj}
%\label{example:circleci}
%\end{exmp}
Fully specified HMap's
statically enforce the private key is not accidentally left in a supposedly
encrypted keypair.

\begin{minted}{clojure}
(ann enc-keypair-bad [RawKeyPair -> EncKeyPair])
(defn enc-keypair-bad
  [{pk :private-key :as kp}]
  (assoc kp :enc-private-key (encrypt pk))
  ; Type Error:
  ;   Expected EncKeyPair, given 
  ;   (HMap :mandatory {:enc-private-key EncKey
  ;                     :private-key RawKey
  ;                     :public-key RawKey}
  ;         :complete? true)
  )
\end{minted}

The extra \clj{:private-key} entry disagrees with \clj{EncKeyPair}, so a type error
is given.

%\clj{enc-keypair} takes an unencrypted keypair and returns an encrypted keypair by
%dissociating the raw \clj{:private-key} entry with \clj{dissoc}
%and associating an encrypted private key
%as \clj{:enc-private-key} on an immutable map with \clj{assoc}.
%The expression \clj{(:private-key kp)} shows that keywords are also 
%functions that look themselves up in a map returning the associated value or \nil{} if the key is missing.
%Since \clj{EncKeyPair} is \clj{:complete?}, Typed Clojure enforces the return type
%does not contain an entry \clj{:private-key}, and would complain if the \clj{dissoc}
%operation forgot to remove it.

%\egref{example:absentkeys}
%is like \egref{example:circleci}
%except the \clj{:absent-keys} HMap option is used
%instead of \clj{:complete?},
%which takes a \emph{set literal} of keywords that do not appear in the map, written 
%with \emph{\#}-prefixed braces.
%The syntax \clj{(fn [{pkey :private-key, :as kp}] ...)}
%aliases \clj{kp} to the first argument and \clj{pkey} to \clj{(:private-key m)}
%in the function body.
%
%\begin{exmp}
%\inputminted[firstline=10,lastline=21]{clojure}{code/demo/src/demo/key2.clj}
%\label{example:absentkeys}
%\end{exmp}
%
%Since this example enforces that \clj{:private-key} must not appear
%in a \clj{EncKeyPair}
%Typed Clojure would still complain if we forgot to \clj{dissoc} \clj{:private-key}
%from the return value.
%Now, however we could stash the raw private key in another entry
%like \clj{:secret-key} which is not mentioned by the partial HMap \clj{EncKeyPair}
%without Typed Clojure noticing.

%\paragraph{Branching on HMaps} Finally, testing on HMap properties
%allows us to refine its type down branches. \clj{dec-map} takes an
%\clj{Expr}, traverses to its nodes and decrements their values by \clj{dec}, then
%builds the \clj{Expr} back up with the decremented nodes.
%
%\begin{exmp}
%\inputminted[linenos,firstnumber=1,firstline=15,lastline=27]{clojure}{code/demo/src/demo/hmap.clj}
%\label{example:decmap}
%\end{exmp}
%
%If we go down the then branch (line 4), since \clj{(= (:op m) :if)} is true
%we remove
%the \clj{:do} and \clj{:const}
%Expr's from the type of \clj{m} (because their respective \clj{:op} entries disagrees with \clj{(= (:op m) :if)})
%and we are left with an \clj{:if} Expr.
%On line 8,
%we instead strike out the \clj{:if} Expr since it contradicts \clj{(= (:op m) :if)} being false. 
%Line 9 know we can
%remove the \clj{:const} Expr from the type of \clj{m} because it contradicts \clj{(= (:op m) :do)} being true,
%and we know \clj{m} is a \clj{:do} Expr.
%Line 12 we strike out \clj{:do} because \clj{(= (:op m) :do)} is false,
%so we are left with \clj{m} being a \clj{:const} Expr.
%
%Section~\ref{sec:formalpaths} discusses how this automatic reasoning is achieved.

\subsection{HMaps and multimethods, joined at the hip}

Since HMaps are the primary way of specifying the structure of data in Clojure,
and multimethods are the primary tool for dispatching on data, they are inevitably
linked.
There are infinite ways of both structuring and dispatching on data, so there is no
hope predicting them all, or even manually covering useful a subset.

Thankfully, occurrence typing is both extensible \emph{and} compositional.
By extending occurrence typing with
a handful of rules based on HMaps and other functions, 
we can automatically cover almost all the common cases---as well others
that compose simple rules in arbitrary ways.

Futhermore, no special attention is needed for multimethod dispatch---the primitive branching
mechanism is still the humble \clj{if} conditional. Only a small number of rules are needed
to encode the \clj{isa?}-based dispatch, themselves made of small, simple pieces.
In practice, this means all conditional-based control flow understood by occurrence typing
also extends to the context of multimethod dispatch, and vice-versa.

We first demonstrate a very common, simple dispatch style,
then move on to deeper structural dispatching where occurrence typing's
compositionality shines.

\paragraph{HMaps and unions} Partially specified HMap's with a common dispatch key
combine naturally with ad-hoc unions.
An \clj{Order} is one of three kinds of HMaps.

\begin{minted}{clojure}
(defalias Order   ; define type abbreviation
  "A meal order, tracking dessert quantities."
  (U '{:Meal ':lunch ; keyword singleton type
       :desserts Int}
     '{:Meal ':dinner :desserts Int}
     '{:Meal ':combo :meal1 Order :meal2 Order}))
\end{minted}

The \clj{:Meal} entry is common to each HMap, always mapped to a known keyword singleton
type.
It's natural to dispatch on the \clj{class} of an instance---it's similarly
natural to dispatch on a known entry like \clj{:Meal}.

\begin{minted}{clojure}
(ann desserts [Order -> Int])
(defmulti desserts "Number of desserts per order."
          :Meal)  ; dispatch on :Meal entry
; map destructuring backwards, d is :desserts entry
(defmethod desserts :lunch [{d :desserts}] d)
(defmethod desserts :dinner [{d :desserts}] d)
(defmethod desserts :combo [{m1 :meal1 m2 :meal2}]
  (+ (desserts m1) (desserts m2)))

(desserts {:Meal :combo 
           :meal1 {:Meal :lunch :desserts 1}
           :meal2 {:Meal :dinner :desserts 2}})
;=> 3
\end{minted}

The \clj{:combo} method is verified to only structurally recur
on \clj{Order}s. This is achieved because we learn the argument
must be at least of type
\clj{'{:Meal :combo}}
since
\clj{(isa? (:Meal o) :combo)}
must be true. This eliminates possibility of \clj{:lunch} and \clj{:dinner}
orders, deducing
\begin{minted}{clojure}
'{:Meal ':combo :meal1 Order :meal2 Order}
\end{minted}
as the argument type, which can safely structurally recur.


\paragraph{Arbitrary dispatch}
An equally valid dispatch mechanism for \clj{desserts}
would be on the \clj{class} of the \clj{:desserts} key.
We have already seen dispatch on \clj{class} and on keywords
in isolation---occurrence typing automatically understands
control flow that combines its simple building blocks.


\begin{minted}{clojure}
(ann desserts' [Order -> Int])
(defmulti desserts' 
          (fn [o :- Order] (class (:desserts o))))
(defmethod desserts' Long [{d :desserts}] d)
(defmethod desserts' nil [{m1 :meal1 m2 :meal2}]
  (+ (desserts' m1) (desserts' m2)))

(desserts' {:Meal :combo 
            :meal1 {:Meal :lunch :desserts 1}
            :meal2 {:Meal :dinner :desserts 2}})
;=> 3
\end{minted}

In the \clj{Long} method, Typed Clojure learns that
its argument is at least of type \clj{'{:desserts Long}}---since
\begin{minted}{clojure}
(isa? (class (:desserts o)) Long)
\end{minted}
must be true.

Combined with the knowledge that it is also an
\clj{Order}, we can eliminate the possibility of
a \clj{:combo} meal, and infer the type
\begin{minted}{clojure}
(U '{:Meal ':lunch :desserts Int}
   '{:Meal ':dinner :desserts Int})
\end{minted}
Looking up \clj{:desserts} on this type returns an \clj{Int}.

In the \clj{nil} method, we know
\begin{minted}{clojure}
(isa? (class (:desserts o)) nil)
\end{minted}
is true---which implies \clj{(class (:desserts o))} is \clj{nil}.

Since lookups on missing keys return \clj{nil}, there are
now two possibilities:
\begin{itemize}
  \item either \clj{o} has a \clj{:desserts} entry to \clj{nil}, or
  \item \clj{o} is missing a \clj{:desserts} entry.
\end{itemize}
In a type, we learn \clj{o} is at least
\begin{minted}{clojure}
(U '{:desserts nil}
   ; :absent-keys, a set of known absent entries
   (HMap :absent-keys #{:desserts}))
\end{minted}

This is enough to infer \clj{o} is a \clj{:combo} meal, thus verifying both
structural recursions.

%\begin{exmp}
%\inputminted[firstline=6,lastline=13]{clojure}{code/demo/src/demo/hmap.clj}
%\label{example:decleaf}
%\end{exmp}
%
%The \clj{defn} macro defines a top-level function, with syntax like the typed \clj{fn}.
%The function \clj{an-exp} is verified to return an \clj{Expr}.
%
%Here \clj{defalias} defines \clj{Expr}, a type abbreviation
%that describes the structure of a recursively-defined AST as a union of HMaps.
%Keyword singleton types are quoted---\clj{':lunch}.
%A type that is a quoted map like \clj{'{:op ':if}} is a
%HMap type with a fixed number of keyword entries of the specified types
%known to be \emph{present},
%zero entries known to absolutely be \emph{absent},
%and an infinite number of \emph{unknown} entries entries.
%Since only keyword keys are allowed, they do not require quoting.

%\paragraph{HMap dispatch} The flexibility of \clj{isa?} is key to the generality of multimethods. 
%In \egref{example:incmap} we
%dispatch on the \clj{:op} key 
%of our HMap AST \clj{Expr}.
%Since keywords are functions that look themselves up in their argument, we simply
%use \clj{:op} as the dispatch function.
%
%\begin{exmp}
%\inputminted[firstline=5,lastline=18]{clojure}{code/demo/src/demo/eg5.clj}
%\label{example:incmap}
%\end{exmp}
%
%The function \clj{inc-leaf} is like \egref{example:decmap} except the nodes are incremented.
%The reasoning is similar, except we only consider one branch (the current method) by
%locally considering the current \emph{dispatch value} and reasoning about how it relates
%to the \emph{dispatch function}.
%For example, 
%in the \clj{:do} method we learn the \clj{:op} key is a \clj{:do}, which
%narrows our argument type to the \clj{:do} Expr, and similarly for the \clj{:if}
%and \clj{:const} methods.
%
%
%\subsection{Final example}
%
%\egref{example:final}
%combines everything we will cover for the rest of the paper:
%multimethod dispatch, reflection resolution via type hints, Java method
%and constructor calls, conditional and exceptional flow reasoning,
%and HMaps. 
%
%
%\begin{figure}
%\begin{exmp}
%\inputminted[firstline=6,lastline=23]{clojure}{code/demo/src/demo/eg7.clj}
%\label{example:multidispatch}
%\end{exmp}
%\begin{exmp}
%\inputminted[firstline=6,lastline=20]{clojure}{code/demo/src/demo/eg8.clj}
%\label{example:final}
%\end{exmp}
%\caption{Multimethod Examples}
%\end{figure}
%
%We dispatch on \clj{:p} to distinguish the two cases of \clj{FSM}---for example on \clj{:F}
%we know the \clj{:file} is a file.
%The body of the first method uses type hints to resolve reflection
%and conditional control flow to prove null-pointer exceptions are impossible.
%The second method is similar except it uses exceptional control flow.


\section{A Formal Model of \lambdatc{}}

\label{sec:formal}

Now that we have demonstrated the core features Typed Clojure
provides, we link them together in a formal model called
\lambdatc{}.
Our presentation will start with a review of
occurrence typing~\cite{TF10}.
Then for the rest of the section we incrementally add each
novel feature of Typed Clojure to the formalism,
interleaving presentation of syntax, typing rules, operational semantics
and subtyping.

The first insight about occurrence typing is that
logical formulas
can be used to represent type information about our programs
by relating parts of the runtime environment to types
via propositional logic.
\emph{Type Propositions} \prop{} make assertions like ``variable \x{} is of type \NumberFull{}'' or
``variable \x{} is not \nil{}''---in our logical system we write these as
{\isprop{\NumberFull}{\x{}}}
and {\notprop{\Nil{}}{\x{}}}. 
The other propositions are standard logical connectives: implications, conjunctions,
disjunctions, and the trivial (\topprop{}) and impossible (\botprop{}) propositions
(\figref{main:figure:termsyntax}).


The particular part of the runtime environment we reference in a
type proposition is called the \emph{object}.
The typing judgment relates an object to every expression in the language.
An object is either \emph{empty}, written \emptyobject{}, 
which says 
this expression is not known to evaluate to a particular part
  of the current runtime environment, or a 
variable with some \emph{path}, written \path{\pathelem{}}{\x{}},
that exactly indicates how the value of this
expression can be derived from the current runtime environment.
Type propositions can only reference non-empty objects.

The second insight is that we can replace the traditional 
representation of a
type environment (eg., a map from variables to types)
with a set of propositions, written \propenv{}. 
Instead of mapping \x{} to
the type \NumberFull{}, we use the proposition {\isprop{\NumberFull}{\x{}}}.

Given a set of propositions, we can use logical reasoning to derive
new information about our programs
with the judgment \inpropenv{\propenv{}}{\prop{}}.
In addition to the standard rules for the logical connectives, the key
rule is L-Update, which combines multiple propositions about the same variable,
allowing us to refine its type.
$$
  {\LUpdate}
$$
For example, with L-Update we can use the knowledge of
\inpropenv{\propenv{}}{\isprop{\UnionNilNum}{\x{}}}
and 
\inpropenv{\propenv{}}{\notprop{\Nil{}}{\x{}}}
to derive \inpropenv{\propenv{}}{\isprop{\Number}{\x{}}}.
(The metavariable \propisnotmeta{} ranges over \t{} and \nottype{\t{}} (without variables).)
We cover L-Update in more detail in \secref{sec:formalpaths}.

Finally, this approach allows the type system to track
programming idioms from 
dynamic languages
using implicit type-based reasoning based on the result of
conditional tests.
For instance, \egref{example:conditionalflow}
only utilizes \clj{x} once
the programmer is convinced it is safe to do so based whether
\clj{(number? x)}
is 
true or false. 
To express this in the type system, every expression 
is described by two propositions: a `then' proposition
for when it reduces to a true value, and an `else' proposition
when it reduces to a false value---for \clj{(number? x)}
the then proposition is {\isprop{\NumberFull}{\x{}}} and 
the else proposition is {\notprop{\NumberFull}{\x{}}}.
%\ref{main:figure:typingrules}

\begin{figure}
  \footnotesize
$$
\begin{array}{lrll}
  \expd{}, \e{} &::=& \x{}
                      \alt \v{} 
                      \alt {\comb {\e{}} {\e{}}} 
                      \alt {\abs {\x{}} {\t{}} {\e{}}} &\mbox{Expressions} \\
                      &\alt& {\ifexp {\e{}} {\e{}} {\e{}}}
                      %\alt {\trdiff{\doexp {\e{}} {\e{}}}}
                      \alt {\letexp {\x{}} {\e{}} {\e{}}}\\
                      %\alt {\errorvalv{}}
  \v{} &::=&          \singletonmeta{}
                      \alt {\num{}}
                      \alt {\const{}}
                      \alt {\closure {\openv{}} {\abs {\x{}} {\t{}} {\e{}}}}
                &\mbox{Values} \\
                \constantssyntax{}\\
  \s{}, \t{}    &::=& \Top 
                      \alt {\Unionsplice {\overrightarrow{\t{}}}}
                      \alt
                      {\ArrowOne {\x{}} {\t{}}
                                   {\t{}}
                                   {\filterset {\prop{}} {\prop{}}}
                                   {\object{}}}
                &\mbox{Types} \\
                      &\alt& {\Value \singletonmeta{}} 
                      \alt \trdiff{\class{}}\\
  \singletonallsyntax{}
                \\ \\
  \occurrencetypingsyntax{}\\
  \propenvsyntax{}\\
  \openvsyntax{}
  %\\
  %\classliteralallsyntax{}
\end{array}
$$
\caption{Syntax of Terms, Types, Propositions and Objects}
\label{main:figure:termsyntax}
\end{figure}


We formalize our type system following~\citet{TF10}
(with differences highlighted in $\trdiff{\text{blue}}$).
The typing judgment 
$$
{\judgement   {\propenv}
              {\e{}} {\t{}}
  {\filterset {\thenprop {\prop{}}}
              {\elseprop {\prop{}}}}
  {\object{}}}
$$
says expression \e{} is of type \t{} in the 
proposition environment $\propenv{}$, with 
`then' proposition {\thenprop {\prop{}}}, `else' proposition {\elseprop {\prop{}}}
and object \object{}. We write 
{\judgementtwo{\propenv}{\e{}} {\t{}} if we are only interested in the type.

The syntax is given in \figref{main:figure:termsyntax}. Expressions include variables, values,
application, abstractions, conditionals and let expressions.
All binding forms introduce fresh variables.
Values include booleans, \nil{}, class literals, keywords, 
numbers,
constants and closures. 
Value environments map local bindings to values.

Types include the top type, \emph{untagged} unions, functions, singleton types
and class instances. 
We abbreviate \Booleanlong{} as \Boolean{}, \Keywordlong{} as \Keyword{}
and \NumberFull{} and \Number{}.
The type \Value{\Keyword} is inhabited by the class literal \Keyword{} and \clj{:a} is of type \Keyword{}.
We abbreviate \EmptyUnion{} as \Bot{}, {\ValueNil} as \Nil{}, 
{\ValueTrue} as \True and {\ValueFalse} as {\False}.
Function types contain \emph{latent} (terminology from~\cite{Lucassen88polymorphiceffect}) propositions and object, which, along with the return type,
may refer to the function argument.
%Latent means they are relevant when the function is applied rather than evaluated.
They are latent because they are instantiated with the
actual object of the argument in applications before they are used in the proposition environment.

\figref{main:figure:typingrules} contains the core typing rules.
The key rule for reasoning about conditional control flow is
T-If. 

\begin{mathpar}
  {\TIf}
\end{mathpar}

The propositions of the test expression \e{1}, \thenprop{\prop{1}} and \elseprop{\prop{1}}, are 
used as assumptions in the then and else branch respectively.
If the result of the \ifliteral{} is a true value, then it either
came from \e{2}, in which case \thenprop{\prop{2}} is true, or from \e{3},
which implies \thenprop{\prop{3}} is true. 
The else proposition is \orprop{\elseprop{\prop{2}}}{\elseprop{\prop{3}}} 
similarly.
The T-Local rule connects the type system to the proof system over type propositions
via \inpropenv {\propenv{}} {\isprop {\t{}} {\x{}}}
to derive a type for a variable.
Using this rule, the type system can then appeal to L-Update to refine the type
assigned to \x{}.

We are now equipped to type check
\egref{example:conditionalflow}, starting at body:
$$
\clj{... (if (number? x) (inc x) 0) ...}
$$

We know {\propenv{}} = {\isprop{\UnionNilNum{}}{\x{}}}.
The test expression uses T-App, 
$$
\judgement{\propenv{}}{\appexp{\numberhuh{}}{\x{}}}{\Boolean}{\filterset{\isprop{\Number}{\x{}}}{\notprop{\Number}{\x{}}}}{\emptyobject{}}
$$
since \numberhuh{} has type
{\ArrowOne{\x{}}{\Top}{\Boolean}
        {\filterset{\isprop{\Number}{\x{}}}{\notprop{\Number}{\x{}}}}{\emptyobject{}}}
      and \x{} has object \x{}.

Finally we check both branches using the extended proposition environment as specified by T-If.
Going down the then branch, our new assumption {\isprop{\Number}{\x{}}} is crucial to check
$$
\judgement{{\propenv{}},{\isprop{\Number}{\x{}}}}{\x{}}{\Number{}}{\filterset{\notprop{\falsy{}}{\x{}}}{\isprop{\falsy{}}{\x{}}}}{\emptyobject{}}
$$
because we can now satisfy the premise of T-Local:
$$
\inpropenv{{\propenv{}},\isprop{\Number}{\x{}}}{\isprop{\Number}{\x{}}}.
$$
%\judgement{{\propenv{}},\isprop{\Number}{\x{}}}{\hastype{\appexp{\inc{}}{\x{}}}{\Number{}}}{\filterset{\topprop{}}{\botprop{}}}{\emptyobject{}}
%$$
%$$
%\judgement{{\propenv{}},\notprop{\Number}{\x{}}}{\hastype{\zeroliteral{}}{\Number}}{\filterset{\topprop{}}{\botprop{}}}{\emptyobject{}}
%$$

%\inc{} has type
%$$
%{\ArrowOne{\x{}}{\Number}{\Number}
%        {\filterset{\topprop{}}{\topprop{}}}{\emptyobject{}}}
%$$
%We can now check the conditional with T-If.
%$$
%\judgement{\isprop{\Number}{\x{}}}{\hastype{\ifexp{\appexp{\numberhuh{}}{\x{}}}{\appexp{\inc{}}{\x{}}}{\zeroliteral{}}}{\Number}}{\filterset{\orprop{\isprop{\Number}{\x{}}}{\topprop{}}}{\orprop{\notprop{\Number}{\x{}}}{\topprop{}}}}{\emptyobject{}}
%$$
%Finally the function can be checked with T-Abs
%$$
%\judgement{}{\hastype{\abs{\x{}}{\UnionNilNum}{\ ...}}
%                                             {\ArrowOne{\x{}}{\UnionNilNum}{\Number}
%        {\filterset{\orprop{\isprop{\Number}{\x{}}}{\topprop{}}}{\orprop{\notprop{\Number}{\x{}}}{\topprop{}}}}{\emptyobject{}}}}
%  {\filterset{\topprop{}}{\botprop{}}}{\emptyobject{}}
%$$

\paragraph{Operational semantics} We define the dynamic semantics for \lambdatc{}
in a big-step style using an environment, following~\citet{TF10}.
We include both errors and a \wrong{} value, which is provably ruled out by the
type system.
The main judgment is \opsem{\openv{}}{\e{}}{\definedreduction{}}
which states that \e{} evaluates to answer \definedreduction{} in environment
\openv{}. We chose to omit the core rules (see \figref{appendix:figure:opsem})
however a notable difference is \nil{} is a false value, which affects the
semantics of \ifliteral{}:

\begin{mathpar}
    \BIfTrue{}

    \BIfFalse{}
\end{mathpar}

Subtyping (\figref{main:figure:subtyping}) 
is a reflexive and transitive relation with top type \Top. 
Singleton types are instances of their respective classes---boolean singleton types
are of type \Boolean{}, class literals are instances of \Class{} and keywords are
instances of \Keyword{}.
Instances of classes \class{} are subtypes of \Object{}. Since function types 
are subtypes of \IFn{}, all types except for \Nil{} are subtypes of \Object{},
so \Top{} = {\Union{\Nil}{\Object}}.
Function subtyping is contravariant left of the arrow---latent propositions, object
and the result type are covariant.
Subtyping for untagged unions is standard.

\begin{figure*}
  \footnotesize
  \begin{mathpar}
    %{\TDo}
    %{\TClass}
    %{\TIf}
    {\TAbs}
    \begin{array}{c}
      {\TSubsume}\\\\
      {\TNum}
    \end{array}
    \begin{array}{c}
      {\TConst}\\\\
      {\TKw}\\\\
      {\TClass}
    \end{array}
    \begin{array}{c}
      {\TTrue}\\\\
      {\TFalse}\\\\
      {\TNil}
    \end{array}

    {\TLet}
    {\TLocal}

    {\TApp}
    %{\TError}

  \end{mathpar}
  \caption{Typing rules}
  \label{main:figure:typingrules}
\end{figure*}

%\begin{figure}
%  \footnotesize
%  \begin{mathpar}
%    {\BLocal}
%
%    %{\BDo}
%
%    {\BLet}
%
%    \BVal{}
%
    %\BIfTrue{}

%    \BIfFalse{}
%
%    \BAbs{}
%
%    \BBetaClosure{}
%
%    \BDelta{}
%  \end{mathpar}
%  \caption{Operational Semantics}
%  \label{main:figure:standardopsem}
%\end{figure}

\begin{figure}
  \footnotesize
  \begin{mathpar}
    \SRefl{}

    \STop{}

\SUnionSuper{}

\SUnionSub{}

\SFunMono{}

\SObject{}

\SClass{}

\begin{array}{l}
\SSBool{}\\\\
\SSKw{}
\end{array}

\SFun{}


  \end{mathpar}
  \caption{Core Subtyping rules}
  \label{main:figure:subtyping}
\end{figure}

\subsection{Reasoning about Exceptional Control Flow}
\label{sec:doformal}

We extend our model with sequencing expressions and errors, where {\errorvalv{}}
models the result of calling Clojure's \clj{throw} special form
with some \clj{Throwable}.

\smallskip
$
\begin{altgrammar}
  \e{} &::=& \ldots \alt {\errorvalv{}} \alt {\doexp {\e{}} {\e{}}} &\mbox{Expressions} 
\end{altgrammar}
$

\smallskip
%
%B-Do simply evaluates its arguments sequentially and returns the right argument.
%Since errors are not values, we define error propagation semantics
%like BE-Do1 (figure~\ref{appendix:figure:errorstuck} for the full rules).
%
%\begin{mathpar}
%    {\BDo}
%
%\infer [BE-Error]
%{}
%{ \opsem {\openv{}} 
%         {\errorvalv{}}
%         {\errorvalv{}}}
%
%\infer [BE-Do1]
%{ \opsem {\openv{}} {\e{1}} {\errorvalv{}} }
%{ \opsem {\openv{}} {\doexp{\e{1}}{\e{}}} {\errorvalv{}}}
%\end{mathpar}

Our main insight is as follows: 
if the first subexpression in a sequence reduces to a value, then it is either true or false.
If we learn some proposition in both cases then we can use that proposition as an assumption to check the second subexpression.
T-Do formalizes this intuition.

\begin{mathpar}
    {\TDo}  
\end{mathpar}

The introduction of errors, 
which do not evaluate to either
a true or false value,
makes our insight interesting.

\begin{mathpar}
    {\TError}
\end{mathpar}

Recall \egref{example:doexception}.
\begin{minted}{clojure}
...
  (do (if (number? x) nil (throw (new Exception)))
      (inc x)) 
...
\end{minted}

As before, checking \appexp{\numberhuh{}}{\x{}} allows us to use the proposition \isprop{\Number}{\x{}}
when checking the then branch.

By T-Nil and subsumption we can propagate this  information to both propositions.
$$
\judgement{\isprop{\Number}{\x{}}}{\nil{}}{\Nil{}}{\filterset{\isprop{\Number}{\x{}}}{\isprop{\Number}{\x{}}}}{\emptyobject{}}
$$
Furthermore, using T-Error
and subsumption we can conclude anything in the else branch.
$$
\judgement{\notprop{\Number}{\x{}}}{\errorvalv{}}{\Bot}{\filterset{\isprop{\Number}{\x{}}}{\isprop{\Number}{\x{}}}}{\emptyobject{}}
$$
Using the above as premises to T-If we conclude that if the first
expression in the \doliteral{} evaluates successfully, \isprop{\Number}{\x{}} must be true.
$$
\judgement{\isprop{\UnionNilNum}{\x{}}}
          {\ifexp{\appexp{\numberhuh{}}{\x{}}}{\nil{}}{\errorvalv{}}}{\Boolean}
          {\filterset{\isprop{\Number}{\x{}}}{\isprop{\Number}{\x{}}}}{\emptyobject{}}
$$
We can now use \isprop{\Number}{\x{}} in the environment to check the second subexpression
{\appexp{\inc{}}{\x{}}}, completing the example.

\subsection{Precise Types for Heterogeneous maps}
\label{sec:hmapformal}

\begin{figure}
  \footnotesize
  $$
  \begin{altgrammar}
    \e{} &::=& \ldots \alt \hmapexpressionsyntax{}
    &\mbox{Expressions} \\
    \v{} &::=& \ldots \alt {\emptymap{}}
    &\mbox{Values} \\
    \t{} &::=& \ldots \alt {\HMapgeneric {\mandatory{}} {\absent{}}}
    &\mbox{Types} \\
    \auxhmapsyntax{}\\
    \pesyntax{}   &::=& \ldots \alt {\keype{\k{}}}
                  &\mbox{Path Elements}
  \end{altgrammar}
  $$
  \begin{mathpar}
    {\TGetHMap}

    {\TGetAbsent}

    {\TGetHMapPartialDefault}

    {\TAssoc}
  \end{mathpar}
  \begin{mathpar}
    \HMapsubtyping{}
  \end{mathpar}
  \begin{mathpar}
    {\BAssoc}
    {\BGet}
    {\BGetMissing}
  \end{mathpar}
  \caption{HMap Syntax, Typing and Operational Semantics}
  \label{main:figure:hmapsyntax}
\end{figure}


\figref{main:figure:hmapsyntax} presents syntax, typing rules
and dynamic semantics in detail.
%
The type \HMapgeneric{\mandatory{}}{\absent{}}
includes {\mandatory{}}, a map of \emph{present} entries (mapping keywords to types),
\absent{}, a set of keyword keys that are known to be \emph{absent}
and
tag \completenessmeta{} which is either {\complete{}} (``complete'') if the map is fully specified by \mandatory{},
and {\partial{}} (``partial'') if there are \emph{unknown} entries.
To ease presentation, 
if an HMap has completeness tag \complete{} then \absent{} implicitly contains all keywords not in the domain of \mandatory{}.
%\HMapcwithabsent{\mandatory{}}{\absent{}} is abbreviated to \HMapc{\mandatory{}}. 
Keys cannot be both present and absent.

The expressions \clj{(get m :a)} and \clj{(:a m)} are semantically identical, though
we only model the former to avoid the added complexity of keywords being functions.
To simplify presentation, we only provide syntax for the empty map literal and
restrict lookup and extension to keyword keys. The metavariable \mapval{}
ranges over the runtime value of maps {\curlymapvaloverright{\k{}}{\v{}}},
usually written {\curlymapvaloverrightnoarrow{\k{}}{\v{}}}.

Subtyping for HMaps
designate \MapLiteral{} as a common supertype for all HMaps.
S-HMap says that an HMap is a subtype of another HMap if they agree
on \completenessmeta{}, agree on mandatory entries with subtyping
and at least cover the absent keys of the supertype.
Complete maps are subtypes of partial maps
as long as they agree on the mandatory entries of the partial map via subtyping (S-HMapP).

The typing rules for \getliteral{} consider three possible cases. T-GetHMap models a lookup
that will certainly succeed, T-GetHMapAbsent a lookup that will certainly fail
and T-GetHMapPartialDefault a lookup with unknown results.
Lookups on unions of HMaps are only supported in T-GetHMap, 
in particular to support
looking up \clj{:op} on a map of type \clj{Expr} (\egref{example:decleaf})
where every element in the union
contains the key we are looking up.
The objects in the T-Get rules are more complicated than those in T-Local---the 
next section discusses this in detail.
Finally T-AssocHMap extends an HMap with a mandatory entry while preserving completeness
and absent entries, and enforcing ${\k{}} \not\in {\absent{}}$ to prevent badly
formed types.

The semantics for \getliteral{} and \assocliteral{} are straightforward.
If the entry is missing, B-GetMissing produces \nil{}.

\subsection{Paths}
\label{sec:formalpaths}

Recall the first insight of occurrence typing---we can reason
about specific \emph{parts} of the runtime environment
using propositions.
We refer to parts of the runtime environment via 
a \emph{path} that consists of a series of
\emph{path elements} applied right-to-left to a variable
written \path{\pathelem{}}{\x{}}.
\citet{TF10} introduce the path elements \carpe{} and \cdrpe{}
to reason about selector operations on cons cells.
We instead want to reason about HMap lookups and calls to \classconst{}.

\paragraph{Key path element} We introduce our first path element
{\keype{\k{}}}, which represents the operation of looking up
a key \k{}.
We directly relate this to our typing rule T-GetHMap
(\figref{main:figure:hmapsyntax}) by
checking the then branch of the first conditional test is checked in 
an equivalent version of \egref{example:decleaf}.
\begin{minted}{clojure}
  (fn [m :- Expr]
    (if (= (get m :op) :if)
      {:op :if, ...}
      (if ...)))
\end{minted}

We do not specifically support \equivliteral{} in our calculus, 
but on keyword arguments it works identically to \clj{isa?} which we model
in \secref{sec:isaformal}.
Intuitively, if {\judgement{\propenv{}}{\e{}}{\t{}}{\filterset{\thenprop{\prop{}}}{\elseprop{\prop{}}}}{\object{}}}
then \equivapp{\e{}}{\makekw{if}} has the true and false propositions
$$
{\replacefor{\filterset{\isprop{\Value{\makekw{if}}}{\x{}}}{\notprop{\Value{\makekw{if}}}{\x{}}}}{\object{}}{\x{}}}
$$
where substitution reduces to \topprop{} if \object{} = \emptyobject{}.

We start with proposition environment \propenv{} = {\isprop{\Expr{}}{m}}.
Since {\Expr{}} is a union of HMaps, each with the entry \makekw{op}, we can use T-GetHMap.
$$
\judgement{\propenv{}}{\getexp{m}{\makekw{op}}}{\Keyword}{\filterset{\topprop{}}{\topprop{}}}{\path{\keype{\makekw{op}}}{m}}
$$
Using our intuitive definition of \equivliteral{} above, we know
$$
\judgement{\propenv{}}{\equivapp{\getexp{m}{\makekw{op}}}{\makekw{if}}}{\Boolean}{\filterset{\isprop{\Value{\makekw{if}}}{\path{\keype{\makekw{op}}}{m}}}{\notprop{\Value{\makekw{if}}}{\path{\keype{\makekw{op}}}{m}}}}{\emptyobject{}}
$$
Going down the then branch gives us the extended environment
\propenvp{} = {\isprop{\Expr{}}{m}},{\isprop{\Value{\makekw{if}}}{\path{\keype{\makekw{op}}}{m}}}.
Using L-Update we can combine what we know about object $m$ and object
{\path{\keype{\makekw{op}}}{m}}
to derive
$$
\inpropenv{\propenvp{}}{\isprop{\HMapp{\mandatoryset{{\mandatoryentrykwnoarrow{op}{\makekw{if}}}, {\mandatoryentrykwnoarrow{test}{\Expr{}}},
                                       {\mandatoryentrykwnoarrow{then}{\Expr{}}},   {\mandatoryentrykwnoarrow{else}{\Expr{}}}}}
                                   {\emptyabsent{}}}{m}}
$$

The full definition of \updateliteral{} is given in \figref{main:figure:update}
which considers both keys a path elements as well as the \classconst{}
path element described below.
In the absence of paths, update simply performs set-theoretic operations
on types; see \figref{main:figure:restrictremove} for details.

\paragraph{Class path element} Our second path element \classpe{} is used in the latent
object of the constant \classconst{} function. Like Clojure's \clj{class}
function \classconst{} returns the argument's class or \nil{}
if passed \nil{}.
$$
\begin{array}{lrlr}
  \pesyntax{}   &::=& \ldots \alt {\classpe{}}
                &\mbox{Path Elements}
\end{array}
$$
\begin{mathpar}
\constanttypefigure{}
\end{mathpar}
The dynamic semantics are given in \figref{main:figure:primitivesem}.
The definition of \updateliteral{} supports various idioms relating to \classpe{}
which we introduce in \secref{sec:isaformal}.

\subsection{Java Interoperability}

\begin{figure}[t]
  \footnotesize
  $$
  \begin{altgrammar}
    \e{} &::=& \ldots   
            \alt \reflectiveexpsyntax{} &\mbox{Expressions}\\
     &\alt& \mininonreflectiveexpsyntax{}
    \\

    \v{} &::=& \ldots \alt {\classvalue{\classhint{}} {\overrightarrow {\classfieldpair{\fld{}} {\v{}}}}}
    &\mbox{Values} \\

    \classtableallsyntax{}
  \end{altgrammar}
  $$
  \begin{mathpar}
    {\TNew}

    {\TMethod}

    %{\TInstance}
  \end{mathpar}
 %\classtablelookupsyntax{}
 \begin{mathpar}
  \begin{altgrammar}
    \convertjavatypenil{}
  \end{altgrammar}
  \begin{altgrammar}
    \convertjavatypenonnil{}
  \end{altgrammar}
  \\
  \begin{altgrammar}
    \converttctype{}
  \end{altgrammar}
\end{mathpar}
  \begin{mathpar}
    \BField{}

    \BNew{}

    \BMethod{}
  \end{mathpar}
  \caption{Java Interoperability Syntax, Typing and Operational Semantics}
  \label{main:figure:javatyping}
\end{figure}

We present Java interoperability in a restricted setting without class inheritance,
overloading or Java Generics.

We extend the syntax in \figref{main:figure:javatyping} with Java field lookups and calls to
methods and constructors. 
We model the syntax after the `dot' special
form to prevent ambiguity---\clj{(.fld e)} is now \fieldexp{\fld{}}{\e{}},
\clj{(.mth e es*)} is $\methodexp{\mth{}}{\e{}}{\overrightarrow{es}}$
and \clj{(new class es*)} is $\newexp{\class{}}{\overrightarrow{es}}$.

T-FieldStatic checks a resolved field expression by ensuring the target has
the correct static type, then returns a nilable type corresponding the Java type.

\begin{mathpar}
    {\TField}
\end{mathpar}

The rules T-MethodStatic and T-NewStatic work similarly (\figref{main:figure:javatyping}), varying
in the choice of nilability in the conversion function---methods
can return nil but constructors cannot. Void also does not have a constructor.

The evaluation rules B-Field, B-New and B-Method (\figref{main:figure:javatyping}) simply evaluate their
arguments and call the relevant JVM operation, which we do not model---\secref{sec:metatheory}
states our exact assumptions.

\subsection{Multimethod preliminaries: \isaliteral}

\label{sec:isaformal}

We now consider the \isaliteral{} operation, a core part of the dispatch mechanism for multimethods. 
Recalling the examples in \secref{sec:multioverview},
\isaliteral{} is
a subclassing test for classes, otherwise an equality test---we do not model
the semantics for vectors.

The key component of the T-IsA rule is the \isacompareliteral{} 
metafunction
(\figref{main:figure:mmsyntax}), used to calculate the propositions for
\isaliteral{} tests.
\begin{mathpar}
  \TIsA{}
\end{mathpar}

As an example,
\isaapp{\appexp{\classconst{}}{\x{}}}{\Keyword}
has the true and false propositions 
\isacompare{\s{}}{\path{\classpe{}}{\x{}}}{\Value{\Keyword}}{\filterset{\isprop{\Keyword}{\x{}}}{\notprop{\Keyword}{\x{}}}},
meaning that if this expression produces true, \x{} is a keyword, otherwise it is not.

The operational behavior of \isaliteral{} is given by B-IsA (\figref{main:figure:mmsyntax}). \isaopsemliteral{} explicitly handles classes in the second case.

%The definition of \isacompareliteral{} (figure~\ref{main:figure:mmsyntax}) is deliberately conservative.
%The first line considers the case where the object of the left argument
%is a non-empty path ending in \classpe{} and the type of the right argument is a singleton class.

\constantsemfigure{main}

\subsection{Multimethods}

\begin{figure}[t!]
  \footnotesize
$$
\begin{altgrammar}
  \e{} &::=& \ldots \alt {\createmultiexp {\t{}} {\e{}}} &\mbox{Expressions} \\
             &\alt& {\extendmultiexp {\e{}} {\e{}} {\e{}}}
             \alt {\isaapp {\e{}} {\e{}}}\\
  \v{} &::=& \ldots \alt {\multi {\v{}} {\disptable{}}}
                &\mbox{Values} \\
  \s{}, \t{} &::=& \ldots \alt {\MultiFntype{\t{}}{\t{}}}
                &\mbox{Types} \\

 \disptablesyntax{} \\
\end{altgrammar}
$$
  \begin{mathpar}
    \TDefMulti{}

    \TDefMethod{}
  \end{mathpar}
  \begin{mathpar}
    \isapropsfigure{}
  \end{mathpar}
  \begin{mathpar}
    \Multisubtyping{}
  \end{mathpar}
  \begin{mathpar}
    \BDefMethod{}
    \BDefMulti{}
    \BBetaMulti{}
  \end{mathpar}
  \getmethodfigure{}
$$
\begin{array}{ll}
  \vcenter{\hbox{\BIsA{}}}
  &
  \vcenter{\hbox{\isaopsemfigure{}}}
\end{array}
$$
\caption{Multimethod Syntax, Typing and Operational Semantics}
\label{main:figure:mmsyntax}
\end{figure}

To ease presentation, we present \emph{immutable}
multimethods, with syntax and semantics given in \figref{main:figure:mmsyntax}. 
\defmethodliteral{} returns a new extended multimethod
without changing the original multimethod. \egref{example:rep} is now written
\begin{minted}{clojure}
(let [path (defmulti [Any -> (U nil String)] class)]
  (let [path (defmethod path String [x] x)]
    (let [path (defmethod path File [^File x] 
                 (.getPath x))]
      (path (File. "dir/a"))))) ;=> "dir/a"
\end{minted}

The type {\MultiFntype {\s{}} {\sp{}}} characterizes multimethods with \emph{interface type}
{\s{}} and \emph{dispatch function type} {\sp{}}.
The expression {\createmultiexp {\s{}} {\e{}}} defines a multimethod
with interface type \s{} and dispatch function \e{}.
The expression {\extendmultiexp {\e{m}} {\e{v}}{\e{f}}}
extends multimethod \e{m} and to map
dispatch value {\e{v}} to {\e{f}} in an extended dispatch table.
The value {\multi {\v{}} {\disptable{}}} is the runtime value of a multimethod
with dispatch function {\v{}} and dispatch table {\disptable{}}.

The T-DefMulti rule ensures that the type of the dispatch function
has at least as permissive a parameter type
as the interface type.
%
For example, we can check the definition from our translation above of \egref{example:rep}
using T-DefMulti.
$$
\judgement{}%{\propenv{}}
{\createmultiexp 
      {\s{}}
      {\classconst{}}}
  {\MultiFntype {\s{}}{\sp{}}}{\filterset{\topprop{}}{\botprop{}}}{\emptyobject{}}
$$
where \s{}  = {\ArrowOne {\x{}} {\Top{}} {\t{}} {\filterset {\topprop{}} {\topprop{}}} {\emptyobject{}}}
  and \sp{} = {\ArrowOne {\x{}} {\Top{}} {\Union{\Nil}{\Class}} {\filterset {\topprop{}} {\topprop{}}} {\path{\classpe{}}{\x{}}}}.
  Since the parameter types agree, this is well-typed.

The T-DefMethod rule is carefully constructed to ensure we have a syntactic
lambda expression as the right-most subexpression.
This way we can manually check the body of the lambda under an extended
environment as sketched in \egref{example:incmap}.
We use \isacompareliteral{} to compute the proposition for this method,
since \isaliteral{} is used at runtime in multimethod dispatch.

We continue with the next line of the translation of \egref{example:rep}.
From the previous line we have \propenv{} = {\isprop{\MultiFntype {\s{}}{\sp{}}}{path}},
so
$$
\judgement{\propenv{}}
  {\extendmultiexp {prop} {\String}
                   {\abs {\x{}} {\Top{}} {\x{}}}}
  {\MultiFntype {\s{}}{\sp{}}}{\filterset{\topprop{}}{\botprop{}}}{\emptyobject{}}
$$
We know \emph{prop} is a multimethod by \propenv{}, so now we check the body
of this method.
$$
\judgement{\propenv{},{\isprop{\Top}{\x{}}},{\isprop{\String}{\x{}}}}
  {\x{}}
  {\String}{\filterset{\topprop{}}{\botprop{}}}{\emptyobject{}}
$$
%This is checked by T-Local since {\inpropenv{\propenv{},{\isprop{\Top}{\x{}}},{\isprop{\String}{\x{}}}}{\isprop{\String}{\x{}}}}.
The new proposition {\isprop{\String}{\x{}}} is derived by 
$$
  \isacompare{\Top{}}{\path{\classpe{}}{\x{}}}{\Value{\File{}}}
             {\filterset{\isprop{\String}{\x{}}}
                        {\notprop{\String}{\x{}}}}.
$$
%
The body of the \clj{let} is checked by T-App because
{\MultiFntype {\s{}}{\sp{}}} is a subtype of its interface type {\s{}}.

Multimethod definition semantics are straightforward.
B-DefMulti creates a multimethod with the given dispatch function and an empty dispatch table.
B-DefMethod produces a new multimethod with an extended dispatch table.
B-BetaMulti invokes the dispatch function with the evaluated argument to obtain the dispatch value,
and uses \getmethodliteral{} (which models Clojure's \clj{get-method}) 
to extract the appropriate method. The call to \getmethodliteral{} only returns a value
if there is \emph{exactly one} method such that the corresponding dispatch value
is compatible, using \isaopsemliteral{}, with the result of the dispatch function.
Finally we return the result of applying the extracted method and the original argument.

\begin{figure*}
  $$
\begin{array}{lllll}
\updatefigure
\end{array}
$$
\caption{Type Update}
\label{main:figure:update}
\end{figure*}

\begin{figure}
  $$
\begin{array}{lllll}
  \restrictremovefigure{}
\end{array}
  $$
  \caption{Restrict and Remove}
  \label{main:figure:restrictremove}
\end{figure}


\section{Metatheory}
\label{sec:metatheory}

We prove type soundness follow using the same technique as~\citet{TF10}. 
We also include errors and a \wrong{} value and prove
well-typed programs do not go wrong.

Rather than modeling Java's dynamic semantics, we instead
make our assumptions about Java explicit. We concede that
method and constructor calls may diverge or error, but we assume they can
never go wrong. (Assumptions for other operations are given in the supplemental
material).

{\javanewassumption{main}}



%For readability we define logical truth in Clojure.

%{\istruefalsedefinitions{main}}

For the purposes of our soundness proof, we require that all values
are \emph{consistent}.
Consistency
ensures that occurrence typing does not refer to variables
hidden inside a closure.

{\consistentwithonlydef{main}}

Our main lemma says if there is a defined reduction, then the propositions, object
and type are correct.
The metavariable \definedreduction{} ranges over \v{}, \errorvalv{} and \wrong{}.

\begin{lemma}\label{main:lemma:soundness}

  {\soundnesslemmahypothesis}
  \begin{proof}
    By induction on the derivation of the typing judgment. 
    (Full proof given as lemma~\ref{appendix:lemma:soundness}).
  \end{proof}
\end{lemma}


We can now state our soundness theorem.

{\soundnesstheoremnoproof{main}}

{\wrongtheoremnoproof{main}}
%
%{\nilinvoketheoremnoproof{main}}


\section{Experience}
\label{sec:experience}

Typed Clojure is implemented as \coretyped{}~\cite{coretyped},
which has seen wide usage.

\subsection{Implementation}

\coretyped{} provides preliminary integration with the Clojure compilation
pipeline, primarily to resolve Java interoperability.
%, however
%most usages are entirely optionally typed.

%In contrast to Racket, Clojure does not provide extension
%points to the macroexpander. 
%To satisfy our goals of providing
%Typed Clojure as a library that works with the latest version of the Clojure
%compiler, \coretyped{} is implemented as an external static analysis pass
%that must be explicitly invoked by the programmer, and not as an
%integral part of the Clojure compilation process. 
%%Therefore, \coretyped{} is in a sense a linter.

The \coretyped{} implementation extends this paper in several key areas 
to handle checking real Clojure code, including an implementation
of Typed Racket's variable-arity polymorphism~\cite{stf-esop}, 
and support for other Clojure idioms like datatypes and protocols.
%
There is no integration with Java Generics, so only Java 1.4-style erased types are ``trusted''
by \coretyped{}.
Casts are needed to recover the discarded information, which---for collections---are 
then tracked via Clojure's universal sequence interface~\cite{CljSeqDoc}.

%Recently, steps have been taken to integrate Typed Clojure into 

%This means that type checking is  optional. 
%On the positive side, \coretyped{} is flexible to the needs of a dynamically
%typed programmer, encouraging experimentation with programs that may not
%type check.
%On the negative side, programmers must remember to type check their namespaces.
%Also, the compiler cannot depend on types, making
%type-based optimisation is impossible. 
%If this were not the case, we could dispose of type-hints
%altogether, and simply use static types to resolve reflection.

%\subsection{Let-aliasing}
%
%\begin{mathpar}
%  \footnotesize
%\infer [T-LocalAlias]
%{ \Theta[\x{}] = \object{}
%  \\
%  \inpropenv {\propenv{}} {\isprop {\t{}} {\object{}}}
%  \\\\
%  \s{} = {\falsy} }
%{ \judgement {\Theta; \propenv{}} 
%             {\hastype {\x{}} {\t{}}}
%             {\filterset {{\notprop {\s{}} {\object{}}}} {{\isprop {\s{}} {\object{}}}}}
%             {\object{}}
%                   }
%
%\infer [T-LetAlias]
%{ \judgement {\Theta; \propenv{}} {\hastype {\e{1}} {\s{}}} {\filterset {\thenprop {\prop{1}}} {\elseprop {\prop{1}}}}
%             {\object{1}}
%  \\\\
%  \object{1} \notequal \emptyobject{}
%  \\\\
%  \judgement
%       {\Theta[\x{} \mapsto \object{1}];
%         \propenv{}}
%             {\hastype {\e{}} {\t{}}} {\filterset {\thenprop {\prop{}}} {\elseprop {\prop{}}}}
%             {\object{}} 
%             }
%{ \judgement {\Theta; \propenv{}} {\hastype {\letexp {\x{}} {\e{1}} {\e{}}} {\t{}}}
%             {\filterset {\thenprop {\prop{}}} {\elseprop {\prop{}}}}
%             {\object{}} 
%             }
%\end{mathpar}

%\subsection{Further Extensions}
%
%In addition to the key features we present in this paper,
%\coretyped{} supports other extensions to handle additional Clojure
%features. 
%
%\smallsection{Datatypes, Records and Protocols}
%Clojure features datatypes and protocols. Datatypes are Java classes
%declared final with public final fields. They can implement Java interfaces
%or protocols, which are similar to interfaces but already-defined classes
%and \nil{} may extend protocols.
%%
%Typed Clojure can reason about most of these features,
%including the ability to define polymorphic datatypes and protocols and
%utilising the Java type system to help check implemented interface methods.

%\smallsection{Intersection Types}
%Typed Clojure includes simple intersection types which do no sophisticated
%reasoning with the dual subtyping rules to unions.
%
%In some cases this makes types more expressive. Say we know \clj{x} has some
%universally quantified type \clj{a} and we learn \clj{x} is a \clj{Number}.
%Without intersection types, we must choose which piece of information to forget.
%In Typed Clojure, \clj{x} is simply of type \clj{(I x Number)}.
%
%\smallsection{Mutation and Polymorphism}
%Clojure supports mutable references with software-transactional-memory
%which Typed Clojure defines \emph{bivariantly}---with write and read type parameters
%as in the atomic reference \clj{(Atom2 Int Int)} which can write and read \clj{Int}.
%Typed Clojure also supports parametric polymorphism, including
%Typed Racket's variable-arity polymorphism~\cite{stf-esop}, 
%which enables us to assign a type to functions like \clj{swap!} (\figref{main:fig:swap!}),
%which takes a mutable \emph{atom},
%a function and extra arguments, and swaps into the atom the result of
%applying the function to the atom's current value and the extra arguments.
%
%\begin{figure}
%\begin{minted}{clojure}
%(ann clojure.core/swap! (All [w r b ...] 
%                          [(Atom2 w r) [r b ... b -> w] b ... b -> w]))
%(swap! (atom :- Num 1) + 2 3);=> 6 (atom contains 6)
%\end{minted}
%%\inputminted[firstline=5,lastline=5]{clojure}{code/demo/src/demo/atom.clj}
%\caption{Type annotation and example call of \clj{swap!}}
%\label{main:fig:swap!}
%\end{figure}

\subsection{Evaluation}
\label{sec:casestudy}

Throughout this paper, we have focused on three interrelated type
system features: heterogenous maps, Java interoperability, and
multimethods. Our hypothesis is that these features are widely used in
existing Clojure programs in interconnecting ways, and that handling
them as we have done is required to type check realistic Clojure
programs.



To evaluate this hypothesis, we analyzed two existing \coretyped{}
code bases, one from the open-source community, and one from a company
that uses \coretyped{} in production. For our data gathering, we
instrumented the \coretyped{} type checker to record how often
various features were used (summarized in 
\figref{experience:featuretable}). 

\begin{figure*}[t]

\begin{tabular}{lll}
      \toprule


  & feeds2imap & CircleCI \\
  \midrule
  Total number of typed namespaces & 11 (825 LOC) & 87 (19,000 LOC) \\
  Total number of \clj{def} expressions & 93  & 1834 \\
  \tabitem
  checked & 52 (56\%) & 407 (22\%) \\
  \tabitem
  unchecked & 41 (44\%) & 1427 (78\%) \\
  Total number of Java interactions & 32 & 105 \\
  \tabitem
  static methods & 5 (16\%) & 26 (25\%) \\ 
  \tabitem
  instance methods & 20 (62\%) & 36 (34\%) \\
  \tabitem
  constructors & 6 (19\%) & 38 (36\%) \\
  \tabitem
  static fields & 1 (3\%) & 5 (5\%) \\
  Methods overriden to return non-nil & 0 & 35 \\
  Methods overriden to accept nil arguments & 0 & 1 \\
  Total HMap lookups & 27  & 328  \\
  \tabitem
  resolved to mandatory key & 20 (74\%) & 208 (64\%) \\
  \tabitem
  resolved to optional key & 6 (22\%) & 70 (21\%) \\
  \tabitem
  resolved of absent key & 0 (0\%) & 20 (6\%) \\
  \tabitem
  unresolved key & 1 (4\%) & 30 (9\%) \\
  Total number of \clj{defalias} expressions & 18  & 95 \\
  \tabitem
  contained HMap or union of HMap type & 7 (39\%)  & 62 (65\%) \\
  Total number of checked \clj{defmulti} expressions & 0  & 11 \\
  Total number of checked \clj{defmethod} expressions & 0  & 89 \\


\end{tabular}
\caption{Typed Clojure Features used in Practice}
\label{experience:featuretable}
\end{figure*}


\paragraph{feeds2imap}
feeds2imap\footnote{https://github.com/frenchy64/feeds2imap.clj}
is an open source library written in Typed Clojure. 
It provides an RSS reader using the \emph{javax.mail} framework.

% static call (:check/:static-call) = 74
% - user   5
% - inlined (:check/static-call-clojure-lang-probably-inline) 69
% static field = 13
% - user 1
% - inlined (:check/static-field-clojure-lang-probably-inline) 12
% new (:check/:new) = 11
% - user 6
% - inlined (:check/new-clojure-lang-probably-inline) 5
% instance call = 53
% - body  20
% - inlined (:check/instance-call-clojure-lang-probably-inline) 33
% total 151
% - user  32
Of 11 typed namespaces containing 825 lines of code, there are 32 Java interactions.
The majority are method calls, consisting of 20 (62\%) instance methods and 5 (16\%) static methods. 
The rest consists of 1 (3\%) static field access, and 6 (19\%) constructor calls---there are no instance field accesses.

%  from :check/find-val-type-with-hmap* numbers
There are 27 lookup operations on HMap types, of which 20 (74\%) resolve to mandatory entries, 6 (22\%) to optional entries, and 1 (4\%) is an unresolved lookup. 
No lookups involved fully specified maps.

% :collect/:def     93
% :check/checked-def 52
From 93 \clj{def} expressions in typed code, 52 (56\%) are checked, with a rate of 1 Java interaction for 1.6 checked top-level definitions, and 1 HMap lookup to 1.9 checked top-level definitions.
That leaves 41 (44\%) unchecked vars, mainly due to partially complete porting to Typed Clojure, but in some cases due to unannotated third-party libraries.

No typed multimethods are defined or used. 
% :collect/defalias-is-HMap      7
% :invoke-special-collect/(quote clojure.core.typed/def-alias*)     18
Of 18 total type aliases, 7 (39\%) contained one HMap type, and none contained unions of HMaps---on further inspection there was no HMap entry used to dictate control flow, often handled by multimethods.
This is unusual in our experience, and is perhaps explained by feeds2imap mainly wrapping existing \emph{javax.mail} functionality.

\paragraph{CircleCI}
CircleCI~\cite{CircleCI}
provides continuous integration services built with a mixture of open-
and closed-source.
Typed Clojure was used at CircleCI in production systems for two years \cite{CircleCIUsesTC},
maintaining
87 namespaces and 19,000 lines of code.
%an experience we summarise in \secref{sec:limitations}.
%
%CircleCI provided the first author access to the main closed-source backend system written in Clojure
%and Typed Clojure.
%We conducted a study of the effectiveness of Typed Clojure in practice.
%There is no clear metric for quantifying typed Clojure code, since untyped code
%can be freely mixed and some seemingly typed namespaces are not checked
%regularly. 
%We manually type checked all namespaces that depend on \clj{clojure.core.typed}
%and considered those with type errors as untyped.
%We then searched the remaining typed code for unsafe Typed Clojure operations like
%var annotations with \clj{:no-check} and the \clj{tc-ignore} macro,
%which instruct Typed Clojure to ignore the specified code,
%and also considered those untyped.
%Furthermore, we manually collected and inspected all top-level annotations and
%classified them.
%
%We determined that

%% Out of 588 top-level var annotations, 270 (46\%) were checked annotations of
%% functions defined in typed code,
%% 129 (22\%) annotations assigned types to external libraries 
%% and the remaining 189 (32\%) annotated `unchecked' user code.
%Some of the type-annotated definitions were so annotated by the first
%author and contributed back to CircleCI.
%HMaps were a valuable feature, with 38 (59\%) out of 64 total type aliases
%featuring them; see \egref{example:circleci} for an instance.
%
%Because of various shortcomings of \coretyped{}, all 57 \clj{defmethod}
%expressions in typed namespaces were unchecked.
%
%811 top-level var annotations
%
%% Due to a lack of checked multimethods,
%% the first author ported 11 previously-untyped multimethods to Typed Clojure, also checking 
%% 89 methods.

The CircleCI code base contains 11 checked multimethods.
 All 11 dispatch functions
are on a HMap key containing a keyword, in a similar style to
\egref{example:desserts-on-meal}.
Correspondingly, all 89 methods are associated with a keyword dispatch value.
The argument type was in all cases a single HMap type, however,
rather than a union type.
In our experience from porting other libraries, this is unusual.

% 87 typed namespaces
% :check/gen-analysis     87

% :check/find-val-type-with-hmap    328
% :check/find-val-type-with-hmap-present    208
% :check/find-val-type-with-hmap-with-optional     70
% :check/find-val-type-with-hmap-fall-through     30
% :check/find-val-type-with-hmap-absent     20
% :check/find-val-has-complete      2
% :merge/complete-used-on-right      5

Of 328 lookup operations on HMaps,
208 (64\%) resolve to mandatory keys,
70 (21\%) to optional keys,
20 (6\%) to absent keys, and
30 (9\%) lookups are unresolved.
%
% :collect/defalias-is-HMap     62
% :invoke-special-collect/(quote clojure.core.typed/def-alias*)     95 
Of 95 total type aliases defined with \clj{defalias},
62 (65\%) involved one or more HMap types.
%
%% :new-special/(quote clojure.lang.MultiFn)     11
%
%% :check/:static-call    525
%% :check/static-call-clojure-lang-probably-inline    499
%% = 26 user
%
%% :check/:instance-call    510
%% :check/instance-call-clojure-lang-probably-inline    474
%% = 36 user
%
%% :check/:new    159
%% :check/new-clojure-lang-probably-inline    121
%% = 38
%
%% :check/:static-field     92
%% :check/static-field-clojure-lang-probably-inline     87
%% = 5
%
%% 26 + 36 + 38 + 5 = 105
%
%% :invoke-special-collect/(quote clojure.core.typed/non-nil-return*) 35
%% :invoke-special-collect/(quote clojure.core.typed/nilable-param*)  1
%
Out of 105 Java interactions, 26 (25\%) are static methods, 36 (34\%)
are instance methods, 38 (36\%) are constructors, and 5 (5\%) are static
fields. 35 methods are overriden to return non-nil, and 1 method 
overridden to accept nil---suggesting that
\coretyped{} disallowing \clj{nil} as a method argument by default
is justified.

% :check/checked-def  407
% :check/checked-MultiFn-addMethod 57
% :instance-method-special/(quote clojure.lang.MultiFn/addMethod)     89
% = 464 checked definitions
Of 464 checked top-level definitions (which consists of
57 \clj{defmethod} calls and 407 \clj{def} expressions),
1 HMap lookup occurs per 1.4 top-level definitions,
and 1 Java interaction occurs every 4.4 top-level definitions.

% :check/def-not-checking-definition   1352
% :check/checked-def  407
% = 1759
% :collect/:def   1834
% = 1427 unchecked
From 1834 \clj{def} expressions in typed code,
%87 typed namespaces,
only 407 (22\%) were checked.
That leaves 1427 (78\%) which have unchecked definitions, either by an explicit \clj{:no-check} annotation
or \clj{tc-ignore} to suppress type checking,
or the \clj{warn-on-unannotated-vars} option, which skips \clj{def} expressions
that lack expected types via \clj{ann}.
From a brief investigation,
reasons include unannotated third-party libraries,
work-in-progress conversions to Typed Clojure,
unsupported Clojure idioms, 
and hard-to-check code.

\paragraph{Lessons}
Based on our empirical survey, HMaps and Java interoperability support
are vital features used on average more than once per typed
function. 
%
Multimethods are less common
in our case studies. The CircleCI code base contains only 26 multimethods total
in 55,000 lines of mixed untyped-typed Clojure code,
a low number in our experience.

%The
%data therefore validates our choice of a type system that supports
%expressive multimethod definition and acknowledges the relationship
%between these seemingly-distinct features. 

%
%The other lesson from our case studies and from other interactions
%with Typed Clojure users, it is clear the main barrier to entry to
%Typed Clojure for large systems is the requirement to annotate
%functions outside the borders of typed code.  We hope that this
%can be addressed by making annotations available for popular
%libraries.

\subsection{Further challenges}
\label{sec:limitations}

After a 2 year trial, the second case study decided to disabled type checking~\cite{CircleCIBlog}.
They were supportive of the fundamental ideas presented in this paper, but primarily
cited issues with the checker implementation in practice and would reconsider
type checking if they were resolved. This is also supported by \figref{experience:featuretable},
where 78\% of \clj{def} expressions are unchecked.

\smallsection{Performance}
Rechecking files with transitive dependencies is expensive since all dependencies must be rechecked.
We conjecture caching type state will significantly
improve re-checking performance,
though preserving static soundness in the context of arbitrary code reloading is a largely unexplored area.

\smallsection{Library annotations}
Annotations for external code are rarely available, so a large part of the
untyped-typed porting process is reverse engineering libraries.

\smallsection{Unsupported idioms}
While the current set of features is vital to checking Clojure code,
there is still much work to do.
For example, common Clojure functions are often too polymorphic for the current implementation
or theory to account for. The post-mortem~\cite{CircleCIBlog} contains more details.

%\smallsection{Java Arrays}
%Java arrays are known to be statically unsound.
%\cite{Bra98} summarises the approach taken to regain runtime soundness, which involves
%checking array writes at runtime.
%
%Typed Clojure implements an experimental partial solution, making arrays \emph{bivariant},
%separating the write and read types into contravariant and covariant parameters.
%If the array originates from typed code, then we may track the write and read
%parameters statically. Currently arrays from foreign sources
%have their write parameter set to to \Bot{}, protecting typed code from writing
%something of incorrect type. However there are currently no casting mechanisms to 
%convince Typed Clojure the foreign array is writeable.

%\smallsection{Array-backed sequences}
%Typed Clojure assumes sequences are immutable. This is almost always true, however
%for performance reasons,
%sequences created from Java arrays (and Iterables) reflect future writes to the array 
%in the `immutable' sequence. While disturbing and a clear unsoundness in Typed Clojure,
%this has not yet been an issue in practice and is strongly discouraged as undefined behavior:
%``Robust programs should not mutate arrays or Iterables that have seqs on them.''~\cite{CljSeqDoc}.
%
%\smallsection{Typed-untyped interoperation}
%Currently, interactions between typed and untyped Clojure code are unchecked
%which can violate the expectations of Typed Clojure.
%Gradual typing~\cite{thf06,siek06:_gradual} ensures sound interoperability between typed and untyped code by enforcing
%invariants of the type system via run-time contracts.
% We hope to add support
%for gradual typing in the future.



%OLD

%\subsection{Using negative filters}
%
%Occurrence typing plays an important role in Typed Racket and Typed Clojure.
%By maintaining a \emph{proposition environment} of propositions relating types to
%bindings, we can update bindings with more accurate types as programs progress.
%It follows that there is some correspondence between propositions and types,
%characterised by the \emph{update} function, which takes a type and a proposition
%and returns a type which updates the input type using the proposition.
%
%There is a straightforward relationship between ``positive'' propositions and types.
%For example 
%{\tt (update Number (is Integer 0))}
%updates Number by Integer, which is Integer, because Integer <: Number.
%
%The relationship between ``negative'' propositions and types is not always obvious.
%A common proposition in Typed Clojure is (! (U nil false) a): the proposition that
%local binding ``a'' is \emph{not} of type (U nil false).
%This problem is most visible in expressions like {\tt (filter identity coll)}, where
%``identity'' has a ``then'' proposition that has negative information: (! (U nil false) 0),
%which reads, the 0th argument of identity does not contain (U nil false).
%
%\subsubsection{Arrays}
%\label{sec:arrays}
%
%Supporting statically sound interactions with Java arrays is a goal
%of Typed Clojure. This is complicated by Java's decision to make
%arrays covariant in their argument, a well documented source of static
%unsoundness. Bracha~\cite{Bra98} summarises Java's approach to maintaining
%soundness at runtime, which involves all array writes being checked by
%runtime assertions.
%
%This approach fits Java's type system, but we can do better in a more powerful
%type system like Typed Clojure. Our goal is to catch all type-incorrect array
%writes at compile time so the type system can do more to help Clojure programmers
%use arrays, especially those being passed from foreign Java code.
%
%Our basic approach is to make our array types \emph{bivariant}. Array types
%look like {\ArrayTwo {\t{w}} {\t{r}}} and
%are reminiscent of functions or pipes: having a contravariant parameter for input (writing)
%and a covariant parameter for output (reading).
%This type can write type {\t{w}} and read type {\t{r}}.
%
%Most commonly, an array type is invariant in its parameter; it can
%write and read input of the same type.
%We can get the same effect by setting our input and output
%parameters to the same type. For example, {\ArrayTwo {\Number} {\Number}}
%(or equivalently, {\Array {\Number}})
%in Typed Clojure is similar to invariant array types of \Number in languages like Scala.
%
%The biggest gain in using a separate input parameter is the ability
%to specify \emph{read-only} arrays. Crucially, our type system features an
%explicit bottom type \lstinline|Nothing|, enabling a read-only \lstinline|Number| array
%to be of type \lstinline|(Array2 Nothing Number)|.
%
%To realise why defining read-only arrays are useful, we need to examine
%what makes array covariance unsound in Java.
%\begin{verbatim}
%FIXME
%Array covariance about the type of an array so the consumer
%of an array cannot tell the actual type of the array when examining a type
%signature.
%\end{verbatim}
%
%\begin{lstlisting}
%...
%public static Number[] getNumberArray() {
%  Number[] n = new Integer[10];
%  return n;
%}
%...
%\end{lstlisting}
%
%To the casual consumer \emph{getNumberArray} returns an array that can both
%read and write \lstinline|Number|s. However it is clear from the implementation
%that attempting to write say a \lstinline|Double| to this array will result
%in a runtime error.
%
%\begin{verbatim}
%...
%Number[] myArray = getNumberArray();
%myArray[0] = 1.1;
%/* Exception in thread "main" 
%   java.lang.ArrayStoreException: 
%   java.lang.Double */
%...
%\end{verbatim}
%
%Notice that this is a runtime error, and Java's type system has not helped
%statically prevent it.
%This could cause a similar issue for other statically-typed languages offering
%interoperability with Java. 
%
%To prevent these sorts of runtime exceptions in Typed Clojure, we declare
%all arrays from unknown sources to be \emph{read-only}. Put differently,
%the only way to define a writeable array is to create it in type-checked Clojure
%code.
%
%\begin{lstlisting}
%(let [n (CovariantArray/getNumberArray)]
%  (aset n 0 1.1))
%
%; Polymorphic static method clojure.lang.RT/aset could not be 
%; applied to arguments:
%; Domains: 
%;         (Array2 i o) clojure.core.typed/AnyInteger i
%; 
%; Arguments:
%;         (Array2 Nothing java.lang.Number) int (Value 1.1)
%; 
%; with expected type:
%;         Any
%\end{lstlisting}
%
%The type inferred for the local \lstinline|n| is \lstinline|(Array2 Nothing Number)|
%which tells the type system: it is never safe to write to this array, but
%it is safe to assume \lstinline|Number|s can be read from this array.
%
%To emphasise, Typed Clojure throws a static type error. Errors like this help Clojure programmers
%use foreign Java libraries more correctly.
%
%\begin{verbatim}
%Note that Java libraries are often large 
%and complex and programmers will probably
%enjoy the extra help from the type system.
%\end{verbatim}


\section{Related Work}

% Cite a few of the early papers here.
%http://www.cs.washington.edu/research/projects/cecil/www/pubs/
\paragraph{Multimethods} 
\citeauthor{MS02} and collaborators present a sequence of
systems~\cite{Chambers:1992:OMC, Chambers:1994:TMM, MS02} with statically-typed multimethods
and modular type checking.  In contrast to Typed Clojure, in these
system methods declare the types of arguments that they expect which
corresponds to exclusively using \clj{class} as the dispatch function
in Typed Clojure. However, Typed Clojure does not attempt to rule out
failed dispatches at runtime.

% one sentence
% TC based on TR, already covered

%\paragraph{Occurrence Typing} 
%Occurrence typing~\cite{TF08,TF10} extends the type 
%system with a \emph{proposition environment} that represents 
%the information on the types of bindings down conditional branches.
%These propositions are then used to update the types associated
%with bindings in the \emph{type environment} down branches
%so binding occurrences are given different types 
%depending on the branches they appear in, and the conditionals
%that lead to that branch.

% What's diff about TC from the related work
% small summary for deisel....
% - diesel supports x
%- - calculus supports some subset of x
% we support y, which covers most of x but also foo

% eg. multiple dispatch
%     nominal vs structural

% eg. run abritrary metaprogramming over dispatch in CLOS
%  more expressive

% type systems for mm or rows
% rows vs HMap
% - no poly in HMap
% - based on subtyping
% - rows based on polymorphism

\paragraph{Record Types} Row polymorphism~\cite{Wand89typeinference,CM91,HP91}, used
in systems such as the OCaml object system, provides many of the
features of HMap types, but defined using universally-quantified row
variables. HMaps in Typed Clojure are instead designed to be used with
subtyping, but nonetheless provide similar expressiveness, including
the ability to require presence and absence of certain keys. 

Dependent JavaScript~\cite{Chugh:2012:DTJ} can track similar
invariants as HMaps with types for JS objects. They must deal with
mutable objects, they feature refinement types and strong updates to
the heap to track changes to objects.

Typed Lua~\cite{Maidl:2014:TLO} has \emph{table types} which track
entries in a mutable Lua table.  Typed Lua changes the dynamic
semantics of Lua to accommodate mutability: Typed Lua raises a runtime
error for lookups on missing keys---HMaps consider lookups on missing
keys normal.

The integration of completeness information, crucial for many examples
in Typed Clojure, is not provided by any of these systems.

\paragraph{Java Interoperability in Statically Typed Languages}
Scala~\cite{OCD+} has nullable references for compatibility with Java.
Programmers must manually check for
\java{null} as in Java to avoid null-pointer exceptions. 


\paragraph{Optional type systems}
Reticulated Python~\cite{Vitousek14} is a gradually typed
system for Python, implemented as a source-to-source
translation that inserts dynamic checks at language boundaries
and supporting Python's first-class object system.
Typed Clojure is implemented as a syntactic check on Clojure
code with no option to insert dynamic checks automatically.
Typed Clojure does not support a first-class object system
because Java has nominal classes, however HMaps offer a subset 
of the structural features offered by Reticulated.

%  \item GradualTalk
%  \item Flow
%\end{itemize}




\section{Conclusion}
\label{sec:conclusion}

Optional type systems must be designed with close attention to the
language that they are intended to work for.
We have therefore designed Typed Clojure, an optionally-typed version of
Clojure, with a type system that works with a wide variety of distinctive
Clojure idioms and features. Although based on the foundation of Typed
Racket's occurrence typing approach, Typed Clojure both extends the
fundamental control-flow based reasoning as well as applying it to
handle seemingly unrelated features such as multi-methods. In
addition, Typed Clojure supports crucial features such as
heterogeneous maps and Java interoperability while integrating these
features into the core type system. Not only are each of these
features important in isolation to Clojure and Typed Clojure
programmers, but they must fit together smoothly to ensure that
existing Clojure programs are easy to convert to Typed Clojure.

The result is a sound, expressive, and useful type system which, as
implemented in \coretyped with appropriate extensions, is suitable for
typechecking a significant amount of existing Clojure programs.
%
As a result, Typed Clojure is already successful: it is used in
the Clojure community among both enthusiasts and professional
programmers.% and receives contributions from many developers.

%Our empirical analysis of existing Typed Clojure programs bears out
%our design choices. Multimethods, Java interoperation, and
%heterogeneous maps are indeed common in both Clojure and Typed Clojure,
%meaning that our type system must accommodate them. Furthermore, they
%are commonly used together, and the features of each are mutually
%reinforcing. Additionally, the choice to make Java's \clj{null}
%explicit in the type system is validated by the many Typed Clojure
%programs that  specify non-nullable types.


% Delete the following paragraphs if space is needed.

%However, there is much more that Typed Clojure can provide. Most
%significantly, Typed Clojure currently does not provide \emph{gradual
%  typing}---interaction between typed and untyped code is unchecked and
%thus unsound. We hope to explore the possibilities of using existing
%mechanisms for contracts and proxies in Java and
%Clojure to enable sound gradual typing for Clojure.
%
%Additionally, the Clojure compiler is unable to use Typed Clojure's
%wealth of static information to optimize programs. Addressing this
%requires not only  enabling sound gradual typing, but also
%integrating Typed Clojure into the Clojure tool so
%that its information can be communicated to the compiler. 

%Finally, our case study, evaluation, and broader experience indicate that Clojure
%programmers still find themselves unable to use Typed Clojure on some
%of their programs for lack of expressiveness. This requires continued
%effort to analyze and understand the features and idioms and
%develop new type checking approaches.


\subsection*{Acknowledgements}

Thanks to Andrew Kent and Andre Kuhlenschmidt for comment on drafts of this paper.


% We recommend abbrvnat bibliography style.

\bibliographystyle{abbrvnat}

% The bibliography should be embedded for final submission.

\bibliography{bibliography}

\clearpage

\counterwithin{figure}{section}
\counterwithin{assumption}{section}
\counterwithin{theorem}{section}
\counterwithin{lemma}{section}
\counterwithin{definition}{section}

\onecolumn
\appendix

\section{Soundness for Typed Clojure}

% TODO
\begin{assumption}[\newjavaliteral] \label{assumption:new}
  If\ $\forall i.\ {\v{i}} = {\classvalue{\classhint{i}}{\overrightarrow{\classfieldpair{\fld{j}} {\v{j}}}}}\ or\ {\v{i}}= {\nil}$
  and $\newjava {\classhint{}}
                {\overrightarrow{\classhint{i}}}
                {\overrightarrow{\v{i}}}
                {\v{}}$
                  then either
                  \begin{itemize}
                    \item
                      \v{} = 
                  ${\classvalue{\classhint{}}{\overrightarrow {\classfieldpair{\fld{k}} {\v{k}}}}}$
                  or
                \item
                  \v{} = 
                  {\errorval{\v{e}}}.
                  \end{itemize}
\end{assumption}

\begin{assumption}[\getfieldliteral] \label{assumption:field}
  If\ {\v{1}} = ${\classvalue{\classhint{1}}{{\classfieldpair{\fld{}}{\v{f}}, {\overrightarrow{\classfieldpair{\fld{l}} {\v{l}}}}}}}$,
         and \getfieldjava{\classhint{1}} {\v{1}} {\fld{}} {\classhint{2}} {\v{}}
                  then either
                  \begin{itemize}
                    \item \v{} = \v{f} and either 
                  \v{f} = ${\classvalue{\classhint{2}}{\overrightarrow{\classfieldpair{\fld{m}} {\v{m}}}}}$
                  \ or\ 
                  \v{f} = \nil,  or
                \item
                  \v{} = {\errorval{\v{e}}}.
                  \end{itemize}
\end{assumption}

\begin{assumption}[\invokejavamethodliteral] \label{assumption:method}
  If\ {\v{1}} = {\classvalue{\classhint{1}}{\overrightarrow{\classfieldpair{\fld{l}} {\v{l}}}}},
  $\forall i.\ {\v{i}}={\classvalue{\classhint{i}}{\overrightarrow{\classfieldpair{\fld{j}} {\v{j}}}}}\ or\ {\v{i}}={\nil}$
         and 
  \invokejavamethod {\classhint{1}} {\v{m}} {mth}
                    {\overrightarrow{\classhint{i}}} {\overrightarrow{\v{i}}}
                    {\classhint{2}}
                    {\v{}}
                  then either
                  \begin{itemize}
                    \item
                  \v{} = ${\classvalue{\classhint{2}}{\overrightarrow{\classfieldpair{\fld{m}} {\v{m}}}}}$
                  or\ 
                  \v{} = \nil,  or
                \item
                  \v{} = {\errorval{\v{e}}}.
                  \end{itemize}
\end{assumption}

\begin{lemma}
If \judgement{\propenv{}}{\hastype{\e{}}{\t{}}}{\filterset{\thenprop{\prop{}}}{\elseprop{\prop{}}}}{\object{}},
\satisfies{\openv{}}{\propenv{}} 
and \opsem {\openv{}} {\e{}} {\v{}} 
then all of the following hold:
\begin{enumerate}
  \item either \object{} = \emptyobject{} or \inopenv {\openv{}} {\object{}} {\v{}},
  \item either \v{} $\not=$ \false\ (or \nil) and {\satisfies{\openv{}}{\thenprop{\prop{}}}} or 
               \v{}       = \false\ (or \nil) and {\satisfies{\openv{}}{\elseprop{\prop{}}}}, and
  \item \judgement{}{\hastype{\v{}}{\t{}}}{\filterset{\thenprop{\propp{}}}{\elseprop{\propp{}}}}{\objectp{}}
        for some \thenprop{\propp{}}, \elseprop{\propp{}} and {\objectp{}}.

\begin{proof}
By induction on the derivation of the typing judgement.

\begin{case}[T-True]
\e{} = \true, \t{} = \True, \thenprop{\prop{}} = \topprop{}, \elseprop{\prop{}} = \botprop{}, \object{} = \emptyobject{}

\begin{itemize}
  \item[] 
    \begin{subcase}[B-Val]
      \v{} = \true{}

Proving part 1 is trivial: \object{} is \emptyobject. 
To prove part 2, we note that \v{} = \true\ 
and \thenprop{\prop{}} = \topprop{}, so \satisfies{\openv{}}{\thenprop{\prop{}}} by M-Top.
Part 3 holds as \e{} can only be reduced to itself via B-Val.
\end{subcase}

\end{itemize}

\begin{case}[T-EmptyMap]

  \begin{itemize}
    \item[]
      \begin{subcase}[B-Val]
        Similar to T-True.
      \end{subcase}
  \end{itemize}
\end{case}

\begin{case}[T-Kw]

  \begin{itemize}
    \item[]
      \begin{subcase}[B-Val]
        Similar to T-True.
      \end{subcase}
  \end{itemize}
\end{case}

\end{case}

\begin{case}[T-False]
\e{} = \false, \t{} = \False, \thenprop{\prop{}} = \botprop{}, \elseprop{\prop{}} = \topprop{}, \object{} = \emptyobject{}

\begin{itemize}
  \item[] 
    \begin{subcase}[B-Val]
      \v{} = \false{}

Proving part 1 is trivial: \object{} is \emptyobject. To prove part 2, we note that \v{} = \false\ 
and \elseprop{\prop{}} = \topprop{}, so \satisfies{\openv{}}{\elseprop{\prop{}}} by M-Top. 
Part 3 holds as \e{} can only be reduced to itself via B-Val.
\end{subcase}

\end{itemize}
\end{case}

\begin{case}[T-Nil]
\e{} = \nil, \t{} = \Nil, \thenprop{\prop{}} = \botprop{}, \elseprop{\prop{}} = \topprop{}, \object{} = \emptyobject{},

\begin{itemize}
  \item[] 
    \begin{subcase}[B-Val] 
      \v{} = \nil{}

Proving part 1 is trivial: \object{} is \emptyobject. To prove part 2, we note that \v{} = \nil\ 
and \elseprop{\prop{}} = \topprop{}, so \satisfies{\openv{}}{\elseprop{\prop{}}} by M-Top. 
Part 3 holds as \e{} can only be reduced to itself via B-Val.
\end{subcase}

\end{itemize}

\end{case}

\begin{case}[T-Local]
  \e{} = \x{}, \thenprop{\prop{}} = {\notprop {\falsy{}} {\x{}}},
  \elseprop{\prop{}} = {\isprop {\falsy{}} {\x{}}},
\object{} = \x{}, 
\inpropenv{\propenv{}}{\isprop{\t{}}{\x{}}},

\begin{itemize}
  \item[]
\begin{subcase}[B-Local]
{ \inopenv {\openv{}} {\x{}} {\v{}} },
{ \opsem {\openv{}} {\x{}} {\v{}} }

Part 1 follows from \inopenv{\openv{}}{\x{}} {\v{}} by B-Local.
Part 2 considers two cases: if \v{} is not \false\ or \nil, then 
\satisfies{\openv{}}{\notprop{\falsy}{\x{}}} holds by M-NotType; if \v{} is \false\ or \nil, then 
\satisfies{\openv{}}{\isprop{\falsy}{\x{}}} holds by M-Type.
We prove part 3 by observing
\inpropenv{\propenv{}}{\isprop{\t{}}{\x{}}}
and
\satisfies{\openv{}}{\propenv{}},
so
{ \inopenv {\openv{}} {\x{}} {\v{}} }
by B-Local
gives us the desired result.
\end{subcase}
\end{itemize}

\end{case}

\begin{case}[T-Do]
\e{} = {\doexp {\e1} {\e2}},
  \judgement {\propenv{}} 
             {\hastype {\e1} {\t1}} 
             {\filterset {\thenprop {\prop{1}}} {\elseprop {\prop1}}} 
             {\object{1}},
\judgement {\propenv{}, {\orprop {\thenprop {\prop{1}}} {\elseprop {\prop{1}}}}}
           {\hastype {\e{2}} {\t{}}} 
           {\filterset {\thenprop {\prop{}}} {\elseprop {\prop{}}}} 
           {\object{}},

%TODO
\begin{itemize}
  \item[] \begin{subcase}[B-Do]
  \opsem {\openv{}} {\e{1}} {\v{1}},
  \opsem {\openv{}} {\e{2}} {\v{}}

For all parts we note 
    since {\e{1}} can be either a true or false value
    then
    {\satisfies{\openv{}}{{\propenv{}},{\orprop {\thenprop {\prop{1}}} {\elseprop {\prop{1}}}}}}
    by M-Or,
    which together with 
\judgement {\propenv{}, {\orprop {\thenprop {\prop{1}}} {\elseprop {\prop{1}}}}}
           {\hastype {\e{2}} {\t{}}} 
           {\filterset {\thenprop {\prop{}}} {\elseprop {\prop{}}}} 
           {\object{}},
    and
  \opsem {\openv{}} {\e{2}} {\v{}}
    allows us to apply the induction hypothesis on \e{2}.

To prove part 1 we use the induction hypothesis on \e{2}
to show either \object{} = \emptyobject{} 
or \inopenv {\openv{}} {\object{}} {\v{}}.

For part 2 we use the induction hypothesis on \e{2}
to show if \v{} $\not=$ \false\ (or \nil) then
        {\satisfies{\openv{}}{\thenprop{\prop{}}}}
        or
  if \v{} = \false\ (or \nil) then
        {\satisfies{\openv{}}{\elseprop{\prop{}}}}.

Part 3 follows from the induction hypothesis on \e{2}.
    \end{subcase}
  \item[]
\begin{subcase}[BE-Do]
  \v{} = {\errorval{\v{e}}},
\opsem {\openv{}} {\e{1}} {\errorval{\v{e}}},



\end{subcase}
\end{itemize}
\end{case}

\begin{case}[T-NewStatic]
  \e{} = {\newstaticexp {\overrightarrow{\classhint{i}}} {\classhint{}} 
                                                          {\class{}} {\overrightarrow{\e{i}}}},
  \object{} = \emptyobject{},
\thenprop{\prop{}} = \topprop{},
\elseprop{\prop{}} = \botprop{},
   $\overrightarrow{
\javatotc {\classhint{i}}
          {\t{i}}
          }$,
  \javatotc {\classhint{}}
            {\t{}},
            $
  \overrightarrow{
  \judgementtwo {\propenv{}}
                    {\hastype {\e{i}} {\t{i}}}
                  }$

\begin{itemize}
  \item[]
\begin{subcase}[B-New]
  $
  \overrightarrow{
  \opsem {\openv{}}
         {\e{i}}
         {\v{i}}
       }$,
         $\newjava {\classhint{1}}
                  {\overrightarrow{\classhint{i}}}
                  {\overrightarrow{\v{i}}}
                  {\v{}}$

Part 1 follows \object{} = \emptyobject{}.
Part 2 requires some explanation. The two false values in Typed Clojure
cannot be constructed with \newliteral, so the only case is \v{} $\not=$ \false\ (or \nil)
where \thenprop{\prop{}} = \topprop{} so \satisfies{\openv{}}{\thenprop{\prop{}}}.
Part 3 holds as B-New reduces to a \emph{non-nilable}
instance of \class{} via \newjavaliteral (by Assumption \ref{assumption:new}), and \javatotc{\classhint{}}{\t{}}.
\end{subcase}
  \item[]
\begin{subcase}[BE-New]
\end{subcase}
\end{itemize}
\end{case}

\begin{case}[T-FieldStatic]
  \e{} = {\fieldstaticexp {\classhint{1}} {\classhint{2}} {\fld{}} {\e{1}}},
  \javatotc {\classhint{1}} {\class{}},
  \javatotcnil {\classhint{2}} {\t{}},
  \judgementtwo {\propenv{}} {\hastype {\e{1}} {\class{}}}

\begin{itemize}
  \item[]
\begin{subcase}[B-Field]
  \opsem {\openv{}}
         {\e{1}} 
         {\classvalue{\classhint{1}} {\classfieldpair{\fld{}} {\v{}}}}


Part 1 is trivial as \object{} is always \emptyobject{}.
Part 2 holds trivially, \v{} can be either a true or false value
and both {\thenprop{\prop{}}} and {\elseprop{\prop{}}}
are \topprop{}.
Part 3 relies on the semantics of \getfieldliteral 
in B-Field, which returns a \emph{nilable} instance of \classhint{2},
and \javatotcnil{\classhint{2}} {\t{}}.
\end{subcase}
  \item[]
\begin{subcase}[BE-Field]
\end{subcase}
\end{itemize}
\end{case}

\begin{case}[T-MethodStatic]
  \e{} = {\methodstaticexp {\classhint{1}} 
                          {\overrightarrow {\classhint{i}}} 
                          {\classhint{2}}
                          {\mth{}} {\e{}} {\overrightarrow{\e{i}}}}

\begin{itemize}
  \item[]
\begin{subcase}[B-Method]

Part 1 is trivial as \object{} is always \emptyobject{}.
Part 2 holds trivially, \v{} can be either a true or false value
and both {\thenprop{\prop{}}} and {\elseprop{\prop{}}}
are \topprop{}.
Part 3 relies on the semantics of \invokejavamethodliteral 
in B-Method, which returns a \emph{nilable} instance of \classhint{2},
and \javatotcnil{\classhint{2}} {\t{}}.
\end{subcase}
  \item[]
\begin{subcase}[BE-Method1]
\end{subcase}
  \item[]
\begin{subcase}[BE-Method2]
\end{subcase}
\end{itemize}

\end{case}

\begin{case}[T-DefMulti]
  \e{} = {\createmultiexp {\s{}} {\e{d}}},
  \t{} = {\MultiFntype {\s{}} {\t{d}}},
  \thenprop{\prop{}} = {\topprop{}},
  \elseprop{\prop{}} = {\botprop{}},
  \s{} = {\ArrowOne {\x{}} {\t{1}} {\t{2}}
                          {\filterset {\thenprop {\prop{1}}}
                                      {\elseprop {\prop{1}}}}
                          {\object{1}}},
  \t{d} = {\ArrowOne {\x{}} {\t{1}} {\t{3}}
                          {\filterset {\thenprop {\prop{2}}}
                                      {\elseprop {\prop{2}}}}
                          {\object{2}}},
  \judgementtwo {\propenv{}} {\hastype {\e{d}} {\t{d}}}


\begin{itemize}
  \item[]
\begin{subcase}[B-DefMulti]
  \v{} = {\multi {\v{d}} {\emptydisptable}},
  \opsem {\openv{}} {\e{d}} {\v{d}}


Part 1 and 2 hold for the same reasons as T-True.
For part 3 we show \judgementtwo{}{\hastype{\multi {\v{d}} {\emptydisptable}}{\MultiFntype {\s{}} {\t{d}}}}
by T-Multi, since \judgementtwo {} {\hastype {\v{d}} {\t{d}}} by the inductive hypothesis on {\e{d}}
and {\emptydisptable} vacuously satisfies the other premises of T-Multi, so we are done.

\end{subcase}
  \item[]
\begin{subcase}[BE-DefMulti]
\end{subcase}
\end{itemize}
\end{case}

\begin{case}[T-DefMethod]
  \e{} = {\extendmultiexp {\e{m}} {\e{v}} 
                          {\abs {\x{}} {\t{1}} {\e{b}}}},
  \t{} = {\MultiFntype {\t{m}} {\t{d}}},
  \thenprop{\prop{}} = {\topprop{}},
  \elseprop{\prop{}} = {\botprop{}},
  \object{} = {\emptyobject{}},
  \t{m} = {\ArrowOne {\x{}} {\t{1}} {\s{}}
                     {\filterset {\thenprop {\prop{m}}}
                                 {\elseprop {\prop{m}}}}
                     {\object{m}}},
  \t{d} = {\ArrowOne {\x{}} {\t{1}} {\sp{}}
                     {\filterset {\thenprop {\prop{d}}}
                                 {\elseprop {\prop{d}}}}
                     {\object{d}}},
  \judgementtwo {\propenv{}} {\hastype {\e{m}} {\MultiFntype {\t{m}} {\t{d}}}},
  \isacompare{\sp{}}{\object{d}}{\t{v}}{\filterset {\thenprop {\prop{i}}} {\elseprop {\prop{i}}}},
  \judgementtwo {\propenv{}}
               {\hastype {\e{v}} {\t{v}}},
  \judgement {\propenv{}, {\isprop{\t{1}} {\x{}}}, {\thenprop {\prop{i}}}}
           {\hastype {\e{b}} {\s{}}}
           {\filterset {\thenprop {\prop{m}}}
                       {\elseprop {\prop{m}}}}
           {\object{m}}

  \begin{itemize}
    \item[]
      \begin{subcase}[B-DefMethod]
       \v{} = {\multi {\v{d}} {\disptablep{}}},
        \opsem {\openv{}}
               {\e{m}}
               {\multi {\v{d}} {\disptable{}}},
  \opsem {\openv{}}
         {\e{v}}
         {\v{v}},
  \opsem {\openv{}}
         {\e{f}}
         {\v{f}},
         \disptablep{} = {\extenddisptable {\disptable{}} 
                                {\v{v}}
                                {\v{f}}}

                                Part 1 and 2 hold for the same reasons as T-True, noting that the propositions
                                and object agree with T-Multi.

For part 3 we show
\judgementtwo{}{\hastype{\multi {\v{d}} {\extenddisptable {\disptable{}}{\v{v}}{\v{f}}}}{\MultiFntype {\t{m}} {\t{d}}}}
by noting \judgementtwo {} {\hastype {\v{d}} {\t{d}}},
  \judgementtwo{}{\hastype{\v{v}}{\Top{}}}
  and
  \judgementtwo{}{\hastype{\v{f}}{\t{m}}}, and since \disptable{} is in the correct form by the inductive
  hypothesis on {\e{m}} we can satisfy all premises of T-Multi, so we are done.


      \end{subcase}

    \item[]
      \begin{subcase}[BE-DefMethod1]
      \end{subcase}
    \item[]
      \begin{subcase}[BE-DefMethod2]
      \end{subcase}
    \item[]
      \begin{subcase}[BE-DefMethod3]
      \end{subcase}
  \end{itemize}
\end{case}

\begin{case}[T-App]
  \e{} = {\appexp {\e{1}} {\e{2}}},
  \t{} = {\replacefor {\t{f}}
                      {\object{2}}
                      {\x{}}},
  {\thenprop {\prop{}}} = 
                 {\replacefor {\thenprop {\prop{f}}}
                              {\object{2}}
                              {\x{}}},
  {\elseprop {\prop{}}} = 
                 {\replacefor {\elseprop {\prop{f}}}
                              {\object{2}}
                              {\x{}}},
  \object{} = {\replacefor {\object{f}}
                              {\object{2}}
                              {\x{}}},
  \judgement {\propenv{}} {\hastype {\e{1}} {\ArrowOne {\x{}} {\s{}}
                                                       {\t{f}}
                                                       {\filterset {\thenprop {\prop{f}}}
                                                                   {\elseprop {\prop{f}}}}
                                                       {\object{f}}}}
                {\filterset {\thenprop {\prop{1}}}
                            {\elseprop {\prop{1}}}}
                {\object{1}},
  \judgement {\propenv{}}
                 {\hastype {\e{2}} {\s{}}}
                 {\filterset {\thenprop {\prop{2}}}
                             {\elseprop {\prop{2}}}}
                 {\object{2}} ,
                 IHe1 = $\forall{\openv{1}},
                               {\v{1}},
                               {\thenprop {\prop{1}}},
                               {\elseprop {\prop{1}}},
                               {\object{1}}.
                               {\judgement
                                 {\propenv{}}
                                 {\hastype
                                   {\e{1}}
                                   {\ArrowOne {\x{}} {\s{}}
                                     {\t{f}}
                                     {\filterset 
                                       {\thenprop {\prop{f}}}
                                       {\elseprop {\prop{f}}}}
                                     {\object{f}}}}
                                {\filterset {\thenprop {\prop{1}}}
                                            {\elseprop {\prop{1}}}}
                                {\object{1}}}
\Rightarrow
                                                       {\satisfies{\openv{1}}{\propenv{}}}
                                                       \Rightarrow
  \opsem {\openv{}}
         {\e{1}}
         {\v{1}}
         \Rightarrow
         (1: \object{1} = \emptyobject{} \vee \inopenv{\openv{1}}{\object{1}}{\v{1}})
         \wedge
       (2: (\v{1} \not= \false\ (or\ \nil) \wedge \satisfies{\openv{1}}{\thenprop{\prop{1}}})
          \vee
            (\v{1} = \false\ (or\ \nil) \wedge \satisfies{\openv{1}}{\elseprop{\prop{1}}}))
            \wedge
          (3: \exists {\thenprop{\propp{1}}},{\elseprop{\propp{1}}},{\objectp{1}}.
          \judgement{}{\hastype{\v{1}}{\ArrowOne {\x{}} {\s{}}
                                     {\t{f}}
                                     {\filterset 
                                       {\thenprop {\prop{f}}}
                                       {\elseprop {\prop{f}}}}
                                   {\object{f}}}}
                        {\filterset{\thenprop{\propp{1}}}{\elseprop{\propp{1}}}}
                      {\objectp{1}})
                                                       $









\begin{itemize}
  \item[]
\begin{subcase}[B-BetaClosure]
  \opsem {\openv{}}
         {\e{1}}
         {\closure {\openv{c}} {\abs {\x{}} {\s{}} {\e{b}}}},
  \opsem {\openv{}}
         {\e{2}}
         {\v{2}},
  \opsem {\extendopenv {\openv{c}} {\x{}} {\v{2}}}
         {\e{b}}
         {\v{}}

         To prove part 1 we note by inversion of the typing relation (by T-Closure) if
         \begin{itemize}
           \item 
  \judgement {\propenv{}} {\hastype {\e{1}} {\ArrowOne {\x{}} {\s{}}
                                                       {\t{f}}
                                                       {\filterset {\thenprop {\prop{f}}}
                                                                   {\elseprop {\prop{f}}}}
                                                               {\object{f}}}}
                {\filterset {\thenprop {\prop{1}}}
                            {\elseprop {\prop{1}}}}
                          {\object{1}} and
              \item 
                \opsem {\openv{}}
                       {\e{1}}
                       {\closure {\openv{c}} {\abs {\x{}} {\s{}} {\e{b}}}},
         \end{itemize}

         then there is some environment {\propenvc{}} such that
         \begin{itemize}
           \item
              \satisfies{\openv{c}}{\propenvc{}} and
            \item
              \judgement {\propenvc{}} {\hastype {\abs {\x{}} {\s{}} {\e{b}}} {\t{f}}}
                               {\filterset {\thenprop {\prop{f}}}
                                           {\elseprop {\prop{f}}}}
                               {\object{f}}.
         \end{itemize}

         By \judgementtwo{\propenv{}}{\hastype {\e{2}} {\s{}}},
            {\opsem {\openv{}}
                    {\e{2}}
                    {\v{2}}}
         and
         the inductive hypothesis 
         \begin{itemize}
           \item
             \opsem {\openv{}}
                    {\e{2}}
                    {\v{2}},
         \end{itemize}
         
         \satisfies{\propenv_{c}, {\isprop {\s{}} {\x{}}}}{\extendopenv {\openv{c}} {\x{}} {\v{2}}}.
         Since 
  \opsem {\extendopenv {\openv{c}} {\x{}} {\v{2}}}
         {\e{b}}
         {\v{}},
  \judgement {\propenv{}}
                 {\hastype {\e{2}} {\s{}}}
                 {\filterset {\thenprop {\prop{2}}}
                             {\elseprop {\prop{2}}}}
                 {\object{2}} 
         and
  \opsem {\openv{}}
         {\e{2}}
         {\v{2}}
         then either 
                              \object{} = {\replacefor {\object{f}}
                                                {\object{2}}
                                                {\x{}}}
                                                where
   \inopenv {\openv{}} {\object{}} {\v{}}
                              or \object{} = \emptyobject{}.

\end{subcase}
  \item[]
\begin{subcase}[B-BetaMulti]
  \opsem {\openv{}}
         {\e{1}}
         {\multi {\v{d}} {m}},
  \opsem {\openv{}}
         {\e{2}}
         {\v{2}},
  \opsem {\openv{}}
         {\appexp {\v{d}} {\v{2}}}
         {\v{e}},
  \getmethod {\disptable{}}
             {\v{e}}
             {\v{f}},
  \opsem {\openv{}}
         {\appexp {\v{f}} {\v{2}}}
         {\v{}}

\end{subcase}
  \item[]
\begin{subcase}[B-Delta]
  \opsem {\openv{}} {\e{1}} {\const{}},
  \opsem {\openv{}} {\e{2}} {\v{2}},
  \constantopsem{\const{}}{\v{2}} = \v{}

  % TODO do I need to prove anything about the argument in the definition
  % of the constant being under \s{}?

  Prove by induction on \const{} and \v{2}.
  \begin{itemize}
    \item[] \begin{subcase}[\const{} = \classconst]
    ${\ArrowOne {\x{}} {\s{}}
                                                       {\t{f}}
                                                       {\filterset {\thenprop {\prop{f}}}
                                                                   {\elseprop {\prop{f}}}}
                                                       {\object{f}}}$
                                                       =
  ${\ArrowOne {\x{}} {\Top{}}
                                      {\Union{\nil{}}{\class{}}}
                                      {\filterset {\topprop{}}
                                                  {\topprop{}}}
                                      {\path {\classpe{}} {\x{}}}}$

        \begin{itemize}
          \item[] \begin{subcase}[\v{2} = \classvalue{\class{}} {\overrightarrow {\classfieldpair{\fld{i}} {\v{i}}}}]
                    \v{} = \class{}

                    To prove part 1, note \object{} = {\replacefor {\object{f}}{\object{2}}{\x{}}} 
                    and \object{f} = {\path {\classpe{}} {\x{}}}.
                    There are two cases defined by substitution: if \object{2} = \emptyobject{} then \object{} = \emptyobject{}
                    and we are done,
                    or if \object{2} = {\path {\pathelem{}} {\xp{}}} then \object{} = 
                    {\path {\classpe{}}{\object{2}}},
                     by the induction hypothesis \inopenv {\openv{}} {\object{2}} {\v{2}}
                    and by the definition of path translation
                    {\openv{}}({\path {\classpe{}} {\object{2}}}) = {\appexp {\classconst{}} {{\openv{}}(\object{2})}},
                    so we can deduce \inopenv {\openv{}} {\object{}} {\appexp {\classconst{}} {{\openv{}}(\object{2})}}
                    where {\appexp {\classconst{}} {{\openv{}}(\object{2})}} = \v{}, and we are done.

                    Part 2 is trivial since both propositions are \topprop{} by substitution.
                    
                    Part 3 
                  \end{subcase}
        \end{itemize}
            \end{subcase}
  \end{itemize}

\end{subcase}
  \item[]
\begin{subcase}[BE-Beta1]
\end{subcase}
  \item[]
\begin{subcase}[BE-Beta2]
\end{subcase}
  \item[]
\begin{subcase}[BE-BetaClosure]
\end{subcase}
  \item[]
\begin{subcase}[BE-BetaMulti1]
\end{subcase}
  \item[]
\begin{subcase}[BE-BetaMulti2]
\end{subcase}
\end{itemize}
\end{case}

\begin{case}[T-IsA]\e{} = {\isaapp {\e{1}} {\e{2}}},
  \t{} = {\Boolean{}},
  \judgement {\propenv{}} {\hastype {\e{1}} {\t{1}}}
             {\filterset {\thenprop {\propp{}}}
                         {\elseprop {\propp{}}}}
                       {\object{1}},
  \judgementtwo {\propenv{}} {\hastype {\e{2}} {\t{2}}},
  \isacompare{\t{1}}{\object{1}}{\t{2}}{\filterset {\thenprop {\prop{}}} {\elseprop {\prop{}}}},
  \object{} = \emptyobject{}

  \begin{itemize}
    \item[]
      \begin{subcase}[B-IsA]
  \opsem {\openv{}} {\e{1}} {\v{1}},
  \opsem {\openv{}} {\e{2}} {\v{2}},
  \isaopsem{\v{1}}{\v{2}} = {\v{}}

  From the definition of \isacompareliteral,
  \thenprop{\prop{}} = {\replacefor{\isprop{\t{2}}{\x{}}}{\object{1}}{\x{}}} and
  \elseprop{\prop{}} = {\replacefor{\notprop{\t{2}}{\x{}}}{\object{1}}{\x{}}}.

  Part 1 holds trivially with \object{} = \emptyobject{}.
  For part 2, if \v{} $\not=$ \false\ (or \nil)
  then {\satisfies{\openv{}}{\replacefor{\isprop{\t{2}}{\x{}}}{\object{1}}{\x{}}}}, as by the definition
  of \isaopsemliteral either 
  \begin{itemize}
    \item \v{1} = \v{2} so {\v{2}} must be the same type as down {\object{1}}, or
    \item \v{1} = \classvaluemeta{1} and \v{2} = \classvaluemeta{2} where \classvaluemeta{1}
  is a subclass of \classvaluemeta{2}, so {\v{2}} may be be safely upcast to the same type as down {\object{1}}.
  \end{itemize}
  Part 3 holds because by the definition of \isaopsemliteral
  \v{} can only be \true or \false, which are both subtypes of
  \t{}.


      \end{subcase}
    \item[]
      \begin{subcase}[BE-IsA1]
      \end{subcase}
    \item[]
      \begin{subcase}[BE-IsA2]
      \end{subcase}
  \end{itemize}
\end{case}

\begin{case}[T-GetHMap]
  \e{} = {\getexp {\e{m}} {\e{k}}},
  $\t{} = {\Unionsplice {\overrightarrow {\t{i}}}}$
  \thenprop{\prop{}} = {\topprop{}},
  \elseprop{\prop{}} = {\topprop{}},
  \object{} = {\replacefor {\path {\keype{k}} {\x{}}}
                          {\object{m}}
                          {\x{}}},
  $\judgementtwo {\propenv{}} {\hastype {\e{m}} {\Unionsplice {\overrightarrow {\HMapgeneric {\mandatory{}} {\absent{}}}}}}$,
  $\judgementtwo {\propenv{}} {\hastype {\e{k}} {\Value {k}}}$,
  $\overrightarrow{\inmandatory{\k{}}{\t{i}}{\mandatory{}}}$


  \begin{itemize}
    \item[]
      \begin{subcase}[B-Get]
      $\opsem {\openv{}} {\e{m}}{\v{m}}$,
        $\v{m} = {\curlymap{\overrightarrow{({\v{a}}\ {\v{b}})}}}$,
         \opsem {\openv{}} {\e{k}} {\k{}},
         $\keyinmap{\k{}}{\curlymap{\overrightarrow{({\v{a}}\ {\v{b}})}}}$,
         \getmap{\curlymap{\overrightarrow{({\v{a}}\ {\v{b}})}}} {\k{}} = {\v{}}

         To prove part 1 we consider two cases on the form of \object{m}: 
         \begin{itemize}
           \item
         if {\object{m}} = \emptyobject{}
         then \object{} = \emptyobject{} by substitution, which gives the desired result;
           \item
         if \object{m} = {\path {\pathelem{m}} {\x{m}}}
         then \object{} = {\path {\keype{k}} {\object{m}}} by substitution.
         We note by the definition of path translation
         {\openv{}}({\path {\keype{k}} {\object{m}}}) =
         {\getexp {{\openv{}}(\object{m})}{\k{}}}
         and by the induction hypothesis on \e{m}
         {{\openv{}}(\object{m})} = {\curlymap{\overrightarrow{({\v{a}}\ {\v{b}})}}},
         which together imply 
         \inopenv {\openv{}} {\object{}} {\getexp {\curlymap{\overrightarrow{({\v{a}}\ {\v{b}})}}} {\k{}}}.
         Since this is the same form as B-Get, we can apply the premise
         \getmap{\curlymap{\overrightarrow{({\v{a}}\ {\v{b}})}}} {\k{}} = {\v{}}
         to derive \inopenv {\openv{}} {\object{}} {\v{}}.
         \end{itemize}
         
         Part 2 holds trivially as \thenprop{\prop{}} = {\topprop{}}
         and \elseprop{\prop{}} = {\topprop{}}.

         To prove part 3 we note that (by the induction hypothesis on \e{m})
         $\judgementtwo{}{\hastype{\v{m}}{\Unionsplice{\overrightarrow {\HMapgeneric {\mandatory{}} {\absent{}}}}}}$,
         where $\overrightarrow{\inmandatory{\k{}}{\t{i}}{\mandatory{}}}$, and 
         both
         $\keyinmap{\k{}}{\curlymap{\overrightarrow{({\v{a}}\ {\v{b}})}}}$
         and
         \getmap{\curlymap{\overrightarrow{({\v{a}}\ {\v{b}})}}} {\k{}} = {\v{}}
         imply \judgementtwo{}{\hastype{\v{}}{\Unionsplice {\overrightarrow {\t{i}}}}}.

      \end{subcase}
    \item[]
      \begin{subcase}[B-GetMissing]
        \v{} = \nil,
        $\opsem {\openv{}}
        {\e{m}} {\curlymap{\overrightarrow{({\v{a}}\ {\v{b}})}}}$,
       \opsem {\openv{}} {\e{k}} {\k{}},
       \keynotinmap{\k{}}{\curlymap{\overrightarrow{({\v{a}}\ {\v{b}})}}}

       Unreachable subcase because 
       \keynotinmap{\k{}}{\curlymap{\overrightarrow{({\v{a}}\ {\v{b}})}}}
       contradicts ${\inmandatory{\k{}}{\t{}}{\mandatory{}}}$.
      \end{subcase}
    \item[]
      \begin{subcase}[BE-Get1]
      \end{subcase}
    \item[]
      \begin{subcase}[BE-Get2]
      \end{subcase}
  \end{itemize}
\end{case}

\begin{case}[T-GetHMapAbsent]
  \e{} = {\getexp {\e{m}} {\e{k}}},
  \t{} = \Nil,
  \thenprop{\prop{}} = {\topprop{}},
  \elseprop{\prop{}} = {\topprop{}},
  \object{} = {\replacefor
               {\path {\keype{k}} {\x{}}}
                          {\object{m}}
                          {\x{}}},
  \judgement {\propenv{}} {\hastype {\e{m}} {\HMapgeneric {\mandatory{}} {\absent}}}
           {\filterset {\thenprop {\prop{m}}} {\elseprop {\prop{m}}}}
           {\object{m}},
  \judgementtwo {\propenv{}} {\hastype {\e{k}} {\Value {k}}},
  {\inabsent{\k{}}{\absent{}}}


  \begin{itemize}
    \item[]
      \begin{subcase}[B-Get]
        $\opsem {\openv{}}
        {\e{m}} {\curlymap{\overrightarrow{({\v{a}}\ {\v{b}})}}}$
        ,
         \opsem {\openv{}} {\e{k}} {\k{}},
         $\keyinmap{\k{}}{\curlymap{\overrightarrow{({\v{a}}\ {\v{b}})}}}$,
         \getmap{\curlymap{\overrightarrow{({\v{a}}\ {\v{b}})}}} {\k{}} = {\v{}}

       Unreachable subcase because 
         $\keyinmap{\k{}}{\curlymap{\overrightarrow{({\v{a}}\ {\v{b}})}}}$,
         contradicts
                {\inabsent{\k{}}{\absent{}}}.
      \end{subcase}
    \item[]
      \begin{subcase}[B-GetMissing]
        \v{} = \nil,
        $\opsem {\openv{}}
        {\e{m}} {\curlymap{\overrightarrow{({\v{a}}\ {\v{b}})}}}$,
       \opsem {\openv{}} {\e{k}} {\k{}},
       \keynotinmap{\k{}}{\curlymap{\overrightarrow{({\v{a}}\ {\v{b}})}}}

         To prove part 1 we consider two cases on the form of \object{m}: 
         \begin{itemize}
           \item
         if {\object{m}} = \emptyobject{}
         then \object{} = \emptyobject{} by substitution, which gives the desired result;
           \item
         if \object{m} = {\path {\pathelem{m}} {\x{m}}}
         then \object{} = {\path {\keype{k}} {\object{m}}} by substitution.
         We note by the definition of path translation
         {\openv{}}({\path {\keype{k}} {\object{m}}}) =
         {\getexp {{\openv{}}(\object{m})}{\k{}}}
         and by the induction hypothesis on \e{m}
         {{\openv{}}(\object{m})} = {\curlymap{\overrightarrow{({\v{a}}\ {\v{b}})}}},
         which together imply 
         \inopenv {\openv{}} {\object{}} {\getexp {\curlymap{\overrightarrow{({\v{a}}\ {\v{b}})}}} {\k{}}}.
         Since this is the same form as B-GetMissing, we can apply the premise
        \v{} = \nil\ 
         to derive \inopenv {\openv{}} {\object{}} {\v{}}.
         \end{itemize}
         
         Part 2 holds trivially as \thenprop{\prop{}} = {\topprop{}}
         and \elseprop{\prop{}} = {\topprop{}}.
         To prove part 3 we note that \e{m} has type {\HMapgeneric {\mandatory{}} {\absent{}}}
         where {\inabsent{\k{}}{\absent{}}}, and
         the premises of B-GetMissing
         \keynotinmap{\k{}}{\curlymap{\overrightarrow{({\v{a}}\ {\v{b}})}}}
         and
          \v{} = \nil\ 
         tell us {\v{}} must be of type {\t{}}.
      \end{subcase}
    \item[]
      \begin{subcase}[BE-Get1]
      \end{subcase}
    \item[]
      \begin{subcase}[BE-Get2]
      \end{subcase}
  \end{itemize}
\end{case}

\begin{case}[T-GetHMapPartialDefault]
  \e{} = {\getexp {\e{m}} {\e{k}}},
  \t{} = \Top,
  \thenprop{\prop{}} = {\topprop{}},
  \elseprop{\prop{}} = {\topprop{}},
  \object{} = {\replacefor
               {\path {\keype{k}} {\x{}}}
                          {\object{m}}
                          {\x{}}},
 \judgement {\propenv{}} {\hastype {\e{m}} {\HMapp {\mandatory{}} {\absent}}}
           {\filterset {\thenprop {\prop{m}}} {\elseprop {\prop{m}}}}
           {\object{m}},
  \judgementtwo {\propenv{}} {\hastype {\e{k}} {\Value {k}}},
             ${\notinmandatory{\k{}}{\t{}}{\mandatory{}}}$,
             {\notinabsent{\k{}}{\absent{}}}

  \begin{itemize}
    \item[]
      \begin{subcase}[B-Get]
        $\opsem {\openv{}}
        {\e{m}} {\curlymap{\overrightarrow{({\v{a}}\ {\v{b}})}}}$
        ,
         \opsem {\openv{}}
                 {\e{k}} {\k{}},
         $\keyinmap{\k{}}{\curlymap{\overrightarrow{({\v{a}}\ {\v{b}})}}}$,
         \getmap{\curlymap{\overrightarrow{({\v{a}}\ {\v{b}})}}} {\k{}} = {\v{}}

      \end{subcase}
    \item[]
      \begin{subcase}[B-GetMissing]
        \v{} = \nil,
        $\opsem {\openv{}}
        {\e{m}} {\curlymap{\overrightarrow{({\v{a}}\ {\v{b}})}}}$,
       \opsem {\openv{}} {\e{k}} {\k{}},
       \keynotinmap{\k{}}{\curlymap{\overrightarrow{({\v{a}}\ {\v{b}})}}}

      \end{subcase}
    \item[]
      \begin{subcase}[BE-Get1]
      \end{subcase}
    \item[]
      \begin{subcase}[BE-Get2]
      \end{subcase}
  \end{itemize}
\end{case}

\begin{case}[T-AssocHMap]
  \e{} = {\assocexp {\e{m}} {\e{k}} {\e{v}}},
  \t{} = ${\HMapgeneric {\extendmandatoryset {\mandatory{}}{\k{}}{\t{}}} {\absent}}$,
  \thenprop{\prop{}} = {\topprop{}},
  \elseprop{\prop{}} = {\botprop{}},
  \object{} = \emptyobject,
  \judgementtwo {\propenv{}} {\hastype {\e{m}} {\HMapgeneric {\mandatory{}} {\absent}}},
  \judgementtwo {\propenv{}} {\hastype {\e{k}} {\Value{\k{}}}},
  \judgementtwo {\propenv{}} {\hastype {\e{v}} {\t{}}},
  {\k{}} $\not\in$ {\absent{}}

  \begin{itemize}
    \item[]
      \begin{subcase}[B-Assoc]
        \v{} = 
        {\extendmap{\curlymap{\overrightarrow{({\v{a}}\ {\v{b}})}}}
                {\k{}}{\v{v}}},
        \opsem {\openv{}}
        {\e{m}} {\curlymap{\overrightarrow{({\v{a}}\ {\v{b}})}}},
        \opsem {\openv{}} {\e{k}} {\k{}},
        \opsem {\openv{}} {\e{v}} {\v{v}}

        We prove parts 1 and 2 for similar reasons as T-True.
        %TODO part 3
      \end{subcase}
    \item[]
      \begin{subcase}[BE-Assoc1]
      \end{subcase}
    \item[]
      \begin{subcase}[BE-Assoc2]
      \end{subcase}
    \item[]
      \begin{subcase}[BE-Assoc3]
      \end{subcase}
  \end{itemize}
\end{case}

\begin{case}[T-If]

  \begin{itemize}
    \item[]
      \begin{subcase}[B-If]
      \end{subcase}
    \item[]
      \begin{subcase}[BE-If]
      \end{subcase}
  \end{itemize}
\end{case}

\begin{case}[T-Let]
  \e{} = {\letexp {\x{}} {\e{1}} {\e{2}}},
  \judgement {\propenv{}} {\hastype {\e{1}} {\s{}}} {\filterset {\thenprop {\prop{1}}} {\elseprop {\prop{1}}}}
             {\object{1}},
             \propp{} = {\impprop {\notprop {\falsy{}} {\x{}}} {\thenprop {\prop{1}}}},
             \proppp{} = {\impprop {\isprop {\falsy{}} {\x{}}} {\elseprop {\prop{1}}}},
  \judgement
       {\propenv{}, {\isprop {\s{}} {\x{}}},
         {\propp{}},
         {\proppp{}}}
             {\hastype {\e{2}} {\t{}}} {\filterset {\thenprop {\prop{}}} {\elseprop {\prop{}}}}
             {\object{}} 


  \begin{itemize}
    \item[]
      \begin{subcase}[B-Let]
        \opsem {\openv{}} {\e{1}} {\v{1}},
        \opsem {\extendopenv{\openv{}}{\x{}}{\v{1}}} {\e{2}} {\v{}}

        For all the following cases (with a reminder that \x{} is fresh)
        we apply the induction hypothesis on \e{2}. We justify this by noting
        that occurrences of \x{} inside \e{2} have the same type as \e{1} and 
        simulate the propositions of \e{1}
        because 
        \opsem {\openv{}} {\e{1}} {\v{1}},
        and
        \opsem {\extendopenv{\openv{}}{\x{}}{\v{1}}} {\e{2}} {\v{}},
        so \satisfies{\openv{}}{\propenv{}, {\isprop {\s{}} {\x{}}}, \propp{}, \proppp{}},
        by M-And.

        We prove parts 1, 2 and 3 by directly using the induction hypothesis on \e{2}.
      \end{subcase}
    \item[]
      \begin{subcase}[BE-Let]
      \end{subcase}
  \end{itemize}
\end{case}

\begin{case}[T-Clos] \e{} = {\closure {\openv{}} {\abs {\x{}} {\s{}} {\e{1}}}},
  {\thenprop {\prop{}}}
  $\exists {\propenvp{}}. \satisfies{\openv{}}{\propenvp{}}$
  \ \text{and}\ 
\judgement {\propenvp{}} {\hastype {\abs {\x{}} {\s{}} {\e{1}}} {\t{}}}
                 {\filterset {\thenprop {\prop{}}}
                             {\elseprop {\prop{}}}}
                 {\object{}}

  \begin{itemize}
    \item[]
      \begin{subcase}[B-Abs] \v{} = {\closure {\openv{}} {\abs {\x{}} {\s{}} {\e{1}}}}

        We assume some \propenvp{}, such that
        \begin{itemize}
          \item \satisfies{\openv{}}{\propenvp{}}
          \item \judgement {\propenvp{}} {\hastype {\abs {\x{}} {\s{}} {\e{1}}} {\t{}}}
                           {\filterset {\thenprop {\prop{}}}
                                       {\elseprop {\prop{}}}}
                           {\object{}}.
       \end{itemize}
       Note the last rule in the derivation of
          \judgement {\propenvp{}} {\hastype {\abs {\x{}} {\s{}} {\e{1}}} {\t{}}}
                           {\filterset {\thenprop {\prop{}}}
                                       {\elseprop {\prop{}}}}
                           {\object{}}
                           must be T-Abs, so 
                           {\thenprop {\prop{}}} = {\topprop{}},
                           {\elseprop {\prop{}}} = {\botprop{}}
                           and {\object{}} = {\emptyobject{}}.
         Thus parts 1 and 2 hold for the same reasons as T-True.
         Part 3 holds as \v{} has the same type as {\abs {\x{}} {\s{}} {\e{1}}}
         under \propenvp{}.

      \end{subcase} 
  \end{itemize}
\end{case}

\begin{case}[T-Multi] \e{} = {\multi {\v{1}} {\curlymapvaloverright{\v{k}}{\v{v}}}},
  \t{} = {\MultiFntype {\s{}} {\t{1}}},
  {\thenprop {\prop{}}} = {\topprop{}},
  {\elseprop{\prop{}}} = {\botprop{}},
  {\object{}} = {\emptyobject{}},
  \judgementtwo {} {\hastype {\v{1}} {\t{1}}},
  $\overrightarrow{\judgementtwo{}{\hastype{\v{k}}{\Top{}}}}$,
  $\overrightarrow{\judgementtwo{}{\hastype{\v{v}}{\s{}}}}$

  \begin{itemize}
    \item[]
      \begin{subcase}[B-Val]
        Similar to T-True.
      \end{subcase}
  \end{itemize}

\end{case}

\begin{case}[T-Abs] \e{} = {\abs {\x{}} {\s{}} {\e{1}}},
  \t{} = {\ArrowOne {\x{}} {\s{}}
                                                      {\t{1}}
                                                      {\filterset {\thenprop {\prop{1}}}
                                                                  {\elseprop {\prop{1}}}}
                                                      {\object{1}}},
  {\thenprop{\prop{}}}= {\topprop{}},
  {\elseprop{\prop{}}}= {\botprop{}},
  {\object{}}= {\emptyobject{}},
{ \judgement {\propenv{}, {\isprop {\s{}} {\x{}}}}
            {\hastype {\e{1}} {\t{}}}
             {\filterset {\thenprop {\prop{1}}}
                         {\elseprop {\prop{1}}}}
             {\object{1}}},
\judgement {\propenv{}} {\hastype {\abs {\x{}} {\s{}} {\e{1}}} {\t{}}}
                 {\filterset {\thenprop {\prop{}}}
                             {\elseprop {\prop{}}}}
                 {\object{}}

  \begin{itemize}
    \item[]
      %TODO
      \begin{subcase}[B-Abs]
        \v{} = ${\closure {\openv{}} {\abs {\x{}} {\s{}} {\e{1}}}}$,
          { \opsem {\openv{}}
                   {\abs {\x{}} {\t{}} {\e{1}}}
                   {\closure {\openv{}} {\abs {\x{}} {\s{}} {\e{1}}}}}

        Parts 1 and 2 hold for the same reasons as T-True.
        Part 3 holds directly via T-Clos, since \v{} must be a closure.
      \end{subcase}
  \end{itemize}
\end{case}

\begin{case}[T-Error]
  \e{} = \errorval{\v{1}},
  \t{} = \Bot,
  \thenprop{\prop{}} = \botprop{}, \elseprop{\prop{}} = \botprop{}, \object{} = \emptyobject{}


  \begin{itemize}
    \item[]
      \begin{subcase}[BE-Error] 
        \v{} = \errorval{\v{1}}

        Parts 1 holds as \object{} = \emptyobject{}.
        Part 2 holds vaculously as \thenprop{\prop{}} = \botprop{} and \elseprop{\prop{}} = \botprop{}.
        Part 3 holds as \errorval{\v{1}} is of type \Bot by the premises of T-Error.
      \end{subcase}
  \end{itemize}
\end{case}

\begin{case}[T-Subsume]

\end{case}

\begin{case}[T-Const]\e{} = {\const{}},
  \t{} = {\constanttype{\const{}}},
{\thenprop{\prop{}}} = {\topprop{}},
{\elseprop{\prop{}}} = {\botprop{}},
{\object{}} = {\emptyobject{}}

  \begin{itemize}
    \item[] 
      \begin{subcase}[B-Val]
        Parts 1, 2 and 3 hold for the same reasons as T-True. 
      \end{subcase}
  \end{itemize}
\end{case}

\end{proof}

\end{enumerate}
\end{lemma}


\begin{figure*}
$$
\begin{altgrammar}
  \expd{}, \e{} &::=& \x{}
                      \alt
                      \v{} \alt
                      {\comb {\e{}} {\e{}}} \alt {\abs {\x{}} {\t{}} {\e{}}}
                      \alt 
                      {\ifexp {\e{}} {\e{}} {\e{}}}
                      \alt 
                      {\doexp {\e{}} {\e{}}}
                      \alt
                      {\letexp {\x{}} {\e{}} {\e{}}}
                      \alt {\errorvalv}
                      \alt {\ReflectiveExp{}}
                      \alt {\NonReflectiveExp{}}
                      \alt {\MultimethodExp{}}
                      \alt {\HintedExp{}}
                      \alt {\HashMapExp{}}
                &\mbox{Expressions} \\
  \v{} &::=&
                      \true{} \alt \false{} \alt \nil{}
                      \alt \class{} \alt \classvaluemeta{}
                      \alt \k{}
                      \alt {\emptymap{}}
                      \alt {\const{}}
                      \alt {\num{}}
                      \alt {\curlymapvaloverright{\v{}}{\v{}}}
                      \alt {\closure {\openv{}} {\abs {\x{}} {\t{}} {\e{}}}}
                      \alt {\multi {\v{d}} {\disptable{}}}
                &\mbox{Values} \\
  {\const{}}           &::=& \classconst \alt \throwconst

                &\mbox{Constants} \\
  \HintedExp{}             &::=&

                      \hinted{\classhint{}} {\x{}}
                      \alt
                      {\letexp {\hinted {\classhint{}} {\x{}}} {\e{}} {\e{}}}
                &\mbox{Type Hinted Expressions} \\
  \HashMapExp{}                &::=&
                      {\getexp {\e{}} {\e{}}}
                      \alt {\assocexp {\e{}}{\e{}}{\e{}}}
                &\mbox{Hash Maps} \\
  \ReflectiveExp{}     &::=&
                      {\fieldstaticexp {\classhint{}} {\classhint{}} {fld} {\e{}}}
                      \alt {\methodstaticexp {\classhint{}} {\overrightarrow{\classhint{}}} {\classhint{}} {mth} {\e{}} {\overrightarrow{\e{}}}}
                      \alt {\newstaticexp {\overrightarrow {\classhint{}}} {\classhint{}} {\class{}} {\overrightarrow{\e{}}}}
                &\mbox{Non-Reflective Java Interop} \\
  \NonReflectiveExp{}     &::=&
                      {\fieldexp {fld} {\e{}}}
                      \alt {\methodexp {mth} {\e{}} {\overrightarrow{\e{}}}}
                      \alt {\newexp {\class{}} {\overrightarrow{\e{}}}}
                &\mbox{Reflective Java Interop} \\
  \MultimethodExp{}     &::=& {\createmultiexp {\t{}} {\e{}}}
                      \alt
                              {\extendmultiexp {\e{}} {\e{}} {\e{}}}
                      \alt {\isaapp {\e{}} {\e{}}}
                &\mbox{Immutable First-Class Multimethods}
                      \\\\
  \s{}, \t{}    &::=& \Top 
                      \alt \class\ 
                      \alt
                      {\Value \k{}} 
                      \alt {\Nil{}}
                      \alt {\True{}}
                      \alt {\False{}}
                      \alt
                      {\Unionsplice {\overrightarrow{\t{}}}}
                      \alt
                      {\ArrowOne {\x{}} {\t{}}
                                   {\t{}}
                                   {\filterset {\prop{}} {\prop{}}}
                                   {\object{}}}
                                   \alt {\HMapgeneric {\mandatory{}} {\absent{}}}
                      \alt \Number{}
                      \alt \Keyword{}
                      
                &\mbox{Typed Clojure Types} \\
  \completenessmeta{} &::=& {\complete{}} \alt {\partial{}}
                &\mbox{HMap completeness} \\
  \tatype{}     &::=& \unknownhint{} \alt \classhint{}
                &\mbox{tools.analyzer Types} \\ \\
  \prop{}       &::=& {\isprop {\t{}} {\path {\pathelem{}} {\x{}}}}
                      \alt {\notprop {\t{}} {\path {\pathelem{}} {\x{}}}}
                      \alt {\impprop {\prop{}} {\prop{}}}
                      \alt {\andprop {\prop{}} {\prop{}}}
                      \alt {\orprop {\prop{}} {\prop{}}}
                      \alt \topprop{}
                      \alt \botprop{}
                &\mbox{Propositions} \\
  \object{}     &::=& {\path {\pathelem{}} {\x{}}}
                      \alt \emptyobject{}
                &\mbox{Objects} \\
  \pathelem{}   &::=& \overrightarrow{\pesyntax{}}
                &\mbox{Paths} \\
  \pesyntax{}   &::=& \classpe{} \alt \keype{\k{}}
                &\mbox{Path Elements} \\ \\
 \disptable{}   &::=& \{ \overrightarrow{\classvaluemeta{} \Rightarrow \v{}} \}
               &\mbox{Multimethod dispatch table} \\
  \propenv{}   &::=& \overrightarrow{\prop{}}
               &\mbox{Proposition Environment} \\
  \taenv{}    &::=& \{ \overrightarrow{ \hastype{\x{}} {\tatype{}}} \}
               &\mbox{tools.analyzer Environment} \\
  \ctentrymeta{} &::=& \ctentry
               &\mbox{Class descriptors} \\
  \ct{}   &::=& \{ \overrightarrow{\classhint{} \Rightarrow \ctentrymeta{}} \}
               &\mbox{Class Table} \\
  \classvaluemeta{} 
          &::=& {\classvalue{\classhint} {\overrightarrow {\classfieldpair{fld} {\v{}}}}}
               &\mbox{Class Values}
\end{altgrammar}
$$
\caption{Syntax of Terms, Types, Propositions, and Objects}
\end{figure*}


\begin{figure*}
\begin{mathpar}
\infer [T-Local]
{ \inpropenv {\propenv{}} {\isprop {\t{}} {\x{}}}
  \\
  \s{} = {\falsydiff} }
{ \judgementfillcol {\propenv{}} 
                   {\hastype {\x{}} {\t{}}}
                   {\filterset {{\notprop {\s{}} {\x{}}}} {{\isprop {\s{}} {\x{}}}}}
                   {\x{}}
                   }

\infer [T-Const]
{}
{ \judgement {\propenv{}} 
             {\hastype {\const{}} {\constanttype{\const{}}}}
             {\filterset {\topprop{}}{\botprop{}}}
             {\emptyobject{}}
                   }

\infer [T-True]
{}
{ \judgementfillcol {\propenv{}}
      {{\hastype {\true{}} {\True{}}}}
      {{\filterset {\topprop{}} {\botprop{}}}}
      {\emptyobject{}}
    }

\infer [T-False]
{}
{ \judgementfillcol {\propenv{}}
      {{\hastype {\false{}} {\False{}}}}
      {\filterset {\botprop{}} {\topprop{}}}
    {\emptyobject{}}
   }

\infer [T-Nil]
{}
{ \judgementfillcol {\propenv{}} 
      {{\hastype {\nil{}} {\Nil{}}}} 
      {\filterset {\botprop{}} {\topprop{}}}
      {\emptyobject{}} 
    }

\infer [T-Do]
{ 
  \judgement {\propenv{}} 
             {\hastype {\e1} {\t1}} 
             {\filterset {\thenprop {\prop{1}}} {\elseprop {\prop1}}} 
             {\object{1}}
\\\\
\judgement {\propenv{}, {\trdiff {\orprop {\thenprop {\prop{1}}} {\elseprop {\prop{1}}}}}}
           {\hastype {\e{}} {\t{}}} 
           {\filterset {\thenprop {\prop{}}} {\elseprop {\prop{}}}} 
           {\object{}}
  }
{ \judgement
    {\propenv{}} 
    {\hastype {\doexp {\e1} {\e{}}} {\t{}}} 
    {\filterset {\thenprop {\prop{}}} {\elseprop {\prop{}}}} {\object{}} 
             }

\infer [T-If]
{ \judgementfillcol {\propenv{}} {\hastype {\e1} {\t{1}}} {\filterset {\thenprop {\prop{1}}} {\elseprop {\prop{1}}}}
                 {\object{1}} 
             \\\\
  \judgementfillcol {\propenv{}, {\thenprop {\prop{1}}}}
                 {\hastype {\e2} {\t{}}} {\filterset {\thenprop {\prop{2}}} {\elseprop {\prop{2}}}}
                 {\object{}}
  \\\\
  \judgementfillcol {\propenv{}, {\elseprop {\prop{1}}}}
                 {\hastype {\e3} {\t{}}} {\filterset {\thenprop {\prop{3}}} {\elseprop {\prop{3}}}}
                 {\object{}}
             }
{ \judgementfillcol {\propenv{}} {\hastype {\ifexp {\e1} {\e2} {\e3}} {\t{}}} 
                 {\filterset {\orprop {\thenprop {\prop{2}}} {\thenprop {\prop{3}}}} 
                             {\orprop {\elseprop {\prop{2}}} {\elseprop {\prop{3}}}}}
                 {\object{}} }

\infer [T-Let]
{ \judgement {\propenv{}} {\hastype {\e{1}} {\s{}}} {\filterset {\thenprop {\prop{1}}} {\elseprop {\prop{1}}}}
             {\object{1}}
             \\\\
             \propp{} = {\impprop {\notprop {\falsydiff{}} {\x{}}} {\thenprop {\prop{1}}}}
             \\\\
             \proppp{} = {\impprop {\isprop {\falsydiff{}} {\x{}}} {\elseprop {\prop{1}}}}
  \\\\
  \judgement
       {\propenv{}, {\isprop {\s{}} {\x{}}},
         {\propp{}},
         {\proppp{}}}
             {\hastype {\e{}} {\t{}}} {\filterset {\thenprop {\prop{}}} {\elseprop {\prop{}}}}
             {\object{}} 
             }
{ \judgement {\propenv{}} {\hastype {\letexp {\x{}} {\e{1}} {\e{}}} {\t{}}}
             {\replacefor {\filterset {\thenprop {\prop{}}} {\elseprop {\prop{}}}}
                          {\object{1}}
                          {\x{}}}
             {\replacefor {\object{}} 
                          {\object{1}}
                          {\x{}}}
             }

\infer [T-App]
{ \judgement {\propenv{}} {\hastype {\e{}} {\ArrowOne {\x{}} {\s{}}
                                                       {\t{}}
                                                       {\filterset {\thenprop {\prop{f}}}
                                                                   {\elseprop {\prop{f}}}}
                                                       {\object{f}}}}
                {\filterset {\thenprop {\prop{}}}
                            {\elseprop {\prop{}}}}
                {\object{}}
  \\\\
  \judgement {\propenv{}}
                 {\hastype {\ep{}} {\s{}}}
                 {\filterset {\thenprop {\propp{}}}
                             {\elseprop {\propp{}}}}
                 {\objectp{}} 
}
{ \judgementfillcol {\propenv{}} {\hastype {\appexp {\e{}} {\ep{}}}
                                        {\replacefor {\t{}}
                                                     {\objectp{}}
                                                     {\x{}}}}
                 {\replacefor {\filterset {\thenprop {\prop{f}}}
                                          {\elseprop {\prop{f}}}}
                              {\objectp{}}
                              {\x{}}}
                 {\replacefor {\object{f}}
                              {\objectp{}}
                              {\x{}}}
}
                 
                              


\infer [T-Abs]
{ \judgement {\propenv{}, {\isprop {\s{}} {\x{}}}}
            {\hastype {\e{}} {\t{}}}
             {\filterset {\thenprop {\prop{}}}
                         {\elseprop {\prop{}}}}
             {\object{}}}
{ \judgement {\propenv{}} {\hastype {\abs {\x{}} {\s{}} {\e{}}}
                                           {\ArrowOne {\x{}} {\s{}}
                                                      {\t{}}
                                                      {\filterset {\thenprop {\prop{}}}
                                                                  {\elseprop {\prop{}}}}
                                                      {\object{}}}}
                {\filterset {\topprop{}}
                            {\botprop{}}}
                {\emptyobject{}}}

\infer [T-Clos]
{ \exists {\propenv{}}. \satisfies{\openv{}}{\propenv{}}
  \ \text{and}\ 
\judgement {\propenv{}} {\hastype {\abs {\x{}} {\t{}} {\e{}}} {\s{}}}
                 {\filterset {\thenprop {\prop{}}}
                             {\elseprop {\prop{}}}}
                 {\object{}}
              }
{ \judgement {}
            {\hastype {\closure {\openv{}} {\abs {\x{}} {\t{}} {\e{}}}} 
                      {\s{}}}
             {\filterset {\thenprop {\prop{}}}
                         {\elseprop {\prop{}}}}
             {\object{}}}

\infer [T-Error]
{}
{ \judgement {\propenv{}} {\hastype {\errorvalv} {\Bot}}
            {\filterset {\botprop{}} {\botprop{}}}
           {\emptyobject{}}}

\infer [T-Subsume]
{ \judgementfillcol {\propenv{}} {\hastype {\e{}} {\t{}}}
             {\filterset {\thenprop {\prop{}}}
                         {\elseprop {\prop{}}}}
             {\object}
\\\\
\inpropenv {\propenv{}, {\thenprop {\prop{}}}} {\thenprop {\propp{}}}
\\
\inpropenv {\propenv{}, {\elseprop {\prop{}}}} {\elseprop {\propp{}}}
\\\\
\inpropenv {} {\issubtype {\t{}} {\tp{}}}
\\
\inpropenv {} {\issubtype {\object{}} {\objectp{}}}
}
{ \judgementfillcol {\propenv{}} {\hastype {\e{}} {\tp{}}}
             {\filterset {\thenprop {\propp{}}}
                         {\elseprop {\propp{}}}}
             {\objectp}}
\end{mathpar}
\caption{Standard Typing Rules}
\end{figure*}


\begin{figure*}
\begin{mathpar}

\infer [T-NewStatic]
{ 
  \overrightarrow{
\javatotc {\classhint{i}}
          {\t{i}}
          }
  \\\\
  \javatotc {\classhint{}}
            {\t{}}
  \\
  \overrightarrow{
  \judgementtwo {\propenv{}}
                    {\hastype {\e{i}} {\t{i}}}
                  }
           }
{ \judgementfillcol {\propenv{}} {\hastype {\newstaticexp {\overrightarrow{\classhint{i}}} {\classhint{}} 
                                                          {\class{}} {\overrightarrow{\e{i}}}} {\t{}}}
             {\filterset {\topprop{}} {\botprop{}}}
             {\emptyobject}
             }

\infer [T-FieldStatic]
{ \javatotc {\classhint{1}} {\class{}}
             \\\\
  \javatotcnil {\classhint{2}} {\t{}}
  \\\\
  \judgementtwo {\propenv{}} {\hastype {\e{1}} {\class{}}}
           }
{ \judgement {\propenv{}} {\hastype {\fieldstaticexp {\classhint{1}} {\classhint{2}} {\fld{}} {\e{1}}} {\t{}}}
             {\filterset {\topprop{}} {\topprop}}
             {\emptyobject{}}}

\infer [T-MethodStatic]
{ 
  \overrightarrow{\javatotc {\classhint{i}} {\t{i}}}
             \\
             \javatotc {\classhint{1}} {\class{}}
             \\\\
             \javatotcnil {\classhint{2}} {\t{}}
             \\\\
  \judgementtwo {\propenv{}} {\hastype {\e{}} {\class{}}}
             \\
             \overrightarrow{
  \judgementtwo {\propenv{}} {\hastype {\e{i}} {\t{i}}}
                  }
           }
{ \judgementfillcol {\propenv{}} {\hastype {\methodstaticexp {\classhint{1}} 
                                                            {\overrightarrow {\classhint{i}}} 
                                                            {\classhint{2}}
                                                            {\mth{}} {\e{}} {\overrightarrow{\e{i}}}}
                                    {\t{}}}
             {\filterset {\topprop{}} {\topprop{}}}
             {\emptyobject{}}}

\end{mathpar}
\caption{Java Interop Typing Rules}
\end{figure*}

\begin{figure*}
\begin{mathpar}

\infer [T-DefMulti]
{ \s{} = {\ArrowOne {\x{}} {\t{}} {\tp{}}
                          {\filterset {\thenprop {\prop{}}}
                                      {\elseprop {\prop{}}}}
                          {\object{}}}
                          \\
  \sp{} = {\ArrowOne {\x{}} {\t{}} {\tpp{}}
                          {\filterset {\thenprop {\propp{}}}
                                      {\elseprop {\propp{}}}}
                          {\objectp{}}}
                \\\\
\judgementtwo {\propenv{}} {\hastype {\e{}} {\sp{}}}
}
{
  \judgement {\propenv{}}  
      {\hastype {\createmultiexp {\s{}}
                                 {\e{}}} 
                {\MultiFntype {\s{}} {\sp{}}}}
             {\filterset {\topprop{}} {\botprop{}}}
             {\emptyobject{}}
}

\infer [T-DefMethod]
{
  \t{m} = {\ArrowOne {\x{}} {\t{}} {\s{}}
                                    {\filterset {\thenprop {\prop{}}}
                                                {\elseprop {\prop{}}}}
                                    {\object{}}}
\\
\t{d} = {\ArrowOne {\x{}} {\t{}} {\sp{}}
                   {\filterset {\thenprop {\propp{}}}
                               {\elseprop {\propp{}}}}
                   {\objectp{}}}
             \\\\
\judgementtwo {\propenv{}}
                  {\hastype {\e{m}} {\MultiFntype {\t{m}} {\t{d}}}}
\\
  \isacompare{\sp{}}{\objectp{}}{\t{v}}{\filterset {\thenprop {\proppp{}}} {\elseprop {\proppp{}}}}
\\\\
\judgementtwo {\propenv{}}
           {\hastype {\e{v}} {\t{v}}}
           \\
\judgementfillcol {\propenv{}, {\isprop{\t{}} {\x{}}}, {\thenprop {\proppp{}}}}
           {\hastype {\e{b}} {\s{}}}
           {\filterset {\thenprop {\prop{}}}
                       {\elseprop {\prop{}}}}
           {\object{}}
}
{ \judgementfillcol {\propenv{}} {\hastype {\extendmultiexp {\e{m}} {\e{v}} 
                                                            {\abs {\x{}} {\t{}} {\e{b}}}}
                                           {\MultiFntype {\t{m}} {\t{d}}}}
             {\filterset {\topprop{}} {\botprop{}}}
             {\emptyobject}
}

\infer [T-IsA]
{
  \judgement {\propenv{}} {\hastype {\e{}} {\s{}}}
             {\filterset {\thenprop {\propp{}}}
                         {\elseprop {\propp{}}}}
                       {\object{}}
  \\
  \judgementtwo
             {\propenv{}} {\hastype {\ep{}} {\t{}}}
             \\
  \isacompare{\s{}}{\object{}}{\t{}}{\filterset {\thenprop {\prop{}}} {\elseprop {\prop{}}}}
}
{ \judgement {\propenv{}} {\hastype {\isaapp {\e{}} {\ep{}}}
                                    {\Boolean{}}}
             {\filterset {\thenprop {\prop{}}} {\elseprop {\prop{}}}}
             {\emptyobject}}

\infer [T-Multi]
{ \judgementtwo {} {\hastype {\e{}} {\t{}}} 
}
{ \judgement {}
            {\hastype {\multi {\e{}} {\disptable{}}}
                      {\MultiFntype {\s{}} {\t{}}}}
             {\filterset {\topprop{}} {\botprop{}}}
           {\emptyobject{}}}

\end{mathpar}
\caption{Multimethod Typing Rules}
\end{figure*}

\begin{figure*}
\begin{mathpar}

\infer [T-GetHMap]
{ \judgement {\propenv{}} {\hastype {\e{m}} {\HMapgeneric {\mandatory{}} {\absent{}}}}
           {\filterset {\thenprop {\prop{}}} {\elseprop {\prop{}}}}
           {\object{}}
         \\\\
  \judgementtwo {\propenv{}} {\hastype {\e{k}} {\Value {k}}}
             \\
             {\inmandatory{\k{}}{\t{}}{\mandatory{}}}
       }
{ \judgement {\propenv{}} {\hastype {\getexp {\e{m}} {\e{k}}} {\t{}}}
             {\filterset {\topprop{}} {\topprop{}}}
             {\replacefor {\path {\keype{k}} {\x{}}}
                          {\object{}}
                          {\x{}}}
                                  }

\infer [T-GetHMapAbsent]
{ \judgement {\propenv{}} {\hastype {\e{m}} {\HMapgeneric {\mandatory{}} {\absent}}}
           {\filterset {\thenprop {\prop{}}} {\elseprop {\prop{}}}}
           {\object{}}
         \\\\
  \judgementtwo {\propenv{}} {\hastype {\e{k}} {\Value {k}}}
             \\
             {\inabsent{\k{}}{\absent{}}}
       }
{ \judgement {\propenv{}} {\hastype {\getexp {\e{m}} {\e{k}}}
                                    {\Nil{}}}
             {\filterset {\topprop{}} {\topprop{}}}
             {\replacefor
               {\path {\keype{k}} {\x{}}}
                          {\object{}}
                          {\x{}}}
                        }

\infer [T-GetHMapPartialDefault]
{ \judgement {\propenv{}} {\hastype {\e{m}} {\HMapp {\mandatory{}} {\absent}}}
           {\filterset {\thenprop {\prop{}}} {\elseprop {\prop{}}}}
           {\object{}}
         \\\\
  \judgementtwo {\propenv{}} {\hastype {\e{k}} {\Value {k}}}
             \\
             {\notinmandatory{\k{}}{\t{}}{\mandatory{}}}
             \\
             {\notinabsent{\k{}}{\absent{}}}
       }
{ \judgement {\propenv{}} {\hastype {\getexp {\e{m}} {\e{k}}}
                                    {\Top{}}}
             {\filterset {\topprop{}} {\topprop{}}}
             {\replacefor
               {\path {\keype{k}} {\x{}}}
                          {\object{}}
                          {\x{}}}
                        }

\infer [T-EmptyMap]
{}
{ \judgement {\propenv{}} {\hastype {\emptymap{}}
  {\HMapc {\mandatoryset {}}}}
             {\filterset {\topprop{}} {\botprop{}}}
             {\emptyobject}}

\infer [T-Kw]
{}
{ \judgementfillcol {\propenv{}} 
      {{\hastype {\k{}} {\Value{\k{}}}}}
      {\filterset {\topprop{}}{\botprop{}}}
      {\emptyobject{}} 
    }

\infer [T-Assoc]
{ 
  \judgementtwo {\propenv{}} {\hastype {\e{m}} {\HMapgeneric {\mandatory{}} {\absent}}}
  \\
  \judgementtwo {\propenv{}} {\hastype {\e{k}} {\Value{\k{}}}}
  \\
  \judgementtwo {\propenv{}} {\hastype {\e{v}} {\t{}}}
  \\
  {\k{}} \not\in {\absent{}}
}
{ \judgement {\propenv{}} 
             {\hastype {\assocexp {\e{m}} {\e{k}} {\e{v}}}
                       {\HMapgeneric {\extendmandatoryset {\mandatory{}}{\k{}}{\t{}}} {\absent}}}
             {\filterset {\topprop{}} {\botprop{}}}
             {\emptyobject{}}
}

\end{mathpar}
\caption{HMap Typing Rules}
\end{figure*}



\begin{figure*}
\begin{mathpar}

  \TALocal{}

  \TANil{}

\infer [TA-True]
{}
{ \tajudgement {\taenv{}} {\hastype {\true{}} {\Booleanhint{}}}
}

\infer [TA-False]
{}
{ \tajudgement {\taenv{}} {\hastype {\false{}} {\Booleanhint{}}}
}

\infer [TA-Kw]
{}
{ \tajudgement {\taenv{}} {\hastype {\k{}} {\Keyword{}}}
}

\infer [TA-Class]
{}
{ \tajudgement {\taenv{}} {\hastype{\class{}}{\Class{}}}}

% never in user code
%\infer [T-Instance]
%{}
%{ \tajudgement {\taenv{}} {\hastype{\classvalue{\classhint{}} {\overrightarrow {\classfieldpair{\fld{}} {\v{}}}}}{\unknownhint{}}}}

\TANewStatic{}

\infer [TA-NewRefl]
{}
{ \tajudgement {\taenv{}}
  {\hastype {\newexp {\class{}} {\overrightarrow {\e{}}}}
                     {\unknownhint{}}}
}

\infer [TA-Abs]
{}
{ \tajudgement {\taenv{}}
               {\hastype {\abs {\x{}} {\t{}} {\e{}}}
                         {\unknownhint{}}}
                       }

\TALetHint{}

\TALet{}

\infer [TA-FieldRefl]
{}
{ \tajudgement {\taenv{}}
               {\hastype {\fieldexp {\fld} {\e{}}}
                         {\unknownhint{}}}
  }

\infer [TA-FieldStatic]
{}
{ \tajudgement {\taenv{}}
               {\hastype {\fieldstaticexp {\classhint{1}} {\classhint{}} {\fld{}} {\e{}}}
                         {\classhint{}}}
  }

\infer [TA-MethodRefl]
{}
{ \tajudgement {\taenv{}}
               {\hastype {\methodexp {\mth} {\e{}} {\overrightarrow {\e{i}}}}
                         {\unknownhint{}}}
  }

\infer [TA-MethodStatic]
{}
{ \tajudgement {\taenv{}}
               {\hastype {\methodstaticexp {\classhint{1}} 
                                           {\overrightarrow{\classhint{i}}} 
                                           {\classhint{}} 
                                         {\mth{}} {\e{}} {\overrightarrow {\e{i}}}}
                         {\classhint{}}}
  }

\infer [TA-App]
{}
{ \tajudgement {\taenv}
               {\hastype {\appexp {\e{}} {\ep{}}}
                         {\unknownhint{}}}
                       }

\infer [TA-AppLocal]
{ \intaenv {\taenv{}} {\x{}} {\tatype{}}
}
{ \tajudgement {\taenv}
               {\hastype {\appexp {\x{}} {\ep{}}}
                         {\tatype{}}}
                       }

\infer [TA-Do]
{ \tajudgement {\taenv{}} {\hastype {\e{}} {\tatype{}}}
}
{ \tajudgement {\taenv{}} {\hastype {\doexp {\e1} {\e{}}} {\tatype{}}}
}

\infer [TA-DefMulti]
{}
{ \tajudgement {\taenv{}} {\hastype {\createmultiexp {\t{}} {\e{}}} {\unknownhint{}}}}

\infer [TA-DefMethod]
{}
{ \tajudgement {\taenv{}} {\hastype {\extendmultiexp {\e{1}} {\e{2}} {\e{3}}} {\unknownhint{}}}}

\infer [TA-If]
{ \tajudgement {\taenv{}} {\hastype {\e{2}} {\classhint{}}}
  \\\\
  \tajudgement {\taenv{}} {\hastype {\e{3}} {\classhint{}}}
}
{ \tajudgement {\taenv{}} {\hastype {\ifexp {\e{1}} {\e{2}} {\e{3}}} {\classhint{}}}}

\infer [TA-IfUnknown]
{}
{ \tajudgement {\taenv{}} {\hastype {\ifexp {\e{1}} {\e{2}} {\e{3}}} {\unknownhint{}}}}

\infer [TA-Isa]
{}
{ \tajudgement {\taenv{}} {\hastype {\isaapp {\e{}} {\e{}}} {\Booleanhint{}}}}

\infer [TA-Const]
{}
{ \tajudgement {\taenv{}} {\hastype {\const{}} {\unknownhint{}}}}

\end{mathpar}
\caption{Type Hint Inference}
\label{appendix:figure:hintinfer}
\end{figure*}

\begin{figure*}
\begin{mathpar}

\RLocal{}

\infer [R-LocalHint]
{}
{ \rewrite {\taenv{}}
  {\hinted {\classhint{}} {\x{}}}
  {\hinted {\classhint{}} {\x{}}}
         }

\infer [R-Val]
{}
{ \rewrite {\taenv{}}
  {\v{}}
  {\v{}}
}

\infer [R-NewRefl]
{ \overrightarrow{
  \rewrite {\taenv{}}
           {\e{i}}
           {\e{j}}
         }
         }
{ \rewrite {\taenv{}}
           {\newexp {\class{}} {\overrightarrow {\e{i}}}}
           {\newexp {\class{}} {\overrightarrow {\e{j}}}}
}

\RNewElimRefl{}

\infer [R-NewStatic]
{ \overrightarrow
  {\rewrite {\taenv{}}
           {\e{i}}
           {\e{j}}}
         }
{ \rewrite {\taenv{}}
           {\newstaticexp {\overrightarrow {\classhint{p}}} {\classhint{}} {\class{}} {\overrightarrow {\e{i}}}}
           {\newstaticexp {\overrightarrow {\classhint{p}}} {\classhint{}} {\class{}} {\overrightarrow {\e{j}}}}
}

\RAbs{}

         \RLet{}

\RLetHint{}

\infer [R-FieldRefl]
{ \rewrite {\taenv{}}
           {\e{}}
           {\ep{}}
         }
{ \rewrite {\taenv{}}
           {\fieldexp {\fld{}} {\e{}}}
           {\fieldexp {\fld{}} {\ep{}}}
         }

\RFieldElimRefl{}

\infer [R-FieldStatic]
{ \rewrite {\taenv{}}
           {\e{}}
           {\ep{}}
         }
{ \rewrite {\taenv{}}
           {\fieldstaticexp {\classhint{1}} {\classhint{2}} {\fld{}} {\e{}}}
           {\fieldstaticexp {\classhint{1}} {\classhint{2}} {\fld{}} {\ep{}}}
         }

\infer [R-MethodRefl]
{ \rewrite {\taenv{}}
           {\e{}}
           {\ep{}}
           \\
           \overrightarrow
           {
  \rewrite {\taenv{}}
           {\e{i}}
           {\e{j}}}
         }
{ \rewrite {\taenv{}}
           {\methodexp {\mth} {\e{}} {\overrightarrow {\e{i}}}}
           {\methodexp {\mth} {\ep{}} {\overrightarrow {\e{j}}}}
  }

\RMethodElimRefl{}

\infer [R-MethodStatic]
{ \rewrite {\taenv{}}
           {\e{}}
           {\ep{}}
           \\
  \rewrite {\taenv{}}
           {\overrightarrow {\e{i}}}
           {\overrightarrow {\e{j}}}
         }
{ \rewrite {\taenv{}}
           {\methodstaticexp {\classhint{}} {\overrightarrow {\classhint{1}}} {\classhint{2}} {\mth{}} {\e{}} {\overrightarrow {\e{i}}}}
           {\methodstaticexp {\classhint{}} {\overrightarrow {\classhint{1}}} {\classhint{2}} {\mth{}} {\ep{}} {\overrightarrow {\e{j}}}}
  }

\infer [R-App]
{ \rewrite {\taenv{}} {\e{}} {\ep{}}
  \\\\
  \rewrite {\taenv{}} {\e{i}} {\e{j}}
}
{ \rewrite {\taenv{}}
           {\appexp {\e{}} {\e{i}}}
           {\appexp {\ep{}} {\e{j}}}
         }

\end{mathpar}
\caption{Java Reflection Resolution}
\label{appendix:figure:rewrite}
\end{figure*}


\begin{figure*}
\begin{mathpar}

\infer [SO-Refl]
{}
{ \issubobjin {} {\object{}} {\object{}}}


\infer [SO-Top]
{}
{ \issubobjin {} {\object{}} {\emptyobject{}}}

\infer [S-Refl]
{}
{ \issubtypein {} {\t{}} {\t{}}}

\infer [S-Top]
{}
{ \issubtypein {} {\t{}} {\Top{}}}

\infer [S-UnionSuper]
{ \exists i.\ {\issubtypein {} {\t{}} {\s{i}}}}
{ \issubtypein {} {\t{}} {\Unionsplice{{\overrightarrow{\s{}}}^{i}}}}

% FIXME fix overrightarrow caption
\infer [S-UnionSub]
{ {\overrightarrow {\issubtypein {} {\t{i}} {\s{}}}}^{i}}
{ \issubtypein {} {\Unionsplice{\overrightarrow{\t{}}^{i}}} {\s{}}}

\infer [S-Fun]
{ \issubtypein {} {\sp{}} {\s{}}
  \\
  \issubtypein {} {\t{}} {\tp{}}
  \\\\
  \inpropenv {\thenprop {\prop{}}} {\thenprop {\propp{}}}
  \\
  \inpropenv {\elseprop {\prop{}}} {\elseprop {\propp{}}}
  \\
  \inpropenv {\object{}} {\objectp{}}
}
{ \issubtypein {}
    {\ArrowOne {\x{}} {\s{}}
              {\t{}}
              {\filterset {\thenprop {\prop{}}}
                          {\elseprop {\prop{}}}}
              {\object{}}}
    {\ArrowOne {\x{}} {\sp{}}
              {\tp{}}
              {\filterset {\thenprop {\propp{}}}
                          {\elseprop {\propp{}}}}
              {\objectp{}}}}

\infer [S-PMulti]
{ \issubtypein {} {\s{}} {\sp{}}
  \\
  \issubtypein {} {\t{}} {\tp{}}}
{ \issubtypein {} {\MultiFntype {\s{}} {\t{}}}{\MultiFntype {\sp{}} {\tp{}}}}

\infer [S-PMultiFn]
{ \issubtypein {}
  {\s{t}}
  {\ArrowOne {\x{}} {\s{}} {\t{}}
                                  {\filterset {\thenprop {\prop{}}}
                                              {\elseprop {\prop{}}}}
                                      {\object{}}}
                                      \\
  \issubtypein {}
  {\s{d}}
  {\ArrowOne {\x{}} {\s{}} {\tp{}}
                                  {\filterset {\thenprop {\propp{}}}
                                              {\elseprop {\propp{}}}}
                                      {\objectp{}}}
}
{ \issubtypein {} {\MultiFntype{\s{t}}{\s{d}}}
                  {\ArrowOne {\x{}} {\s{}} {\t{}}
                             {\filterset {\thenprop {\prop{}}}
                                         {\elseprop {\prop{}}}}
                             {\object{}}}}

\infer [S-HMap]
{ \forall i.\ ({\k{i}},\ \s{i}) \in \mandatory{} \text{ and } \issubtypein {} {\s{i{}}} {\t{i{}}}
  \\\\
  {\absent{1}} \supseteq {\absent{2}}
}
{\issubtypein {} {\HMapgeneric {\mandatory{}} {\absent{1}}}
  {\HMapgeneric {\overrightarrow{(\k{},\ \t{})}^i} {\absent{2}}}
               }

\infer [S-HMapP]
{ \forall i.\ ({\k{i}},\ \s{i}) \in \mandatory{} \text{ and } \issubtypein {} {\s{i{}}} {\t{i{}}}
}
{\issubtypein {} 
  {\HMapc {\mandatory{}}}
  {\HMapp {\overrightarrow{(\k{},\ \t{})}^i} {\absent{}}}
               }

\infer [S-Class]
{}
{\issubtypein {}
  {\class{}}
  {\Class}}

\infer [S-MultiMono]
{}
{\issubtypein {}
  {\MultiFntype {\ArrowOne {\x{}} {\s{}} {\t{}}
                          {\filterset {\thenprop{\prop{}}}
                                      {\elseprop{\prop{}}}}
                          {\object{}}}
                {\ArrowOne {\x{}} {\s{}} {\tp{}}
                          {\filterset {\thenprop{\propp{}}}
                                      {\elseprop{\propp{}}}}
                          {\objectp{}}}
                }
  {\PMulti{}}
}

\infer [S-FunMono]
{}
{\issubtypein {}
  {\ArrowOne {\x{}} {\s{}}
             {\t{}}
             {\filterset {\thenprop{\prop{}}}
                         {\elseprop{\prop{}}}}
             {\object{}}}
  {\IFn{}}}

\end{mathpar}
\caption{Subtyping rules}
\end{figure*}


%$$
%\begin{tdisplay}{Evaluation Contexts}
%  \begin{altgrammar}
%    \E{} &::=& [ ] % application rules
%              \alt (\c{}\ \overrightarrow{\v{}}\ \E{}\ \overrightarrow{\exp{}}) % eval arguments left-to-right
%              % map rules
%              \alt \{\overrightarrow{\v{}\ \v{}}\ \E{}\ \exp{}\ \overrightarrow{\exp{}\ \exp{}} \} % key first
%              \alt \{\overrightarrow{\v{}\ \v{}}\ \v{}\ \E{}\ \overrightarrow{\exp{}\ \exp{}} \}   % value next
%              &\mbox{Evaluation Contexts}
%  \end{altgrammar}
%\end{tdisplay}
%$$ 

\begin{figure*}
$$
\begin{altgrammar}

  \methodtypealign
  {\ct{}}
  {\classhint{}}
  {\mth{}}
  {\overrightarrow{\classhint{p}}}
  {\classhint{}}
  {\overrightarrow{\classhint{p}}}
  {\classhint{r}}
  & \text{if}\ \ctmthentry{\mth{}}{\overrightarrow{\classhint{p}}}{\classhint{r}} \in {\ctlookupmethods{\ct{}}{\classhint{}}}
      \\\\
  \fieldtypealign 
  {\ct{}}
  {\classhint{}}
  {\fld{}}
  {\classhint{}}
  {\classhint{f}}
  & \text{if}\ \ctfldentry{\fld{}}{\classhint{f}} \in {\ctlookupfields{\ct{}}{\classhint{}}}
      \\\\
  \ctorparamsalign 
      {\ct{}}
      {\classhint{}}
      {\overrightarrow{\class{p{}}}}
      {\overrightarrow{\class{p{}}}}
  & \text{if}\ \ctctorentry{\overrightarrow{\class{p{}}}} \in {\ctlookupctors{\ct{}}{\classhint{}}}

%{\javaspecial {\JavaField {\class{}}}} &=& {\javatotc {\class{}} {\true{}}}
%\\
%{\javaspecial {\JavaMethod {\overrightarrow{{\class{1}}}} {\class{2}}}} &=& 
%    {\Arrow {\overrightarrow {\hastype {\x{}} {\javatotc {\class{1}} {\false{}}}}}
%            {\javatotc {\class{2}} {\true{}}}
%            {\filterset {\topprop{}} {\topprop{}}}
%            {\emptyobject{}}
%            }
%\\
%{\javaspecial {\JavaCtor {\overrightarrow{{\class{1}}}} {\class{2}}}} &=& 
%    {\Arrow {\overrightarrow {\hastype {\x{}} {\javatotc {\class{1}} {\false{}}}}}
%            {\javatotc {\class{2}} {\false{}}}
%            {\filterset {\topprop{}} {\topprop{}}}
%            {\emptyobject{}}
%            }
%\\\\
\\\\

\javatotcalign {\Void{}} {\Nil}
\\
\javatotcalignnil {\Void{}} {\Nil}
\\
\javatotcalign {\class{}} \class{}
\\
\javatotcalignnil {\class{}} {\Union {\Nil} {\class{}}}

\end{altgrammar}
$$
\caption{Converting Java types to Typed Clojure types}
\end{figure*}

\begin{figure*}
\begin{mathpar}

\begin{array}{lllr}

  \constanttype{\classconst} &=& {\ArrowOne {\x{}} {\Top{}}
                                      {\Union{\nil{}}{\Class{}}}
                                      {\filterset {\topprop{}}
                                                  {\topprop{}}}
                                      {\path {\classpe{}} {\x{}}}}
                                      \\

  \constanttype{\throwconst} &=& {\ArrowOne {\x{}} {\Top{}}
                                      {\Bot{}}
                                      {\filterset {\botprop{}}
                                                  {\botprop{}}}
                                      {\emptyobject{}}}

\end{array}
\end{mathpar}
\caption{Constant Typing}
\end{figure*}

\begin{figure*}
\begin{mathpar}

\begin{array}{lllr}

\constantopsem{\classconst}{\classvalue{\class{}} {\overrightarrow {\classfieldpair{\fld{}} {\v{}}}}} &=& \class{}\\
\constantopsem{\classconst}{\class{}} &=& \Class{}\\
\constantopsem{\classconst}{\true{}} &=& \Boolean{}\\
\constantopsem{\classconst}{\false{}} &=& \Boolean{}\\
\constantopsem{\classconst}{\closure {\openv{}} {\abs {\x{}} {\t{}} {\e{}}}} &=& \IFn{}\\
\constantopsem{\classconst}{\multi {\v{d}} {\disptable{}}} &=& \PMulti{}\\
\constantopsem{\classconst}{\curlymapvaloverright{\v{1}}{\v{2}}} &=& \HMapInstance{}\\
\constantopsem{\classconst}{\k{}} &=& \Keyword{}\\
%\constantopsem{\classconst}{\num{}} &=& \Number{}\\
\constantopsem{\classconst}{\nil{}} &=& \nil{}\\
  %\constantopsem{\classconst}{\v{}} &=& \wrong{} & otherwise\\
                                      \\\\

\constantopsem{\throwconst}{\v{}} &=& \errorval{\v{}}\\


\end{array}
\end{mathpar}
\caption{Primitives}
\end{figure*}


\begin{figure*}
$$
\begin{array}{llr}
  \isacompare{\s{}}{\object{}}{\t{}}{\replacefor{\filtersetparen{\isprop{\t{}}{\x{}}}{\notprop{\t{}}{\x{}}}}{\object{}}{\x{}}}
\end{array}
$$
$$
\begin{array}{lclr}
  \isaopsem{\v{}}{\v{}} &=& {\true{}}\\
  \isaopsem{\classvaluemeta{1}}{\classvaluemeta{2}} &=& {\true{}} & \classvaluemeta{1}\ \text{is a subclass of}\ \classvaluemeta{2}\\
  \isaopsem{\v{1}}{\v{2}} &=& {\false{}} & \text{otherwise}
\end{array}
$$
\caption{Definition of isa?}
\end{figure*}

%\begin{figure*}
%$$
%\begin{array}{llrr}
%  \isacompare{\HVec{\overrightarrow{{\t{}};{\prop{}};{\object{}}}^i}}
%             {\object{}}
%             {\HVec{\overrightarrow{{\t{}};{\prop{}};{\object{}}}^j}}
%             {\replacefor
%              {\filtersetparen
%                {\isprop {\HVec{\overrightarrow{\isacomparethree{\t{i}}{\object{i}}{\t{j}}}}}{\x{}}}
%                {\notprop{\HVec{\overrightarrow{\isacomparethree{\t{i}}{\object{i}}{\t{j}}}}}{\x{}}}}
%              {\object{}}
%              {\x{}}}
%              & i = j
%\end{array}
%$$
%$$
%\begin{array}{lclr}
%  \isaopsem{\rtvector{\overrightarrow{\x{}}^i}}{\rtvector{\overrightarrow{\x{}}^j}} &=& {\true{}}
%                                                                                    & i = j, \overrightarrow{\isaopsem{\x{i}}{\x{j}} = {\true{}}}^{i,j}
%  \\
%\end{array}
%$$
%\caption{isa? Vector Extensions}
%\end{figure*}

\begin{figure*}
$$
\begin{array}{lr}

  \getmethod {\disptable{}}
             {\v{e}}
             {\roundpair{\v{v}}{\v{f}}}

             & 
             \text{let } {\overrightarrow{\v{fs}}} = 
                \{ (\v{v}, \v{f}) | 
                    {\roundpair{\v{v}}{\v{f}}} \in \disptable{} 
                  \text{ and } 
                    \opsem{}{\isaapp{\v{v}}{\v{e}}}{\true{}} \}, 
             \text{ if } {\overrightarrow{\v{fs}}} = \{{\roundpair{\v{v}}{\v{f}}}\}

             \\
  \getmethod {\disptable{}}
             {\v{e}}
             {\mmerror{}} & \text{otherwise}

\end{array}
$$
\caption{Definition of get-method}
\end{figure*}


\clearpage

\begin{figure*}
\begin{mathpar}

\BLocal{}

\BDo{}

\BLet{}

\BVal{}

\BIfTrue{}

\BIfFalse{}

\BAbs{}

\BBetaClosure{}

\BDelta{}

\BBetaMulti{}

\infer [B-Field]
{ \opsem {\openv{}}
         {\e{}} 
       {\v{1}}
         \\\\
         \getfieldjava{\classhint{1}} {\v{1}} {\fld{}} {\classhint{2}} {\v{}}
       }
{ \opsem {\openv{}}
         {\fieldstaticexp {\classhint{1}} {\classhint{2}} {\fld{}} {\e{}}}
         {\v{}}
   }

\infer [B-Method]
{ \opsem {\openv{}}
         {\e{m}}
         {\v{m}}
  \\
  \overrightarrow{
  \opsem {\openv{}}
         {\e{a}}
         {\v{a}}
       }
  \\\\
  \invokejavamethod {\classhint{1}} {\v{m}} {mth}
                    {\overrightarrow{\classhint{a}}} {\overrightarrow{\v{a}}}
                    {\classhint{2}}
                    {\v{}}
}
{\opsem {\openv{}}
        {\methodstaticexp {\classhint{1}} {\overrightarrow{\classhint{a}}} {\classhint{2}} {mth} {\e{m}} {\overrightarrow{\e{a}}}}
        {\v{}}
      }

\infer [B-New]
{ 
  \overrightarrow{
  \opsem {\openv{}}
         {\e{i}}
         {\v{i}}
       }
         \\\\
         \newjava {\classhint{1}}
                  {\overrightarrow{\classhint{i}}}
                  {\overrightarrow{\v{i}}}
                  {\v{}}
       }
{ \opsem {\openv{}}
         {\newstaticexp {\overrightarrow{\classhint{i}}} {\classhint{1}} 
                        {\class{}} {\overrightarrow{\e{i}}}}
         {\v{}}}

       \BDefMulti{}

       \BDefMethod{}

       \BIsA{}

\infer [B-Assoc]
{\opsem {\openv{}}
        {\e{m}} {\curlymap{\overrightarrow{({\v{a}}\ {\v{b}})}}}
        \\
 \opsem {\openv{}} {\e{k}} {\k{}}
        \\\\
 \opsem {\openv{}} {\e{v}} {\v{v}}
}
{
 \opsem {\openv{}}
        {\assocexp {\e{m}} {\e{k}} {\e{v}}} 
        {\extendmap{\curlymap{\overrightarrow{({\v{a}}\ {\v{b}})}}}
                {\k{}}{\v{v}}}
                }

\infer [B-Get]
{\opsem {\openv{}}
        {\e{m}} {\curlymap{\overrightarrow{({\v{a}}\ {\v{b}})}}}
        \\\\
 \opsem {\openv{}}
        {\e{k}} {\k{}}
        \\\\
 \keyinmap{\k{}}{\curlymap{\overrightarrow{({\v{a}}\ {\v{b}})}}}
        \\\\
 \getmap{\curlymap{\overrightarrow{({\v{a}}\ {\v{b}})}}}
        {\k{}}
        =
        {\v{}}
}
{
 \opsem {\openv{}}
        {\getexp {\e{m}} {\e{k}}}
        {\v{}}
}

\infer [B-GetMissing]
{\opsem {\openv{}}
        {\e{m}} {\curlymap{\overrightarrow{({\v{a}}\ {\v{b}})}}}
        \\\\
 \opsem {\openv{}}
        {\e{k}} {\k{}}
        \\\\
 \keynotinmap{\k{}}{\curlymap{\overrightarrow{({\v{a}}\ {\v{b}})}}}
}
{
 \opsem {\openv{}}
        {\getexp {\e{m}} {\e{k}}}
        {\nil{}}
}

\end{mathpar}
\caption{Operational Semantics}
\end{figure*}

\begin{figure*}
\begin{mathpar}

\infer [BS-MethodRefl]
{}
{\opsem {\openv{}} {\methodexp {mth} {\e{}} {\overrightarrow{\e{}}}}
        {\wrong{}}}

\infer [BS-FieldRefl]
{}
{\opsem {\openv{}} {\fieldexp {\fld{}} {\e{}}}
        {\wrong{}}}

\infer [BS-NewRefl]
{}
{\opsem {\openv{}} {\fieldexp {\fld{}} {\e{}}}
        {\wrong{}}}


\infer [BS-Beta]
{ \opsem {\openv{}}
         {\e{f}}
         {\v{}}
         \\\\
  {\v{}} \not= {\const{}}
  \\
  {\v{}} \not= {\multi {\v{d}} {\disptable{}}}
  \\\\
  {\v{}} \not= {\closure {\openv{c}} {\abs {\x{}} {\t{}} {\e{b}}}}
       }
{ \opsem {\openv{}}
         {\appexp {\e{f}} {\e{a}}}
         {\wrong{}}
       }

\infer [BS-BetaMulti]
{ \opsem {\openv{}}
         {\e{f}}
         {\multi {\v{}} {\disptable{}}}
         \\\\
  {\v{}} \not= {\const{}}
  \\
  {\v{}} \not= {\multi {\v{d}} {\disptable{}}}
  \\\\
  {\v{}} \not= {\closure {\openv{c}} {\abs {\x{}} {\t{}} {\e{b}}}}
       }
{ \opsem {\openv{}}
         {\appexp {\e{f}} {\e{a}}}
         {\wrong{}}
       }

\infer [BS-FieldTarget]
{ \opsem {\openv{}}
         {\e{}} 
       {\v{1}}
         \\\\
         {\v{}} \not= {\classvalue{\classhint{1}} {\overrightarrow {\classfieldpair{\fld{i}} {\v{i}}}}}
       }
{ \opsem {\openv{}}
         {\fieldstaticexp {\classhint{1}} {\classhint{2}} {\fld{}} {\e{}}}
         {\wrong{}}
   }

\infer [BS-FieldMissing]
{ \opsem {\openv{}}
         {\e{}} 
       {\classvalue{\classhint{1}} {\overrightarrow {\classfieldpair{\fld{i}} {\v{i}}}}}
       \\
       \fld{} \not\in \{\overrightarrow{\fld{i}}\}
       }
{ \opsem {\openv{}}
         {\fieldstaticexp {\classhint{1}} {\classhint{2}} {\fld{}} {\e{}}}
         {\wrong{}}
   }


\infer [BS-MethodTarget]
{ \opsem {\openv{}}
         {\e{m}}
         {\v{}}
  \\
         {\v{}} \not= {\classvalue{\classhint{1}} {\overrightarrow {\classfieldpair{\fld{i}} {\v{i}}}}}
}
{\opsem {\openv{}}
        {\methodstaticexp {\classhint{1}} {\overrightarrow{\classhint{a}}} {\classhint{2}} {mth} {\e{m}} {\overrightarrow{\e{a}}}}
        {\wrong{}}
      }

\infer [BS-MethodArity]
{ i \not= a
}
{\opsem {\openv{}}
        {\methodstaticexp {\classhint{1}} {\overrightarrow{\classhint{i}}} {\classhint{2}} {mth} {\e{m}} {\overrightarrow{\e{a}}}}
        {\wrong{}}
      }

\infer [BS-MethodArg]
{ \opsem {\openv{}}
         {\e{m}}
         {\v{m}}
  \\
  \overrightarrow{
  \opsem {\openv{}}
         {\e{a}}
         {\v{a}}
       }
       \\\\
  \exists a.\ 
    \v{a} \not=\ {\classvalue{\classhint{a}} {\overrightarrow {\classfieldpair{\fld{i}} {\v{i}}}}}\ or\ \v{a} \not= \nil{}
}
{\opsem {\openv{}}
        {\methodstaticexp {\classhint{1}} {\overrightarrow{\classhint{a}}} {\classhint{2}} {mth} {\e{m}} {\overrightarrow{\e{a}}}}
        {\wrong{}}
      }

\infer [BS-NewArg]
{ \overrightarrow{
  \opsem {\openv{}}
         {\e{i}}
         {\v{i}}
     }
       \\\\
  \exists i.\ 
    \v{i} \not=\ {\classvalue{\classhint{i}} {\overrightarrow {\classfieldpair{\fld{i}} {\v{i}}}}}\ or\ \v{i} \not= \nil{}
}
{\opsem {\openv{}}
        {\newstaticexp {\overrightarrow{\classhint{i}}} {\classhint{1}} 
                       {\class{}} {\overrightarrow{\e{i}}}}
        {\wrong{}}
      }

\infer [BS-NewArity]
{ i \not= a
}
{\opsem {\openv{}}
        {\newstaticexp {\overrightarrow{\classhint{i}}} {\classhint{1}} 
                       {\class{}} {\overrightarrow{\e{a}}}}
        {\wrong{}}
      }

\infer [BS-AssocMap]
{\opsem {\openv{}}
        {\e{m}} {\v{}}
        \\
        \v{} \not= {\curlymap{\overrightarrow{({\v{a}}\ {\v{b}})}}}
}
{
 \opsem {\openv{}}
        {\assocexp {\e{m}} {\e{k}} {\e{v}}} 
        {\wrong{}}
                }

\infer [BS-AssocKey]
{\opsem {\openv{}}
        {\e{m}} {\curlymap{\overrightarrow{({\v{a}}\ {\v{b}})}}}
        \\
 \opsem {\openv{}} {\e{k}} {\v{k}}
 \\\\
 {\v{k}} \not= \k{}
}
{
 \opsem {\openv{}}
        {\assocexp {\e{m}} {\e{k}} {\e{v}}} 
        {\wrong{}}
                }

\infer [BS-GetMap]
{ \opsem {\openv{}}
         {\e{m}} {\v{}}
        \\
        \v{} \not= {\curlymap{\overrightarrow{({\v{a}}\ {\v{b}})}}}
}
{\opsem {\openv{}}
        {\getexp {\e{m}} {\e{k}}}
        {\wrong{}}
}

\infer [BS-GetKey]
{ \opsem {\openv{}}
         {\e{m}} {\v{}}
        \\
 \opsem {\openv{}}
        {\e{k}} {\v{k}}
        \\\\
      \v{} \not= {\k{}}
}
{\opsem {\openv{}}
        {\getexp {\e{m}} {\e{k}}}
        {\wrong{}}
}

\infer [BS-Local]
{ \notinopenv {\openv{}} {\x{}}}
{ \opsem {\openv{}} {\x{}} {\wrong{}} }

\infer [BS-DefMethod]
{ \opsem {\openv{}}
         {\e{m}}
         {\v{m}}
         \\
         \v{m} \not= {\multi {\v{d}} {\disptable{}}}
}
{\opsem {\openv{}}
        {\extendmultiexp {\e{m}} {\e{v}} {\e{f}}}
        {\wrong{}}
      }

\end{mathpar}
\caption{Stuck programs}
\end{figure*}

\begin{figure*}
\begin{mathpar}
\infer [BE-ErrorWrong]
{}
{ \opsem {\openv{}} 
         {\wrongorerror{}}
         {\wrongorerror{}}}

\infer [BE-Let]
{ \opsem {\openv{}} {\e{a}} {\wrongorerror{}}
 }
{ \opsem {\openv{}} 
         {\letexp {\x{}} {\e{a}} {\e{}}}
       {\wrongorerror{}}}

\infer [BE-Do1]
{ \opsem {\openv{}} {\e{1}} {\wrongorerror{}} }
{ \opsem {\openv{}} {\doexp{\e{1}}{\e{}}} {\wrongorerror{}}}

\infer [BE-Do2]
{ \opsem {\openv{}} {\e{1}} {\v{1}} 
  \\\\
  \opsem {\openv{}} {\e{}}  {\wrongorerror{}}
}
{ \opsem {\openv{}} {\doexp{\e{1}}{\e{}}} {\wrongorerror{}} }

\infer [BE-If]
{  \opsem {\openv{}} {\e{1}} {\wrongorerror{}}
}
{ \opsem {\openv{}}
         {\ifexp {\e1} {\e2} {\e3}}
         {\wrongorerror{}}
       }

\infer [BE-IfTrue]
{ \opsem {\openv{}} {\e{1}} {\v{1}}
  \\\\
  {\v{1}} \not= {\false{}}
  \\
  {\v{1}} \not= {\nil{}}
  \\\\
  \opsem {\openv{}} {\e{2}} {\wrongorerror{}}
}
{ \opsem {\openv{}}
         {\ifexp {\e1} {\e2} {\e3}}
         {\wrongorerror{}}
       }

\infer [BE-IfFalse]
{  \opsem {\openv{}} {\e{1}} {\false{}}
  \ \ \text{or}\ \ 
  \opsem {\openv{}} {\e{1}} {\nil{}}
  \\\\
  \opsem {\openv{}} {\e{3}} {\wrongorerror{}}
}
{ \opsem {\openv{}}
         {\ifexp {\e1} {\e2} {\e3}}
         {\wrongorerror{}}
       }

\infer [BE-Beta1]
{ \opsem {\openv{}}
         {\e{f}}
         {\wrongorerror{}}
       }
{ \opsem {\openv{}}
         {\appexp {\e{f}} {\e{a}}}
         {\wrongorerror{}}
       }

\infer [BE-Beta2]
{ \opsem {\openv{}}
         {\e{f}}
         {\v{f}}
         \\\\
  \opsem {\openv{}}
         {\e{a}}
         {\wrongorerror{}}
       }
{ \opsem {\openv{}}
         {\appexp {\e{f}} {\e{a}}}
         {\wrongorerror{}}
       }

\infer [BE-BetaClosure]
{ \opsem {\openv{}}
         {\e{f}}
         {\closure {\openv{c}} {\abs {\x{}} {\t{}} {\e{b}}}}
         \\\\
  \opsem {\openv{}}
         {\e{a}}
         {\v{a}}
         \\\\
  \opsem {\extendopenv {\openv{c}} {\x{}} {\v{a}}}
         {\e{b}}
         {\wrongorerror{}}
       }
{ \opsem {\openv{}}
         {\appexp {\e{f}} {\e{a}}}
         {\wrongorerror{}}
       }

\infer [BE-BetaMulti1]
{ \opsem {\openv{}}
         {\e{f}}
         {\multi {\v{d}} {m}}
         \\\\
  \opsem {\openv{}}
         {\e{a}}
         {\v{a}}
         \\\\
  \opsem {\openv{}}
         {\appexp {\v{d}} {\v{a}}}
         {\wrongorerror{}}
       }
{ \opsem {\openv{}}
         {\appexp {\e{f}} {\e{a}}}
         {\wrongorerror{}}
       }

\infer [BE-BetaMulti2]
{ \opsem {\openv{}}
         {\e{f}}
         {\multi {\v{d}} {m}}
         \\\\
  \opsem {\openv{}}
         {\e{a}}
         {\v{a}}
         \\\\
  \opsem {\openv{}}
         {\appexp {\v{d}} {\v{a}}}
         {\v{e}}
         \\\\
  \getmethod {\disptable{}}
             {\v{e}}
             {\errorvalv{}}
       }
{ \opsem {\openv{}}
         {\appexp {\e{f}} {\e{a}}}
         {\errorvalv{}}
       }

\infer [BE-Delta]
{ \opsem {\openv{}} {\e{}} {\const{}}
  \\\\
  \opsem {\openv{}} {\ep{}} {\v{}}
  \\\\
  \constantopsem{\const{}}{\v{}} = \wrongorerror{}
}
{ \opsem {\openv{}}
         {\appexp {\e{}} {\ep{}}}
         {\wrongorerror{}}
       }

\infer [BE-Field]
{ \opsem {\openv{}}
         {\e{}} 
         {\wrongorerror{}}
       }
{ \opsem {\openv{}}
         {\fieldstaticexp {\classhint{1}} {\classhint{2}} {\fld{}} {\e{}}}
         {\wrongorerror{}}
   }

\infer [BE-Method1]
{ \opsem {\openv{}}
         {\e{m}}
         {\wrongorerror{}}
}
{\opsem {\openv{}}
        {\methodstaticexp {\classhint{1}} {\overrightarrow{\classhint{a}}} {\classhint{2}} {mth} {\e{m}} {\overrightarrow{\e{}}}}
        {\wrongorerror{}}
      }

\infer [BE-Method2]
{ \opsem {\openv{}}
         {\e{m}}
         {\v{m}}
  \\\\
  \overrightarrow{
  \opsem {\openv{}}
         {\e{n-1}}
         {\v{n-1}}
       }
         \\\\
  \opsem {\openv{}}
         {\e{n}}
         {\wrongorerror{}}
}
{\opsem {\openv{}}
        {\methodstaticexp {\classhint{1}} {\overrightarrow{\classhint{a}}} {\classhint{2}} {mth} {\e{m}} {\overrightarrow{\e{}}}}
        {\wrongorerror{}}
      }

\infer [BE-Method3]
{ \opsem {\openv{}}
         {\e{m}}
         {\v{m}}
  \\
  \overrightarrow{
  \opsem {\openv{}}
         {\e{a}}
         {\v{a}}
       }
  \\\\
  \invokejavamethod {\classhint{1}} {\v{m}} {mth}
                    {\overrightarrow{\classhint{a}}} {\overrightarrow{\v{a}}}
                    {\classhint{2}}
                    {\errorvalv{}}
}
{\opsem {\openv{}}
        {\methodstaticexp {\classhint{1}} {\overrightarrow{\classhint{a}}} {\classhint{2}} {mth} {\e{m}} {\overrightarrow{\e{a}}}}
        {\errorvalv{}}
      }

\infer [BE-New1]
{ \overrightarrow{
  \opsem {\openv{}}
         {\e{n-1}}
         {\v{n-1}}
       }
       \\\\
  \opsem {\openv{}}
         {\e{n}}
         {\wrongorerror{}}
       }
{ \opsem {\openv{}}
         {\newstaticexp {\overrightarrow{\classhint{i}}} {\classhint{1}} 
                        {\class{}} {\overrightarrow{\e{}}}}
         {\wrongorerror{}}
       }

\infer [BE-New2]
{ 
  \overrightarrow{
  \opsem {\openv{}}
         {\e{i}}
         {\v{i}}
       }
         \\\\
         \newjava {\classhint{1}}
                  {\overrightarrow{\classhint{i}}}
                  {\overrightarrow{\v{i}}}
                  {\errorvalv{}}
       }
{ \opsem {\openv{}}
         {\newstaticexp {\overrightarrow{\classhint{i}}} {\classhint{1}} 
                        {\class{}} {\overrightarrow{\e{i}}}}
         {\errorvalv{}}}

\infer [BE-DefMulti]
{ \opsem {\openv{}} {\e{d}} {\wrongorerror{}}
}
{\opsem {\openv{}}
        {\createmultiexp {\t{}}
                         {\e{d}}}
        {\wrongorerror{}}
}

\infer [BE-DefMethod1]
{ \opsem {\openv{}}
         {\e{m}}
         {\wrongorerror{}}
}
{\opsem {\openv{}}
        {\extendmultiexp {\e{m}} {\e{v}} {\e{f}}}
        {\wrongorerror{}}
      }

\infer [BE-DefMethod2]
{ \opsem {\openv{}}
         {\e{m}}
         {\multi {\v{d}} {\disptable{}}}
         \\\\
  \opsem {\openv{}}
         {\e{v}}
         {\wrongorerror{}}
}
{\opsem {\openv{}}
        {\extendmultiexp {\e{m}} {\e{v}} {\e{f}}}
        {\wrongorerror{}}
      }

\infer [BE-DefMethod3]
{ \opsem {\openv{}}
         {\e{m}}
         {\multi {\v{d}} {\disptable{}}}
         \\\\
  \opsem {\openv{}}
         {\e{v}}
         {\v{v}}
         \\\\
  \opsem {\openv{}}
         {\e{f}}
         {\wrongorerror{}}
}
{\opsem {\openv{}}
        {\extendmultiexp {\e{m}} {\e{v}} {\e{f}}}
         {\wrongorerror{}}
      }

\infer [BE-IsA1]
{ \opsem {\openv{}} {\e{1}} {\wrongorerror{}}
}
{\opsem {\openv{}} {\isaapp {\e{1}} {\e{2}}} {\wrongorerror{}}}

\infer [BE-IsA2]
{ \opsem {\openv{}} {\e{1}} {\v{1}}
  \\\\
  \opsem {\openv{}} {\e{2}} {\wrongorerror{}}
}
{\opsem {\openv{}} {\isaapp {\e{1}} {\e{2}}} {\wrongorerror{}}}

\infer [BE-Assoc1]
{\opsem {\openv{}}
        {\e{m}}{\wrongorerror{}} 
}
{
 \opsem {\openv{}}
        {\assocexp {\e{m}} {\e{k}} {\e{v}}} 
        {\wrongorerror{}}
                }

\infer [BE-Assoc2]
{\opsem {\openv{}}
        {\e{m}} {\curlymap{\overrightarrow{({\v{a}}\ {\v{b}})}}}
        \\
 \opsem {\openv{}}
        {\e{k}}{\wrongorerror{}}
}
{
 \opsem {\openv{}}
        {\assocexp {\e{m}} {\e{k}} {\e{v}}} 
        {\wrongorerror{}}
                }

\infer [BE-Assoc3]
{\opsem {\openv{}}
        {\e{m}} {\curlymap{\overrightarrow{({\v{a}}\ {\v{b}})}}}
        \\
 \opsem {\openv{}}
        {\e{k}} {\v{k}}
        \\
 \opsem {\openv{}}
        {\e{v}} {\wrongorerror{}}
}
{
 \opsem {\openv{}}
        {\assocexp {\e{m}} {\e{k}} {\e{v}}} 
        {\wrongorerror{}}
                }

\infer [BE-Get1]
{\opsem {\openv{}}
        {\e{m}} {\wrongorerror{}}
}
{
 \opsem {\openv{}}
        {\getexp {\e{m}} {\e{k}}}
        {\wrongorerror{}}
}

\infer [BE-Get2]
{\opsem {\openv{}}
        {\e{m}} {\curlymap{\overrightarrow{({\v{a}}\ {\v{b}})}}}
        \\
 \opsem {\openv{}}
        {\e{k}} {\wrongorerror{}}
}
{
 \opsem {\openv{}}
        {\getexp {\e{m}} {\e{k}}}
        {\wrongorerror{}}
}
\end{mathpar}
\caption{Error and stuck propagation}
\end{figure*}





\begin{figure*}
\begin{mathpar}

\begin{array}{lllll}
  {\openv{}}({\x{}}) &=& {\v{}}  & {\roundpair{\x{}}{\v{}}} \in \openv{}\\
  {\openv{}}({\path {\keype{k}} {\object{}}}) &=& {\getexp {{\openv{}}(\object{})}{\k{}}}\\
  {\openv{}}({\path {\classpe{}} {\object{}}}) &=& {\appexp {\classconst{}} {{\openv{}}(\object{})}}

\end{array}

\end{mathpar}
\caption{Path translation}
\end{figure*}

\begin{figure*}
\begin{mathpar}

\begin{array}{lllll}
\update{\HMapgeneric {\mandatory} {\absent}}{\propisnotmeta{}}{\destructpath {\pathelem{}} {\keype{\k{}}}}
&=&
{\HMapgeneric {\extendmandatoryset{\mandatory}{\k{}}{\update{\t{}}{\propisnotmeta{}}{\pathelem{}}}}{\absent}}
& {\inmandatory{\k{}}{\t{}}{\mandatory{}}}
%& Present Key
\\
\update{\HMapgeneric {\mandatory} {\absent}}{\t{}}{\destructpath {\pathelem{}} {\keype{\k{}}}}
&=&
{\Bottom{}}
& {\notsubtypein {} {\Nil{}} {\t{}}}\ \text{and}\ {\inabsent{\k{}}{\absent{}}}
%& Bad absent +
\\
\update{\HMapgeneric {\mandatory} {\absent}}{\nottype{\t{}}}{\destructpath {\pathelem{}} {\keype{\k{}}}}
&=&
{\Bottom{}}
& {\issubtypein {} {\Nil{}} {\t{}}}\ \text{and}\ {\inabsent{\k{}}{\absent{}}}
%& Bad absent -
\\
\update{\HMapgeneric {\mandatory} {\absent}}{\propisnotmeta{}}{\destructpath {\pathelem{}} {\keype{\k{}}}}
&=&
{\HMapgeneric {\mandatory} {\absent}}
& {\inabsent{\k{}}{\absent{}}}
%& Consistent absent
\\
\update{\HMapp {\mandatory} {\absent}}{\t{}}{\destructpath {\pathelem{}} {\keype{\k{}}}}
&=&
{\Union {\HMapp {\extendmandatoryset {\mandatory} {\k{}}
                                     {\t{}}}
                {\absent}}
        {\HMapp {\mandatory} {\extendabsentset{\absent}{\k{}}}}}
& {\issubtypein {} {\Nil{}} {\t{}}},\ 
{\notinmandatory{\k{}}{\s{}}{\mandatory{}}}\ \text{and}\ {\notinabsent{\k{}}{\absent{}}}
\\
\update{\HMapp {\mandatory} {\absent}}{\propisnotmeta{}}{\destructpath {\pathelem{}} {\keype{\k{}}}}
&=&
{\HMapp {\extendmandatoryset {\mandatory} {\k{}}{\update{\Top{}}{\propisnotmeta{}}{\pathelem{}}}} {\absent}}
& 
{\notinmandatory{\k{}}{\s{}}{\mandatory{}}}\ \text{and}\ {\notinabsent{\k{}}{\absent{}}}

% ClassPE
\\
\update{\t{}}{\s{}}{\destructpath {\pathelem{}} {\classpe{}}}
&=& \restrict{\t{}}{\s{}}

\\
\update{\t{}}{\nottype{\s{}}}{\destructpath {\pathelem{}} {\classpe{}}}
&=& \remove{\t{}}{\s{}}

\\
\update{\t{}}{\s{}}{\emptypath{}}
&=&
\restrict{\t{}}{\s{}}
\\
\update{\t{}}{\nottype{\s{}}}{\emptypath{}}
&=&
\remove{\t{}}{\s{}}


\\\\

\restrict{\t{}} {\s{}} &=& \Bot{}
                       & \text{if} \not\exists \v{}.\  
                          {\judgement {} {\hastype {\v{}} {\t{}}} {\prop{1}}{\object{1}}}
                          \ \text{and}\ 
                          {\judgement {} {\hastype {\v{}} {\s{}}} {\prop{2}}{\object{2}}}
                          \\
\restrict{\Unionsplice{\overrightarrow{{\t{}}}}}{\s{}} &=& 
\Unionsplice{\overrightarrow{\restrict{{\t{}}}{\s{}}}}
\\
\restrict{\t{}}{\s{}} &=& {\t{}} & \text{if}\ {\issubtypein {} {\t{}} {\s{}}}
\\
\restrict{\t{}}{\s{}} &=& {\s{}} & \text{otherwise}
\\\\
\remove{\t{}}{\s{}} &=& {\Bot{}} & \text{if}\ {\issubtypein {} {\t{}} {\s{}}}
\\
\remove{\Unionsplice{\overrightarrow{{\t{}}}}}{\s{}} &=& 
\Unionsplice{\overrightarrow{\remove{{\t{}}}{\s{}}}}
\\
\remove{\t{}}{\s{}} &=& {\t{}} & \text{otherwise}


\end{array}

\end{mathpar}
\caption{Type Update}
\end{figure*}


\begin{figure*}
\begin{mathpar}
\infer [M-Or]
{ \satisfies{\propenv{}}{\prop{1}}\ \text{or}\  \satisfies{\propenv{}}{\prop{2}}}
{ \satisfies{\propenv{}}{\orprop{\prop{1}}{\prop{2}}}
                   }

\infer [M-Imp]
{ \satisfies{\propenv{}}{\prop{1}}\ \text{implies}\ \satisfies{\propenv{}}{\prop{2}}}
{ \satisfies{\propenv{}}{\impprop{\prop{1}}{\prop{2}}}
                   }

\infer [M-And]
{ \satisfies{\propenv{}}{\prop{1}}
\\ \satisfies{\propenv{}}{\prop{2}}}
{ \satisfies{\propenv{}}{\andprop{\prop{1}}{\prop{2}}}
                   }


\infer [M-Top]
{}
{ \satisfies{\propenv{}}{\topprop{}}
                   }

                   \\

\infer [M-Type]
{ \judgement {} {\openv{}({\path{\pathelem{}}{\x{}}})} {\t{}}{\filterset{\thenprop{\prop{}}}{\elseprop{\prop{}}}}{\object{}}}
{ \satisfies{\propenv{}}{\isprop{\t{}}{\path{\pathelem{}}{\x{}}}}
                   }

\infer [M-NotType]
{ \judgement {} {\openv{}({\path{\pathelem{}}{\x{}}})} {\s{}}{\filterset{\thenprop{\prop{}}}{\elseprop{\prop{}}}}{\object{}}
\\\\
\text{there is no}\ \v{}\ \text{such that}\ \judgement{}{\v{}}{\t{}}{\filterset{\thenprop{\prop{1}}}{\elseprop{\prop{1}}}}{\object{1}}
\ \text{and}\ \judgement{}{\v{}}{\s{}}{\filterset{\thenprop{\prop{2}}}{\elseprop{\prop{2}}}}{\object{2}}
}
{ \satisfies{\propenv{}}{\notprop{\t{}}{\path{\pathelem{}}{\x{}}}}
                   }
\end{mathpar}
\caption{Satisfaction Relation}
\end{figure*}

\begin{figure*}
\begin{mathpar}
\infer [L-Atom]
{ {\prop{}} \in {\propenv{}}}
{ \inpropenv {\propenv{}} {\prop{}}
}

\infer [L-True]
{}
{ \inpropenv {\propenv{}} {\topprop{}}}

\infer [L-False]
{ \inpropenv {\propenv{}} {\botprop{}}}
{ \inpropenv {\propenv{}} {\prop{}}}

\infer [L-AndI]
{ \inpropenv {\propenv{}} {\prop{1}}
  \\\\
  \inpropenv {\propenv{}} {\prop{2}}}
{ \inpropenv {\propenv{}} {\andprop {\prop{1}}{\prop{2}}}}

\infer [L-AndE]
{ \inpropenv {\propenv{}, {\prop{1}}} {\prop{}}
  \ \text{or}\ 
  \inpropenv {\propenv{}, {\prop{2}}} {\prop{}}}
{ \inpropenv {\propenv{}, {\andprop {\prop{1}}{\prop{2}}}} {\prop{}}}

\infer [L-ImplI]
{ \inpropenv {\propenv{}, {\prop{1}}} {\prop{2}}}
{ \inpropenv {\propenv{}} {\impprop {\prop{1}}{\prop{2}}}}

\infer [L-ImplE]
{ \inpropenv {\propenv{}} {\prop{1}}
  \\\\
  \inpropenv {\propenv{}} {\impprop {\prop{1}}{\prop{2}}}}
{ \inpropenv {\propenv{}} {\prop{2}}}

\infer [L-OrI]
{ \inpropenv {\propenv{}} {\prop{1}}
  \ \text{or}\ 
  \inpropenv {\propenv{}} {\prop{2}}}
{ \inpropenv {\propenv{}} {\orprop {\prop{1}}{\prop{2}}}}


\infer [L-OrE]
{ \inpropenv {\propenv{}, {\prop{1}}}{\prop{}}
  \\\\
  \inpropenv {\propenv{}, {\prop{2}}}{\prop{}}}
{ \inpropenv {\propenv{}, {\orprop {\prop{1}}{\prop{2}}}}{\prop{}}}

\infer [L-Sub]
{ \inpropenv {\propenv{}} {\isprop {\t{}}{\path {\pathelem{}} {\x{}}}}
  \\
  \issubtypein {} {\t{}}{\s{}}
}
{ \inpropenv {\propenv{}} {\isprop {\s{}}{\path {\pathelem{}} {\x{}}}}}

\infer [L-SubNot]
{ \inpropenv {\propenv{}} {\notprop {\s{}}{\path {\pathelem{}} {\x{}}}}
  \\
  \issubtypein {} {\t{}}{\s{}}}
{ \inpropenv {\propenv{}} {\notprop {\t{}}{\path {\pathelem{}} {\x{}}}}}

\infer [L-Bot]
{ \inpropenv {\propenv{}} {\isprop {\Bot} {\path {\pathelem{}} {\x{}}}}}
{ \inpropenv {\propenv{}} {\prop{}}}

{\LUpdate}

\\

\text{(The metavariable \propisnotmeta{} ranges over \t{} and \nottype{\t{}} (without variables).)}

\end{mathpar}
\caption{Proof System}
\label{appendix:figure:proofsystem}
\end{figure*}

\begin{figure*}
$$
\begin{array}{lclr}

{\withpolarity
  {\replacefor
    {\filterset {\thenprop {\prop{}}}{\elseprop {\prop{}}}}
    {\object{}}
    {\x{}}}
  {\polaritymeta{}}}
  &=&
{\filterset 
  {\withpolarity
    {\replacefor
      {\thenprop {\prop{}}}
      {\object{}}
      {\x{}}}
    {\polaritymeta{}}}
  {\withpolarity
    {\replacefor
      {\elseprop {\prop{}}}
      {\object{}}
      {\x{}}}
    {\polaritymeta{}}}}
\\\\
{\withpolarity
  {\replacefor
    {\isprop {\propisnotmeta{}} {\path {\pathelem{}} {\x{}}}}
    {\path {\pathelemp{}} {\y{}}}
    {\x{}}}
  {\polaritymeta{}}}
&=&
  {\isprop {({\withpolarity
              {\replacefor
               {\propisnotmeta{}}
               {\path {\pathelemp{}} {\y{}}}
               {\x{}}}
              {\polaritymeta{}}})}
           {{\pathelem{}}({\path {\pathelemp{}} {\y{}}})}}
           \\

{\pluspolarity
{\replacefor
  {\isprop {\propisnotmeta{}} {\path {\pathelem{}} {\x{}}}}
  {\emptyobject{}}
  {\x{}}}
}
&=&
{\topprop{}}
\\
{\minuspolarity
{\replacefor
  {\isprop {\propisnotmeta{}} {\path {\pathelem{}} {\x{}}}}
  {\emptyobject{}}
  {\x{}}}
}
&=&
{\botprop{}}

\\
{\withpolarity
{\replacefor
  {\isprop {\propisnotmeta{}} {\path {\pathelem{}} {\x{}}}}
  {\object{}}
  {\z{}}}
{\polaritymeta{}}}
&=&
  {\isprop {\propisnotmeta{}} {\path {\pathelem{}} {\x{}}}}
  & \x{} \not= \z{}\ \text{and}\ \z{} \not\in {\fv {\propisnotmeta{}}}

\\
{\pluspolarity
{\replacefor
  {\isprop {\propisnotmeta{}} {\path {\pathelem{}} {\x{}}}}
  {\object{}}
  {\z{}}}
}
&=&
{\topprop{}}
  & \x{} \not= \z{}\ \text{and}\ \z{} \in {\fv {\propisnotmeta{}}}
\\
{\minuspolarity
{\replacefor
  {\isprop {\propisnotmeta{}} {\path {\pathelem{}} {\x{}}}}
  {\object{}}
  {\z{}}}
}
&=&
{\botprop{}}
  & \x{} \not= \z{}\ \text{and}\ \z{} \in {\fv {\propisnotmeta{}}}

\\
{\withpolarity
{\replacefor
  {\topprop{}}
  {\object{}}
  {\x{}}}
{\polaritymeta{}}}
&=&
  {\topprop{}}

\\
{\withpolarity
{\replacefor
  {\botprop{}}
  {\object{}}
  {\x{}}}
{\polaritymeta{}}}
&=&
  {\botprop{}}

\\
{\pluspolarity
{\replacefor
  {({\impprop {\prop{1}} {\prop{2}}})}
  {\object{}}
  {\x{}}}
}
&=&
{\impprop 
  {\minuspolarity {\replacefor {\prop{1}} {\object{}} {\x{}}}}
  {\pluspolarity {\replacefor {\prop{2}} {\object{}} {\x{}}}}}
\\
{\minuspolarity
{\replacefor
  {({\impprop {\prop{1}} {\prop{2}}})}
  {\object{}}
  {\x{}}}
}
&=&
{\impprop 
  {\pluspolarity {\replacefor {\prop{1}} {\object{}} {\x{}}}}
  {\minuspolarity {\replacefor {\prop{2}} {\object{}} {\x{}}}}}
\\
{\withpolarity
{\replacefor
  {({\orprop {\prop{1}} {\prop{2}}})}
  {\object{}}
  {\x{}}}
{\polaritymeta{}}}
&=&
{\orprop 
  {\withpolarity
    {\replacefor {\prop{1}} {\object{}} {\x{}}}
    {\polaritymeta{}}}
  {\withpolarity
    {\replacefor {\prop{2}} {\object{}} {\x{}}}
    {\polaritymeta{}}}}
\\
{\withpolarity
{\replacefor
  {({\andprop {\prop{1}} {\prop{2}}})}
  {\object{}}
  {\x{}}}
{\polaritymeta{}}}
&=&
{\andprop 
{\withpolarity
  {\replacefor {\prop{1}} {\object{}} {\x{}}}
  {\polaritymeta{}}}
{\withpolarity
  {\replacefor {\prop{2}} {\object{}} {\x{}}}
  {\polaritymeta{}}}}

    \\\\

{\withpolarity
{\replacefor
  {\path {\pathelem{}} {\x{}}}
  {\path {\pathelemp{}} {\y{}}}
  {\x{}}}
{\polaritymeta{}}}
           &=&
{\path{\pathelem{}}{\path {\pathelemp{}} {\y{}}}}

    \\

{\withpolarity
{\replacefor
  {\path {\pathelem{}} {\x{}}}
  {\emptyobject{}}
  {\x{}}}
{\polaritymeta{}}}
           &=&
{\emptyobject{}}

    \\

{\withpolarity
{\replacefor
  {\path {\pathelem{}} {\x{}}}
  {\object{}}
  {\z{}}}
{\polaritymeta{}}}
           &=&
{\path {\pathelem{}} {\x{}}}

& \x{} \not= \z{}
    \\

{\withpolarity
{\replacefor
  {\emptyobject{}}
  {\object{}}
  {\x{}}}
{\polaritymeta{}}}
           &=&
{\emptyobject{}}

\end{array}
$$
\center{\text{Substitution on types is capture-avoiding structural recursion.}}
\caption{Substitution}
\end{figure*}



\end{document}

%                       Revision History
%                       -------- -------
%  Date         Person  Ver.    Change
%  ----         ------  ----    ------

%  2013.06.29   TU      0.1--4  comments on permission/copyright notices

