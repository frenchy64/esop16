%-----------------------------------------------------------------------------
%
%               Template for sigplanconf LaTeX Class
%
% Name:         sigplanconf-template.tex
%
% Purpose:      A template for sigplanconf.cls, which is a LaTeX 2e class
%               file for SIGPLAN conference proceedings.
%
% Guide:        Refer to "Author's Guide to the ACM SIGPLAN Class,"
%               sigplanconf-guide.pdf
%
% Author:       Paul C. Anagnostopoulos
%               Windfall Software
%               978 371-2316
%               paul@windfall.com
%
% Created:      15 February 2005
%
%-----------------------------------------------------------------------------


\documentclass[preprint,9pt,twocolumn,nocopyrightspace,authoryear]{sigplanconf}

% The following \documentclass options may be useful:

% preprint      Remove this option only once the paper is in final form.
% 10pt          To set in 10-point type instead of 9-point.
% 11pt          To set in 11-point type instead of 9-point.
% authoryear    To obtain author/year citation style instead of numeric.

\usepackage{common}
\usepackage{minted}
\include{bibliography.bib}
\usepackage{chngcntr}

\newcounter{ex}
\newenvironment{exmp}{
\refstepcounter{ex}%
\par\vspace{.8em}\hfill\framebox{\small Example~\theex}\nopagebreak
\\[-2.4em]\nopagebreak}
{\relax}


\begin{document}

\special{papersize=8.5in,11in}
\setlength{\pdfpageheight}{\paperheight}
\setlength{\pdfpagewidth}{\paperwidth}

\conferenceinfo{CONF 'yy}{Month d--d, 20yy, City, ST, Country} 
\copyrightyear{20yy} 
\copyrightdata{978-1-nnnn-nnnn-n/yy/mm} 
\doi{nnnnnnn.nnnnnnn}

\newcommand{\clj}[1]{\mintinline{clojure}{#1}}
\newcommand{\java}[1]{\mintinline{java}{#1}}
\newcommand{\rkt}[1]{\mintinline{racket}{#1}}

% Uncomment one of the following two, if you are not going for the 
% traditional copyright transfer agreement.

%\exclusivelicense                % ACM gets exclusive license to publish, 
                                  % you retain copyright

%\permissiontopublish             % ACM gets nonexclusive license to publish
                                  % (paid open-access papers, 
                                  % short abstracts)

%\titlebanner{banner above paper title}        % These are ignored unless
%\preprintfooter{short description of paper}   % 'preprint' option specified.

\title{Practical Optional Types for Clojure}

\authorinfo{Ambrose Bonnaire-Sergeant}
           {Indiana University}
           {abonnair@indiana.edu}
\authorinfo{Rowan Davies}
           {University of Western Australia}
           {rowan.davies@uwa.edu.au}
\authorinfo{Sam Tobin-Hochstadt}
           {Indiana University}
           {samth@indiana.edu}

\maketitle

\begin{abstract}
Typed Clojure is an optional type system for Clojure, a dynamic
language in the Lisp family that targets the JVM. Typed Clojure's type
system build on the design of Typed Racket, repurposing in particular
\emph{occurrence typing}, its approach to statically reasoning about
predicate tests. However, in adapting the type system to Clojure,
changes and extensions are required to accommodate both additional
language features and  idioms used by Clojure programmers.

In this paper, we describe Typed Clojure and present these type system
extensions, focusing on three features widely used in Clojure. First, Java
interoperability is central to Clojure's mission but introduces
challenges such as ubiquitous \texttt{null}; Typed Clojure handles
Java interoperability while ensuring the absence of null-pointer
exceptions in typed programs. Second, Clojure programmers
idiomatically use immutable dictionaries for data structures; Typed
Clojure handles this in the type system with multiple forms of
heterogeneous dictionary types. Third, multimethods provide extensible
operations, and their Clojure semantics turns out to have a surprising
synergy with the underlying occurrence typing framework.

We provide a formal model of the Typed Clojure type system
incorporating these and other features, with a proof of
soundness. Additionally, Typed Clojure is now in use by numerous
corporations and developers working with Clojure, and we report on 
experience with the system and its lessons for the future.


%% Clojure is a dynamically typed language hosted on the Java 
%% Virtual Machine.
%% Typed Racket is a valuable starting point for
%% a gradual type system that targets Clojure.
%% Building a similar type system for a new language gives the
%% designer some flexibility to repurpose and extend features.
%% This paper gives an overview of Typed Clojure, concentrating
%% on the extensions and differences from Typed Racket. We also
%% show where Typed Racket's features were particularly useful
%% for type checking non-trivial Clojure idioms.
\end{abstract}

%% \category{CR-number}{subcategory}{third-level}

%% \keywords
%% optional types, Clojure

\section{Clojure with static typing}

% current situation 

The popularity of dynamically-typed languages in software
development, combined with a recognition that types often improve
programmer productivity, software reliability, and performance, has
led to the recent development of a wide variety of optional and
gradual type systems aimed at checking existing programs written in
existing languages.  These include Microsoft's TypeScript for
JavaScript, Facebook's Hack for PHP and Flow for JavaScript, and MyPy
for Python among the optional systems, and Typed Racket, Reticulated
Python, and GradualTalk among gradually-typed systems.\footnote{We
  reserve the term ``gradual typing'' for systems such as Typed Racket which soundly
  interoperate between typed and untyped code; systems like Typed Clojure or TypeScript which do not
  enforce type invariants we describe as ``optionally typed''.}
%FIXME: add citations for all those systems.

One key lesson of these systems, indeed a lesson known to early
developers of optional type systems such as StrongTalk, is that type
systems for existing languages must be designed to work with the
features and idioms of the target language. Often this takes the form
of a core language, be it of functions or classes and objects,
together with extensions to handle distinctive language features.


We synthesize these lessons to present \emph{Typed Clojure}, an
optional type system for Clojure. Typed Clojure builds on the core
type checking approach of Typed Racket, an existing gradual type
system for Racket. However, Typed Clojure extends this basic framework
in multiple ways to accommodate the unique idioms and features of
Clojure, producing an expressive synthesis of ideas and demonstrating
a surprising coincidence between multiple dispatch in
Clojure and Typed Racket's  occurrence typing framework. 

The essence of Typed Clojure, of course, is Clojure, a dynamically
typed language in the Lisp family built to run on the Java Virtual
Machine (JVM) which has recently gained popularity as an alternative
JVM language.  It offers the flexibility of a Lisp dialect, including
macros, emphasizes a functional style via a
standard library of immutable data structures, and provides
interoperability with existing Java code, allowing programmers to use
existing Java libraries without leaving Clojure.
%
Since its initial release in 2007, Clojure has been widely adopted for
``backend'' development in places where its support for parallelism,
functional programming, and Lisp-influenced abstraction is desired on
the JVM. As a result, it now has an extensive base of existing untyped
programs, whose developers can now benefit from Typed Clojure. As
a result, Typed Clojure is used in industry, experience we discuss
in this paper.

\begin{figure}
\inputminted[firstline=5,lastline=7]{clojure}{code/demo/src/demo/parent2.clj}
\caption{A simple Typed Clojure program}
\label{fig:ex1}
\end{figure}

Figure~\ref{fig:ex1} presents a simple program demonstrating many
aspects of our system, from simple type annotations to explicit
handling of Java's \java{null} (written \clj{nil}) in interoperation, as well as an
extended form of occurrence typing and Clojure's \emph{type hints},
which are central to Typed Clojure's approach to interoperability. 

The \clj{parent} function has the type 
$$
\clj{['{:file (U nil File)} -> (U nil Str)]}
$$
which means that it takes a hash table whose \clj{:file} key maps to either
\clj{nil} or a \clj{File}, and it produces either \clj{nil} or a
\clj{String}. The \clj{parent} function uses the \clj{:file} keyword
as an accessor to get the file, checks that it isn't \clj{nil}, and
then obtains the parent by making a Java method call.
%
The annotation \clj{^File f} is a type hint on \clj{f}, which instructs the Clojure
compiler (running prior to Typed Clojure typechecking) to statically
resolve the \clj{getParent} 
call to \clj{File}'s \clj{getParent} method with signature \java{String getParent();}, rather than using reflection at runtime.

In the remainder of this paper, we describe how Typed Clojure's
central innovations, including Java interoperability, multimethods,
and heterogeneously-typed immutable maps, enable this example and many
others. We begin with an example-driven presentation of the main type
system features in \secref{sec:overview}. We then incrementally
present a core calculus for Typed Clojure covering all of these
features together in \secref{sec:formal} and prove type
soundness (\secref{sec:metatheory}). We then discuss the full
implementation of Typed Clojure, dubbed \coretyped{}, which extends
the formal model in many ways, and the experience gained from its use
in \secref{sec:experience}. Finally, we discuss related work and
conclude.


\section{Overview of Typed Clojure}

\label{sec:overview}

We now begin a tour of the central features of Typed Clojure,
beginning with Clojure itself. In our presentation, we will make 
use of the full Typed Clojure system to illustrate the key type system
ideas, before studying the key features in detail in
section~\ref{sec:formal}. 

\subsection{Clojure}

Clojure~\cite{Hic08} is a Lisp built to run on the
Java Virtual Machine with exemplary support for concurrent programming
and immutable data structures. It emphasizes mostly-functional
programming, restricting imperative updates to a limited set of
structures which have specific thread synchronization behaviour. By
default, it provides fast implementations of immutable lists, vectors,
and hash tables, which are used for most data structures, although it
also provides means for defining new records.

One of Clojure's primary advantages is easy interoperation with
existing Java libraries. It automatically generates appropriate JVM
bytecode to make Java method and constructor calls, and treats Java
values as any other Clojure value. However, this smooth
interoperability comes at the cost of pervasive \java{null}, which
leads to the possibility of null pointer exceptions---a drawback we
address in Typed Clojure.

\subsection{Typed Racket and occurrence typing}

\citet{TF10}
presented Typed Racket with occurrence typing,
an inference technique for conditional control flow.
They introduce the concept of occurrence typing 
with the following example.

\inputminted[firstline=1]{racket}{code/tr/example1.rkt}

This function takes a value that is either \emph{\#f} % mintinline really hates #
or a number, represented by an untagged \emph{union type}.
The `then' branch has an implicit invariant
that \rkt{x} is a number, which is automatically inferred with occurrence typing
and type checked without further annotations.
Here is same program in Typed Clojure.

\begin{exmp}
\inputminted[firstline=1]{clojure}{code/demo/src/demo/eg1.clj}
\label{example:conditionalflow}
\end{exmp}

Clojure uses \emph{namespaces}, declared with the
\clj{ns} form, to manage top-level \emph{vars}.
This is a regular Clojure file compiled with
the Clojure compiler except it declares a runtime dependency on
\clj{clojure.core.typed}, Typed Clojure's core namespace.
Programmers import typed versions of certain constructs as needed, like
a \clj{fn} variant that supports annotations,
and use the provided \clj{check-ns} function to type check the current namespace
when convenient.

Typed Clojure can check all the examples in the occurrence typing
paper---the rest of this section describes the extensions necessary
to check Clojure code.

\subsection{Exceptional control flow}

Along with conditional control flow,
Clojure programmers rely on \emph{exceptions}
to assert type-related invariants.

\begin{exmp}
\inputminted[firstline=13,lastline=15]{clojure}{code/demo/src/demo/do.clj}
\label{example:doexception}
\end{exmp}

We combine the sequencing form \clj{do} and a thrown exception with
\clj{throw} to avoid a null-pointer exception from attempting to increment \clj{nil}.
(We omit \clj{ns} forms for the rest of the examples)
\footnote{See <this> github repo for full working examples}.

To check this example, we note that
occurrence typing already provides
valuable information about the conditional expression---namely that it cannot return logically true and if it returns logically false 
then \clj{x} is a \clj{Number}. Equivalently, if it returns a value then \clj{x} is a \clj{Number}.
We use this assumption to check \clj{(inc x)} with an automatic
guarantee that a null-pointer exception is impossible
(modelled formally in section~\ref{sec:doformal} and proved
in section~\ref{sec:metatheory}).

\subsection{Heterogeneous hash-maps}

Hash-maps with keyword keys play a major role in Clojure programming.
HMap types model the most common usages of keyword maps.

\begin{Code}
\begin{exmp}
\inputminted[firstline=6,lastline=8]{clojure}{code/demo/src/demo/hmap.clj}
\inputminted[firstline=27,lastline=29]{clojure}{code/demo/src/demo/hmap.clj}
\end{exmp}
\end{Code}

Here we define a type abbreviation with \clj{defalias} called \clj{Expr}
that describes the structure of a recursively-defined AST as a union of HMaps.
\clj{an-exp} is a function verified to return an \clj{Expr}.

HMaps support various options to track the presence and absence of entries.
\clj{:mandatory} declares entries that must be present.

\begin{Code}
\begin{exmp}
\inputminted[firstline=58,lastline=64]{clojure}{code/demo/src/demo/hmap.clj}
\end{exmp}
\end{Code}
Looking up \clj{:a} simply returns the associated type \clj{Num}.
Keys other than \clj{:a} are unaccounted for in \clj{Mand}, so 
the best we can infer is \clj{Any}, the supertype of all types.
\clj{'{:a Num}} is shorthand for \clj{(HMap :mandatory {:a Num})}.

\clj{:absent-keys} is a set of keywords that are known to not have entries.
\begin{Code}
\begin{exmp}
\inputminted[firstline=66,lastline=73]{clojure}{code/demo/src/demo/hmap.clj}
\end{exmp}
\end{Code}
We can infer a better type for \clj{(:b m)}---\clj{:b} is absent in \clj{Abs}
so it looking up \clj{:b} returns \clj{nil}.

\clj{:optional} is a map that declares entries either absent \emph{or}
present with a specific type. 

\begin{Code}
\begin{exmp}
\inputminted[firstline=5,lastline=12]{clojure}{code/demo/src/demo/hmap_path.clj}
\end{exmp}
\end{Code}

Concretely \clj{(HMap :optional {:a Num})}
is the same as
\begin{minted}{clojure}
      (U '{:a Num} (HMap :absent-keys #{:a})).
\end{minted}


\clj{:complete?} is \clj{true} if there are no further entries
than those declared \clj{:mandatory} or \clj{:optional}.
\begin{Code}
\begin{exmp}
\inputminted[firstline=17,lastline=26]{clojure}{code/demo/src/demo/hmap_path.clj}
\end{exmp}
\end{Code}
\clj{:a} is \clj{:mandatory} in \clj{Comp} so lookups return \clj{Num}.
\clj{:b} is \clj{:optional} so lookups return \clj{(U nil Kw)}.
\clj{:c} is not mentioned, but since \clj{:complete?} is \clj{true}
it must be absent so a lookup returns \clj{nil}.

The next example is a real Typed Clojure program written by CircleCI, who maintain a large production
installation of Clojure (see section~\ref{sec:casestudy} for a case study).

\begin{Code}
\begin{exmp}
\inputminted[firstline=10,lastline=22]{clojure}{code/demo/src/demo/key.clj}
\end{exmp}
\end{Code}

\clj{enc-keypair} takes an unencrypted keypair and returns an encrypted keypair by
removing the raw \clj{:private-key} entry and associating an encrypted private key
as \clj{:enc-private-key}.
Since \clj{EncKeyPair} is \clj{:complete?}, Typed Clojure enforces the return type
does not contain an entry \clj{:private-key}, and would complain if the \clj{dissoc}
operation forgot to remove it.

HMaps interact with occurrence typing in interesting ways.
The next example branches on the result of a lookup, and the types
inferred for \clj{f} are subtle.

\begin{Code}
\begin{exmp}
\inputminted[firstline=14,lastline=15]{clojure}{code/demo/src/demo/hmap_path.clj}
\end{exmp}
\end{Code}

Typed Clojure infers \clj{m} as \clj{'{:a Num}} in the then branch
and \clj{(HMap :optional {:a nil})}
in the else branch.
We cannot infer 
\begin{minted}{clojure}
 (HMap :absent-keys #{:a})
\end{minted}
in the else branch because \clj{m} could be the value \clj{{:a nil}}
by the definiton of \clj{Opt}.
Section~\ref{sec:hmapformal} discusses this in greater detail.

\subsection{Java interoperability}

Clojure supports interoperability with Java, including the ability to
call constructors, methods and access fields.
Method calls use prefix syntax, for example this call to 
\clj{java.io.File}'s \clj{getParent} method.

\begin{minted}{clj}
  (fn [f] (.getParent f))
\end{minted}

We have a problem: Clojure's compiler does not know which method to call.
Instead, \clj{f} is inspected at runtime and an appropriate method is selected based
and then invoked by \emph{Java reflection}.
Unfortunately, reflection is slow and Clojure's algorithm for
choosing methods at runtime is undefined in some cases.

To resolve reflective calls, Clojure supports \emph{type-hints}.

\begin{minted}{clj}
  (fn [^File f] (.getParent f))
\end{minted}

The Clojure compiler uses the type hint on \clj{f}
to statically resolve the method call to the \clj{java.io.File}
method with the Java signature

\begin{minted}{java}
  public String getParent()
\end{minted}

Typed Clojure ignores type-hints, so to check this we must
annotate the parameter's static type.

\begin{minted}{clj}
  (fn [f :- File] (.getParent f))
\end{minted}

However Typed Clojure disallows reflection in typed code, so we
must add a type hint.

\inputminted[firstline=10,lastline=10]{clojure}{code/demo/src/demo/parent3.clj}

Now we have a well-typed expression free of reflection.

Typed Clojure and Java treat \java{null} differently.
In Clojure, where it is known as \clj{nil}, Typed Clojure assigns it an explicit type
called \clj{nil}. In Java \java{null} is implicitly a member of any reference type.
This means the Java static type \java{String} is equivalent to
\clj{(U nil String)} in Typed Clojure.

To guarantee \java{null} is never accidentally leaked into a Typed Clojure program,
we must assume methods are nullable.

\begin{exmp}
\inputminted[firstline=12,lastline=13]{clojure}{code/demo/src/demo/parent3.clj}
\end{exmp}

In contrast, JVM invariants guarantee that  constructor cannot return \java{null},
so we are safe to assume constructors are non-nullable.

\begin{Code}
\begin{exmp}
\inputminted[firstline=15,lastline=16]{clojure}{code/demo/src/demo/parent3.clj}
\end{exmp}
\end{Code}

Notice a type hint is used in an argument position to help choose the \java{File(String pathname)}
constructor. By default Typed Clojure conservatively assumes method and constructor arguments to be \emph{non-nullable},
but can be configured globally for particular positions if needed.

Finally, Typed Clojure guarantees typed code cannot throw a null-pointer exception
by attempting to dereference \clj{nil}. This can happen in several ways, including
calling a method or a field on \clj{nil}.
The next example calls a method on once-nullable local binding.
(\clj{ann} annotates a var with an expected type).

\begin{exmp}
\inputminted[firstline=5,lastline=8]{clojure}{code/demo/src/demo/parent3.clj}
\end{exmp}

The test on \clj{f} is essential to check this example---only then can Typed Clojure
prove null-pointer exceptions are impossible.

\subsection{Multimethods}

A multimethod in Clojure is a function that contains a dispatch
function and methods. Multimethods are created with {\clj{defmulti}}.
\begin{minted}{clojure}
(ann rep [Any -> String])
(defmulti rep class)
\end{minted}
\clj{rep} is a multimethod of type \clj{[Any -> String]} with an empty \emph{dispatch table}
and \emph{dispatch function} \clj{class}, a function that returns the argument's class or \clj{nil} if none.
There are no methods to dispatch to so invocations of \clj{rep} will fail at runtime.
\begin{minted}{clojure}
  (rep :a) ;=> IllegalArgumentException ...
\end{minted}

Methods are installed via {\clj{defmethod}}.
\begin{minted}{clojure}
(defmethod rep Keyword [x] (str (name x)))
\end{minted}
We extend \clj{rep}'s dispatch table, mapping
the \emph{dispatch value} \clj{Number} to the function
\clj{(fn [x] (str (name x)))}. The call \clj{(rep arg)}
uses the value of \clj{(isa? (disp-fn arg) disp-val)}
on each method entry to pick a winner \clj{mth}, and finally returns \clj{(mth arg)}.
\clj{isa?} is a subclassing check when provided with classes
\begin{minted}{clojure}
  (isa? Keyword Object) ;=> true
  (isa? Keyword Number) ;=> false
\end{minted}
otherwise an equality check.
\begin{minted}{clojure}
  (isa? :a :a) ;=> true
  (isa? :a 1) ;=> false
\end{minted}

For example
\clj{(rep :a)}
picks the \clj{Keyword} method because
\clj{(isa? (class :a) Keyword)} returns true,
and finally returns \clj{"a"}, the value of
\clj{((fn [x] (str (name x))) :a)}.
We give the full definition of \clj{rep}.

\begin{Code}
\begin{exmp}
\inputminted[firstline=5,lastline=11]{clojure}{code/demo/src/demo/rep.clj}
\label{example:rep}
\end{exmp}
\end{Code}
Typed Clojure does not statically ensure multimethod calls dispatch successfully---
\clj{(rep "a")} type checks but throws a runtime error.

The flexibility of \clj{isa?} is key to the generality of multimethods. 
We can dispatch on the \clj{:op} key 
of our HMap AST \clj{Expr}.
Keywords are functions that look themselves up in their argument, so \clj{:op}
is our dispatch function.
\begin{Code}
\begin{exmp}
\inputminted[firstline=5,lastline=23]{clojure}{code/demo/src/demo/eg5.clj}
\end{exmp}
\end{Code}
The destructuring syntax \clj{(fn [{:keys [val] :as m}] ...)} binds
\clj{m} to the value of the first argument, and \clj{val} to \clj{(:val m)}.

\clj{isa?} is special with vectors---vectors of the
same length recursively call \clj{isa?} on the elements pairwise.
\begin{minted}{clojure}
  (isa? [Keyword :a] [Object :a]) ;=> true
  (isa? [Keyword Keyword] [Object Object]) ;=> true
  (isa? [Keyword Object] [Object Object]) ;=> false
\end{minted}

Now multiple dispatch is possible---we dispatch on the class of both
arguments simultaneously by defining a dispatch function that returns
a vector containing the classes of the arguments.
\begin{Code}
\begin{exmp}
\inputminted[firstline=6,lastline=23]{clojure}{code/demo/src/demo/eg7.clj}
\end{exmp}
\end{Code}
The dispatch values are also vectors---the first method is picked
when the left argument is a \clj{Number} and the right is a \clj{Keyword},
and is thus safe to increment and extract its name respectively.
The dispatch value \clj{:default} specifies a default
method if no preferred method is found.

No extra annotations are needed to follow type-directed control flow
in multimethod dispatch.

\subsection{Final example}

The final example combines everything we will cover for the rest of the paper:
multimethod dispatch, reflection resolution via type hints, Java method
and constructor calls, conditional and exceptional flow reasoning
and use of HMaps. 
\begin{Code}
\begin{exmp}
\inputminted[firstline=6,lastline=20]{clojure}{code/demo/src/demo/eg8.clj}
\end{exmp}
\end{Code}
\clj{PayLoad} is either a HMap containing a file
or a string. We dispatch on \clj{:p} to distinguish the two cases---for example on \clj{:F}
we know the \clj{:file} is a file.
The body of the first method uses type-hints to resolve reflection
and conditional control flow to prove null-pointer exceptions are impossible.
The second method is similar except it uses exceptional control flow.





%% Old stuff vvvv


% TODO references
% why is TR such a good base?
% - immutability
% - common lisp ancestry
% differences?
% - Clojure is built on JVM
% - interop with JVM
% - Clojure's idiomatic primitives are different
% - multimethods + protocols
% - less sophisticated macro system
%  - not an issue
%  - implementation difference, AST walking vs syntax walking


%Typed Clojure is a gradual type system for Clojure. It is designed
%to type check normal Clojure code by adding annotations. It is implemented
%as a library, and can be seamlessly included in any Clojure project; no
%separate compiler or language is needed.
%
%{\smallsection {Based on Typed Racket}}
%Initially, the similarities between untyped Racket and Clojure and Typed Racket's 
%ability to type check Racket code led us to investigate a similar type system for Clojure.
%After two years of development, the solid basis of Typed Racket 
%helps us type check many Clojure idioms without significant differences
%in implementation or theory. We found that extending Typed Clojure to check
%those idioms that have no obvious Racket equivalent did not significantly alter the structure
%of the type system.
%
%{\smallsection {Occurrence typing}}
%\citet{TF08,TF10} developed \emph{occurrence typing}, which helps improve types at branches.
%Typed Clojure uses occurrence typing in a similar way to Typed Racket, with
%some extensions (discussed in in Section [?]). %FIXME
%
%{\smallsection {Practical Variable-Arity Polymorphism}}
%Functions with non-trivial variable parameters are common in Racket.
%For example, Racket provides \emph{map} which takes a function and a
%variable number of collections and applies the function simultaneously
%to each element of the provided collections, returning a list of results.
%\citet*{STF09} developed a practical system that handles advanced variable parameters
%which can handle applications of functions like \emph{map}.
%
%Clojure has a similar emphasis on variable-arity functions. In some ways,
%Clojure's core library encourages even more complicated variable-parameter schemes.
%The \emph{assoc} core function, for example, takes three parameters and
%then a quantity of variable parameters that is a multiple of two.
%This is beyond what Typed Racket (and Typed Clojure) can currently handle. 
%
%Functions like \emph{map} are common in Clojure, so we provide an implementation
%of variable-arity polymorphism which has similar capabilities as Typed Racket's
%implementation.
%
%{\smallsection {Local Type Inference}}
%We use Pierce and Turner's Local Type Inference~\cite{PT00} to infer some polymorphic
%applications. Our implementation is based on Typed Racket's, which has extensions
%to support applications of polymorphic variable-arity functions like \emph{map}.
%
%{\smallsection {Unions and intersections}}
%Like Typed Racket, we include union and ordered intersection types. Unions define
%a least-upper-bound for a set of types. For example, we can express a type that is
%either \Number or \Symbol by including them in a union: {\Union {\Number} {\Symbol}}
%
%Ordered intersections (described further by \citet{SA+12})
%are used for overloading function types. We can express a function that takes
%a \Number and returns a \Symbol, and vice-versa with an ordered intersection function type:
%
%\begin{lstlisting}[label=lst:ordered]
%(Fn [Number -> Symbol]
%    [Symbol -> Number])
%\end{lstlisting}
%
%As our intersections are \emph{ordered}, we can express fine invariants in the
%case where arity parameter types overlap. Similar to a pattern match, earlier arities 
%are tried first, and the first arity to match ``wins''.
%
%For example, applying an \lstinline|Integer| argument to a function of type
%
%\begin{lstlisting}
%(Fn [Integer -> Integer]
%    [Number -> Number])
%\end{lstlisting}
%
%returns an \lstinline|Integer|. Reversing the arities however gives
%type \lstinline|Number|, because the arity taking a \lstinline|Number|
%always matches first.
%
%{\smallsection {Hosted on the Java Virtual Machine}}
%Clojure is built to run on the Java Virtual Machine (JVM),
%offering good interoperability with existing Java code.
%Typed Clojure helps programmers correctly call Java code
%by integrating with Java's type system.
%
%We give Java arrays and Java's \emph{null} special treatment
%when involved with interoperability. Arrays are treated as \emph{read-only}
%when sourced from Java methods, discussed in Section \ref{sec:arrays}.
%We are explicit, and conservative by default, in the positions where
%Java's \emph{null} can be passed, discussed in Section \ref{sec:null}.



\section{A Formal Model of \lambdatc{}}

\label{sec:formal}

Now that we have demonstrated the core features Typed Clojure
provides, we link them together in a formal model called
\lambdatc{}.
Our presentation will start with a review of
occurrence typing~\cite{TF10}.
Then for the rest of the section we incrementally add each
novel feature of Typed Clojure to the formalism,
interleaving presentation of syntax, typing rules, operational semantics
and subtyping.

The first insight about occurrence typing is that
logical formulas
can be used to represent type information about our programs
by relating parts of the runtime environment to types
via propositional logic.
\emph{Type Propositions} \prop{} make assertions like ``variable \x{} is of type \NumberFull{}'' or
``variable \x{} is not \nil{}''---in our logical system we write these as
{\isprop{\NumberFull}{\x{}}}
and {\notprop{\Nil{}}{\x{}}}. 
The other propositions are standard logical connectives: implications, conjunctions,
disjunctions, and the trivial (\topprop{}) and impossible (\botprop{}) propositions
(\figref{main:figure:termsyntax}).


The particular part of the runtime environment we reference in a
type proposition is called the \emph{object}.
The typing judgement relates an object to every expression in the language.
An object is either \emph{empty}, written \emptyobject{}, 
which says 
this expression is not known to evaluate to a particular part
  of the current runtime environment, or a 
variable with some \emph{path}, written \path{\pathelem{}}{\x{}},
that exactly indicates how the value of this
expression can be derived from the current runtime environment.
Type propositions can only reference non-empty objects.

The second insight is that we can replace the traditional 
representation of a
type environment (eg., a map from variables to types)
with a set of propositions, written \propenv{}. 
Instead of mapping \x{} to
the type \NumberFull{}, we use the proposition {\isprop{\NumberFull}{\x{}}}.

Given a set of propositions, we can use logical reasoning to derive
new information about our programs
with the judgement \inpropenv{\propenv{}}{\prop{}}.
In addition to the standard rules for the logical connectives, the key
rule is L-Update, which combines multiple propositions about the same variable,
allowing us to refine its type.
$$
  {\LUpdate}
$$
For example, with L-Update we can use the knowledge of
\inpropenv{\propenv{}}{\isprop{\UnionNilNum}{\x{}}}
and 
\inpropenv{\propenv{}}{\notprop{\Nil{}}{\x{}}}
to derive \inpropenv{\propenv{}}{\isprop{\Number}{\x{}}}.
(The metavariable \propisnotmeta{} ranges over \t{} and \nottype{\t{}} (without variables).)
We cover L-Update in more detail in \secref{sec:formalpaths}.

Finally, this approach allows the type system to track
programming idioms from 
dynamic languages
using implicit type-based reasoning based on the result of
conditional tests.
For instance, \egref{example:conditionalflow}
only utilises \clj{x} once
the programmer is convinced it is safe to do so based whether
\clj{(number? x)}
is 
true or false. 
To express this in the type system, every expression 
is described by two propositions: a `then' proposition
for when it reduces to a true value, and an `else' proposition
when it reduces to a false value---for \clj{(number? x)}
the then proposition is {\isprop{\NumberFull}{\x{}}} and 
the else proposition is {\notprop{\NumberFull}{\x{}}}.
%\ref{main:figure:typingrules}

\begin{figure}
  \footnotesize
$$
\begin{array}{lrll}
  \expd{}, \e{} &::=& \x{}
                      \alt \v{} 
                      \alt {\comb {\e{}} {\e{}}} 
                      \alt {\abs {\x{}} {\t{}} {\e{}}} &\mbox{Expressions} \\
                      &\alt& {\ifexp {\e{}} {\e{}} {\e{}}}
                      %\alt {\trdiff{\doexp {\e{}} {\e{}}}}
                      \alt {\letexp {\x{}} {\e{}} {\e{}}}\\
                      %\alt {\errorvalv{}}
  \v{} &::=&          \singletonmeta{}
                      \alt {\num{}}
                      \alt {\const{}}
                      \alt {\closure {\openv{}} {\abs {\x{}} {\t{}} {\e{}}}}
                &\mbox{Values} \\
                \constantssyntax{}\\
  \s{}, \t{}    &::=& \Top 
                      \alt {\Unionsplice {\overrightarrow{\t{}}}}
                      \alt
                      {\ArrowOne {\x{}} {\t{}}
                                   {\t{}}
                                   {\filterset {\prop{}} {\prop{}}}
                                   {\object{}}}
                &\mbox{Types} \\
                      &\alt& {\Value \singletonmeta{}} 
                      \alt \trdiff{\class{}}\\
  \singletonallsyntax{}
                \\ \\
  \occurrencetypingsyntax{}\\
  \propenvsyntax{}\\
  \openvsyntax{}
  %\\
  %\classliteralallsyntax{}
\end{array}
$$
\caption{Syntax of Terms, Types, Propositions and Objects}
\label{main:figure:termsyntax}
\end{figure}


We formalise our type system following~\citet{TF10}
(with differences highlighted in $\trdiff{\text{blue}}$).
The typing judgement 
$$
{\judgement   {\propenv}
              {\hastype {\e{}} {\t{}}}
  {\filterset {\thenprop {\prop{}}}
              {\elseprop {\prop{}}}}
  {\object{}}}
$$
says expression \e{} is of type \t{} in the 
proposition environment $\propenv{}$, with 
`then' proposition {\thenprop {\prop{}}}, `else' proposition {\elseprop {\prop{}}}
and object \object{}. We write 
{\judgementtwo{\propenv}{\hastype {\e{}} {\t{}}} if we are only interested in the type.

Syntax is given in \figref{main:figure:termsyntax}. Expressions include variables, values,
application, abstractions, conditionals and let expressions.
All binding forms introduce fresh variables.
Values include booleans, \nil{}, class literals, keywords, 
numbers,
constants and closures. 
Value environments map local bindings to values.

Types include the top type, \emph{untagged} unions, functions, singleton types
and class instances. 
We abbreviate \Booleanlong{} as \Boolean{}, \Keywordlong{} as \Keyword{}
and \NumberFull{} and \Number{}.
The type \Value{\Keyword} is inhabited by the class literal \Keyword{} and \clj{:a} is of type \Keyword{}.
We abbreviate \EmptyUnion{} as \Bot{}, {\ValueNil} as \Nil{}, 
{\ValueTrue} as \True and {\ValueFalse} as {\False}.
Function types contain \emph{latent} (terminology from~\cite{Lucassen88polymorphiceffect}) propositions and object, which, along with the return type,
may refer to the function argument.
%Latent means they are relevant when the function is applied rather than evaluated.
They are latent because they are instantiated with the
actual object of the argument in applications before they are used in the proposition environment.

\figref{main:figure:typingrules} contains the core typing rules.
The key rule for reasoning about conditional control flow is
T-If. 

\begin{mathpar}
  {\TIf}
\end{mathpar}

The propositions of the test expression \e{1}, \thenprop{\prop{1}} and \elseprop{\prop{1}}, are 
used as assumptions in the then and else branch respectively.
If the result of the \ifliteral{} is a true value, then it either
came from \e{2}, in which case \thenprop{\prop{2}} is true, or from \e{3},
which implies \thenprop{\prop{3}} is true. 
The else proposition is \orprop{\elseprop{\prop{2}}}{\elseprop{\prop{3}}} 
similarly.
The T-Local rule connects the type system to the proof system over type propositions
via \inpropenv {\propenv{}} {\isprop {\t{}} {\x{}}}
to derive a type for a variable.
Using this rule, the type system can then appeal to L-Update to refine the type
assigned to \x{}.

We are now equipped to type check
\egref{example:conditionalflow}, starting at body:
$$
\clj{... (if (number? x) (inc x) 0) ...}
$$

We know {\propenv{}} = {\isprop{\UnionNilNum{}}{\x{}}}.
The test expression uses T-App, 
$$
\judgement{\propenv{}}{\hastype{\appexp{\numberhuh{}}{\x{}}}{\Boolean}}{\filterset{\isprop{\Number}{\x{}}}{\notprop{\Number}{\x{}}}}{\emptyobject{}}
$$
since \numberhuh{} has type
{\ArrowOne{\x{}}{\Top}{\Boolean}
        {\filterset{\isprop{\Number}{\x{}}}{\notprop{\Number}{\x{}}}}{\emptyobject{}}}
      and \x{} has object \x{}.

Finally we check both branches using the extended proposition environment as specified by T-If.
Going down the then branch, our new assumption {\isprop{\Number}{\x{}}} is crucial to check
$$
\judgement{{\propenv{}},{\isprop{\Number}{\x{}}}}{\hastype{\x{}}{\Number{}}}{\filterset{\notprop{\falsy{}}{\x{}}}{\isprop{\falsy{}}{\x{}}}}{\emptyobject{}}
$$
because we can now satisify the premise of T-Local:
$$
\inpropenv{{\propenv{}},\isprop{\Number}{\x{}}}{\isprop{\Number}{\x{}}}.
$$
%\judgement{{\propenv{}},\isprop{\Number}{\x{}}}{\hastype{\appexp{\inc{}}{\x{}}}{\Number{}}}{\filterset{\topprop{}}{\botprop{}}}{\emptyobject{}}
%$$
%$$
%\judgement{{\propenv{}},\notprop{\Number}{\x{}}}{\hastype{\zeroliteral{}}{\Number}}{\filterset{\topprop{}}{\botprop{}}}{\emptyobject{}}
%$$

%\inc{} has type
%$$
%{\ArrowOne{\x{}}{\Number}{\Number}
%        {\filterset{\topprop{}}{\topprop{}}}{\emptyobject{}}}
%$$
%We can now check the conditional with T-If.
%$$
%\judgement{\isprop{\Number}{\x{}}}{\hastype{\ifexp{\appexp{\numberhuh{}}{\x{}}}{\appexp{\inc{}}{\x{}}}{\zeroliteral{}}}{\Number}}{\filterset{\orprop{\isprop{\Number}{\x{}}}{\topprop{}}}{\orprop{\notprop{\Number}{\x{}}}{\topprop{}}}}{\emptyobject{}}
%$$
%Finally the function can be checked with T-Abs
%$$
%\judgement{}{\hastype{\abs{\x{}}{\UnionNilNum}{\ ...}}
%                                             {\ArrowOne{\x{}}{\UnionNilNum}{\Number}
%        {\filterset{\orprop{\isprop{\Number}{\x{}}}{\topprop{}}}{\orprop{\notprop{\Number}{\x{}}}{\topprop{}}}}{\emptyobject{}}}}
%  {\filterset{\topprop{}}{\botprop{}}}{\emptyobject{}}
%$$

\paragraph{Operational semantics} We define the dynamic semantics for \lambdatc{}
in a big-step style using an environment, following~\citet{TF10}.
We include both errors and a \wrong{} value, which is provably ruled out by the
type system.
The main judgement is \opsem{\openv{}}{\e{}}{\definedreduction{}}
which states that \e{} evaluates to answer \definedreduction{} in environment
\openv{}. We chose to omit the core rules (see \figref{appendix:figure:opsem})
however a notable difference is \nil{} is a false value, which affects the
semantics of \ifliteral{}:

\begin{mathpar}
    \BIfTrue{}

    \BIfFalse{}
\end{mathpar}

Subtyping (\figref{main:figure:subtyping}) 
is a reflexive and transitive relation with top type \Top. 
Singleton types are instances of their respective classes---boolean singleton types
are of type \Boolean{}, class literals are instances of \Class{} and keywords are
instances of \Keyword{}.
Instances of classes \class{} are subtypes of \Object{}. Since function types 
are subtypes of \IFn{}, all types except for \Nil{} are subtypes of \Object{},
so \Top{} = {\Union{\Nil}{\Object}}.
Function subtyping is contravariant left of the arrow---latent propositions, object
and the result type are covariant.
Subtyping for untagged unions is standard.

\begin{figure*}
  \footnotesize
  \begin{mathpar}
    %{\TDo}
    %{\TClass}
    %{\TIf}
    {\TAbs}
    {\TSubsume}
    \begin{array}{c}
      {\TConst}\\\\
      {\TKw}\\\\
      {\TClass}
    \end{array}
    \begin{array}{c}
      {\TTrue}\\\\
      {\TFalse}\\\\
      {\TNil}
    \end{array}

    {\TLet}
    {\TLocal}

    {\TApp}
    %{\TError}

  \end{mathpar}
  \caption{Typing rules}
  \label{main:figure:typingrules}
\end{figure*}

%\begin{figure}
%  \footnotesize
%  \begin{mathpar}
%    {\BLocal}
%
%    %{\BDo}
%
%    {\BLet}
%
%    \BVal{}
%
    %\BIfTrue{}

%    \BIfFalse{}
%
%    \BAbs{}
%
%    \BBetaClosure{}
%
%    \BDelta{}
%  \end{mathpar}
%  \caption{Operational Semantics}
%  \label{main:figure:standardopsem}
%\end{figure}

\begin{figure*}
  \footnotesize
  \begin{mathpar}
    \standardsubtyping{}
  \end{mathpar}
  \caption{Subtyping rules}
  \label{main:figure:subtyping}
\end{figure*}

\subsection{Reasoning about Exceptional Control Flow}
\label{sec:doformal}

We extend our model with sequencing expressions and errors, where {\errorvalv{}}
models the result of calling Clojure's \clj{throw} special form
with some \clj{Throwable}.

\smallskip
$
\begin{altgrammar}
  \e{} &::=& \ldots \alt {\errorvalv{}} \alt {\doexp {\e{}} {\e{}}} &\mbox{Expressions} 
\end{altgrammar}
$

\smallskip
%
%B-Do simply evaluates its arguments sequentially and returns the right argument.
%Since errors are not values, we define error propagation semantics
%like BE-Do1 (figure~\ref{appendix:figure:errorstuck} for the full rules).
%
%\begin{mathpar}
%    {\BDo}
%
%\infer [BE-Error]
%{}
%{ \opsem {\openv{}} 
%         {\errorvalv{}}
%         {\errorvalv{}}}
%
%\infer [BE-Do1]
%{ \opsem {\openv{}} {\e{1}} {\errorvalv{}} }
%{ \opsem {\openv{}} {\doexp{\e{1}}{\e{}}} {\errorvalv{}}}
%\end{mathpar}

Our main insight is as follows: 
if the first subexpression in a sequence reduces to a value, then it is either true or false.
If we learn some proposition in both cases then we can use that proposition as an assumption to check the second subexpression.
T-Do formalises this intuition.

\begin{mathpar}
    {\TDo}  
\end{mathpar}

The introduction of errors, 
which do not evaluate to either
a true or false value,
makes our insight interesting.

\begin{mathpar}
    {\TError}
\end{mathpar}

Recall \egref{example:doexception}.
\begin{minted}{clojure}
...  (do (if (number? x) nil (throw (Exception.)))
         (inc x)) ...
\end{minted}

As before, checking \appexp{\numberhuh{}}{\x{}} allows us to use the proposition \isprop{\Number}{\x{}}
when checking the then branch.

By T-Nil and subsumption we can propagate this  information to both propositions.
$$
\judgement{\isprop{\Number}{\x{}}}{\hastype{\nil{}}{\Nil{}}}{\filterset{\isprop{\Number}{\x{}}}{\isprop{\Number}{\x{}}}}{\emptyobject{}}
$$
Furthermore, using T-Error
and subsumption we can conclude anything in the else branch.
$$
\judgement{\notprop{\Number}{\x{}}}{\hastype{\errorvalv{}}{\Bot}}{\filterset{\isprop{\Number}{\x{}}}{\isprop{\Number}{\x{}}}}{\emptyobject{}}
$$
Using the above as premises to T-If we conclude that if the first
expression in the \doliteral{} evaluates successfully, \isprop{\Number}{\x{}} must be true.
$$
\judgement{\isprop{\UnionNilNum}{\x{}}}
          {\hastype{\ifexp{\appexp{\numberhuh{}}{\x{}}}{\nil{}}{\errorvalv{}}}{\Boolean}}
          {\filterset{\isprop{\Number}{\x{}}}{\isprop{\Number}{\x{}}}}{\emptyobject{}}
$$
We can now use \isprop{\Number}{\x{}} in the environment to check the second subexpression
{\appexp{\inc{}}{\x{}}}, completing the example.

\subsection{Heterogeneous maps}
\label{sec:hmapformal}

\begin{figure}
  \footnotesize
  $$
  \begin{altgrammar}
    \e{} &::=& \ldots \alt \hmapexpressionsyntax{}
    &\mbox{Expressions} \\
    \v{} &::=& \ldots \alt {\emptymap{}}
    &\mbox{Values} \\
    \t{} &::=& \ldots \alt {\HMapgeneric {\mandatory{}} {\absent{}}}
    &\mbox{Types} \\
    \auxhmapsyntax{}\\
    \pesyntax{}   &::=& \ldots \alt {\keype{\k{}}}
                  &\mbox{Path Elements}
  \end{altgrammar}
  $$
  \begin{mathpar}
    {\TGetHMap}

    {\TGetAbsent}

    {\TGetHMapPartialDefault}

    {\TAssoc}
  \end{mathpar}
  \begin{mathpar}
    {\BAssoc}
    {\BGet}
    {\BGetMissing}
  \end{mathpar}
  \begin{mathpar}
    \HMapsubtyping{}
  \end{mathpar}
  \caption{HMap Terms, Types, Semantics, Subtyping and Typing Rules}
  \label{main:figure:hmapsyntax}
\end{figure}


We now present \HMapliteral{} types in detail.
\figref{main:figure:hmapsyntax} extends the syntax for terms and types.

We write \HMapgeneric{\mandatory{}}{\absent{}}
where {\mandatory{}} is a map of \emph{present} entries (mapping keywords to types),
\absent{} is a set of keyword keys that are known to be \emph{absent}
and
\completenessmeta{} is {\complete{}} (``complete'') if the map is fully specified by \mandatory{},
and {\partial{}} (``partial'') if there are \emph{unknown} entries.
To ease presentation, 
if a HMap is declared as \complete{} then \absent{} implicitly contains all keywords not in the domain of \mandatory{}.
%\HMapcwithabsent{\mandatory{}}{\absent{}} is abbreviated to \HMapc{\mandatory{}}. 
Keys cannot be both present and absent.

The expressions \clj{(get m :a)} and \clj{(:a m)} are semantically identical, though
we only model the former to avoid the added complexity of keywords being functions.
To simplify presentation, we only provide syntax for the empty map literal and
restrict lookup and extension to keyword keys. The metavariable \mapval{}
ranges over the runtime value of maps {\curlymapvaloverright{\k{}}{\v{}}},
usually written {\curlymapvaloverrightnoarrow{\k{}}{\v{}}}.

Subtyping for HMaps (\figref{main:figure:hmapsyntax})
designate \MapLiteral{} as a common supertype for all HMaps.
S-HMap says that a HMap is a subtype of another HMap if they agree
on \completenessmeta{}, agree on mandatory entries with subtyping
and at least cover the absent keys of the supertype.
Complete maps are subtypes of partial maps
as long as they agree on the mandatory entries of the partial map via subtyping (S-HMapP).

\figref{main:figure:hmapsyntax} contains the typing rules. T-GetHMap models a lookup
that will certainly succeed, T-GetHMapAbsent a lookup that will certainly fail
and T-GetHMapPartialDefault a lookup with unknown results.
Lookups on unions of HMaps are only supported in T-GetHMap, 
in particular to support
looking up \clj{:op} on a map of type \clj{Expr} (\egref{example:decleaf})
where every element in the union
contains the key we are looking up.
The objects in the T-Get rules are more complicated than those in T-Local---the 
next section discusses this in detail.
Finally T-AssocHMap extends a HMap with a mandatory entry while preserving completeness
and absent entries, and enforcing ${\k{}} \not\in {\absent{}}$ to prevent badly
formed types.

\figref{main:figure:hmapsyntax} contains the semantics for \getliteral{}
and \assocliteral{}.
The B-Get rule evaluates first evaluates the map and the key, asserts
the key must be present in the map and returns the associated key. If
the entry is missing, B-GetMissing says we return \nil{}.

\subsection{Paths}
\label{sec:formalpaths}

Recall the first insight of occurrence typing---we can reason
about specific \emph{parts} of the runtime environment
using propositions.
The way to refer to \emph{parts} of the runtime environment with
occurrencing is via \emph{path elements}.
A \emph{path} consists of a series of path elements
applied right-to-left to a variable
written
\path{\pathelem{}}{\x{}}.
\citet{TF10} introduce the path elements \carpe{} and \cdrpe{}
to reason about selector operations on cons cells.
We instead want to reason about HMap lookups and calls to \classconst{}.

\paragraph{Key path element} We introduce our first path element
{\keype{\k{}}}, which represents the operation of looking up
a key \k{}.
We directly relate this to our typing rule T-GetHMap
(\figref{main:figure:hmapsyntax}) by
checking the then branch of the first conditional test is checked in 
an equivalent version of \egref{example:decleaf}.
\begin{minted}{clojure}
  (fn [m :- Expr]
    (if (= (get m :op) :if)
      {:op :if, ...}
      (if ...)))
\end{minted}

We do not specifically support \equivliteral{} in our calculus, 
but on keyword arguments it works identically to \clj{isa?} which we model
in \secref{sec:isaformal}.
Intuitively, if {\judgement{\propenv{}}{\hastype{\e{}}{\t{}}}{\filterset{\thenprop{\prop{}}}{\elseprop{\prop{}}}}{\object{}}}
then \equivapp{\e{}}{\makekw{if}} has the proposition set 
$$
{\replacefor{\filterset{\isprop{\Value{\makekw{if}}}{\x{}}}{\notprop{\Value{\makekw{if}}}{\x{}}}}{\object{}}{\x{}}}
$$
where substitution reduces to \topprop{} if \object{} = \emptyobject{}.

We start with proposition environment \propenv{} = {\isprop{\Expr{}}{m}}.
Since {\Expr{}} is a union of HMaps, each with the entry \makekw{op}, we can use T-GetHMap.
$$
\judgement{\propenv{}}{\hastype{\getexp{m}{\makekw{op}}}{\Keyword}}{\filterset{\topprop{}}{\topprop{}}}{\path{\keype{\makekw{op}}}{m}}
$$
Using our intuitive definition of \equivliteral{} above, we know
$$
\judgement{\propenv{}}{\hastype{\equivapp{\getexp{m}{\makekw{op}}}{\makekw{if}}}{\Boolean}}{\filterset{\isprop{\Value{\makekw{if}}}{\path{\keype{\makekw{op}}}{m}}}{\notprop{\Value{\makekw{if}}}{\path{\keype{\makekw{op}}}{m}}}}{\emptyobject{}}
$$
Going down the then branch gives us the extended environment
\propenvp{} = {\isprop{\Expr{}}{m}},{\isprop{\Value{\makekw{if}}}{\path{\keype{\makekw{op}}}{m}}}.
Using L-Update we can combine what we know about object $m$ and object
{\path{\keype{\makekw{op}}}{m}}
to derive
$$
\inpropenv{\propenvp{}}{\isprop{\HMapp{\mandatoryset{{\mandatoryentrykwnoarrow{op}{\makekw{if}}}, {\mandatoryentrykwnoarrow{test}{\Expr{}}},
                                       {\mandatoryentrykwnoarrow{then}{\Expr{}}},   {\mandatoryentrykwnoarrow{else}{\Expr{}}}}}
                                   {\emptyabsent{}}}{m}}
$$

\paragraph{Class path element} Our second path element \classpe{} is used in the latent
object of the constant \classconst{} function. Like Clojure's \clj{class}
function \classconst{} returns the argument's class or \nil{}
if passed \nil{}.
$$
\begin{array}{lrlr}
  \pesyntax{}   &::=& \ldots \alt {\classpe{}}
                &\mbox{Path Elements}
\end{array}
$$
\begin{mathpar}
\constanttypefigure{}
\end{mathpar}
The full semantics are given in \figref{main:figure:primitivesem}.
The definition of \updateliteral{} supports various idioms relating to \classpe{}
which we introduce in \secref{sec:isaformal}.

\subsection{Java Interoperability and Type Hints}

\begin{figure}[h]
  \footnotesize
  $$
  \begin{altgrammar}
    \e{} &::=& \ldots \alt \mininonreflectiveexpsyntax{}
    \\

    \v{} &::=& \ldots \alt {\classvalue{\classhint{}} {\overrightarrow {\classfieldpair{\fld{}} {\v{}}}}}
    &\mbox{Values} \\

    \tatypesyntax{}\\
    \typehintenvsyntax{}\\
    \classtableallsyntax{}
  \end{altgrammar}
  $$
 \classtablelookupsyntax{}
 \begin{mathpar}
  \begin{altgrammar}
    \convertjavatypenil{}
  \end{altgrammar}
  \begin{altgrammar}
    \convertjavatypenonnil{}
  \end{altgrammar}
\end{mathpar}
  \begin{mathpar}
    {\TNewStatic}

    {\TMethodStatic}

    %{\TInstance}
  \end{mathpar}
  \begin{mathpar}
    \BField{}

    \BNew{}

    \BMethod{}
  \end{mathpar}
  \begin{mathpar}
    %\RAbs{}
%
%    \RNewElimRefl{}
%
%    \RMethodElimRefl{}
%
%    \RFieldElimRefl{}
%
%    \RLet{}
%

%    \RLetHint{}
  \end{mathpar}
  \caption{Java Interoperability Syntax, Semantics, Typing Rules and Rewrite Relation}
  \label{main:figure:javatyping}
\end{figure}

In \secref{sec:overviewjavainterop} we discussed the role of type hints
to help eliminate reflective calls.
In this section, we introduce our model of Java and provide user-facing
syntax corresponding to Clojure's Java interoperability forms and type hinted forms.
Then we model the Clojure compiler's compile-time reflection resolution
algorithm.
To achieve this, first we define notation for
\emph{non-reflective} Java forms that 
unambiguously call a field, method or constructor.
Then we define a rewrite relation that uses
type hints to resolve reflection explicitly.
Finally we give typing rules that model how
Typed Clojure interacts with non-reflective calls.

We present Java interoperability in a restricted setting without class inheritance,
overloading or Java Generics.

  $$
  \begin{altgrammar}
    \e{} &::=& \ldots   \localhintsyntax{} \alt \lethintsyntax{} &\mbox{Expressions}\\
            &\alt& \reflectiveexpsyntax{} 
  \end{altgrammar}
  $$

We extend the syntax with type hinted expressions, Java field lookups and calls to
methods and constructors. We model the syntax after the `dot' special
form to prevent ambiguity---\clj{(.fld e)} is now \fieldexp{\fld{}}{\e{}},
\clj{(.mth e es*)} is $\methodexp{\mth{}}{\e{}}{\overrightarrow{es}}$
and \clj{(.class es*)} is $\newexp{\class{}}{\overrightarrow{es}}$.
The Java expressions come without typing rules because Typed Clojure
only reasons about resolved reflection
(as demonstrated in \secref{sec:overviewjavainterop}).

Now we model the compiler's reflection-resolution algorithm.
\figref{main:figure:javatyping} gives the syntax for non-reflective Java calls.
The method ${\methodstaticexp {\classhint{1}}
                             {\overrightarrow {\classhint{i}}}
                             {\classhint{2}}
                             {\mth{}} {\e{}} {\overrightarrow{\e{i}}}}$
is a non-reflective call to the \mth{} method on class {\class{1}}, 
with Java signature 
${\classhint{2}}\ \mth{}\ (\overrightarrow {\classhint{i}});$.
The field {\fieldstaticexp {\classhint{1}} {\classhint{2}} {\fld{}} {\e{}}}
calls the field on class {\classhint{1}} with Java signature
${\classhint{2}}\ \fld{};$.
The constructor {\newstaticexp {\overrightarrow{\classhint{i}}} {\classhint{}} 
                               {\class{}} {\overrightarrow{\e{i}}}}
calls the constructor with Java signature
${\classhint{}}\ ({\overrightarrow{\classhint{i}}});$.

%\begin{figure}
%  \footnotesize
%  \begin{mathpar}
%%    \TALocal{}
%%
%%    \TANil{}
%%
%%    \TANewStatic{}
%%
%%    \TALetHint{}
%
%%    \TALet{}
%  \end{mathpar}
%\caption{Type Hint Inference (select rules, figure~\ref{appendix:figure:hintinfer} for full rules)}
%\label{main:figure:hintinfer}
%\end{figure}


To obtain non-reflective calls we define our rewrite relation
\rewrite {\taenv{}} {\e{}} {\ep{}}
which rewrites \e{} to a possibly-less reflective expression
\ep{} with respect to type hint environment
\taenv{} and Java class table \ct{}.

For example,
R-FieldElimRefl emits a non-reflective field if it can
find a field matching the type hint inferred on \ep{}.
  $$
    \RFieldElimRefl{}
  $$

Type hint inference is modelled by
\tajudgement {\taenv{}} {\hastype {\e{}} {\tatype{}}}
which infers the (possibly-unknown) type hint \tatype{} of expression \e{} in environment \taenv{}.
%$$
%\TALetHint{}
%$$

For demonstration purposes let us rewrite
a simple field expression, with the assumptions that
that we have a class \Point{} with a field \getx{} of Java type \Number
and \intaenv {\taenv{}} {\x{}} {\Point{}}.
In rewriting
the expression \fieldexp{\getx}{\x{}},
{\Point{}} is inferred as the type hint of \x{}
and \Number{} is the field type
by \fieldtypeliteral{} (\figref{main:figure:javatyping}).
Now we just plug in the new information into our new expression
$$
\fieldstaticexp{\Point}{\Number}{\getx}{\x{}}.
$$

Now we present the typing rules for resolved Java interoperability.
T-FieldStatic checks a resolved field expression by ensuring the target has
the correct static type, then returns a nilable type corresponding the Java type.

\begin{mathpar}
    {\TFieldStatic}
\end{mathpar}

To continue our example, let us assume \propenv{} = {\isprop{\Point}{\x{}}}.
T-FieldStatic checks \x{} under the non-nilable type {\Point}
and returns the nilable type {\Union{\Nil}{\Number}}.

The other rules T-MethodStatic and T-NewStatic work similarly (\figref{main:figure:javatyping}), varying
in the choice of nilability in the conversion function---method returns
are nilable and constructor returns are non-nilable.

The operational semantics B-Field, B-New and B-Method (\figref{main:figure:javatyping}) simply evaluate their
arguments and call the relevant JVM operation---\secref{sec:metatheory}
states our exact assumptions about each.

\subsection{Multimethod dispatch mechanism: \isaliteral}

\label{sec:isaformal}

We now consider the core dispatch mechanism for multimethods. 
Recalling the examples in \secref{sec:multioverview},
\isaliteral{} is
a subclassing test for classes, otherwise an equality test---we do not model
the semantics for vectors.

The main interest for the T-IsA typing rule is the \isacompareliteral{} 
metafunction
(\figref{main:figure:mmsyntax}), used to calculate the propositions for
\isaliteral{} expressions.

\begin{mathpar}
  \TIsA{}
\end{mathpar}

To demonstrate,
\isaapp{\appexp{\classconst{}}{\x{}}}{\Keyword}
has the proposition set \isacompare{\s{}}{\path{\classpe{}}{\x{}}}{\Value{\Keyword}}{\filterset{\isprop{\Keyword}{\x{}}}{\notprop{\Keyword}{\x{}}}}.

B-IsA models the semantics. \isaopsemliteral{} explicitly handles classes in the second case.

$$
\begin{array}{ll}
  \vcenter{\hbox{\BIsA{}}}
  &
  \vcenter{\hbox{\isaopsemfigure{}}}
\end{array}
$$

%The definition of \isacompareliteral{} (figure~\ref{main:figure:mmsyntax}) is deliberately conservative.
%The first line considers the case where the object of the left argument
%is a non-empty path ending in \classpe{} and the type of the right argument is a singleton class.

\constantsemfigure{main}

\subsection{Multimethods}

\begin{figure}
  \footnotesize
$$
\begin{altgrammar}
  \e{} &::=& \ldots \alt {\createmultiexp {\t{}} {\e{}}} &\mbox{Expressions} \\
             &\alt& {\extendmultiexp {\e{}} {\e{}} {\e{}}}
             \alt {\isaapp {\e{}} {\e{}}}\\
  \v{} &::=& \ldots \alt {\multi {\v{}} {\disptable{}}}
                &\mbox{Values} \\
  \s{}, \t{} &::=& \ldots \alt {\MultiFntype{\t{}}{\t{}}}
                &\mbox{Types} \\

 \disptablesyntax{} \\
\end{altgrammar}
$$
  \begin{mathpar}
    \Multisubtyping{}
  \end{mathpar}
  \begin{mathpar}
    \isapropsfigure{}
  \end{mathpar}
  \begin{mathpar}
    \TDefMulti{}

    \TDefMethod{}
  \end{mathpar}
  \getmethodfigure{}
  \begin{mathpar}
    \BDefMethod{}
    \BDefMulti{}
    \BBetaMulti{}
  \end{mathpar}
\caption{Multimethod Syntax, Subtyping, Typing Rules and Semantics}
\label{main:figure:mmsyntax}
\end{figure}

To ease presentation, we present \emph{immutable}
multimethods (\figref{main:figure:mmsyntax}). \defmethodliteral{} returns a new extended multimethod
without changing the original multimethod. \egref{example:rep} is now written
\begin{minted}{clojure}
(let [rep (defmulti [Any -> String] class)
      rep (defmethod rep Keyword [x] (str (name x)))
      rep (defmethod rep Number [x] (str (inc x)))]
  (rep :a)) ;=> "a"
\end{minted}

Multimethod semantics are in \figref{main:figure:mmsyntax}.
B-DefMulti creates a multimethod with a dispatch function and an empty dispatch table.
B-DefMethod returns a new multimethod with an extended dispatch table.
B-BetaMulti invokes the dispatch function with the evaluated argument to get the dispatch value,
which \getmethodliteral{} uses to choose the method to invoke.

The typing rules (\figref{main:figure:mmsyntax}) use the multimethod type {\MultiFntype{\s{}}{\t{}}}, 
where \s{} is the interface type for beta reduction, and \t{} is the type for
the dispatch function. T-DefMulti uses the provided interface type and dispatch function
to infer the multimethod type. T-DefMethod uses \isacompareliteral{} to infer the proposition
that must be true if this method is being invoked. The example can be type checked without
changes since we infer \isprop{\Number}{\x{}} and \isprop{\Keyword}{\x{}} for the respective
method bodies. T-App handles beta reduction because multimethods are subtypes of functions
(\figref{main:figure:mmsyntax}).



\begin{figure*}
  $$
\begin{array}{lllll}
\updatefigure
\end{array}
$$
\caption{Type Update}
\label{main:figure:update}
\end{figure*}

\begin{figure}
  $$
\begin{array}{lllll}
  \restrictremovefigure{}
\end{array}
  $$
  \caption{Restrict and Remove}
  \label{main:figure:restrictremove}
\end{figure}


\section{Metatheory}
\label{sec:metatheory}

Our proof for type soundness is similar to~\citet{TF10}. We add
errors and \wrong{} value and prove
well-typed programs do not go wrong.

We make our assumptions about Java explicit. We concede that
method and constructor calls may diverge or error, but we assume they can
never go wrong.

{\javanewassumption{main}}

%For readability we define logical truth in Clojure.

%{\istruefalsedefinitions{main}}

We use an extra lemma to support our main soundness lemma. Consistency
ensures that occurrence typing does not refer to variables
hidden inside a closure.

{\consistentwithonlydef{main}}

Our main lemma says if there is a defined reduction, then the propositions, object
and type are correct.
The metavariable \definedreduction{} ranges over \v{}, \errorvalv{} and \wrong{}.

\begin{lemma}\label{main:lemma:soundness}

  {\soundnesslemmahypothesis}
  \begin{proof}
    By induction on the derivation of the typing judgement. 
    (Full proof given as lemma~\ref{appendix:lemma:soundness}).
  \end{proof}
\end{lemma}


We can now state our soundness theorem.

{\soundnesstheoremnoproof{main}}

{\wrongtheoremnoproof{main}}
\noindent
As a corollary, null-pointer exceptions are ruled-out in typed code.
%
%{\nilinvoketheoremnoproof{main}}


\section{Experience}

\subsection{Higher-rank polymorphism}

Typed Clojure supports f-bounded polymorphism and higher-rank
types.
% why include them?
% examples, involving generic type operators for monads & conduits
% useful for generic programming, similar to Hashell type classes

\subsection{Using negative filters}

Occurrence typing plays an important role in Typed Racket and Typed Clojure.
By maintaining a \emph{proposition environment} of propositions relating types to
bindings, we can update bindings with more accurate types as programs progress.
It follows that there is some correspondence between propositions and types,
characterised by the \emph{update} function, which takes a type and a proposition
and returns a type which updates the input type using the proposition.

There is a straightforward relationship between ``positive'' propositions and types.
For example 
{\tt (update Number (is Integer 0))}
updates Number by Integer, which is Integer, because Integer <: Number.

The relationship between ``negative'' propositions and types is not always obvious.
A common proposition in Typed Clojure is (! (U nil false) a): the proposition that
local binding ``a'' is \emph{not} of type (U nil false).
This problem is most visible in expressions like {\tt (filter identity coll)}, where
``identity'' has a ``then'' proposition that has negative information: (! (U nil false) 0),
which reads, the 0th argument of identity does not contain (U nil false).

\subsubsection{Arrays}
\label{sec:arrays}

Supporting statically sound interactions with Java arrays is a goal
of Typed Clojure. This is complicated by Java's decision to make
arrays covariant in their argument, a well documented source of static
unsoundness. Bracha~\cite{Bra98} summarises Java's approach to maintaining
soundness at runtime, which involves all array writes being checked by
runtime assertions.

This approach fits Java's type system, but we can do better in a more powerful
type system like Typed Clojure. Our goal is to catch all type-incorrect array
writes at compile time so the type system can do more to help Clojure programmers
use arrays, especially those being passed from foreign Java code.

Our basic approach is to make our array types \emph{bivariant}. Array types
look like {\ArrayTwo {\t{w}} {\t{r}}} and
are reminiscent of functions or pipes: having a contravariant parameter for input (writing)
and a covariant parameter for output (reading).
This type can write type {\t{w}} and read type {\t{r}}.

Most commonly, an array type is invariant in its parameter; it can
write and read input of the same type.
We can get the same effect by setting our input and output
parameters to the same type. For example, {\ArrayTwo {\Number} {\Number}}
(or equivalently, {\Array {\Number}})
in Typed Clojure is similar to invariant array types of \Number in languages like Scala.

The biggest gain in using a separate input parameter is the ability
to specify \emph{read-only} arrays. Crucially, our type system features an
explicit bottom type \lstinline|Nothing|, enabling a read-only \lstinline|Number| array
to be of type \lstinline|(Array2 Nothing Number)|.

To realise why defining read-only arrays are useful, we need to examine
what makes array covariance unsound in Java.
\begin{verbatim}
FIXME
Array covariance about the type of an array so the consumer
of an array cannot tell the actual type of the array when examining a type
signature.
\end{verbatim}

\begin{lstlisting}
...
public static Number[] getNumberArray() {
  Number[] n = new Integer[10];
  return n;
}
...
\end{lstlisting}

To the casual consumer \emph{getNumberArray} returns an array that can both
read and write \lstinline|Number|s. However it is clear from the implementation
that attempting to write say a \lstinline|Double| to this array will result
in a runtime error.

\begin{verbatim}
...
Number[] myArray = getNumberArray();
myArray[0] = 1.1;
/* Exception in thread "main" 
   java.lang.ArrayStoreException: 
   java.lang.Double */
...
\end{verbatim}

Notice that this is a runtime error, and Java's type system has not helped
statically prevent it.
This could cause a similar issue for other statically-typed languages offering
interoperability with Java. 

To prevent these sorts of runtime exceptions in Typed Clojure, we declare
all arrays from unknown sources to be \emph{read-only}. Put differently,
the only way to define a writeable array is to create it in type-checked Clojure
code.

\begin{lstlisting}
(let [n (CovariantArray/getNumberArray)]
  (aset n 0 1.1))

; Polymorphic static method clojure.lang.RT/aset could not be 
; applied to arguments:
; Domains: 
;         (Array2 i o) clojure.core.typed/AnyInteger i
; 
; Arguments:
;         (Array2 Nothing java.lang.Number) int (Value 1.1)
; 
; with expected type:
;         Any
\end{lstlisting}

The type inferred for the local \lstinline|n| is \lstinline|(Array2 Nothing Number)|
which tells the type system: it is never safe to write to this array, but
it is safe to assume \lstinline|Number|s can be read from this array.

To emphasise, Typed Clojure throws a static type error. Errors like this help Clojure programmers
use foreign Java libraries more correctly.

\begin{verbatim}
Note that Java libraries are often large 
and complex and programmers will probably
enjoy the extra help from the type system.
\end{verbatim}


\section{Related Work}

\paragraph{Multimethods} 
Millstein and Chambers~\cite{MS02}
describe Dubious, a simple statically typed core language including multimethods that
dispatch on the type of its arguments. They tackle a key challenge for statically typing
multimethods: ``it is possible for two modules containing arbitrary multimethods to typecheck
successfully in isolation but generate type errors when linked together.''~\cite{MS02}

% one sentence
% TC based on TR, already covered

%\paragraph{Occurrence Typing} 
%Occurrence typing~\cite{TF08,TF10} extends the type 
%system with a \emph{proposition environment} that represents 
%the information on the types of bindings down conditional branches.
%These propositions are then used to update the types associated
%with bindings in the \emph{type environment} down branches
%so binding occurrences are given different types 
%depending on the branches they appear in, and the conditionals
%that lead to that branch.

% What's diff about TC from the related work
% small summary for deisel....
% - diesel supports x
%- - calculus supports some subset of x
% we support y, which covers most of x but also foo

% eg. multiple dispatch
%     nominal vs structural

% eg. run abritrary metaprogramming over dispatch in CLOS
%  more expressive

% type systems for mm or rows
% rows vs HMap
% - no poly in HMap
% - based on subtyping
% - rows based on polymorphism

\paragraph{Record Types} 
O'Caml-style extensible record types have been the subject of extensive research 
(eg. Wand~\cite{Wan89}, Cardelli and Mitchell~\cite{CM91}, Harper and Pierce~\cite{HP91})
most inference

Dependent JavaScript~\cite{Chugh:2012:DTJ} can track similar invariants as HMaps with types
for JS objects. They must deal with mutable objects, they feature refinement types and strong updates to the heap
to track changes to objects.

Typed Lua~\cite{Maidl:2014:TLO} has \emph{table types} which track entries in a mutable Lua table.
Typed Lua changes the dynamic semantics of Lua to accommodate mutability: Typed Lua raises a runtime error
for lookups on missing keys---HMaps consider lookups on missing keys normal.


\paragraph{Optional type systems}
\begin{itemize}
  \item Reticulated Python~\cite{Vitousek14}
  \item GradualTalk
  \item Flow
\end{itemize}

Typed Racket is a statically typed sister language of Racket. It
attempts to preserve existing Racket idioms and aims to type check
existing Racket code by simply adding top level type annotations~\cite{Tob10}.

Typed Racket fully expands all macro calls before type checking~\cite{Tob10},
avoiding the complex semantics of type checking macro definitions, an ongoing research area summarised
by~\citet{Her10}. Typed Clojure also expands macros before type checking.

\paragraph{Variable-arity Polymorphism}

~\citet{STF09} invented a type system supporting variable-arity polymorphism  % doesn't fit on a line
a version of which is included in the current implementation of Typed Racket.
Their main innovation centres around \emph{dotted type variables}, which represent a heterogeneous sequence
of types. Dotted type variables allow \emph{non-uniform} variable-arity function types,
which are used to check definitions and usages of functions with non-trivial rest parameters

\paragraph{Java Interoperability in Statically Typed Languages}

Scala~\cite{OCD+} has nullable references for compatibility with Java.
Programmers must manually check for
\java{null} like in Java to avoid null-pointer exceptions. 

% def checkNullOrEmpty(v:Seq[Any]):Boolean = {
%   return v.getClass
% }
% 
% checkNullOrEmpty(null) ;=> NPE
%
% null.getClass


\section{Conclusion}
\label{sec:conclusion}

We have presented Typed Clojure, an optionally-typed version of
Clojure whose type system works with a wide variety of distinctive
Clojure idioms and features. Although based on the foundation of Typed
Racket's occurrence typing approach, Typed Clojure both extends the
fundamental control-flow based reasoning as well as applying it to
handle seemingly unrelated features such as multi-methods. In
addition, Typed Clojure supports crucial features such as heterogenous
maps and Java interoperability while integrating these features into
the core type system.

The result is a sound, expressive, and useful type system which, when
implemented in \coretyped with appropriate extensions, suitable for
typechecking significant amount of existing Clojure programs.
%
As a result, Typed Clojure is already successful: it is widely used in
the Clojure community among both enthusiasts and professional
programmers and recieves contributions from many developers.

However, there is much more that Typed Clojure can provide. Most
significantly, Typed Clojure currently does not provide \emph{gradual
  typing}---ineraction between typed and untyped code is unchecked and
thus unsound. We hope to explore the possibilites of using existing
mechanisms for contracts and proxies in Java and
Clojure~\cite{some-stuff} to enable sound gradual typing for Clojure.

Additionally, the Clojure compiler is unable to use Typed Clojure's
wealth of static information to optimze programs. Addressing this
requires not only first enabling sound gradual typing, but also
integrating Typed Clojure into the Clojure tool chain more deeply, so
that its information can be passed on to the compiler. 

Finally, our case study and broader experience indicate that Clojure
programmers still find themselves unable to use Typed Clojure on some
of their programs for lack of expressiveness. This requires continued
effort to analyze and understand the relevant features and idioms and
develop new type checking approaches for them.


\section{Acknowledgements}

Thanks to Andrew Kent.


% We recommend abbrvnat bibliography style.

\bibliographystyle{abbrvnat}

% The bibliography should be embedded for final submission.

\bibliography{bibliography}

\clearpage

\counterwithin{figure}{section}
\counterwithin{assumption}{section}
\counterwithin{theorem}{section}
\counterwithin{lemma}{section}
\counterwithin{definition}{section}

\onecolumn
\appendix

\section{Soundness for Typed Clojure}

\begin{assumption}[\newjavaliteral] \label{assumption:new}
  If\ $\forall i.\ {\v{i}} = {\classvalue{\classhint{i}}{\overrightarrow{\classfieldpair{\fld{j}} {\v{j}}}}}\ or\ {\v{i}}= {\nil}$
  then
  \begin{itemize}
  \item if $\newjava {\classhint{}}
                {\overrightarrow{\classhint{i}}}
                {\overrightarrow{\v{i}}}
                {\v{}}$
                  then
                  \begin{itemize}
                    \item
                      \v{} = 
                  ${\classvalue{\classhint{}}{\overrightarrow {\classfieldpair{\fld{k}} {\v{k}}}}}$
                  \end{itemize}
  \item
                  otherwise 
        $\newjava {\classhint{}}
                {\overrightarrow{\classhint{i}}}
                {\overrightarrow{\v{i}}}
                {\errorval{\v{e}}}$.
            \end{itemize}
\end{assumption}

\begin{assumption}[\getfieldliteral] \label{assumption:field}
  If\ {\v{1}} = ${\classvalue{\classhint{1}}{{\classfieldpair{\fld{}}{\v{f}}, {\overrightarrow{\classfieldpair{\fld{l}} {\v{l}}}}}}}$,
  then 
  \begin{itemize}
    \item
         if \getfieldjava{\classhint{1}} {\v{1}} {\fld{}} {\classhint{2}} {\v{}}
                  then \v{} = \v{f} and either
                  \begin{itemize}
                    \item
                  \v{f} = ${\classvalue{\classhint{2}}{\overrightarrow{\classfieldpair{\fld{m}} {\v{m}}}}}$
                   or 
                    \item
                  \v{f} = \nil,
                  \end{itemize}
    \item
      otherwise \getfieldjava{\classhint{1}} {\v{1}} {\fld{}} {\classhint{2}} {\errorval{\v{e}}}.
  \end{itemize}
\end{assumption}

\begin{assumption}[\invokejavamethodliteral] \label{assumption:method}
  If\ {\v{1}} = {\classvalue{\classhint{1}}{\overrightarrow{\classfieldpair{\fld{l}} {\v{l}}}}},
  $\forall i.\ {\v{i}}={\classvalue{\classhint{i}}{\overrightarrow{\classfieldpair{\fld{j}} {\v{j}}}}}\ or\ {\v{i}}={\nil}$
         then 
         \begin{itemize}
           \item if
  \invokejavamethod {\classhint{1}} {\v{m}} {mth}
                    {\overrightarrow{\classhint{i}}} {\overrightarrow{\v{i}}}
                    {\classhint{2}}
                    {\v{}}
                  then either
                  \begin{itemize}
                    \item
                  \v{} = ${\classvalue{\classhint{2}}{\overrightarrow{\classfieldpair{\fld{m}} {\v{m}}}}}$
                  or\ 
                  \v{} = \nil,  or
                  \end{itemize}
          \item
            otherwise
  \invokejavamethod {\classhint{1}} {\v{m}} {mth}
                    {\overrightarrow{\classhint{i}}} {\overrightarrow{\v{i}}}
                    {\classhint{2}}
                    {\errorval{\v{e}}}.
        \end{itemize}
\end{assumption}

\begin{lemma} \label{lemma:soundness}
If \judgement{\propenv{}}{\hastype{\e{}}{\t{}}}{\filterset{\thenprop{\prop{}}}{\elseprop{\prop{}}}}{\object{}}
and
\satisfies{\openv{}}{\propenv{}} then
\begin{itemize}
  \item
    if
    \opsem {\openv{}} {\e{}} {\v{}} 
then of the following hold:
\begin{enumerate}
  \item either \object{} = \emptyobject{} or \inopenv {\openv{}} {\object{}} {\v{}},
  \item either \v{} $\not=$ \false\ (or \nil) and {\satisfies{\openv{}}{\thenprop{\prop{}}}} or 
               \v{}       = \false\ (or \nil) and {\satisfies{\openv{}}{\elseprop{\prop{}}}}, and
  \item \judgement{}{\hastype{\v{}}{\t{}}}{\filterset{\thenprop{\propp{}}}{\elseprop{\propp{}}}}{\objectp{}}
        for some \thenprop{\propp{}}, \elseprop{\propp{}} and {\objectp{}}
\end{enumerate}

  \item
otherwise \opsem {\openv{}} {\e{}} {\errorvalv{}}.
\end{itemize}


\begin{proof}
By induction on the derivation of the typing judgement.

\begin{case}[T-True]
\e{} = \true, \t{} = \True, \thenprop{\prop{}} = \topprop{}, \elseprop{\prop{}} = \botprop{}, \object{} = \emptyobject{}

\begin{itemize}
  \item[] 
    \begin{subcase}[B-Val]
      \v{} = \true{}

Proving part 1 is trivial: \object{} is \emptyobject. 
To prove part 2, we note that \v{} = \true\ 
and \thenprop{\prop{}} = \topprop{}, so \satisfies{\openv{}}{\thenprop{\prop{}}} by M-Top.
Part 3 holds as \e{} can only be reduced to itself via B-Val.
\end{subcase}

\end{itemize}

\begin{case}[T-HMap] \e{} = {\curlymapvaloverright{\v{k}}{\v{v}}},
  \t{} = {\HMapc {\mandatory{}}},
  \thenprop{\prop{}} = {\topprop{}},
  \elseprop{\prop{}} = {\botprop{}},
  \object{} = {\emptyobject{}},
  $\overrightarrow{\judgementtwo {} {\hastype{\v{k}}{\Value \k{}}}}$,
  $\overrightarrow{\judgementtwo {} {\hastype{\v{v}}{\t{v}}}}$,
  \mandatory{} = \mandatorysetoverright{\k{}}{\t{v}}

  \begin{itemize}
    \item[]
      \begin{subcase}[B-Val] \v{} = {\curlymapvaloverright{\v{k}}{\v{v}}}

        Similar to T-True.
      \end{subcase}
  \end{itemize}
\end{case}

\begin{case}[T-Kw] \e{} = {\k{}},
  \t{} = {\Value{\k{}}},
  \thenprop{\prop{}} = {\topprop{}},
  \elseprop{\prop{}} = {\botprop{}},
  \object{} = {\emptyobject{}}

  \begin{itemize}
    \item[]
      \begin{subcase}[B-Val] \v{} = {\k{}}

        Similar to T-True.
      \end{subcase}
  \end{itemize}
\end{case}

\end{case}

\begin{case}[T-False]
\e{} = \false, \t{} = \False, \thenprop{\prop{}} = \botprop{}, \elseprop{\prop{}} = \topprop{}, \object{} = \emptyobject{}

\begin{itemize}
  \item[] 
    \begin{subcase}[B-Val]
      \v{} = \false{}

Proving part 1 is trivial: \object{} is \emptyobject. To prove part 2, we note that \v{} = \false\ 
and \elseprop{\prop{}} = \topprop{}, so \satisfies{\openv{}}{\elseprop{\prop{}}} by M-Top. 
Part 3 holds as \e{} can only be reduced to itself via B-Val.
\end{subcase}

\end{itemize}
\end{case}

\begin{case}[T-Nil]
\e{} = \nil, \t{} = \Nil, \thenprop{\prop{}} = \botprop{}, \elseprop{\prop{}} = \topprop{}, \object{} = \emptyobject{},

\begin{itemize}
  \item[] 
    \begin{subcase}[B-Val] 
      \v{} = \nil{}

      Similar to T-False.
\end{subcase}

\end{itemize}

\end{case}

\begin{case}[T-Local]
  \e{} = \x{}, \thenprop{\prop{}} = {\notprop {\falsy{}} {\x{}}},
  \elseprop{\prop{}} = {\isprop {\falsy{}} {\x{}}},
\object{} = \x{}, 
\inpropenv{\propenv{}}{\isprop{\t{}}{\x{}}},

\begin{itemize}
  \item[]
\begin{subcase}[B-Local]
{ \inopenv {\openv{}} {\x{}} {\v{}} },
{ \opsem {\openv{}} {\x{}} {\v{}} }

Part 1 follows from \inopenv{\openv{}}{\x{}} {\v{}} by B-Local.
Part 2 considers two cases: if \v{} is not \false\ or \nil, then 
\satisfies{\openv{}}{\notprop{\falsy}{\x{}}} holds by M-NotType; if \v{} is \false\ or \nil, then 
\satisfies{\openv{}}{\isprop{\falsy}{\x{}}} holds by M-Type.
We prove part 3 by observing
\inpropenv{\propenv{}}{\isprop{\t{}}{\x{}}}
and
\satisfies{\openv{}}{\propenv{}},
so
{ \inopenv {\openv{}} {\x{}} {\v{}} }
by B-Local
gives us the desired result.
\end{subcase}
\end{itemize}

\end{case}

\begin{case}[T-Do]
\e{} = {\doexp {\e1} {\e2}},
  \judgement {\propenv{}} 
             {\hastype {\e1} {\t1}} 
             {\filterset {\thenprop {\prop{1}}} {\elseprop {\prop1}}} 
             {\object{1}},
\judgement {\propenv{}, {\orprop {\thenprop {\prop{1}}} {\elseprop {\prop{1}}}}}
           {\hastype {\e{2}} {\t{}}} 
           {\filterset {\thenprop {\prop{}}} {\elseprop {\prop{}}}} 
           {\object{}},

\begin{itemize}
  \item[] \begin{subcase}[B-Do]
  \opsem {\openv{}} {\e{1}} {\v{1}},
  \opsem {\openv{}} {\e{2}} {\v{}}

For all parts we note 
    since {\e{1}} can be either a true or false value
    then
    {\satisfies{\openv{}}{{\propenv{}},{\orprop {\thenprop {\prop{1}}} {\elseprop {\prop{1}}}}}}
    by M-Or,
    which together with 
\judgement {\propenv{}, {\orprop {\thenprop {\prop{1}}} {\elseprop {\prop{1}}}}}
           {\hastype {\e{2}} {\t{}}} 
           {\filterset {\thenprop {\prop{}}} {\elseprop {\prop{}}}} 
           {\object{}},
    and
  \opsem {\openv{}} {\e{2}} {\v{}}
    allows us to apply the induction hypothesis on \e{2}.

To prove part 1 we use the induction hypothesis on \e{2}
to show either \object{} = \emptyobject{} 
or \inopenv {\openv{}} {\object{}} {\v{}}, since \e{} always
evaluates to the result of \e{2}.

For part 2 we use the induction hypothesis on \e{2}
to show if \v{} $\not=$ \false\ (or \nil) then
        {\satisfies{\openv{}}{\thenprop{\prop{}}}}
        or
  if \v{} = \false\ (or \nil) then
        {\satisfies{\openv{}}{\elseprop{\prop{}}}}.

Part 3 follows from the induction hypothesis on \e{2}.
    \end{subcase}
  \item[]

\begin{subcase}[BE-Do1]
  \opsem {\openv{}} {\e{1}} {\errorval{\v{e}}},
  \opsem {\openv{}} {\e{}} {\errorval{\v{}}}

        Trivially reduces to an error.
\end{subcase}
  \item[]
\begin{subcase}[BE-Do2]
  \opsem {\openv{}} {\e{1}} {\v{1}},
  \opsem {\openv{}} {\e{2}} {\errorvalv{}},
  \opsem {\openv{}} {\e{}} {\errorvalv{}}

        As above.
\end{subcase}
  \item[]

\begin{subcase}[BW-Do1]
  Unreachable by the induction hypothesis on \e{1}.
\end{subcase}
  \item[]

\begin{subcase}[BW-Do2]
  Unreachable by the induction hypothesis on \e{2}.
\end{subcase}
\end{itemize}
\end{case}

\begin{case}[T-NewStatic]
  \e{} = {\newstaticexp {\overrightarrow{\classhint{i}}} {\classhint{}} 
                                                          {\class{}} {\overrightarrow{\e{i}}}},
  \object{} = \emptyobject{},
\thenprop{\prop{}} = \topprop{},
\elseprop{\prop{}} = \botprop{},
   $\overrightarrow{
\javatotc {\classhint{i}}
          {\t{i}}
          }$,
  \javatotc {\classhint{}}
            {\t{}},
            $
  \overrightarrow{
  \judgementtwo {\propenv{}}
                    {\hastype {\e{i}} {\t{i}}}
                  }$

\begin{itemize}
  \item[]
\begin{subcase}[B-New]
  $
  \overrightarrow{
  \opsem {\openv{}}
         {\e{i}}
         {\v{i}}
       }$,
         $\newjava {\classhint{1}}
                  {\overrightarrow{\classhint{i}}}
                  {\overrightarrow{\v{i}}}
                  {\v{}}$

Part 1 follows \object{} = \emptyobject{}.
Part 2 requires some explanation. The two false values in Typed Clojure
cannot be constructed with \newliteral, so the only case is \v{} $\not=$ \false\ (or \nil)
where \thenprop{\prop{}} = \topprop{} so \satisfies{\openv{}}{\thenprop{\prop{}}}.
Part 3 holds as B-New reduces to a \emph{non-nilable}
instance of \class{} via \newjavaliteral (by assumption \ref{assumption:new}), and \javatotc{\classhint{}}{\t{}}.
\end{subcase}
  \item[]

\begin{subcase}[BE-New1] $\overrightarrow{
  \opsem {\openv{}}
         {\e{i-1}}
         {\v{i-1}}
       }$,
  \opsem {\openv{}}
         {\e{i}}
         {\errorvalv{}},
  \opsem {\openv{}} {\e{}} {\errorvalv{}}

        Trivially reduces to an error.

\end{subcase}
  \item[]

\begin{subcase}[BE-New2] 
  $\overrightarrow{
  \opsem {\openv{}}
         {\e{i}}
         {\v{i}}
       }$,
         \newjava {\classhint{1}}
                  {\overrightarrow{\classhint{i}}}
                  {\overrightarrow{\v{i}}}
                  {\errorvalv{}},
        \opsem {\openv{}} {\e{}} {\errorvalv{}}

        As above.

\end{subcase}
  \item[]

\begin{subcase}[BW-New1]
  Unreachable by the induction hypothesis on {\e{i}}.
\end{subcase}
\end{itemize}
\end{case}

\begin{case}[T-FieldStatic]
  \e{} = {\fieldstaticexp {\classhint{1}} {\classhint{2}} {\fld{}} {\e{1}}},
  \javatotc {\classhint{1}} {\class{}},
  \javatotcnil {\classhint{2}} {\t{}},
  \judgementtwo {\propenv{}} {\hastype {\e{1}} {\class{}}}

\begin{itemize}
  \item[]
\begin{subcase}[B-Field]
  \opsem {\openv{}}
         {\e{1}} 
         {\classvalue{\classhint{1}} {\classfieldpair{\fld{}} {\v{}}}}


Part 1 is trivial as \object{} is always \emptyobject{}.
Part 2 holds trivially, \v{} can be either a true or false value
and both {\thenprop{\prop{}}} and {\elseprop{\prop{}}}
are \topprop{}.
Part 3 relies on the semantics of \getfieldliteral (assumption \ref{assumption:field})
in B-Field, which returns a \emph{nilable} instance of \classhint{2},
and \javatotcnil{\classhint{2}} {\t{}}.
\end{subcase}
  \item[]
\begin{subcase}[BE-Field]
  \opsem {\openv{}}
         {\e{1}} 
         {\errorvalv{}},
  \opsem {\openv{}}
         {\e{}}
         {\errorvalv{}}

         Trivially reduces to an error.

\end{subcase}

\item[]
\begin{subcase}[BW-Field]
  Unreachable by the induction hypothesis on \e{1}.
\end{subcase}
\end{itemize}
\end{case}

\begin{case}[T-MethodStatic]
  \e{} = ${\methodstaticexp {\classhint{1}} 
                          {\overrightarrow {\classhint{i}}} 
                          {\classhint{2}}
                          {\mth{}} {\e{m}} {\overrightarrow{\e{a}}}}$,
  \thenprop{\prop{}} = \topprop{}, \elseprop{\prop{}} = \topprop{}, \object{} = \emptyobject{},
  $\overrightarrow{\javatotc {\classhint{i}} {\t{i}}}$,
             \javatotc {\classhint{1}} {\class{}},
             \javatotcnil {\classhint{2}} {\t{}},
  \judgementtwo {\propenv{}} {\hastype {\e{m}} {\class{}}},
             $\overrightarrow{
  \judgementtwo {\propenv{}} {\hastype {\e{a}} {\t{a}}}
                  }$

\begin{itemize}
  \item[]
\begin{subcase}[B-Method]
  \opsem {\openv{}}
         {\e{m}}
         {\v{m}},
  $\overrightarrow{
  \opsem {\openv{}}
         {\e{a}}
         {\v{a}}}$,
  \invokejavamethod {\classhint{1}} {\v{m}} {mth}
                    {\overrightarrow{\classhint{a}}} {\overrightarrow{\v{a}}}
                    {\classhint{2}}
                    {\v{}}


Part 1 is trivial as \object{} is always \emptyobject{}.
Part 2 holds trivially, \v{} can be either a true or false value
and both {\thenprop{\prop{}}} and {\elseprop{\prop{}}}
are \topprop{}.
Part 3 relies on the semantics of \invokejavamethodliteral (assumption \ref{assumption:method})
in B-Method, which returns a \emph{nilable} instance of \classhint{2},
and \javatotcnil{\classhint{2}} {\t{}}.
\end{subcase}

  \item[]
\begin{subcase}[BE-Method1]
  \opsem {\openv{}}
         {\e{m}}
         {\errorval{\v{}}},
  \opsem {\openv{}}
         {\e{}}
         {\errorval{\v{}}}

         Trivially reduces to an error.
\end{subcase}
  \item[]
\begin{subcase}[BE-Method2]
  \opsem {\openv{}}
         {\e{m}}
         {\v{m}},
 $\overrightarrow{
  \opsem {\openv{}}
         {\e{n-1}}
         {\v{n-1}}
       }$,
  \opsem {\openv{}}
         {\e{n}}
         {\errorval{\v{}}},
  \opsem {\openv{}}
         {\e{}}
         {\errorval{\v{}}}

  As above.
\end{subcase}
  \item[]
\begin{subcase}[BE-Method3]
  \opsem {\openv{}}
         {\e{m}}
         {\v{m}},
  $\overrightarrow{
  \opsem {\openv{}}
         {\e{a}}
         {\v{a}}
       }$,
  \invokejavamethod {\classhint{1}} {\v{m}} {mth}
                    {\overrightarrow{\classhint{a}}} {\overrightarrow{\v{a}}}
                    {\classhint{2}}
                    {\errorvalv{}},
  \opsem {\openv{}} {\e{}} {\errorvalv{}}

  As above.

\end{subcase}
  \item[]
\begin{subcase}[BW-Method1]
\end{subcase}
  \item[]
\begin{subcase}[BW-Method2]
\end{subcase}
\end{itemize}

\end{case}

\begin{case}[T-DefMulti]
  \e{} = {\createmultiexp {\s{}} {\e{d}}},
  \t{} = {\MultiFntype {\s{}} {\t{d}}},
  \thenprop{\prop{}} = {\topprop{}},
  \elseprop{\prop{}} = {\botprop{}},
  \s{} = {\ArrowOne {\x{}} {\t{1}} {\t{2}}
                          {\filterset {\thenprop {\prop{1}}}
                                      {\elseprop {\prop{1}}}}
                          {\object{1}}},
  \t{d} = {\ArrowOne {\x{}} {\t{1}} {\t{3}}
                          {\filterset {\thenprop {\prop{2}}}
                                      {\elseprop {\prop{2}}}}
                          {\object{2}}},
  \judgementtwo {\propenv{}} {\hastype {\e{d}} {\t{d}}}


\begin{itemize}
  \item[]
\begin{subcase}[B-DefMulti]
  \v{} = {\multi {\v{d}} {\emptydisptable}},
  \opsem {\openv{}} {\e{d}} {\v{d}}


Part 1 and 2 hold for the same reasons as T-True.
For part 3 we show \judgementtwo{}{\hastype{\multi {\v{d}} {\emptydisptable}}{\MultiFntype {\s{}} {\t{d}}}}
by T-Multi, since \judgementtwo {} {\hastype {\v{d}} {\t{d}}} by the inductive hypothesis on {\e{d}}
and {\emptydisptable} vacuously satisfies the other premises of T-Multi, so we are done.

\end{subcase}
  \item[]
\begin{subcase}[BE-DefMulti] \opsem {\openv{}} {\e{d}} {\errorvalv{}},
        \opsem {\openv{}} {\e{}} {\errorvalv{}}

        Trivially reduces to an error.

\end{subcase}
\end{itemize}
\end{case}

\begin{case}[T-DefMethod]
  \e{} = {\extendmultiexp {\e{m}} {\e{v}} {\e{f}}},
  \e{f} = {\abs {\x{}} {\t{1}} {\e{b}}},
  \t{} = {\MultiFntype {\t{m}} {\t{d}}},
  \thenprop{\prop{}} = {\topprop{}},
  \elseprop{\prop{}} = {\botprop{}},
  \object{} = {\emptyobject{}},
  \t{m} = {\ArrowOne {\x{}} {\t{1}} {\s{}}
                     {\filterset {\thenprop {\prop{m}}}
                                 {\elseprop {\prop{m}}}}
                     {\object{m}}},
  \t{d} = {\ArrowOne {\x{}} {\t{1}} {\sp{}}
                     {\filterset {\thenprop {\prop{d}}}
                                 {\elseprop {\prop{d}}}}
                     {\object{d}}},
  \judgementtwo {\propenv{}} {\hastype {\e{m}} {\MultiFntype {\t{m}} {\t{d}}}},
  \isacompare{\sp{}}{\object{d}}{\t{v}}{\filterset {\thenprop {\prop{i}}} {\elseprop {\prop{i}}}},
  \judgementtwo {\propenv{}}
               {\hastype {\e{v}} {\t{v}}},
  \judgement {\propenv{}, {\isprop{\t{1}} {\x{}}}, {\thenprop {\prop{i}}}}
           {\hastype {\e{b}} {\s{}}}
           {\filterset {\thenprop {\prop{m}}}
                       {\elseprop {\prop{m}}}}
           {\object{m}}

  \begin{itemize}
    \item[]
      \begin{subcase}[B-DefMethod]
       \v{} = {\multi {\v{d}} {\disptablep{}}},
        \opsem {\openv{}}
               {\e{m}}
               {\multi {\v{d}} {\disptable{}}},
  \opsem {\openv{}}
         {\e{v}}
         {\v{v}},
  \opsem {\openv{}}
         {\e{f}}
         {\v{f}},
         \disptablep{} = {\extenddisptable {\disptable{}} 
                                {\v{v}}
                                {\v{f}}}

                                Part 1 and 2 hold for the same reasons as T-True, noting that the propositions
                                and object agree with T-Multi.

For part 3 we show
\judgementtwo{}{\hastype{\multi {\v{d}} {\extenddisptable {\disptable{}}{\v{v}}{\v{f}}}}{\MultiFntype {\t{m}} {\t{d}}}}
by noting \judgementtwo {} {\hastype {\v{d}} {\t{d}}},
  \judgementtwo{}{\hastype{\v{v}}{\Top{}}}
  and
  \judgementtwo{}{\hastype{\v{f}}{\t{m}}}, and since \disptable{} is in the correct form by the inductive
  hypothesis on {\e{m}} we can satisfy all premises of T-Multi, so we are done.


      \end{subcase}

    \item[]
      \begin{subcase}[BE-DefMethod1]
        \opsem {\openv{}}
               {\e{m}}
               {\errorval{\v{}}},
        \opsem {\openv{}}
                  {\e{}}
                {\errorval{\v{}}}

                Trivially reduces to an error.

      \end{subcase}
    \item[]
      \begin{subcase}[BE-DefMethod2]
        \opsem {\openv{}}
         {\e{m}}
         {\multi {\v{d}} {\disptable{}}},
  \opsem {\openv{}}
         {\e{v}}
         {\errorval{\v{}}},
        \opsem {\openv{}}
                  {\e{}}
                {\errorval{\v{}}}

                As above.
      \end{subcase}
    \item[]
      \begin{subcase}[BE-DefMethod3]
        \opsem {\openv{}}
         {\e{m}}
         {\multi {\v{d}} {\disptable{}}},
  \opsem {\openv{}}
         {\e{v}}
         {\v{v}},
  \opsem {\openv{}}
         {\e{f}}
         {\errorval{\v{}}},
        \opsem {\openv{}}
                  {\e{}}
                {\errorval{\v{}}}

                As above.

      \end{subcase}
  \end{itemize}
\end{case}

      %TODO
\begin{case}[T-App]
  \e{} = {\appexp {\e{1}} {\e{2}}},
  \t{} = {\replacefor {\t{f}}
                      {\object{2}}
                      {\x{}}},
  {\thenprop {\prop{}}} = 
                 {\replacefor {\thenprop {\prop{f}}}
                              {\object{2}}
                              {\x{}}},
  {\elseprop {\prop{}}} = 
                 {\replacefor {\elseprop {\prop{f}}}
                              {\object{2}}
                              {\x{}}},
  \object{} = {\replacefor {\object{f}}
                           {\object{2}}
                           {\x{}}},
  \judgement {\propenv{}} {\hastype {\e{1}} {\ArrowOne {\x{}} {\s{}}
                                                       {\t{f}}
                                                       {\filterset {\thenprop {\prop{f}}}
                                                                   {\elseprop {\prop{f}}}}
                                                       {\object{f}}}}
                {\filterset {\thenprop {\prop{1}}}
                            {\elseprop {\prop{1}}}}
                {\object{1}},
  \judgement {\propenv{}}
                 {\hastype {\e{2}} {\s{}}}
                 {\filterset {\thenprop {\prop{2}}}
                             {\elseprop {\prop{2}}}}
                 {\object{2}},
  \opsem{\openv{}}{\e{}}{\v{}}
%                 IHe1 = $\forall{\openv{1}},
%                               {\v{1}},
%                               {\thenprop {\prop{1}}},
%                               {\elseprop {\prop{1}}},
%                               {\object{1}}.
%                               {\judgement
%                                 {\propenv{}}
%                                 {\hastype
%                                   {\e{1}}
%                                   {\ArrowOne {\x{}} {\s{}}
%                                     {\t{f}}
%                                     {\filterset 
%                                       {\thenprop {\prop{f}}}
%                                       {\elseprop {\prop{f}}}}
%                                     {\object{f}}}}
%                                {\filterset {\thenprop {\prop{1}}}
%                                            {\elseprop {\prop{1}}}}
%                                {\object{1}}}
%\Rightarrow
%                                                       {\satisfies{\openv{1}}{\propenv{}}}
%                                                       \Rightarrow
%  \opsem {\openv{}}
%         {\e{1}}
%         {\v{1}}
%         \Rightarrow
%         (1: \object{1} = \emptyobject{} \vee \inopenv{\openv{1}}{\object{1}}{\v{1}})
%         \wedge
%       (2: (\v{1} \not= \false\ (or\ \nil) \wedge \satisfies{\openv{1}}{\thenprop{\prop{1}}})
%          \vee
%            (\v{1} = \false\ (or\ \nil) \wedge \satisfies{\openv{1}}{\elseprop{\prop{1}}}))
%            \wedge
%          (3: \exists {\thenprop{\propp{1}}},{\elseprop{\propp{1}}},{\objectp{1}}.
%          \judgement{}{\hastype{\v{1}}{\ArrowOne {\x{}} {\s{}}
%                                     {\t{f}}
%                                     {\filterset 
%                                       {\thenprop {\prop{f}}}
%                                       {\elseprop {\prop{f}}}}
%                                   {\object{f}}}}
%                        {\filterset{\thenprop{\propp{1}}}{\elseprop{\propp{1}}}}
%                      {\objectp{1}})
%                                                       $









\begin{itemize}
  \item[]
\begin{subcase}[B-BetaClosure]
  \opsem {\openv{}}
         {\e{1}}
         {\closure {\openv{c}} {\abs {\x{}} {\s{}} {\e{b}}}},
  \opsem {\openv{}}
         {\e{2}}
         {\v{2}},
  \opsem {\extendopenv {\openv{c}} {\x{}} {\v{2}}}
         {\e{b}}
         {\v{}}

         To prove part 1, we know by inversion on \e{1} from T-Clos
         there is some environment {\propenvc{}} such that
         \begin{itemize}
           \item
              \satisfies{\openv{c}}{\propenvc{}} and
            \item
  \judgement {\propenvc{}} {\hastype {\abs {\x{}} {\s{}} {\e{b}}} {\ArrowOne {\x{}} {\s{}}
                                                       {\t{f}}
                                                       {\filterset {\thenprop {\prop{f}}}
                                                                   {\elseprop {\prop{f}}}}
                                                       {\object{f}}}}
                {\filterset {\thenprop {\prop{1}}}
                            {\elseprop {\prop{1}}}}
                {\object{1}},
         \end{itemize}
         and also by inversion on \e{1} from T-Abs
         \begin{itemize}
           \item
  { \judgement {\propenvc{}, {\isprop {\s{}} {\x{}}}}
              {\hastype {\e{b}} {\t{f}}}
               {\filterset {\thenprop {\prop{f}}}
                           {\elseprop {\prop{f}}}}
               {\object{f}}}.
         \end{itemize}

         We consider cases on the form of \object{f}:

         \begin{itemize}
           \item if \object{f} = \emptyobject{}, then \object{} = \emptyobject{} by substitution,
           \item if \object{f} = {\path {\pathelem{f}} {\x{}}}, 
             we consider cases on the form of \object{2}:
             \begin{itemize}
               \item if \object{2} = \emptyobject{}, then \object{} = \emptyobject{},
               \item otherwise \object{2} = {\path {\pathelem{2}} {\y{}}}, where,
                 because
                \object{} = {\replacefor {\path {\pathelem{f}} {\x{}}}
                                         {\path {\pathelem{2}} {\y{}}}
                                         {\x{}}},
                                         by substitution
                                         \object{} = {\path {\pathelem{f}} {\path {\pathelem{2}} {\y{}}}}.
                                         To prove \inopenv{\openv{}}{\path {\pathelem{f}} {\path {\pathelem{2}} {\y{}}}}{\v{}}
                                         we note by the induction hypothesis on {\e{2}} that 
                                         \inopenv{\openv{}}{\path {\pathelem{2}} {\y{}}}{\v{2}}
  \opsem {\extendopenv {\openv{c}} {\x{}} {\v{2}}}
         {\e{b}}
         {\v{}}

             \end{itemize}
           \item otherwise \object{f} = {\path {\pathelem{f}} {\y{}}} where $\x{} \not= \y{}$
         \end{itemize}


         To prove part 2, if \v{} $\not=$ \false\ (or \nil)
\end{subcase}
  \item[]
\begin{subcase}[B-BetaMulti]
  \opsem {\openv{}}
         {\e{1}}
         {\multi {\v{d}} {m}},
  \opsem {\openv{}}
         {\e{2}}
         {\v{2}},
  \opsem {\openv{}}
         {\appexp {\v{d}} {\v{2}}}
         {\v{e}},
  \getmethod {\disptable{}}
             {\v{e}}
             {\v{f}},
  \opsem {\openv{}}
         {\appexp {\v{f}} {\v{2}}}
         {\v{}}

\end{subcase}
  \item[]
\begin{subcase}[B-Delta]
  \opsem {\openv{}} {\e{1}} {\const{}},
  \opsem {\openv{}} {\e{2}} {\v{2}},
  \constantopsem{\const{}}{\v{2}} = \v{}

  % TODO do I need to prove anything about the argument in the definition
  % of the constant being under \s{}?

  Prove by cases on \const{}.
  \begin{itemize}
    \item[] \begin{subcase}[\const{} = \classconst]
    ${\ArrowOne {\x{}} {\s{}}
                                                       {\t{f}}
                                                       {\filterset {\thenprop {\prop{f}}}
                                                                   {\elseprop {\prop{f}}}}
                                                       {\object{f}}}$
                                                       =
  ${\ArrowOne {\x{}} {\Top{}}
                                      {\Union{\nil{}}{\Class{}}}
                                      {\filterset {\topprop{}}
                                                  {\topprop{}}}
                                      {\path {\classpe{}} {\x{}}}}$

    Prove by cases on \v{2}.

        \begin{itemize}
          \item[] \begin{subcase}[\v{2} = \classvalue{\class{}} {\overrightarrow {\classfieldpair{\fld{i}} {\v{i}}}}]
                    \v{} = \class{}

                    To prove part 1, note \object{} = {\replacefor {\object{f}}{\object{2}}{\x{}}} 
                    and \object{f} = {\path {\classpe{}} {\x{}}}.
                    There are two cases defined by substitution: if \object{2} = \emptyobject{} then \object{} = \emptyobject{}
                    and we are done,
                    or if \object{2} = {\path {\pathelem{}} {\xp{}}} then \object{} = 
                    {\path {\classpe{}}{\object{2}}},
                    by the induction hypothesis on \e{2} \inopenv {\openv{}} {\object{2}} {\v{2}}
                    and by the definition of path translation
                    {\openv{}}({\path {\classpe{}} {\object{2}}}) = {\appexp {\classconst{}} {{\openv{}}(\object{2})}},
                    which evaluates to \v{}.

                    Part 2 is trivial since both propositions are \topprop{} by substitution.
                    
                    Part 3 holds because {\judgementtwo{}{\hastype{\v{2}}{\Class{}}}}.
                  \end{subcase}
          \item[] \begin{subcase}[\v{2} = \class{}] \v{} = \Class{}

              As above.
                  \end{subcase}
          \item[] \begin{subcase}[\v{2} = \true{}] \v{} = \Boolean{}

              As above.
                  \end{subcase}
          \item[] \begin{subcase}[\v{2} = \false{}] \v{} = \Boolean{}


              As above.
                  \end{subcase}
          \item[] \begin{subcase}[\v{2} = {\closure {\openv{}} {\abs {\x{}} {\t{}} {\e{}}}}] \v{} = \IFn{}


              As above.
                  \end{subcase}
          \item[] \begin{subcase}[\v{2} = {\multi {\v{d}} {\disptable{}}}] \v{} = \HMapInstance{}


              As above.
                  \end{subcase}
          \item[] \begin{subcase}[\v{2} = {\curlymapvaloverright{\v{1}}{\v{2}}}] \v{} = \Keyword{}


              As above.
                  \end{subcase}
          \item[] \begin{subcase}[\v{2} = {\nil{}}] \v{} = \nil{}

             Parts 1 and 2 as above.
                    Part 3 holds because {\judgementtwo{}{\hastype{\v{2}}{\Nil}}}.
                  \end{subcase}
        \end{itemize}
      \end{subcase}
    \item[]
      \begin{subcase}[\const{} = \throwconst]
        {\ArrowOne {\x{}} {\s{}}
                                                       {\t{f}}
                                                       {\filterset {\thenprop {\prop{f}}}
                                                                   {\elseprop {\prop{f}}}}
                                                       {\object{f}}}
                                                       =
        {\ArrowOne {\x{}} {\Top{}}
                                      {\Bot{}}
                                      {\filterset {\botprop{}}
                                                  {\botprop{}}}
                                      {\emptyobject{}}}

                                      Part 1 is trivial since \object{} = \emptyobject{} after substition.
                                      Part 2 holds vacuously as both propositions are \botprop{} after substition.
                                      Finally part 3 holds since {\judgementtwo{}{\hastype{\errorval{\v{2}}}{\Bot{}}}}.

      \end{subcase}
  \end{itemize}

\end{subcase}
  \item[]
\begin{subcase}[BE-Beta1]
\end{subcase}
  \item[]
\begin{subcase}[BE-Beta2]
\end{subcase}
  \item[]
\begin{subcase}[BE-BetaClosure]
\end{subcase}
  \item[]
\begin{subcase}[BE-BetaMulti1]
\end{subcase}
  \item[]
\begin{subcase}[BE-BetaMulti2]
\end{subcase}
\end{itemize}
\end{case}

\begin{case}[T-IsA]\e{} = {\isaapp {\e{1}} {\e{2}}},
  \t{} = {\Boolean{}},
  \judgement {\propenv{}} {\hastype {\e{1}} {\t{1}}}
             {\filterset {\thenprop {\propp{}}}
                         {\elseprop {\propp{}}}}
                       {\object{1}},
  \judgementtwo {\propenv{}} {\hastype {\e{2}} {\t{2}}},
  \isacompare{\t{1}}{\object{1}}{\t{2}}{\filterset {\thenprop {\prop{}}} {\elseprop {\prop{}}}},
  \object{} = \emptyobject{}

  \begin{itemize}
    \item[]
      \begin{subcase}[B-IsA]
  \opsem {\openv{}} {\e{1}} {\v{1}},
  \opsem {\openv{}} {\e{2}} {\v{2}},
  \isaopsem{\v{1}}{\v{2}} = {\v{}}

  From the definition of \isacompareliteral,
  \thenprop{\prop{}} = {\replacefor{\isprop{\t{2}}{\x{}}}{\object{1}}{\x{}}} and
  \elseprop{\prop{}} = {\replacefor{\notprop{\t{2}}{\x{}}}{\object{1}}{\x{}}}.

  Part 1 holds trivially with \object{} = \emptyobject{}.
  For part 2, if \v{} $\not=$ \false\ (or \nil)
  then {\satisfies{\openv{}}{\replacefor{\isprop{\t{2}}{\x{}}}{\object{1}}{\x{}}}}, as by the definition
  of \isaopsemliteral either 
  \begin{itemize}
    \item \v{1} = \v{2} so {\v{2}} must be the same type as down {\object{1}}, or
    \item \v{1} = \classvaluemeta{1} and \v{2} = \classvaluemeta{2} where \classvaluemeta{1}
  is a subclass of \classvaluemeta{2}, so {\v{2}} may be be safely upcast to the same type as down {\object{1}}.
  \end{itemize}
  Part 3 holds because by the definition of \isaopsemliteral
  \v{} can only be \true or \false, which are both subtypes of
  \t{}.


      \end{subcase}
    \item[]
      \begin{subcase}[BE-IsA1]
      \end{subcase}
    \item[]
      \begin{subcase}[BE-IsA2]
      \end{subcase}
  \end{itemize}
\end{case}

\begin{case}[T-GetHMap]
  \e{} = {\getexp {\e{m}} {\e{k}}},
  $\t{} = {\Unionsplice {\overrightarrow {\t{i}}}}$
  \thenprop{\prop{}} = {\topprop{}},
  \elseprop{\prop{}} = {\topprop{}},
  \object{} = {\replacefor {\path {\keype{k}} {\x{}}}
                          {\object{m}}
                          {\x{}}},
  $\judgementtwo {\propenv{}} {\hastype {\e{k}} {\Value {k}}}$,
  $\judgement {\propenv{}} {\hastype {\e{m}} {\Unionsplice {\overrightarrow {\HMapgeneric {\mandatory{}} {\absent{}}}}}}
           {\filterset {\thenprop {\prop{m}}} {\elseprop {\prop{m}}}}
           {\object{m}}$,
  $\overrightarrow{\inmandatory{\k{}}{\t{i}}{\mandatory{}}}$


  \begin{itemize}
    \item[]
      \begin{subcase}[B-Get]
      $\opsem {\openv{}} {\e{m}}{\v{m}}$,
        $\v{m} = {\curlymap{\overrightarrow{({\v{a}}\ {\v{b}})}}}$,
         \opsem {\openv{}} {\e{k}} {\k{}},
         $\keyinmap{\k{}}{\curlymap{\overrightarrow{({\v{a}}\ {\v{b}})}}}$,
         \getmap{\curlymap{\overrightarrow{({\v{a}}\ {\v{b}})}}} {\k{}} = {\v{}}

         To prove part 1 we consider two cases on the form of \object{m}: 
         \begin{itemize}
           \item
         if {\object{m}} = \emptyobject{}
         then \object{} = \emptyobject{} by substitution, which gives the desired result;
           \item
         if \object{m} = {\path {\pathelem{m}} {\x{m}}}
         then \object{} = {\path {\keype{k}} {\object{m}}} by substitution.
         We note by the definition of path translation
         {\openv{}}({\path {\keype{k}} {\object{m}}}) =
         {\getexp {{\openv{}}(\object{m})}{\k{}}}
         and by the induction hypothesis on \e{m}
         {{\openv{}}(\object{m})} = {\curlymap{\overrightarrow{({\v{a}}\ {\v{b}})}}},
         which together imply 
         \inopenv {\openv{}} {\object{}} {\getexp {\curlymap{\overrightarrow{({\v{a}}\ {\v{b}})}}} {\k{}}}.
         Since this is the same form as B-Get, we can apply the premise
         \getmap{\curlymap{\overrightarrow{({\v{a}}\ {\v{b}})}}} {\k{}} = {\v{}}
         to derive \inopenv {\openv{}} {\object{}} {\v{}}.
         \end{itemize}
         
         Part 2 holds trivially as \thenprop{\prop{}} = {\topprop{}}
         and \elseprop{\prop{}} = {\topprop{}}.

         To prove part 3 we note that (by the induction hypothesis on \e{m})
         $\judgementtwo{}{\hastype{\v{m}}{\Unionsplice{\overrightarrow {\HMapgeneric {\mandatory{}} {\absent{}}}}}}$,
         where $\overrightarrow{\inmandatory{\k{}}{\t{i}}{\mandatory{}}}$, and 
         both
         $\keyinmap{\k{}}{\curlymap{\overrightarrow{({\v{a}}\ {\v{b}})}}}$
         and
         \getmap{\curlymap{\overrightarrow{({\v{a}}\ {\v{b}})}}} {\k{}} = {\v{}}
         imply \judgementtwo{}{\hastype{\v{}}{\Unionsplice {\overrightarrow {\t{i}}}}}.

      \end{subcase}
    \item[]
      \begin{subcase}[B-GetMissing]
        \v{} = \nil,
        $\opsem {\openv{}}
        {\e{m}} {\curlymap{\overrightarrow{({\v{a}}\ {\v{b}})}}}$,
       \opsem {\openv{}} {\e{k}} {\k{}},
       \keynotinmap{\k{}}{\curlymap{\overrightarrow{({\v{a}}\ {\v{b}})}}}

       Unreachable subcase because 
       \keynotinmap{\k{}}{\curlymap{\overrightarrow{({\v{a}}\ {\v{b}})}}}
       contradicts ${\inmandatory{\k{}}{\t{}}{\mandatory{}}}$.
      \end{subcase}
    \item[]
      \begin{subcase}[BE-Get1]
      \end{subcase}
    \item[]
      \begin{subcase}[BE-Get2]
      \end{subcase}
  \end{itemize}
\end{case}

\begin{case}[T-GetHMapAbsent]
  \e{} = {\getexp {\e{m}} {\e{k}}},
  \t{} = \Nil,
  \thenprop{\prop{}} = {\topprop{}},
  \elseprop{\prop{}} = {\topprop{}},
  \object{} = {\replacefor
               {\path {\keype{k}} {\x{}}}
                          {\object{m}}
                          {\x{}}},
  \judgementtwo {\propenv{}} {\hastype {\e{k}} {\Value {k}}},
  \judgement {\propenv{}} {\hastype {\e{m}} {\HMapgeneric {\mandatory{}} {\absent}}}
           {\filterset {\thenprop {\prop{m}}} {\elseprop {\prop{m}}}}
           {\object{m}},
  {\inabsent{\k{}}{\absent{}}}


  \begin{itemize}
    \item[]
      \begin{subcase}[B-Get]
        $\opsem {\openv{}}
        {\e{m}} {\curlymap{\overrightarrow{({\v{a}}\ {\v{b}})}}}$
        ,
         \opsem {\openv{}} {\e{k}} {\k{}},
         $\keyinmap{\k{}}{\curlymap{\overrightarrow{({\v{a}}\ {\v{b}})}}}$,
         \getmap{\curlymap{\overrightarrow{({\v{a}}\ {\v{b}})}}} {\k{}} = {\v{}}

       Unreachable subcase because 
         $\keyinmap{\k{}}{\curlymap{\overrightarrow{({\v{a}}\ {\v{b}})}}}$,
         contradicts
                {\inabsent{\k{}}{\absent{}}}.
      \end{subcase}
    \item[]
      \begin{subcase}[B-GetMissing]
        \v{} = \nil,
        $\opsem {\openv{}}
        {\e{m}} {\curlymap{\overrightarrow{({\v{a}}\ {\v{b}})}}}$,
       \opsem {\openv{}} {\e{k}} {\k{}},
       \keynotinmap{\k{}}{\curlymap{\overrightarrow{({\v{a}}\ {\v{b}})}}}

         To prove part 1 we consider two cases on the form of \object{m}: 
         \begin{itemize}
           \item
         if {\object{m}} = \emptyobject{}
         then \object{} = \emptyobject{} by substitution, which gives the desired result;
           \item
         if \object{m} = {\path {\pathelem{m}} {\x{m}}}
         then \object{} = {\path {\keype{k}} {\object{m}}} by substitution.
         We note by the definition of path translation
         {\openv{}}({\path {\keype{k}} {\object{m}}}) =
         {\getexp {{\openv{}}(\object{m})}{\k{}}}
         and by the induction hypothesis on \e{m}
         {{\openv{}}(\object{m})} = {\curlymap{\overrightarrow{({\v{a}}\ {\v{b}})}}},
         which together imply 
         \inopenv {\openv{}} {\object{}} {\getexp {\curlymap{\overrightarrow{({\v{a}}\ {\v{b}})}}} {\k{}}}.
         Since this is the same form as B-GetMissing, we can apply the premise
        \v{} = \nil\ 
         to derive \inopenv {\openv{}} {\object{}} {\v{}}.
         \end{itemize}
         
         Part 2 holds trivially as \thenprop{\prop{}} = {\topprop{}}
         and \elseprop{\prop{}} = {\topprop{}}.
         To prove part 3 we note that \e{m} has type {\HMapgeneric {\mandatory{}} {\absent{}}}
         where {\inabsent{\k{}}{\absent{}}}, and
         the premises of B-GetMissing
         \keynotinmap{\k{}}{\curlymap{\overrightarrow{({\v{a}}\ {\v{b}})}}}
         and
          \v{} = \nil\ 
         tell us {\v{}} must be of type {\t{}}.
      \end{subcase}
    \item[]
      \begin{subcase}[BE-Get1]
      \end{subcase}
    \item[]
      \begin{subcase}[BE-Get2]
      \end{subcase}
  \end{itemize}
\end{case}

\begin{case}[T-GetHMapPartialDefault]
  \e{} = {\getexp {\e{m}} {\e{k}}},
  \t{} = \Top,
  \thenprop{\prop{}} = {\topprop{}},
  \elseprop{\prop{}} = {\topprop{}},
  \object{} = {\replacefor
               {\path {\keype{k}} {\x{}}}
                          {\object{m}}
                          {\x{}}},
  \judgementtwo {\propenv{}} {\hastype {\e{k}} {\Value {k}}},
 \judgement {\propenv{}} {\hastype {\e{m}} {\HMapp {\mandatory{}} {\absent}}}
           {\filterset {\thenprop {\prop{m}}} {\elseprop {\prop{m}}}}
           {\object{m}},
             ${\notinmandatory{\k{}}{\t{}}{\mandatory{}}}$,
             {\notinabsent{\k{}}{\absent{}}}

  \begin{itemize}
    \item[]
      \begin{subcase}[B-Get]
        $\opsem {\openv{}}
        {\e{m}} {\curlymap{\overrightarrow{({\v{a}}\ {\v{b}})}}}$
        ,
         \opsem {\openv{}}
                 {\e{k}} {\k{}},
         $\keyinmap{\k{}}{\curlymap{\overrightarrow{({\v{a}}\ {\v{b}})}}}$,
         \getmap{\curlymap{\overrightarrow{({\v{a}}\ {\v{b}})}}} {\k{}} = {\v{}}

         Parts 1 and 2 are the same as the B-Get subcase of T-GetHMap.
         Part 3 is trivial as \t{} = \Top.


      \end{subcase}
    \item[]
      \begin{subcase}[B-GetMissing]
        \v{} = \nil,
        $\opsem {\openv{}}
        {\e{m}} {\curlymap{\overrightarrow{({\v{a}}\ {\v{b}})}}}$,
       \opsem {\openv{}} {\e{k}} {\k{}},
       \keynotinmap{\k{}}{\curlymap{\overrightarrow{({\v{a}}\ {\v{b}})}}}

         Parts 1 and 2 are the same as the B-GetMissing subcase of T-GetHMapAbsent.
         Part 3 is trivial as \t{} = \Top.

      \end{subcase}
    \item[]
      \begin{subcase}[BE-Get1]
      \end{subcase}
    \item[]
      \begin{subcase}[BE-Get2]
      \end{subcase}
  \end{itemize}
\end{case}

\begin{case}[T-AssocHMap]
  \e{} = {\assocexp {\e{m}} {\e{k}} {\e{v}}},
  \t{} = ${\HMapgeneric {\extendmandatoryset {\mandatory{}}{\k{}}{\t{}}} {\absent}}$,
  \thenprop{\prop{}} = {\topprop{}},
  \elseprop{\prop{}} = {\botprop{}},
  \object{} = \emptyobject,
  \judgementtwo {\propenv{}} {\hastype {\e{m}} {\HMapgeneric {\mandatory{}} {\absent}}},
  \judgementtwo {\propenv{}} {\hastype {\e{k}} {\Value{\k{}}}},
  \judgementtwo {\propenv{}} {\hastype {\e{v}} {\t{}}},
  {\k{}} $\not\in$ {\absent{}}

  \begin{itemize}
    \item[]
      \begin{subcase}[B-Assoc]
        \v{} = 
        {\extendmap{\curlymap{\overrightarrow{({\v{a}}\ {\v{b}})}}}
                {\k{}}{\v{v}}},
        \opsem {\openv{}}
        {\e{m}} {\curlymap{\overrightarrow{({\v{a}}\ {\v{b}})}}},
        \opsem {\openv{}} {\e{k}} {\k{}},
        \opsem {\openv{}} {\e{v}} {\v{v}}

        Parts 1 and 2 hold for the same reasons as T-True.
        %TODO part 3
      \end{subcase}
    \item[]
      \begin{subcase}[BE-Assoc1]
      \end{subcase}
    \item[]
      \begin{subcase}[BE-Assoc2]
      \end{subcase}
    \item[]
      \begin{subcase}[BE-Assoc3]
      \end{subcase}
  \end{itemize}
\end{case}

\begin{case}[T-If] \e{} = {\ifexp {\e1} {\e2} {\e3}},
  \judgement {\propenv{}} {\hastype {\e1} {\t{1}}} {\filterset {\thenprop {\prop{1}}} {\elseprop {\prop{1}}}}
                 {\object{1}},
  \judgement {\propenv{}, {\thenprop {\prop{1}}}}
                 {\hastype {\e2} {\t{}}} {\filterset {\thenprop {\prop{2}}} {\elseprop {\prop{2}}}}
                 {\object{}},
  \judgement {\propenv{}, {\elseprop {\prop{1}}}}
                 {\hastype {\e3} {\t{}}} {\filterset {\thenprop {\prop{3}}} {\elseprop {\prop{3}}}}
                 {\object{}},
  \thenprop{\prop{}} = {\orprop {\thenprop {\prop{2}}} {\thenprop {\prop{3}}}},
  \elseprop{\prop{}} = {\orprop {\elseprop {\prop{2}}} {\elseprop {\prop{3}}}}

  \begin{itemize}
    \item[]
      \begin{subcase}[B-IfTrue]
        \opsem {\openv{}} {\e{1}} {\v{1}},
              ${\v{1}} \not= {\false}$,
              ${\v{1}} \not= {\nil}$,
              \opsem {\openv{}} {\e{2}} {\v{}}

              For part 1, either \object{} = \emptyobject{}, or \e{} evaluates to the
              result of \e{2}.

              To prove part 2, we consider two cases:
              \begin{itemize}
                \item if \v{} = \false\ (or \nil)
                  then \e{2} evaluates to a false value so {\satisfies{\openv{}}{\elseprop {\prop{2}}}}, and thus
                  {\satisfies{\openv{}}{\orprop {\elseprop {\prop{2}}} {\elseprop {\prop{3}}}}} by M-Or, 
                \item otherwise
                  ${\v{}} \not= {\false}$ and
                  ${\v{}} \not= {\nil}$,
                  so \e{2} evaluates to a true value so {\satisfies{\openv{}}{\thenprop {\prop{2}}}}, and thus
                  {\satisfies{\openv{}}{\orprop {\thenprop {\prop{2}}} {\thenprop {\prop{3}}}}} by M-Or.
              \end{itemize}

              Part 3 is trivial as
              \opsem {\openv{}} {\e{2}} {\v{}}
              and {\judgementtwo{}{\hastype{\v{}}{\t{}}}} by the induction hypothesis on {\e{2}}.


      \end{subcase}
    \item[]
      \begin{subcase}[B-IfFalse]
        \opsem {\openv{}} {\e{1}} {\false}
        \ \ \text{or}\ \ 
        \opsem {\openv{}} {\e{1}} {\nil},
        \opsem {\openv{}} {\e{3}} {\v{}}

              For part 1, either \object{} = \emptyobject{}, or \e{} evaluates to the
              result of \e{3}.

              To prove part 2, we consider two cases:
              \begin{itemize}
                \item if \v{} = \false\ (or \nil)
                  then \e{3} evaluates to a false value so {\satisfies{\openv{}}{\elseprop {\prop{3}}}}, and thus
                  {\satisfies{\openv{}}{\orprop {\elseprop {\prop{2}}} {\elseprop {\prop{3}}}}} by M-Or, 
                \item otherwise
                  ${\v{}} \not= {\false}$ and
                  ${\v{}} \not= {\nil}$,
                  so \e{3} evaluates to a true value so {\satisfies{\openv{}}{\thenprop {\prop{3}}}}, and thus
                  {\satisfies{\openv{}}{\orprop {\thenprop {\prop{2}}} {\thenprop {\prop{3}}}}} by M-Or.
              \end{itemize}

              Part 3 is trivial as
              \opsem {\openv{}} {\e{3}} {\v{}}
              and {\judgementtwo{}{\hastype{\v{}}{\t{}}}} by the induction hypothesis on {\e{3}}.
      \end{subcase}
    \item[]
      \begin{subcase}[BE-If]
      \end{subcase}
  \end{itemize}
\end{case}

\begin{case}[T-Let]
  \e{} = {\letexp {\x{}} {\e{1}} {\e{2}}},
  \judgement {\propenv{}} {\hastype {\e{1}} {\s{}}} {\filterset {\thenprop {\prop{1}}} {\elseprop {\prop{1}}}}
             {\object{1}},
             \propp{} = {\impprop {\notprop {\falsy{}} {\x{}}} {\thenprop {\prop{1}}}},
             \proppp{} = {\impprop {\isprop {\falsy{}} {\x{}}} {\elseprop {\prop{1}}}},
  \judgement
       {\propenv{}, {\isprop {\s{}} {\x{}}},
         {\propp{}},
         {\proppp{}}}
             {\hastype {\e{2}} {\t{}}} {\filterset {\thenprop {\prop{}}} {\elseprop {\prop{}}}}
             {\object{}} 


  \begin{itemize}
    \item[]
      \begin{subcase}[B-Let]
        \opsem {\openv{}} {\e{1}} {\v{1}},
        \opsem {\extendopenv{\openv{}}{\x{}}{\v{1}}} {\e{2}} {\v{}}

        For all the following cases (with a reminder that \x{} is fresh)
        we apply the induction hypothesis on \e{2}. We justify this by noting
        that occurrences of \x{} inside \e{2} have the same type as \e{1} and 
        simulate the propositions of \e{1}
        because 
        \opsem {\openv{}} {\e{1}} {\v{1}},
        and
        \opsem {\extendopenv{\openv{}}{\x{}}{\v{1}}} {\e{2}} {\v{}},
        so \satisfies{\openv{}}{\propenv{}, {\isprop {\s{}} {\x{}}}, \propp{}, \proppp{}},
        by M-And.

        We prove parts 1, 2 and 3 by directly using the induction hypothesis on \e{2}.
      \end{subcase}
    \item[]
      \begin{subcase}[BE-Let]
      \end{subcase}
  \end{itemize}
\end{case}

\begin{case}[T-Clos] \e{} = {\closure {\openv{}} {\abs {\x{}} {\s{}} {\e{1}}}},
  {\thenprop {\prop{}}}
  $\exists {\propenvp{}}. \satisfies{\openv{}}{\propenvp{}}$
  \ \text{and}\ 
\judgement {\propenvp{}} {\hastype {\abs {\x{}} {\s{}} {\e{1}}} {\t{}}}
                 {\filterset {\thenprop {\prop{}}}
                             {\elseprop {\prop{}}}}
                 {\object{}}

  \begin{itemize}
    \item[]
      \begin{subcase}[B-Abs] \v{} = {\closure {\openv{}} {\abs {\x{}} {\s{}} {\e{1}}}}

        We assume some \propenvp{}, such that
        \begin{itemize}
          \item \satisfies{\openv{}}{\propenvp{}}
          \item \judgement {\propenvp{}} {\hastype {\abs {\x{}} {\s{}} {\e{1}}} {\t{}}}
                           {\filterset {\thenprop {\prop{}}}
                                       {\elseprop {\prop{}}}}
                           {\object{}}.
       \end{itemize}
       Note the last rule in the derivation of
          \judgement {\propenvp{}} {\hastype {\abs {\x{}} {\s{}} {\e{1}}} {\t{}}}
                           {\filterset {\thenprop {\prop{}}}
                                       {\elseprop {\prop{}}}}
                           {\object{}}
                           must be T-Abs, so 
                           {\thenprop {\prop{}}} = {\topprop{}},
                           {\elseprop {\prop{}}} = {\botprop{}}
                           and {\object{}} = {\emptyobject{}}.
         Thus parts 1 and 2 hold for the same reasons as T-True.
         Part 3 holds as \v{} has the same type as {\abs {\x{}} {\s{}} {\e{1}}}
         under \propenvp{}.

      \end{subcase} 
  \end{itemize}
\end{case}

\begin{case}[T-Multi] \e{} = {\multi {\v{1}} {\curlymapvaloverright{\v{k}}{\v{v}}}},
  \t{} = {\MultiFntype {\s{}} {\t{1}}},
  {\thenprop {\prop{}}} = {\topprop{}},
  {\elseprop{\prop{}}} = {\botprop{}},
  {\object{}} = {\emptyobject{}},
  \judgementtwo {} {\hastype {\v{1}} {\t{1}}},
  $\overrightarrow{\judgementtwo{}{\hastype{\v{k}}{\Top{}}}}$,
  $\overrightarrow{\judgementtwo{}{\hastype{\v{v}}{\s{}}}}$

  \begin{itemize}
    \item[]
      \begin{subcase}[B-Val] \v{} = {\multi {\v{1}} {\curlymapvaloverright{\v{k}}{\v{v}}}}

        Similar to T-True.
      \end{subcase}
  \end{itemize}

\end{case}

\begin{case}[T-Abs] \e{} = {\abs {\x{}} {\s{}} {\e{1}}},
  \t{} = {\ArrowOne {\x{}} {\s{}}
                                                      {\t{1}}
                                                      {\filterset {\thenprop {\prop{1}}}
                                                                  {\elseprop {\prop{1}}}}
                                                      {\object{1}}},
  {\thenprop{\prop{}}}= {\topprop{}},
  {\elseprop{\prop{}}}= {\botprop{}},
  {\object{}}= {\emptyobject{}},
{ \judgement {\propenv{}, {\isprop {\s{}} {\x{}}}}
            {\hastype {\e{1}} {\t{}}}
             {\filterset {\thenprop {\prop{1}}}
                         {\elseprop {\prop{1}}}}
             {\object{1}}},
\judgement {\propenv{}} {\hastype {\abs {\x{}} {\s{}} {\e{1}}} {\t{}}}
                 {\filterset {\thenprop {\prop{}}}
                             {\elseprop {\prop{}}}}
                 {\object{}}

  \begin{itemize}
    \item[]
      %TODO
      \begin{subcase}[B-Abs]
        \v{} = ${\closure {\openv{}} {\abs {\x{}} {\s{}} {\e{1}}}}$,
          { \opsem {\openv{}}
                   {\abs {\x{}} {\t{}} {\e{1}}}
                   {\closure {\openv{}} {\abs {\x{}} {\s{}} {\e{1}}}}}

        Parts 1 and 2 hold for the same reasons as T-True.
        Part 3 holds directly via T-Clos, since \v{} must be a closure.
      \end{subcase}
  \end{itemize}
\end{case}

\begin{case}[T-Error]
  \e{} = \errorval{\v{1}},
  \t{} = \Bot,
  \thenprop{\prop{}} = \botprop{}, \elseprop{\prop{}} = \botprop{}, \object{} = \emptyobject{}


  \begin{itemize}
    \item[]
      \begin{subcase}[BE-Error] 
        \opsem {\openv{}} {\e{}} {\errorval{\v{1}}}

        Trivially reduces to an error.
      \end{subcase}
  \end{itemize}
\end{case}

\begin{case}[T-Subsume]
  \judgement {\propenv{}} {\hastype {\e{}} {\tp{}}}
             {\filterset {\thenprop {\propp{}}}
                         {\elseprop {\propp{}}}}
             {\objectp{}},
\inpropenv {\propenv{}, {\thenprop {\propp{}}}} {\thenprop {\prop{}}},
\inpropenv {\propenv{}, {\elseprop {\propp{}}}} {\elseprop {\prop{}}},
\inpropenv {} {\issubtype {\tp{}} {\t{}}},
\inpropenv {} {\issubtype {\objectp{}} {\object{}}}

Part 1 holds because \objectp{} is the object of \e{}
and \inpropenv {} {\issubtype {\objectp{}} {\object{}}}.
Part 2 holds because 
the then and else propositions of \e{} are
{\thenprop {\propp{}}} and 
{\elseprop {\propp{}}} respectively, so
{\thenprop {\prop{}}} and
{\elseprop {\prop{}}} are also respectively as 
\inpropenv {\propenv{}, {\thenprop {\propp{}}}} {\thenprop {\prop{}}} and
\inpropenv {\propenv{}, {\elseprop {\propp{}}}} {\elseprop {\prop{}}}.
Part 3 holds because \e{} is of type \tp{} and 
\inpropenv {} {\issubtype {\tp{}} {\t{}}}.

\end{case}

\begin{case}[T-Const]\e{} = {\const{}},
  \t{} = {\constanttype{\const{}}},
{\thenprop{\prop{}}} = {\topprop{}},
{\elseprop{\prop{}}} = {\botprop{}},
{\object{}} = {\emptyobject{}}

  \begin{itemize}
    \item[] 
      \begin{subcase}[B-Val]
        Parts 1, 2 and 3 hold for the same reasons as T-True. 
      \end{subcase}
  \end{itemize}
\end{case}

\end{proof}

\end{lemma}

\begin{theorem}[Well-typed programs don't go wrong] \label{theorem:wrong}
  If {\judgement {} {\hastype {\e{}} {\t{}}} {\filterset {\thenprop{\prop{}}}{\elseprop{\prop{}}}} {\object{}}}
  then \notopsem{}{\e{}}{\wrong{}}.

\begin{proof}
  Corollary of lemma \ref{lemma:soundness}, since 
  when {\judgement {} {\hastype {\e{}} {\t{}}} {\filterset {\thenprop{\prop{}}}{\elseprop{\prop{}}}} {\object{}}}
    , by lemma \ref{lemma:soundness}, either
    \opsem {} {\e{}} {\v{}} or
    \opsem {} {\e{}} {\errorvalv{}}, therefore \notopsem{}{\e{}}{\wrong{}}.
\end{proof}
\end{theorem}

\begin{theorem}[Type soundness] \label{theorem:soundness}
If \judgement{\propenv{}}{\hastype{\e{}}{\t{}}}{\filterset{\thenprop{\prop{}}}{\elseprop{\prop{}}}}{\object{}}
and \opsem {\openv{}} {\e{}} {\v{}} 
then \judgement{}{\hastype{\v{}}{\t{}}}{\filterset{\thenprop{\propp{}}}{\elseprop{\propp{}}}}{\objectp{}}
        for some \thenprop{\propp{}}, \elseprop{\propp{}} and {\objectp{}}

\begin{proof}
  Corollary of lemma \ref{lemma:soundness}.
\end{proof}

\end{theorem}


\begin{figure*}
$$
\begin{altgrammar}
  v &::=&
                      \x{}
                      \alt
                      \hinted{\tatype{}} {\x{}}
                      \alt \true{} \alt \false{} \alt \nil{}
                      \alt \class{}
                &\mbox{Values} \\
  \expd{}, \e{} &::=& v \alt
                      {\comb {\e{}} {\e{}}} \alt {\abs {\x{}} {\t{}} {\e{}}}
                      \alt
                      {\tryexp {\e{}} {\class} {\x} {\e{}} {\e{}}}
                      \alt 
                      {\ifexp {\e{}} {\e{}} {\e{}}}
                      \\ &\alt& 
                      {\doexp {\e{}} {\e{}}}
                      %{\fnexp {\overrightarrow{\hastype {\x{}} {\t{}}}} {\e{}}}
                      \alt
                      {\letexp {\x{}} {\e{}} {\e{}}}
                      \alt {\fieldexp {fld} {\e{}}}
                      \alt {\methodexp {mth} {\e{}} {\overrightarrow{\e{}}}}
                      \alt {\newexp {\class{}} {\overrightarrow{\e{}}}}
                &\mbox{External Language} \\
  \expd{}, \e{} &::=& \ldots\ {\methodstaticexp {\tatype{}} {\overrightarrow{\tatype{}}} {\tatype{}} {mth} {\e{}} {\overrightarrow{\e{}}}}
                      \alt {\newstaticexp {\overrightarrow {\tatype{}}} {\tatype{}} {\class{}} {\overrightarrow{\e{}}}}
                      \alt {\fieldstaticexp {\tatype{}} {\tatype{}} {fld} {\e{}}}
                &\mbox{Internal Language} \\ \\
  \s{}, \t{}    &::=& \Top \alt (\class{}\ \overrightarrow{\t{}})
                      \alt \class\ 
                      \alt \protocol\ 
                      \alt {\Instance {\class}}
                      \alt {\Satisfies {\protocol}}
                      \\ &\alt&
                      {\Value \v{}}
                      \alt {\Unionsplice {\overrightarrow{\t{}}}}
                      \alt
                      {\In {\t{}\ \overrightarrow{\t{}}}}
                      \alt 
                      %\alt (\Not\ \t{})
                      {\All {\overrightarrow{\x{}}} {\t{}}}
                      \alt
                      {\TFn {\overrightarrow{\x{}}} \t{}}
                      \\ &\alt&
                      {\ArrowOverright {\x{}} {\t{}}
                                   {\t{}}
                                   {\filterset {\prop{}} {\prop{}}}
                                   {\object{}}}
                      \alt 
                      {\ArrayTwo {\t{}} {\t{}}}
                      \alt {\HMapp {\mandatory} {\absent}}
                      \alt
                      {\HMapc {\mandatory}}
                      
                &\mbox{Typed Clojure Types} \\
  \tatype{}     &::=& \classhint{} \alt \primitivehint{}
                &\mbox{Class and Primitive Hints} \\
  \tatypeunknown{}     &::=& \unknownhint{} \alt \tatype{}
                &\mbox{tools.analyzer Types} \\ \\
  \prop{}       &::=& {\isprop {\t{}} {\path {\pathelem{}} {\x{}}}}
                      \alt {\notprop {\t{}} {\path {\pathelem{}} {\x{}}}}
                      \alt {\impprop {\prop{}} {\prop{}}}
                      \alt {\andprop {\prop{}} {\prop{}}}
                      \alt {\orprop {\prop{}} {\prop{}}}
                      \alt \topprop{}
                      \alt \botprop{}
                &\mbox{Propositions} \\
  \object{}     &::=& {\path {\pathelem{}} {\x{}}}
                      \alt \emptyobject{}
                &\mbox{Objects} \\
  \pathelem{}   &::=& \overrightarrow{\pesyntax{}}
                &\mbox{Paths} \\
  \pesyntax{}   &::=& \classpe{} \alt \keype{k} \alt \countpe{}
                &\mbox{Path Elements} \\ \\
                % TODO objects with path
%  \c{} &::=& \assoc{} \alt \dissoc{} \alt \get{}
%              &\mbox{Constants}
  \propenv{}   &::=& \overrightarrow{\prop{}}
               &\mbox{Proposition Environment} \\
  \taenv{}    &::=& \{ \overrightarrow{ \hastype{\x{}} {\tatype{}}} \}
               &\mbox{tools.analyzer Environment} \\
  \ctentrymeta{} &::=& \ctentry
               &\mbox{Class descriptors} \\
  \ct{}   &::=& \{ \overrightarrow{\classhint{} \Rightarrow \ctentrymeta{}} \}
               &\mbox{Class Table}
\end{altgrammar}
$$
\caption{Syntax of Terms, Types, Propositions, and Objects}
\end{figure*}


\begin{figure*}
\begin{mathpar}
\infer [T-Local]
{ \inpropenv {\propenv{}} {\isprop {\t{}} {\x{}}}
  \\
  \s{} = {\falsydiff} }
{ \judgementfillcol {\propenv{}} 
                   {\hastype {\x{}} {\t{}}}
                   {\filterset {{\notprop {\s{}} {\x{}}}} {{\isprop {\s{}} {\x{}}}}}
                   {\x{}}
                   }

\infer [T-Const]
{}
{ \judgement {\propenv{}} 
             {\hastype {\const{}} {\constanttype{\const{}}}}
             {\filterset {\topprop{}}{\botprop{}}}
             {\emptyobject{}}
                   }

\infer [T-True]
{}
{ \judgementfillcol {\propenv{}}
      {{\hastype {\true{}} {\True{}}}}
      {{\filterset {\topprop{}} {\botprop{}}}}
      {\emptyobject{}}
    }

\infer [T-False]
{}
{ \judgementfillcol {\propenv{}}
      {{\hastype {\false{}} {\False{}}}}
      {\filterset {\botprop{}} {\topprop{}}}
    {\emptyobject{}}
   }

\infer [T-Nil]
{}
{ \judgementfillcol {\propenv{}} 
      {{\hastype {\nil{}} {\Nil{}}}} 
      {\filterset {\botprop{}} {\topprop{}}}
      {\emptyobject{}} 
    }

\infer [T-Do]
{ 
  \judgement {\propenv{}} 
             {\hastype {\e1} {\t1}} 
             {\filterset {\thenprop {\prop{1}}} {\elseprop {\prop1}}} 
             {\object{1}}
\\\\
\judgement {\propenv{}, {\trdiff {\orprop {\thenprop {\prop{1}}} {\elseprop {\prop{1}}}}}}
           {\hastype {\e{}} {\t{}}} 
           {\filterset {\thenprop {\prop{}}} {\elseprop {\prop{}}}} 
           {\object{}}
  }
{ \judgement
    {\propenv{}} 
    {\hastype {\doexp {\e1} {\e{}}} {\t{}}} 
    {\filterset {\thenprop {\prop{}}} {\elseprop {\prop{}}}} {\object{}} 
             }

\infer [T-If]
{ \judgement {\propenv{}} {\hastype {\e1} {\t{1}}} {\filterset {\thenprop {\prop{1}}} {\elseprop {\prop{1}}}}
                 {\object{1}} 
             \\\\
  \judgement {\propenv{}, {\thenprop {\prop{1}}}}
                 {\hastype {\e2} {\t{}}} {\filterset {\thenprop {\prop{2}}} {\elseprop {\prop{2}}}}
                 {\object{}}
  \\\\
  \judgement {\propenv{}, {\elseprop {\prop{1}}}}
                 {\hastype {\e3} {\t{}}} {\filterset {\thenprop {\prop{3}}} {\elseprop {\prop{3}}}}
                 {\object{}}
             }
{ \judgement {\propenv{}} {\hastype {\ifexp {\e1} {\e2} {\e3}} {\t{}}} 
                 {\filterset {\orprop {\thenprop {\prop{2}}} {\thenprop {\prop{3}}}} 
                             {\orprop {\elseprop {\prop{2}}} {\elseprop {\prop{3}}}}}
                 {\object{}} }

\infer [T-Let]
{ \judgement {\propenv{}} {\hastype {\e{1}} {\s{}}} {\filterset {\thenprop {\prop{1}}} {\elseprop {\prop{1}}}}
             {\object{1}}
             \\\\
             \propp{} = {\impprop {\notprop {\falsydiff{}} {\x{}}} {\thenprop {\prop{1}}}}
             \\\\
             \proppp{} = {\impprop {\isprop {\falsydiff{}} {\x{}}} {\elseprop {\prop{1}}}}
  \\\\
  \judgement
       {\propenv{}, {\isprop {\s{}} {\x{}}},
         {\propp{}},
         {\proppp{}}}
             {\hastype {\e{}} {\t{}}} {\filterset {\thenprop {\prop{}}} {\elseprop {\prop{}}}}
             {\object{}} 
             }
{ \judgement {\propenv{}} {\hastype {\letexp {\x{}} {\e{1}} {\e{}}} {\t{}}}
             {\replacefor {\filterset {\thenprop {\prop{}}} {\elseprop {\prop{}}}}
                          {\object{1}}
                          {\x{}}}
             {\replacefor {\object{}} 
                          {\object{1}}
                          {\x{}}}
             }

\infer [T-App]
{ \judgement {\propenv{}} {\hastype {\e{}} {\ArrowOne {\x{}} {\s{}}
                                                       {\t{}}
                                                       {\filterset {\thenprop {\prop{f}}}
                                                                   {\elseprop {\prop{f}}}}
                                                       {\object{f}}}}
                {\filterset {\thenprop {\prop{}}}
                            {\elseprop {\prop{}}}}
                {\object{}}
  \\\\
  \judgement {\propenv{}}
                 {\hastype {\ep{}} {\s{}}}
                 {\filterset {\thenprop {\propp{}}}
                             {\elseprop {\propp{}}}}
                 {\objectp{}} 
}
{ \judgementfillcol {\propenv{}} {\hastype {\appexp {\e{}} {\ep{}}}
                                        {\replacefor {\t{}}
                                                     {\objectp{}}
                                                     {\x{}}}}
                 {\replacefor {\filterset {\thenprop {\prop{f}}}
                                          {\elseprop {\prop{f}}}}
                              {\objectp{}}
                              {\x{}}}
                 {\replacefor {\object{f}}
                              {\objectp{}}
                              {\x{}}}
}
                 
                              


\infer [T-Abs]
{ \judgement {\propenv{}, {\isprop {\s{}} {\x{}}}}
            {\hastype {\e{}} {\t{}}}
             {\filterset {\thenprop {\prop{}}}
                         {\elseprop {\prop{}}}}
             {\object{}}}
{ \judgement {\propenv{}} {\hastype {\abs {\x{}} {\s{}} {\e{}}}
                                           {\ArrowOne {\x{}} {\s{}}
                                                      {\t{}}
                                                      {\filterset {\thenprop {\prop{}}}
                                                                  {\elseprop {\prop{}}}}
                                                      {\object{}}}}
                {\filterset {\topprop{}}
                            {\botprop{}}}
                {\emptyobject{}}}

\infer [T-Clos]
{ \exists {\propenv{}}. \satisfies{\openv{}}{\propenv{}}
  \ \text{and}\ 
\judgement {\propenv{}} {\hastype {\abs {\x{}} {\t{}} {\e{}}} {\s{}}}
                 {\filterset {\thenprop {\prop{}}}
                             {\elseprop {\prop{}}}}
                 {\object{}}
              }
{ \judgement {}
            {\hastype {\closure {\openv{}} {\abs {\x{}} {\t{}} {\e{}}}} 
                      {\s{}}}
             {\filterset {\thenprop {\prop{}}}
                         {\elseprop {\prop{}}}}
             {\object{}}}

\infer [T-Error]
{}
{ \judgement {\propenv{}} {\hastype {\errorvalv} {\Bot}}
            {\filterset {\botprop{}} {\botprop{}}}
           {\emptyobject{}}}

\infer [T-Subsume]
{ \judgement {\propenv{}} {\hastype {\e{}} {\t{}}}
             {\filterset {\thenprop {\prop{}}}
                         {\elseprop {\prop{}}}}
             {\object}
\\\\
\inpropenv {\propenv{}, {\thenprop {\prop{}}}} {\thenprop {\propp{}}}
\\
\inpropenv {\propenv{}, {\elseprop {\prop{}}}} {\elseprop {\propp{}}}
\\\\
\inpropenv {} {\issubtype {\t{}} {\tp{}}}
\\
\inpropenv {} {\issubtype {\object{}} {\objectp{}}}
}
{ \judgement {\propenv{}} {\hastype {\e{}} {\tp{}}}
             {\filterset {\thenprop {\propp{}}}
                         {\elseprop {\propp{}}}}
             {\objectp{}}}
\end{mathpar}
\caption{Standard Typing Rules}
\end{figure*}


\begin{figure*}
\begin{mathpar}

\infer [T-NewStatic]
{ 
  \overrightarrow{
\javatotc {\classhint{i}}
          {\t{i}}
          }
  \\\\
  \javatotc {\classhint{}}
            {\t{}}
  \\
  \overrightarrow{
  \judgementtwo {\propenv{}}
                    {\hastype {\e{i}} {\t{i}}}
                  }
           }
{ \judgementfillcol {\propenv{}} {\hastype {\newstaticexp {\overrightarrow{\classhint{i}}} {\classhint{}} 
                                                          {\class{}} {\overrightarrow{\e{i}}}} {\t{}}}
             {\filterset {\topprop{}} {\botprop{}}}
             {\emptyobject}
             }

\infer [T-FieldStatic]
{ \javatotc {\classhint{1}} {\class{}}
             \\\\
  \javatotcnil {\classhint{2}} {\t{}}
  \\\\
  \judgementtwo {\propenv{}} {\hastype {\e{1}} {\class{}}}
           }
{ \judgement {\propenv{}} {\hastype {\fieldstaticexp {\classhint{1}} {\classhint{2}} {\fld{}} {\e{1}}} {\t{}}}
             {\filterset {\topprop{}} {\topprop}}
             {\emptyobject{}}}

\infer [T-MethodStatic]
{ 
  \overrightarrow{\javatotc {\classhint{i}} {\t{i}}}
             \\
             \javatotc {\classhint{1}} {\class{}}
             \\\\
             \javatotcnil {\classhint{2}} {\t{}}
             \\\\
  \judgementtwo {\propenv{}} {\hastype {\e{}} {\class{}}}
             \\
             \overrightarrow{
  \judgementtwo {\propenv{}} {\hastype {\e{i}} {\t{i}}}
                  }
           }
{ \judgementfillcol {\propenv{}} {\hastype {\methodstaticexp {\classhint{1}} 
                                                            {\overrightarrow {\classhint{i}}} 
                                                            {\classhint{2}}
                                                            {\mth{}} {\e{}} {\overrightarrow{\e{i}}}}
                                    {\t{}}}
             {\filterset {\topprop{}} {\topprop{}}}
             {\emptyobject{}}}

\infer [T-Class]
{}
{ \judgement {\propenv{}} {\hastype{\class{}}{\Value{\class{}}}}{\filterset{\topprop{}}{\botprop{}}}{\emptyobject{}}}

\infer [T-Instance]
{}
{ \judgement {\propenv{}} {\hastype{\classvalue{\classhint{}} {\overrightarrow {\classfieldpair{\fld{}} {\v{}}}}}{\class{}}}{\filterset{\topprop{}}{\botprop{}}}{\emptyobject{}}}

\end{mathpar}
\caption{Java Interop Typing Rules}
\end{figure*}

\begin{figure*}
\begin{mathpar}

\infer [T-DefMulti]
{ \s{} = {\ArrowOne {\x{}} {\t{}} {\tp{}}
                          {\filterset {\thenprop {\prop{}}}
                                      {\elseprop {\prop{}}}}
                          {\object{}}}
                          \\
  \sp{} = {\ArrowOne {\x{}} {\t{}} {\tpp{}}
                          {\filterset {\thenprop {\propp{}}}
                                      {\elseprop {\propp{}}}}
                          {\objectp{}}}
                \\\\
\judgementtwo {\propenv{}} {\hastype {\e{}} {\sp{}}}
}
{
  \judgement {\propenv{}}  
      {\hastype {\createmultiexp {\s{}}
                                 {\e{}}} 
                {\MultiFntype {\s{}} {\sp{}}}}
             {\filterset {\topprop{}} {\botprop{}}}
             {\emptyobject{}}
}

\infer [T-DefMethod]
{
  \t{m} = {\ArrowOne {\x{}} {\t{}} {\s{}}
                                    {\filterset {\thenprop {\prop{}}}
                                                {\elseprop {\prop{}}}}
                                    {\object{}}}
\\
\t{d} = {\ArrowOne {\x{}} {\t{}} {\sp{}}
                   {\filterset {\thenprop {\propp{}}}
                               {\elseprop {\propp{}}}}
                   {\objectp{}}}
             \\\\
\judgementtwo {\propenv{}}
                  {\hastype {\e{m}} {\MultiFntype {\t{m}} {\t{d}}}}
\\
  \isacompare{\sp{}}{\objectp{}}{\t{v}}{\filterset {\thenprop {\proppp{}}} {\elseprop {\proppp{}}}}
\\\\
\judgementtwo {\propenv{}}
           {\hastype {\e{v}} {\t{v}}}
           \\
\judgement {\propenv{}, {\isprop{\t{}} {\x{}}}, {\thenprop {\proppp{}}}}
           {\hastype {\e{b}} {\s{}}}
           {\filterset {\thenprop {\prop{}}}
                       {\elseprop {\prop{}}}}
           {\object{}}
}
{ \judgementfillcol {\propenv{}} {\hastype {\extendmultiexp {\e{m}} {\e{v}} 
                                                            {\abs {\x{}} {\t{}} {\e{b}}}}
                                           {\MultiFntype {\t{m}} {\t{d}}}}
             {\filterset {\topprop{}} {\botprop{}}}
             {\emptyobject{}}
}

\infer [T-IsA]
{
  \judgement {\propenv{}} {\hastype {\e{}} {\s{}}}
             {\filterset {\thenprop {\propp{}}}
                         {\elseprop {\propp{}}}}
                       {\object{}}
  \\
  \judgementtwo
             {\propenv{}} {\hastype {\ep{}} {\t{}}}
             \\
  \isacompare{\s{}}{\object{}}{\t{}}{\filterset {\thenprop {\prop{}}} {\elseprop {\prop{}}}}
}
{ \judgement {\propenv{}} {\hastype {\isaapp {\e{}} {\ep{}}}
                                    {\Boolean{}}}
             {\filterset {\thenprop {\prop{}}} {\elseprop {\prop{}}}}
             {\emptyobject}}

\infer [T-Multi]
{ \judgementtwo {} {\hastype {\v{}} {\t{}}}
  \\
  \overrightarrow{\judgementtwo{}{\hastype{\v{k}}{\Top{}}}}
  \\
  \overrightarrow{\judgementtwo{}{\hastype{\v{v}}{\s{}}}}
}
{ \judgement {}
  {\hastype {\multi {\v{}} {\curlymapvaloverright{\v{k}}{\v{v}}}}
                      {\MultiFntype {\s{}} {\t{}}}}
             {\filterset {\topprop{}} {\botprop{}}}
           {\emptyobject{}}}

\end{mathpar}
\caption{Multimethod Typing Rules}
\end{figure*}

\begin{figure*}
\begin{mathpar}

%\infer [T-Get]
%{ \judgementtwo {\propenv{}} {\hastype {\e{m}} {\Map {\t{k}}{\t{v}}}}
%  \\
%  \judgementtwo {\propenv{}} {\hastype {\e{k}} {\Top}}}
%{ \judgement {\propenv{}} {\hastype {\getexp {\e{m}} {\e{k}}} {\Union{\t{v}}{\nil{}}}}
%             {\filterset {\topprop{}} {\topprop{}}}
%           {\emptyobject{}}}
%
%\infer [T-Assoc]
%{ 
%  \judgementtwo {\propenv{}} {\hastype {\e{m}} {\Map{\t{k}}{\t{v}}}}
%  \\
%  \judgementtwo {\propenv{}} {\hastype {\e{k}} {\t{k}}}
%  \\
%  \judgementtwo {\propenv{}} {\hastype {\e{v}} {\t{v}}}
%}
%{ \judgement {\propenv{}} 
%             {\hastype {\assocexp {\e{m}} {\e{k}} {\e{v}}} {\Map {\t{k}}{\t{v}}}}
%             {\filterset {\topprop{}} {\botprop{}}}
%             {\emptyobject{}}
%}

\infer [T-HMap]
{ \overrightarrow{\judgementtwo {} {\hastype{\v{k}}{\Value \k{}}}}\\
  \overrightarrow{\judgementtwo {} {\hastype{\v{v}}{\t{v}}}}\\
  \mandatory{} = \mandatorysetoverright{\k{}}{\t{v}}
}
{ \judgement {}
             {\hastype {\curlymapvaloverright{\v{k}}{\v{v}}}
                       {\HMapc {\mandatory{}}}}
             {\filterset {\topprop{}} {\botprop{}}}
             {\emptyobject{}}}

\infer [T-Kw]
{}
{ \judgementfillcol {\propenv{}} 
      {\hastype {\k{}} {\Value{\k{}}}}
      {\filterset {\topprop{}}{\botprop{}}}
      {\emptyobject{}} 
    }

\infer [T-GetHMap]
{ \judgement {\propenv{}} 
  {\hastype {\e{m}} {\Unionsplice {{\overrightarrow {\HMapgeneric {\mandatory{}} {\absent{}}}}^i}}}
           {\filterset {\thenprop {\prop{}}} {\elseprop {\prop{}}}}
           {\object{}}
           \\\\
  \judgementtwo {\propenv{}} {\hastype {\e{k}} {\Value {k}}}
             \\
             {\overrightarrow{\inmandatory{\k{}}{\t{}}{\mandatory{}}}}^i
       }
{ \judgement {\propenv{}} {\hastype {\getexp {\e{m}} {\e{k}}} {\Unionsplice {{\overrightarrow {\t{}}}^i}}}
             {\filterset {\topprop{}} {\topprop{}}}
             {\replacefor {\path {\keype{k}} {\x{}}}
                          {\object{}}
                          {\x{}}}
                                  }

\infer [T-GetHMapAbsent]
{ \judgement {\propenv{}} {\hastype {\e{m}} {\HMapgeneric {\mandatory{}} {\absent}}}
           {\filterset {\thenprop {\prop{}}} {\elseprop {\prop{}}}}
           {\object{}}
         \\\\
  \judgementtwo {\propenv{}} {\hastype {\e{k}} {\Value {k}}}
             \\
             {\inabsent{\k{}}{\absent{}}}
       }
{ \judgement {\propenv{}} {\hastype {\getexp {\e{m}} {\e{k}}}
                                    {\Nil{}}}
             {\filterset {\topprop{}} {\topprop{}}}
             {\replacefor
               {\path {\keype{k}} {\x{}}}
                          {\object{}}
                          {\x{}}}
                        }

\infer [T-GetHMapPartialDefault]
{ \judgement {\propenv{}} {\hastype {\e{m}} {\HMapp {\mandatory{}} {\absent}}}
           {\filterset {\thenprop {\prop{}}} {\elseprop {\prop{}}}}
           {\object{}}
         \\\\
  \judgementtwo {\propenv{}} {\hastype {\e{k}} {\Value {k}}}
             \\
             {\notinmandatory{\k{}}{\t{}}{\mandatory{}}}
             \\
             {\notinabsent{\k{}}{\absent{}}}
       }
{ \judgement {\propenv{}} {\hastype {\getexp {\e{m}} {\e{k}}}
                                    {\Top{}}}
             {\filterset {\topprop{}} {\topprop{}}}
             {\replacefor
               {\path {\keype{k}} {\x{}}}
                          {\object{}}
                          {\x{}}}
                        }

\infer [T-AssocHMap]
{ 
  \judgementtwo {\propenv{}} {\hastype {\e{m}} {\HMapgeneric {\mandatory{}} {\absent}}}
  \\
  \judgementtwo {\propenv{}} {\hastype {\e{k}} {\Value{\k{}}}}
  \\
  \judgementtwo {\propenv{}} {\hastype {\e{v}} {\t{}}}
  \\
  {\k{}} \not\in {\absent{}}
}
{ \judgement {\propenv{}} 
             {\hastype {\assocexp {\e{m}} {\e{k}} {\e{v}}}
                       {\HMapgeneric {\extendmandatoryset {\mandatory{}}{\k{}}{\t{}}} {\absent}}}
             {\filterset {\topprop{}} {\botprop{}}}
             {\emptyobject{}}
}

\end{mathpar}
\caption{Map Typing Rules}
\end{figure*}



\begin{figure*}
\begin{mathpar}

\infer [TA-Local]
{ \intaenv {\taenv{}} {\x{}} {\tatype{}}
}
{ \tajudgement {\taenv{}} {\hastype {\x{}} {\tatype{}}}
}

\infer [TA-Nil]
{}
{ \tajudgement {\taenv{}} {\hastype {\nil{}} {\unknownhint{}}}
}

\infer [TA-True]
{}
{ \tajudgement {\taenv{}} {\hastype {\true{}} {\Booleanhint{}}}
}

\infer [TA-False]
{}
{ \tajudgement {\taenv{}} {\hastype {\false{}} {\Booleanhint{}}}
}

\infer [TA-Kw]
{}
{ \tajudgement {\taenv{}} {\hastype {\k{}} {\Keyword{}}}
}

\infer [TA-Class]
{}
{ \tajudgement {\taenv{}} {\hastype{\class{}}{\Class{}}}}

% never in user code
%\infer [T-Instance]
%{}
%{ \tajudgement {\taenv{}} {\hastype{\classvalue{\classhint{}} {\overrightarrow {\classfieldpair{\fld{}} {\v{}}}}}{\unknownhint{}}}}

\infer [TA-NewStatic]
{}
{ \tajudgement {\taenv{}}
  {\hastype {\newstaticexp {\overrightarrow {\classhint{i}}} {\classhint{1}} {\class{}} {\overrightarrow {\e{}}}}
                           {\classhint{1}}}
}

\infer [TA-NewRefl]
{}
{ \tajudgement {\taenv{}}
  {\hastype {\newexp {\class{}} {\overrightarrow {\e{}}}}
                     {\unknownhint{}}}
}

\infer [TA-Abs]
{}
{ \tajudgement {\taenv{}}
               {\hastype {\abs {\x{}} {\t{}} {\e{}}}
                         {\unknownhint{}}}
                       }

\infer [TA-LetHint]
{ \tajudgement {\taenv{}, {\hastype {\x{}} {\classhint{}}}}
               {\hastype {\e{}} {\tatype{}}}
}
{ \tajudgement {\taenv{}}
      {\hastype {\letexp {\hinted{\classhint{}}{\x{}}}
                                  {\e{1}}
                                  {\e{}}}
                         {\tatype{}}}
}

\infer [TA-Let]
{ \tajudgement {\taenv{}} 
               {\hastype {\e{1}} {\tatype{1}}}
               \\\\
  \tajudgement {\taenv{}, {\hastype {\x{}} {\tatype{1}}}}
               {\hastype {\e{}} {\tatype{}}}
}
{ \tajudgement {\taenv{}}
               {\hastype {\letexp {\x{}}
                                  {\e{1}}
                                  {\e{}}}
                         {\tatype{}}}
}

\infer [TA-FieldRefl]
{}
{ \tajudgement {\taenv{}}
               {\hastype {\fieldexp {\fld} {\e{}}}
                         {\unknownhint{}}}
  }

\infer [TA-FieldStatic]
{}
{ \tajudgement {\taenv{}}
               {\hastype {\fieldstaticexp {\classhint{1}} {\classhint{}} {\fld{}} {\e{}}}
                         {\classhint{}}}
  }

\infer [TA-MethodRefl]
{}
{ \tajudgement {\taenv{}}
               {\hastype {\methodexp {\mth} {\e{}} {\overrightarrow {\e{i}}}}
                         {\unknownhint{}}}
  }

\infer [TA-MethodStatic]
{}
{ \tajudgement {\taenv{}}
               {\hastype {\methodstaticexp {\classhint{1}} 
                                           {\overrightarrow{\classhint{i}}} 
                                           {\classhint{}} 
                                         {\mth{}} {\e{}} {\overrightarrow {\e{i}}}}
                         {\classhint{}}}
  }

\infer [TA-App]
{}
{ \tajudgement {\taenv}
               {\hastype {\appexp {\e{}} {\ep{}}}
                         {\unknownhint{}}}
                       }

\infer [TA-AppLocal]
{ \intaenv {\taenv{}} {\x{}} {\tatype{}}
}
{ \tajudgement {\taenv}
               {\hastype {\appexp {\x{}} {\ep{}}}
                         {\tatype{}}}
                       }

\infer [TA-Do]
{ \tajudgement {\taenv{}} {\hastype {\e{}} {\tatype{}}}
}
{ \tajudgement {\taenv{}} {\hastype {\doexp {\e1} {\e{}}} {\tatype{}}}
}

\infer [TA-DefMulti]
{}
{ \tajudgement {\taenv{}} {\hastype {\createmultiexp {\t{}} {\e{}}} {\unknownhint{}}}}

\infer [TA-DefMethod]
{}
{ \tajudgement {\taenv{}} {\hastype {\extendmultiexp {\e{1}} {\e{2}} {\e{3}}} {\unknownhint{}}}}

\infer [TA-If]
{ \tajudgement {\taenv{}} {\hastype {\e{2}} {\classhint{}}}
  \\\\
  \tajudgement {\taenv{}} {\hastype {\e{3}} {\classhint{}}}
}
{ \tajudgement {\taenv{}} {\hastype {\ifexp {\e{1}} {\e{2}} {\e{3}}} {\classhint{}}}}

\infer [TA-IfUnknown]
{}
{ \tajudgement {\taenv{}} {\hastype {\ifexp {\e{1}} {\e{2}} {\e{3}}} {\unknownhint{}}}}

\infer [TA-Isa]
{}
{ \tajudgement {\taenv{}} {\hastype {\isaapp {\e{}} {\e{}}} {\Booleanhint{}}}}

\infer [TA-Const]
{}
{ \tajudgement {\taenv{}} {\hastype {\const{}} {\unknownhint{}}}}

\end{mathpar}
\caption{tools.analyzer Type Hints}
\end{figure*}

\begin{figure*}
\begin{mathpar}

\RLocal{}

\infer [R-LocalHint]
{}
{ \rewrite {\taenv{}}
  {\hinted {\classhint{}} {\x{}}}
  {\hinted {\classhint{}} {\x{}}}
         }

\infer [R-Val]
{}
{ \rewrite {\taenv{}}
  {\v{}}
  {\v{}}
}

\infer [R-NewRefl]
{ \overrightarrow{
  \rewrite {\taenv{}}
           {\e{i}}
           {\e{j}}
         }
         }
{ \rewrite {\taenv{}}
           {\newexp {\class{}} {\overrightarrow {\e{i}}}}
           {\newexp {\class{}} {\overrightarrow {\e{j}}}}
}

\RNewElimRefl{}

\infer [R-NewStatic]
{ \overrightarrow
  {\rewrite {\taenv{}}
           {\e{i}}
           {\e{j}}}
         }
{ \rewrite {\taenv{}}
           {\newstaticexp {\overrightarrow {\classhint{p}}} {\classhint{}} {\class{}} {\overrightarrow {\e{i}}}}
           {\newstaticexp {\overrightarrow {\classhint{p}}} {\classhint{}} {\class{}} {\overrightarrow {\e{j}}}}
}

\RAbs{}

         \RLet{}

\RLetHint{}

\infer [R-FieldRefl]
{ \rewrite {\taenv{}}
           {\e{}}
           {\ep{}}
         }
{ \rewrite {\taenv{}}
           {\fieldexp {\fld{}} {\e{}}}
           {\fieldexp {\fld{}} {\ep{}}}
         }

\RFieldElimRefl{}

\infer [R-FieldStatic]
{ \rewrite {\taenv{}}
           {\e{}}
           {\ep{}}
         }
{ \rewrite {\taenv{}}
           {\fieldstaticexp {\classhint{1}} {\classhint{2}} {\fld{}} {\e{}}}
           {\fieldstaticexp {\classhint{1}} {\classhint{2}} {\fld{}} {\ep{}}}
         }

\infer [R-MethodRefl]
{ \rewrite {\taenv{}}
           {\e{}}
           {\ep{}}
           \\
           \overrightarrow
           {
  \rewrite {\taenv{}}
           {\e{i}}
           {\e{j}}}
         }
{ \rewrite {\taenv{}}
           {\methodexp {\mth} {\e{}} {\overrightarrow {\e{i}}}}
           {\methodexp {\mth} {\ep{}} {\overrightarrow {\e{j}}}}
  }

\RMethodElimRefl{}

\infer [R-MethodStatic]
{ \rewrite {\taenv{}}
           {\e{}}
           {\ep{}}
           \\
  \rewrite {\taenv{}}
           {\overrightarrow {\e{i}}}
           {\overrightarrow {\e{j}}}
         }
{ \rewrite {\taenv{}}
           {\methodstaticexp {\classhint{}} {\overrightarrow {\classhint{1}}} {\classhint{2}} {\mth{}} {\e{}} {\overrightarrow {\e{i}}}}
           {\methodstaticexp {\classhint{}} {\overrightarrow {\classhint{1}}} {\classhint{2}} {\mth{}} {\ep{}} {\overrightarrow {\e{j}}}}
  }

\infer [R-App]
{ \rewrite {\taenv{}} {\e{}} {\ep{}}
  \\\\
  \rewrite {\taenv{}} {\e{i}} {\e{j}}
}
{ \rewrite {\taenv{}}
           {\appexp {\e{}} {\e{i}}}
           {\appexp {\ep{}} {\e{j}}}
         }

\end{mathpar}
\caption{Java Reflection Resolution}
\end{figure*}


\begin{figure*}
\begin{mathpar}
\objectsub{}

\standardsubtyping{}
\Multisubtyping{}

\HMapsubtyping{}
\end{mathpar}
\caption{Subtyping rules}
\end{figure*}


%$$
%\begin{tdisplay}{Evaluation Contexts}
%  \begin{altgrammar}
%    \E{} &::=& [ ] % application rules
%              \alt (\c{}\ \overrightarrow{\v{}}\ \E{}\ \overrightarrow{\exp{}}) % eval arguments left-to-right
%              % map rules
%              \alt \{\overrightarrow{\v{}\ \v{}}\ \E{}\ \exp{}\ \overrightarrow{\exp{}\ \exp{}} \} % key first
%              \alt \{\overrightarrow{\v{}\ \v{}}\ \v{}\ \E{}\ \overrightarrow{\exp{}\ \exp{}} \}   % value next
%              &\mbox{Evaluation Contexts}
%  \end{altgrammar}
%\end{tdisplay}
%$$ 

\begin{figure*}
$$
\begin{altgrammar}

  \methodtypealign {\ \{ methods \Rightarrow \{ \mth{} \Rightarrow [ \overrightarrow{\classhint{p}}, \classhint{r} ] \} \} }
              {\mth{}}
              {\overrightarrow{\classhint{p}}}
              {\classhint{r}}
      \\\\
  \fieldtypealign {\ \{ fields \Rightarrow \{ \fld{} \Rightarrow \classhint{} \} \} }
             {\fld{}}
             {\classhint{}}
      \\\\
  \ctorparamsalign {\ \{ constructors \Rightarrow \{ ctor \Rightarrow \classhint{} \} \} }
             {\fld{}}
             {\classhint{}}

%{\javaspecial {\JavaField {\class{}}}} &=& {\javatotc {\class{}} {\true{}}}
%\\
%{\javaspecial {\JavaMethod {\overrightarrow{{\class{1}}}} {\class{2}}}} &=& 
%    {\Arrow {\overrightarrow {\hastype {\x{}} {\javatotc {\class{1}} {\false{}}}}}
%            {\javatotc {\class{2}} {\true{}}}
%            {\filterset {\topprop{}} {\topprop{}}}
%            {\emptyobject{}}
%            }
%\\
%{\javaspecial {\JavaCtor {\overrightarrow{{\class{1}}}} {\class{2}}}} &=& 
%    {\Arrow {\overrightarrow {\hastype {\x{}} {\javatotc {\class{1}} {\false{}}}}}
%            {\javatotc {\class{2}} {\false{}}}
%            {\filterset {\topprop{}} {\topprop{}}}
%            {\emptyobject{}}
%            }
%\\\\
\\\\

\javatotcalign {\Void{}} {\Nil}
\\
\javatotcalignnil {\Void{}} {\Nil}
\\
\javatotcalign {\class{}} \class{}
\\
\javatotcalignnil {\class{}} {\Union {\Nil} {\class{}}}

\end{altgrammar}
$$
\caption{Converting Java types to Typed Clojure types}
\end{figure*}

\begin{figure*}
\begin{mathpar}

\begin{array}{lllr}

  \constanttype{\classconst} &=& {\ArrowOne {\x{}} {\Top{}}
                                      {\Union{\nil{}}{\Class{}}}
                                      {\filterset {\topprop{}}
                                                  {\topprop{}}}
                                      {\path {\classpe{}} {\x{}}}}
                                      \\

  \constanttype{\throwconst} &=& {\ArrowOne {\x{}} {\Top{}}
                                      {\Bot{}}
                                      {\filterset {\botprop{}}
                                                  {\botprop{}}}
                                      {\emptyobject{}}}

\end{array}
\end{mathpar}
\caption{Constant Typing}
\end{figure*}

\begin{figure*}
\begin{mathpar}

\begin{array}{lllr}

\constantopsem{\classconst}{\classvalue{\class{}} {\overrightarrow {\classfieldpair{\fld{}} {\v{}}}}} &=& \class{}\\
\constantopsem{\classconst}{\true{}} &=& \Boolean{}\\
\constantopsem{\classconst}{\false{}} &=& \Boolean{}\\
\constantopsem{\classconst}{\const{}} &=& \IFn{}\\
\constantopsem{\classconst}{\curlymap{\overrightarrow{({\v{}}\ {\v{}})}}} &=& \HMapInstance{}\\
\constantopsem{\classconst}{\class{}} &=& \Class{}\\
\constantopsem{\classconst}{\k{}} &=& \Keyword{}\\
\constantopsem{\classconst}{\nil{}} &=& \nil{}
                                      \\\\

\constantopsem{\throwconst}{\v{}} &=& \errorval{\v{}}


\end{array}
\end{mathpar}
\caption{Primitives}
\end{figure*}


\begin{figure*}
$$
\begin{array}{lclr}

\isacomparealign {\s{}} {\object{}} {\t{}}
                {\replacefor
                  {\filterset {\isprop{\t{}}  {\x{}}}
                              {\notprop{\t{}} {\x{}}}}
                  {\object{}}
                  {\x{}}}
                            %\\
                            % Other case: 2 equal sized HVecs
%\isacomparealign {} {\x{}} {\t{}}
%                {\filterset {\isprop{\t{}}  {\x{}}}
%                            {\notprop{\t{}} {\x{}}}}
\end{array}
$$
$$
\begin{array}{lclr}
  \isaopsem{\classvaluemeta{1}}{\classvaluemeta{2}} &=& {\true{}} & \classvaluemeta{1}\ \text{is a subclass of}\ \classvaluemeta{2}\\
  \isaopsem{\classvaluemeta{1}}{\classvaluemeta{2}} &=& {\false{}}\\
  \isaopsem{\v{}}{\v{}} &=& {\true{}}\\
  \isaopsem{\v{1}}{\v{2}} &=& {\false{}} & \v{1} \not= \v{2}
\end{array}
$$
\caption{Definition of isa}
\end{figure*}

\begin{figure*}
$$
\begin{array}{lr}

  \getmethod {\disptable{}}
             {\v{e}}
             {\v{f}}

             & 
             \text{let } {\overrightarrow{\v{fs}}} = \{ \v{f} | (\v{v}, \v{f}) \in \disptable{} \text{ and } \isaopsem{\v{v}}{\v{e}} = \true{} \}, \text{ if } {\overrightarrow{\v{fs}}} = \{\v{f}\}

             \\
  \getmethod {\disptable{}}
             {\v{e}}
             {\mmerror{}} & \text{otherwise}

\end{array}
$$
\caption{Definition of get-method}
\end{figure*}


\clearpage

\begin{figure*}
\begin{mathpar}

\infer [B-Local]
{ \inopenv {\openv{}} {\x{}} {\v{}} }
{ \opsem {\openv{}} {\x{}} {\v{}} }

\infer [B-Do]
{ \opsem {\openv{}} {\e{1}} {\v{1}} 
  \\\\
  \opsem {\openv{}} {\e{}} {\v{}} 
}
{ \opsem {\openv{}} {\doexp{\e{1}}{\e{}}} {\v{}} }

\infer [B-Let]
{ \opsem {\openv{}} {\e{a}} {\v{a}}
  \\\\
  \opsem {\extendopenv{\openv{}}{\x{}}{\v{a}}}
         {\e{}}
         {\v{}}
 }
{ \opsem {\openv{}} 
         {\letexp {\x{}} {\e{a}} {\e{}}}
         {\v{}}}

\infer [B-Val]
{}
{ \opsem {\openv{}} {\v{}} {\v{}}}

\infer [B-IfTrue]
{ \opsem {\openv{}} {\e{1}} {\v{1}}
  \\\\
  {\v{1}} \not= {\false{}}
  \\
  {\v{1}} \not= {\nil{}}
  \\\\
  \opsem {\openv{}} {\e{2}} {\v{}}
}
{ \opsem {\openv{}}
         {\ifexp {\e1} {\e2} {\e3}}
         {\v{}}
       }

\infer [B-IfFalse]
{  \opsem {\openv{}} {\e{1}} {\false{}}
  \ \ \text{or}\ \ 
  \opsem {\openv{}} {\e{1}} {\nil{}}
  \\\\
  \opsem {\openv{}} {\e{3}} {\v{}}
}
{ \opsem {\openv{}}
         {\ifexp {\e1} {\e2} {\e3}}
         {\v{}}
       }

\infer [B-Abs]
{}
{ \opsem {\openv{}}
         {\abs {\x{}} {\t{}} {\e{}}}
         {\closure {\openv{}} {\abs {\x{}} {\t{}} {\e{}}}}}

\infer [B-BetaClosure]
{ \opsem {\openv{}}
         {\e{f}}
         {\closure {\openv{c}} {\abs {\x{}} {\t{}} {\e{b}}}}
         \\\\
  \opsem {\openv{}}
         {\e{a}}
         {\v{a}}
         \\\\
  \opsem {\extendopenv {\openv{c}} {\x{}} {\v{a}}}
         {\e{b}}
         {\v{}}
       }
{ \opsem {\openv{}}
         {\appexp {\e{f}} {\e{a}}}
         {\v{}}
       }

\infer [B-Delta]
{ \opsem {\openv{}} {\e{}} {\const{}}
  \\\\
  \opsem {\openv{}} {\ep{}} {\v{}}
  \\\\
  \constantopsem{\const{}}{\v{}} = \vp{}
}
{ \opsem {\openv{}}
         {\appexp {\e{}} {\ep{}}}
         {\vp{}}
       }

\infer [B-BetaMulti]
{ \opsem {\openv{}}
         {\e{f}}
         {\multi {\v{d}} {\disptable{}}}
         \\\\
  \opsem {\openv{}}
         {\e{a}}
         {\v{a}}
         \\\\
  \opsem {\openv{}}
         {\appexp {\v{d}} {\v{a}}}
         {\v{e}}
         \\\\
  \getmethod {\disptable{}}
             {\v{e}}
             {\v{f}}
         \\\\
  \opsem {\openv{}}
         {\appexp {\v{f}} {\v{a}}}
         {\v{}}
       }
{ \opsem {\openv{}}
         {\appexp {\e{f}} {\e{a}}}
         {\v{}}
       }

\infer [B-Field]
{ \opsem {\openv{}}
         {\e{}} 
       {\v{1}}
         \\\\
         \getfieldjava{\classhint{1}} {\v{1}} {\fld{}} {\classhint{2}} {\v{}}
       }
{ \opsem {\openv{}}
         {\fieldstaticexp {\classhint{1}} {\classhint{2}} {\fld{}} {\e{}}}
         {\v{}}
   }

\infer [B-Method]
{ \opsem {\openv{}}
         {\e{m}}
         {\v{m}}
  \\
  \overrightarrow{
  \opsem {\openv{}}
         {\e{a}}
         {\v{a}}
       }
  \\\\
  \invokejavamethod {\classhint{1}} {\v{m}} {mth}
                    {\overrightarrow{\classhint{a}}} {\overrightarrow{\v{a}}}
                    {\classhint{2}}
                    {\v{}}
}
{\opsem {\openv{}}
        {\methodstaticexp {\classhint{1}} {\overrightarrow{\classhint{a}}} {\classhint{2}} {mth} {\e{m}} {\overrightarrow{\e{a}}}}
        {\v{}}
      }

\infer [B-New]
{ 
  \overrightarrow{
  \opsem {\openv{}}
         {\e{i}}
         {\v{i}}
       }
         \\\\
         \newjava {\classhint{1}}
                  {\overrightarrow{\classhint{i}}}
                  {\overrightarrow{\v{i}}}
                  {\v{}}
       }
{ \opsem {\openv{}}
         {\newstaticexp {\overrightarrow{\classhint{i}}} {\classhint{1}} 
                        {\class{}} {\overrightarrow{\e{i}}}}
         {\v{}}}

\infer [B-DefMulti]
{ \opsem {\openv{}} {\e{d}} {\v{d}}
}
{\opsem {\openv{}}
        {\createmultiexp {\t{}}
                         {\e{d}}}
        {\multi {\v{d}} {\emptydisptable}}
}

\infer [B-DefMethod]
{ \opsem {\openv{}}
         {\e{m}}
         {\multi {\v{d}} {\disptable{}}}
         \\
  \opsem {\openv{}}
         {\e{v}}
         {\v{v}}
         \\\\
  \opsem {\openv{}}
         {\e{f}}
         {\v{f}}
         \\
         \disptablep{} = {\extenddisptable {\disptable{}} 
                                {\v{v}}
                                {\v{f}}}
}
{\opsem {\openv{}}
        {\extendmultiexp {\e{m}} {\e{v}} {\e{f}}}
        {\multi {\v{d}} {\disptablep{}}}
      }

\infer [B-IsA]
{ \opsem {\openv{}} {\e{1}} {\v{1}}
  \\\\
  \opsem {\openv{}} {\e{2}} {\v{2}}
  \\\\
  \isaopsem{\v{1}}{\v{2}} = {\v{}}
}
{\opsem {\openv{}} {\isaapp {\e{1}} {\e{2}}} {\v{}}}

\infer [B-MethodRefl]
{}
{\opsem {\openv{}} {\methodexp {mth} {\e{}} {\overrightarrow{\e{}}}}
        {\reflerror{}}}

\infer [B-FieldRefl]
{}
{\opsem {\openv{}} {\fieldexp {\fld{}} {\e{}}}
        {\reflerror{}}}

\infer [B-NewRefl]
{}
{\opsem {\openv{}} {\fieldexp {\fld{}} {\e{}}}
        {\reflerror{}}}

\infer [B-Assoc]
{\opsem {\openv{}}
        {\e{m}} {\curlymap{\overrightarrow{({\v{a}}\ {\v{b}})}}}
        \\
 \opsem {\openv{}} {\e{k}} {\v{k}}
        \\\\
 \opsem {\openv{}} {\e{v}} {\v{v}}
}
{
 \opsem {\openv{}}
        {\assocexp {\e{m}} {\e{k}} {\e{v}}} 
        {\extendmap{\curlymap{\overrightarrow{({\v{a}}\ {\v{b}})}}}
                {\v{k}}{\v{v}}}
                }

\infer [B-Get]
{\opsem {\openv{}}
        {\e{m}} {\curlymap{\overrightarrow{({\v{a}}\ {\v{b}})}}}
        \\\\
 \opsem {\openv{}}
        {\e{k}} {\v{k}}
        \\\\
 \keyinmap{\v{k}}{\curlymap{\overrightarrow{({\v{a}}\ {\v{b}})}}}
        \\\\
 \getmap{\curlymap{\overrightarrow{({\v{a}}\ {\v{b}})}}}
        {\v{k}}
        =
        {\v{}}
}
{
 \opsem {\openv{}}
        {\getexp {\e{m}} {\e{k}}}
        {\v{}}
}

\infer [B-GetMissing]
{\opsem {\openv{}}
        {\e{m}} {\curlymap{\overrightarrow{({\v{a}}\ {\v{b}})}}}
        \\\\
 \opsem {\openv{}}
        {\e{k}} {\v{k}}
        \\\\
 \keynotinmap{\v{k}}{\curlymap{\overrightarrow{({\v{a}}\ {\v{b}})}}}
}
{
 \opsem {\openv{}}
        {\getexp {\e{m}} {\e{k}}}
        {\nil{}}
}

\end{mathpar}
\caption{Operational Semantics}
\end{figure*}
\begin{figure*}
\begin{mathpar}
\infer [BE-ErrorWrong]
{}
{ \opsem {\openv{}} 
         {\wrongorerror{}}
         {\wrongorerror{}}}

\infer [BE-Let]
{ \opsem {\openv{}} {\e{a}} {\wrongorerror{}}
 }
{ \opsem {\openv{}} 
         {\letexp {\x{}} {\e{a}} {\e{}}}
       {\wrongorerror{}}}

\infer [BE-Do1]
{ \opsem {\openv{}} {\e{1}} {\wrongorerror{}} }
{ \opsem {\openv{}} {\doexp{\e{1}}{\e{}}} {\wrongorerror{}}}

\infer [BE-Do2]
{ \opsem {\openv{}} {\e{1}} {\v{1}} 
  \\\\
  \opsem {\openv{}} {\e{}}  {\wrongorerror{}}
}
{ \opsem {\openv{}} {\doexp{\e{1}}{\e{}}} {\wrongorerror{}} }

\infer [BE-If]
{  \opsem {\openv{}} {\e{1}} {\wrongorerror{}}
}
{ \opsem {\openv{}}
         {\ifexp {\e1} {\e2} {\e3}}
         {\wrongorerror{}}
       }

\infer [BE-IfTrue]
{ \opsem {\openv{}} {\e{1}} {\v{1}}
  \\\\
  {\v{1}} \not= {\false{}}
  \\
  {\v{1}} \not= {\nil{}}
  \\\\
  \opsem {\openv{}} {\e{2}} {\wrongorerror{}}
}
{ \opsem {\openv{}}
         {\ifexp {\e1} {\e2} {\e3}}
         {\wrongorerror{}}
       }

\infer [BE-IfFalse]
{  \opsem {\openv{}} {\e{1}} {\false{}}
  \ \ \text{or}\ \ 
  \opsem {\openv{}} {\e{1}} {\nil{}}
  \\\\
  \opsem {\openv{}} {\e{3}} {\wrongorerror{}}
}
{ \opsem {\openv{}}
         {\ifexp {\e1} {\e2} {\e3}}
         {\wrongorerror{}}
       }

\infer [BE-Beta1]
{ \opsem {\openv{}}
         {\e{f}}
         {\wrongorerror{}}
       }
{ \opsem {\openv{}}
         {\appexp {\e{f}} {\e{a}}}
         {\wrongorerror{}}
       }

\infer [BE-Beta2]
{ \opsem {\openv{}}
         {\e{f}}
         {\v{f}}
         \\\\
  \opsem {\openv{}}
         {\e{a}}
         {\wrongorerror{}}
       }
{ \opsem {\openv{}}
         {\appexp {\e{f}} {\e{a}}}
         {\wrongorerror{}}
       }

\infer [BE-BetaClosure]
{ \opsem {\openv{}}
         {\e{f}}
         {\closure {\openv{c}} {\abs {\x{}} {\t{}} {\e{b}}}}
         \\\\
  \opsem {\openv{}}
         {\e{a}}
         {\v{a}}
         \\\\
  \opsem {\extendopenv {\openv{c}} {\x{}} {\v{a}}}
         {\e{b}}
         {\wrongorerror{}}
       }
{ \opsem {\openv{}}
         {\appexp {\e{f}} {\e{a}}}
         {\wrongorerror{}}
       }

\infer [BE-BetaMulti1]
{ \opsem {\openv{}}
         {\e{f}}
         {\multi {\v{d}} {m}}
         \\\\
  \opsem {\openv{}}
         {\e{a}}
         {\v{a}}
         \\\\
  \opsem {\openv{}}
         {\appexp {\v{d}} {\v{a}}}
         {\wrongorerror{}}
       }
{ \opsem {\openv{}}
         {\appexp {\e{f}} {\e{a}}}
         {\wrongorerror{}}
       }

\infer [BE-BetaMulti2]
{ \opsem {\openv{}}
         {\e{f}}
         {\multi {\v{d}} {m}}
         \\\\
  \opsem {\openv{}}
         {\e{a}}
         {\v{a}}
         \\\\
  \opsem {\openv{}}
         {\appexp {\v{d}} {\v{a}}}
         {\v{e}}
         \\\\
  \getmethod {\disptable{}}
             {\v{e}}
             {\wrongorerror{}}
       }
{ \opsem {\openv{}}
         {\appexp {\e{f}} {\e{a}}}
         {\wrongorerror{}}
       }


\infer [BE-Field]
{ \opsem {\openv{}}
         {\e{}} 
         {\wrongorerror{}}
       }
{ \opsem {\openv{}}
         {\fieldstaticexp {\classhint{1}} {\classhint{2}} {\fld{}} {\e{}}}
         {\wrongorerror{}}
   }

\infer [BE-Method1]
{ \opsem {\openv{}}
         {\e{m}}
         {\wrongorerror{}}
}
{\opsem {\openv{}}
        {\methodstaticexp {\classhint{1}} {\overrightarrow{\classhint{a}}} {\classhint{2}} {mth} {\e{m}} {\overrightarrow{\e{}}}}
        {\wrongorerror{}}
      }

\infer [BE-Method2]
{ \opsem {\openv{}}
         {\e{m}}
         {\v{m}}
  \\\\
  \overrightarrow{
  \opsem {\openv{}}
         {\e{n-1}}
         {\v{n-1}}
       }
         \\\\
  \opsem {\openv{}}
         {\e{n}}
         {\wrongorerror{}}
}
{\opsem {\openv{}}
        {\methodstaticexp {\classhint{1}} {\overrightarrow{\classhint{a}}} {\classhint{2}} {mth} {\e{m}} {\overrightarrow{\e{}}}}
        {\wrongorerror{}}
      }

\infer [BE-Method3]
{ \opsem {\openv{}}
         {\e{m}}
         {\v{m}}
  \\
  \overrightarrow{
  \opsem {\openv{}}
         {\e{a}}
         {\v{a}}
       }
  \\\\
  \invokejavamethod {\classhint{1}} {\v{m}} {mth}
                    {\overrightarrow{\classhint{a}}} {\overrightarrow{\v{a}}}
                    {\classhint{2}}
                    {\wrongorerror{}}
}
{\opsem {\openv{}}
        {\methodstaticexp {\classhint{1}} {\overrightarrow{\classhint{a}}} {\classhint{2}} {mth} {\e{m}} {\overrightarrow{\e{a}}}}
        {\wrongorerror{}}
      }

\infer [BE-New1]
{ \overrightarrow{
  \opsem {\openv{}}
         {\e{n-1}}
         {\v{n-1}}
       }
       \\\\
  \opsem {\openv{}}
         {\e{n}}
         {\wrongorerror{}}
       }
{ \opsem {\openv{}}
         {\newstaticexp {\overrightarrow{\classhint{i}}} {\classhint{1}} 
                        {\class{}} {\overrightarrow{\e{}}}}
         {\wrongorerror{}}
       }

\infer [BE-New2]
{ 
  \overrightarrow{
  \opsem {\openv{}}
         {\e{i}}
         {\v{i}}
       }
         \\\\
         \newjava {\classhint{1}}
                  {\overrightarrow{\classhint{i}}}
                  {\overrightarrow{\v{i}}}
                  {\wrongorerror{}}
       }
{ \opsem {\openv{}}
         {\newstaticexp {\overrightarrow{\classhint{i}}} {\classhint{1}} 
                        {\class{}} {\overrightarrow{\e{i}}}}
         {\wrongorerror{}}}

\infer [BE-DefMulti]
{ \opsem {\openv{}} {\e{d}} {\wrongorerror{}}
}
{\opsem {\openv{}}
        {\createmultiexp {\t{}}
                         {\e{d}}}
        {\wrongorerror{}}
}

\infer [BE-DefMethod1]
{ \opsem {\openv{}}
         {\e{m}}
         {\wrongorerror{}}
}
{\opsem {\openv{}}
        {\extendmultiexp {\e{m}} {\e{v}} {\e{f}}}
        {\wrongorerror{}}
      }

\infer [BE-DefMethod2]
{ \opsem {\openv{}}
         {\e{m}}
         {\multi {\v{d}} {\disptable{}}}
         \\\\
  \opsem {\openv{}}
         {\e{v}}
         {\wrongorerror{}}
}
{\opsem {\openv{}}
        {\extendmultiexp {\e{m}} {\e{v}} {\e{f}}}
        {\wrongorerror{}}
      }

\infer [BE-DefMethod3]
{ \opsem {\openv{}}
         {\e{m}}
         {\multi {\v{d}} {\disptable{}}}
         \\\\
  \opsem {\openv{}}
         {\e{v}}
         {\v{v}}
         \\\\
  \opsem {\openv{}}
         {\e{f}}
         {\wrongorerror{}}
}
{\opsem {\openv{}}
        {\extendmultiexp {\e{m}} {\e{v}} {\e{f}}}
         {\wrongorerror{}}
      }

\infer [BE-IsA1]
{ \opsem {\openv{}} {\e{1}} {\wrongorerror{}}
}
{\opsem {\openv{}} {\isaapp {\e{1}} {\e{2}}} {\wrongorerror{}}}

\infer [BE-IsA2]
{ \opsem {\openv{}} {\e{1}} {\v{1}}
  \\\\
  \opsem {\openv{}} {\e{2}} {\wrongorerror{}}
}
{\opsem {\openv{}} {\isaapp {\e{1}} {\e{2}}} {\wrongorerror{}}}

\infer [BE-Assoc1]
{\opsem {\openv{}}
        {\e{m}}{\wrongorerror{}} 
}
{
 \opsem {\openv{}}
        {\assocexp {\e{m}} {\e{k}} {\e{v}}} 
        {\wrongorerror{}}
                }

\infer [BE-Assoc2]
{\opsem {\openv{}}
        {\e{m}} {\curlymap{\overrightarrow{({\v{a}}\ {\v{b}})}}}
        \\
 \opsem {\openv{}}
        {\e{k}}{\wrongorerror{}}
}
{
 \opsem {\openv{}}
        {\assocexp {\e{m}} {\e{k}} {\e{v}}} 
        {\wrongorerror{}}
                }

\infer [BE-Assoc3]
{\opsem {\openv{}}
        {\e{m}} {\curlymap{\overrightarrow{({\v{a}}\ {\v{b}})}}}
        \\
 \opsem {\openv{}}
        {\e{k}} {\v{k}}
        \\
 \opsem {\openv{}}
        {\e{v}} {\wrongorerror{}}
}
{
 \opsem {\openv{}}
        {\assocexp {\e{m}} {\e{k}} {\e{v}}} 
        {\wrongorerror{}}
                }

\infer [BE-Get1]
{\opsem {\openv{}}
        {\e{m}} {\wrongorerror{}}
}
{
 \opsem {\openv{}}
        {\getexp {\e{m}} {\e{k}}}
        {\wrongorerror{}}
}

\infer [BE-Get2]
{\opsem {\openv{}}
        {\e{m}} {\curlymap{\overrightarrow{({\v{a}}\ {\v{b}})}}}
        \\
 \opsem {\openv{}}
        {\e{k}} {\wrongorerror{}}
}
{
 \opsem {\openv{}}
        {\getexp {\e{m}} {\e{k}}}
        {\wrongorerror{}}
}
\end{mathpar}
\caption{Error and wrong propagation}
\end{figure*}





\begin{figure*}
\begin{mathpar}

\begin{array}{lllll}
  {\openv{}}({\x{}}) &=& {\v{}}  & (\x{}\ \v{}) \in \openv{}\\
  {\openv{}}({\path {\keype{k}} {\object{}}}) &=& {\getexp {{\openv{}}(\object{})}{\k{}}}\\
  {\openv{}}({\path {\classpe{}} {\object{}}}) &=& {\appexp {\classconst{}} {{\openv{}}(\object{})}}

\end{array}

\end{mathpar}
\caption{Path translation}
\end{figure*}

\begin{figure*}
\begin{mathpar}

\begin{array}{lllr}
\update{\HMapp {\mandatory} {\absent}}{\propisnotmeta{}}{\destructpath {\pathelem{}} {\keype{\k{}}}}
&=&
{\HMapp {\mandatory} {\absent}}
\end{array}

\end{mathpar}
\caption{Type Update}
\end{figure*}


\begin{figure*}
\begin{mathpar}
\infer [M-Or]
{ \satisfies{\propenv{}}{\prop{1}}\ \text{or}\  \satisfies{\propenv{}}{\prop{2}}}
{ \satisfies{\propenv{}}{\orprop{\prop{1}}{\prop{2}}}
                   }

\infer [M-Imp]
{ \satisfies{\propenv{}}{\prop{1}}\ \text{implies}\ \satisfies{\propenv{}}{\prop{2}}}
{ \satisfies{\propenv{}}{\impprop{\prop{1}}{\prop{2}}}
                   }

\infer [M-And]
{ \satisfies{\propenv{}}{\prop{1}}
\\ \satisfies{\propenv{}}{\prop{2}}}
{ \satisfies{\propenv{}}{\andprop{\prop{1}}{\prop{2}}}
                   }


\infer [M-Top]
{}
{ \satisfies{\propenv{}}{\topprop{}}
                   }

                   \\

\infer [M-Type]
{ \judgement {} {\hastype {\openv{}({\path{\pathelem{}}{\x{}}})} {\t{}}}{\filterset{\thenprop{\prop{}}}{\elseprop{\prop{}}}}{\object{}}}
{ \satisfies{\propenv{}}{\isprop{\t{}}{\path{\pathelem{}}{\x{}}}}
                   }

\infer [M-NotType]
{ \judgement {} {\hastype {\openv{}({\path{\pathelem{}}{\x{}}})} {\s{}}}{\filterset{\thenprop{\prop{}}}{\elseprop{\prop{}}}}{\object{}}
\\\\
\text{there is no}\ \v{}\ \text{such that}\ \judgement{}{\hastype{\v{}}{\t{}}}{\filterset{\thenprop{\prop{1}}}{\elseprop{\prop{1}}}}{\object{1}}
\ \text{and}\ \judgement{}{\hastype{\v{}}{\s{}}}{\filterset{\thenprop{\prop{2}}}{\elseprop{\prop{2}}}}{\object{2}}
}
{ \satisfies{\propenv{}}{\notprop{\t{}}{\path{\pathelem{}}{\x{}}}}
                   }
\end{mathpar}
\caption{Satisfaction Relation}
\end{figure*}

\begin{figure*}
\begin{mathpar}
\infer [L-Atom]
{ {\prop{}} \in {\propenv{}}}
{ \inpropenv {\propenv{}} {\prop{}}
}

\infer [L-True]
{}
{ \inpropenv {\propenv{}} {\topprop{}}}

\infer [L-False]
{ \inpropenv {\propenv{}} {\botprop{}}}
{ \inpropenv {\propenv{}} {\prop{}}}

\infer [L-AndI]
{ \inpropenv {\propenv{}} {\prop{1}}
  \\\\
  \inpropenv {\propenv{}} {\prop{2}}}
{ \inpropenv {\propenv{}} {\andprop {\prop{1}}{\prop{2}}}}

\infer [L-AndE]
{ \inpropenv {\propenv{}, {\prop{1}}} {\prop{}}
  \ \text{or}\ 
  \inpropenv {\propenv{}, {\prop{2}}} {\prop{}}}
{ \inpropenv {\propenv{}, {\andprop {\prop{1}}{\prop{2}}}} {\prop{}}}

\infer [L-ImplI]
{ \inpropenv {\propenv{}, {\prop{1}}} {\prop{2}}}
{ \inpropenv {\propenv{}} {\impprop {\prop{1}}{\prop{2}}}}

\infer [L-ImplE]
{ \inpropenv {\propenv{}} {\prop{1}}
  \\\\
  \inpropenv {\propenv{}} {\impprop {\prop{1}}{\prop{2}}}}
{ \inpropenv {\propenv{}} {\prop{2}}}

\infer [L-OrI]
{ \inpropenv {\propenv{}} {\prop{1}}
  \ \text{or}\ 
  \inpropenv {\propenv{}} {\prop{2}}}
{ \inpropenv {\propenv{}} {\orprop {\prop{1}}{\prop{2}}}}


\infer [L-OrE]
{ \inpropenv {\propenv{}, {\prop{1}}}{\prop{}}
  \\\\
  \inpropenv {\propenv{}, {\prop{2}}}{\prop{}}}
{ \inpropenv {\propenv{}, {\orprop {\prop{1}}{\prop{2}}}}{\prop{}}}

\infer [L-Sub]
{ \inpropenv {\propenv{}} {\isprop {\t{}}{\path {\pathelem{}} {\x{}}}}
  \\
  \issubtypein {} {\t{}}{\s{}}
}
{ \inpropenv {\propenv{}} {\isprop {\s{}}{\path {\pathelem{}} {\x{}}}}}

\infer [L-SubNot]
{ \inpropenv {\propenv{}} {\notprop {\s{}}{\path {\pathelem{}} {\x{}}}}
  \\
  \issubtypein {} {\t{}}{\s{}}}
{ \inpropenv {\propenv{}} {\notprop {\t{}}{\path {\pathelem{}} {\x{}}}}}

\infer [L-Bot]
{ \inpropenv {\propenv{}} {\isprop {\Bot} {\path {\pathelem{}} {\x{}}}}}
{ \inpropenv {\propenv{}} {\prop{}}}

\infer [L-Update]
{ \inpropenv {\propenv{}} {\isprop {\t{}} {\path {\pathelemp{}} {\x{}}}}
\\
  \inpropenv {\propenv{}} {\isprop {\propisnotmeta{}} {{\pathelem{}}({\path {\pathelemp{}} {\x{}}})}}}
{ \inpropenv {\propenv{}} {\isprop {\update {\t{}} {\propisnotmeta{}} {\pathelem{}}} {\path {\pathelemp{}} {\x{}}}}}
\\

\text{(The metavariable \propisnotmeta{} ranges over \t{} and \nottype{\t{}} (without variables).)}

\end{mathpar}
\caption{Proof System}
\end{figure*}

\begin{figure*}
$$
\begin{array}{lclr}

{\replacefor
  {\filterset {\thenprop {\prop{}}}{\elseprop {\prop{}}}}
  {\object{}}
  {\x{}}}
  &=&
{\filterset 
  {\replacefor
    {\thenprop {\prop{}}}
    {\object{}}
    {\x{}}}
  {\replacefor
    {\elseprop {\prop{}}}
    {\object{}}
    {\x{}}}}
\\\\
{\replacefor
  {\isprop {\propisnotmeta{}} {\path {\pathelem{}} {\x{}}}}
  {\path {\pathelemp{}} {\y{}}}
  {\x{}}}
&=&
  {\isprop {({\replacefor
               {\propisnotmeta{}}
               {\path {\pathelemp{}} {\y{}}}
               {\x{}}})}
           {{\pathelem{}}({\path {\pathelemp{}} {\y{}}})}}
           \\

{\pluspolarity
{\replacefor
  {\isprop {\propisnotmeta{}} {\path {\pathelem{}} {\x{}}}}
  {\emptyobject{}}
  {\x{}}}
}
&=&
{\topprop{}}
\\
{\minuspolarity
{\replacefor
  {\isprop {\propisnotmeta{}} {\path {\pathelem{}} {\x{}}}}
  {\emptyobject{}}
  {\x{}}}
}
&=&
{\botprop{}}

\\
{\replacefor
  {\isprop {\propisnotmeta{}} {\path {\pathelem{}} {\x{}}}}
  {\object{}}
  {\z{}}}
&=&
  {\isprop {\propisnotmeta{}} {\path {\pathelem{}} {\x{}}}}
  & \x{} \not= \z{}\ \text{and}\ \z{} \not\in {\fv {\propisnotmeta{}}}

\\
{\pluspolarity
{\replacefor
  {\isprop {\propisnotmeta{}} {\path {\pathelem{}} {\x{}}}}
  {\object{}}
  {\z{}}}
}
&=&
{\topprop{}}
  & \x{} \not= \z{}\ \text{and}\ \z{} \in {\fv {\propisnotmeta{}}}
\\
{\minuspolarity
{\replacefor
  {\isprop {\propisnotmeta{}} {\path {\pathelem{}} {\x{}}}}
  {\object{}}
  {\z{}}}
}
&=&
{\botprop{}}
  & \x{} \not= \z{}\ \text{and}\ \z{} \in {\fv {\propisnotmeta{}}}

\\
{\replacefor
  {\topprop{}}
  {\object{}}
  {\x{}}}
&=&
  {\topprop{}}

\\
{\replacefor
  {\botprop{}}
  {\object{}}
  {\x{}}}
&=&
  {\botprop{}}

\\
{\pluspolarity
{\replacefor
  {({\impprop {\prop{1}} {\prop{2}}})}
  {\object{}}
  {\x{}}}
}
&=&
{\impprop 
  {\minuspolarity {\replacefor {\prop{1}} {\object{}} {\x{}}}}
  {\pluspolarity {\replacefor {\prop{2}} {\object{}} {\x{}}}}}
\\
{\minuspolarity
{\replacefor
  {({\impprop {\prop{1}} {\prop{2}}})}
  {\object{}}
  {\x{}}}
}
&=&
{\impprop 
  {\pluspolarity {\replacefor {\prop{1}} {\object{}} {\x{}}}}
  {\minuspolarity {\replacefor {\prop{2}} {\object{}} {\x{}}}}}
\\
{\replacefor
  {({\orprop {\prop{1}} {\prop{2}}})}
  {\object{}}
  {\x{}}}
&=&
{\orprop 
  {\replacefor {\prop{1}} {\object{}} {\x{}}}
  {\replacefor {\prop{2}} {\object{}} {\x{}}}}
\\
{\replacefor
  {({\andprop {\prop{1}} {\prop{2}}})}
  {\object{}}
  {\x{}}}
&=&
{\andprop 
  {\replacefor {\prop{1}} {\object{}} {\x{}}}
  {\replacefor {\prop{2}} {\object{}} {\x{}}}}

    \\\\

{\replacefor
  {\path {\pathelem{}} {\x{}}}
  {\path {\pathelemp{}} {\y{}}}
  {\x{}}}
           &=&
{\path{\pathelem{}}{\path {\pathelemp{}} {\y{}}}}

    \\

{\replacefor
  {\path {\pathelem{}} {\x{}}}
  {\emptyobject{}}
  {\x{}}}
           &=&
{\emptyobject{}}

    \\

{\replacefor
  {\path {\pathelem{}} {\x{}}}
  {\object{}}
  {\z{}}}
           &=&
{\path {\pathelem{}} {\x{}}}

& \x{} \not= \z{}
    \\

{\replacefor
  {\emptyobject{}}
  {\object{}}
  {\x{}}}
           &=&
{\emptyobject{}}

\end{array}
$$
\center{\text{Substitution on types is capture-avoiding structural recursion.}}
\caption{Substitution}
\end{figure*}



\end{document}

%                       Revision History
%                       -------- -------
%  Date         Person  Ver.    Change
%  ----         ------  ----    ------

%  2013.06.29   TU      0.1--4  comments on permission/copyright notices

