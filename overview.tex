\section{Overview of Typed Clojure}

Typed Clojure is a gradual type system for Clojure. It is designed
to type check normal Clojure code by adding annotations. It is implemented
as a library, and can be seamlessly included in any Clojure project; no
separate compiler or language is needed.

{\smallsection {Based on Typed Racket}}
Initially, the similarities between untyped Racket and Clojure and Typed Racket's 
ability to type check Racket code led us to investigate a similar type system for Clojure.
After two years of development, the solid basis of Typed Racket 
helps us type check many Clojure idioms without significant differences
in implementation or theory. We found that extending Typed Clojure to check
those idioms that have no obvious Racket equivalent did not significantly alter the structure
of the type system.

{\smallsection {Occurrence typing}}
\citet{TF08,TF10} developed \emph{occurrence typing}, which helps improve types at branches.
Typed Clojure uses occurrence typing in a similar way to Typed Racket, with
some extensions (discussed in in Section [?]). %FIXME

{\smallsection {Practical Variable-Arity Polymorphism}}
Functions with non-trivial variable parameters are common in Racket.
For example, Racket provides \emph{map} which takes a function and a
variable number of collections and applies the function simultaneously
to each element of the provided collections, returning a list of results.
\citet*{STF09} developed a practical system that handles advanced variable parameters
which can handle applications of functions like \emph{map}.

Clojure has a similar emphasis on variable-arity functions. In some ways,
Clojure's core library encourages even more complicated variable-parameter schemes.
The \emph{assoc} core function, for example, takes three parameters and
then a quantity of variable parameters that is a multiple of two.
This is beyond what Typed Racket (and Typed Clojure) can currently handle. 

Functions like \emph{map} are common in Clojure, so we provide an implementation
of variable-arity polymorphism which has similar capabilities as Typed Racket's
implementation.

{\smallsection {Local Type Inference}}
We use Pierce and Turner's Local Type Inference~\cite{PT00} to infer some polymorphic
applications. Our implementation is based on Typed Racket's, which has extensions
to support applications of polymorphic variable-arity functions like \emph{map}.

{\smallsection {Unions and intersections}}
Like Typed Racket, we include union and ordered intersection types. Unions define
a least-upper-bound for a set of types. For example, we can express a type that is
either \Number or \Symbol by including them in a union: {\Union {\Number} {\Symbol}}

Ordered intersections (described further by \citet{SA+12})
are used for overloading function types. We can express a function that takes
a \Number and returns a \Symbol, and vice-versa with an ordered intersection function type:

\begin{lstlisting}[label=lst:ordered]
(Fn [Number -> Symbol]
    [Symbol -> Number])
\end{lstlisting}

As our intersections are \emph{ordered}, we can express fine invariants in the
case where arity parameter types overlap. Similar to a pattern match, earlier arities 
are tried first, and the first arity to match ``wins''.

For example, applying an \lstinline|Integer| argument to a function of type

\begin{lstlisting}
(Fn [Integer -> Integer]
    [Number -> Number])
\end{lstlisting}

returns an \lstinline|Integer|. Reversing the arities however gives
type \lstinline|Number|, because the arity taking a \lstinline|Number|
always matches first.

{\smallsection {Hosted on the Java Virtual Machine}}
Clojure is built to run on the Java Virtual Machine (JVM),
offering good interoperability with existing Java code.
Typed Clojure helps programmers correctly call Java code
by integrating with Java's type system.

We give Java arrays and Java's \emph{null} special treatment
when involved with interoperability. Arrays are treated as \emph{read-only}
when sourced from Java methods, discussed in Section \ref{sec:arrays}.
We are explicit, and conservative by default, in the positions where
Java's \emph{null} can be passed, discussed in Section \ref{sec:null}.


\inputminted[firstline=1]{clojure}{code/demo/src/demo/eg1.clj}
\inputminted[firstline=4]{clojure}{code/demo/src/demo/eg2.clj}
\inputminted[firstline=4]{clojure}{code/demo/src/demo/eg3.clj}
\inputminted[firstline=4]{clojure}{code/demo/src/demo/eg4.clj}
\inputminted[firstline=4,lastline=15]{clojure}{code/demo/src/demo/eg5.clj}
\inputminted[firstline=4,lastline=15]{clojure}{code/demo/src/demo/eg6.clj}
\inputminted[firstline=6,lastline=23]{clojure}{code/demo/src/demo/eg7.clj}

% TODO references
% why is TR such a good base?
% - immutability
% - common lisp ancestry
% differences?
% - Clojure is built on JVM
% - interop with JVM
% - Clojure's idiomatic primitives are different
% - multimethods + protocols
% - less sophisticated macro system
%  - not an issue
%  - implementation difference, AST walking vs syntax walking
