\section{Metatheory}

Our proof for type soundness is similar to~\cite{TF10}. We add
errors and prove well-typed programs can never go wrong.

We make our assumptions about Java explicit. We concede that
method and constructor calls may diverge or error, but we assume they can
never go wrong.

{\javanewassumption{main}}

%For readability we define logical truth in Clojure.

%{\istruefalsedefinitions{main}}

We use an extra lemma to support our main soundness lemma. Consistency
ensures that occurrence typing does not claim the outer 
\mintinline{clojure}{x} is a keyword. 

\begin{minted}{clojure}
(let [x 1]
  ((let [x :a]
     (fn [] x))))
\end{minted}

{\consistentwithonlydef{main}}

Our main lemma says if there is a defined reduction, then the propositions, object
and type are correct.
The metavariable \definedreduction{} ranges over \v{}, \errorvalv{} and \wrong{}.

\begin{lemma}\label{main:lemma:soundness}

  {\soundnesslemmahypothesis}
  \begin{proof}
    By induction on the derivation of the typing judgement. 
    (Full proof given as theorem~\ref{appendix:lemma:soundness}).

  \end{proof}
\end{lemma}


We can now state our soundness theorem.

{\soundnesstheoremnoproof{main}}

{\wrongtheoremnoproof{main}}

Null-pointer exceptions are ruled-out in typed code.

{\nilinvoketheoremnoproof{main}}
